\chapter{Lezione 1 - Informatica giuridica e diritto dell'informatica.}


Gli argomenti di oggi saranno:
\begin{itemize}
    \item il diritto e la società dell'informazione
    \item l'ambito dell'informatica giuridica
\end{itemize}
Iniziamo dal primo argomento. 
\section{Il diritto e la società dell'informazione.} Intanto cerchiamo di definire cosa intendiamo per società dell'informazione. Molti hanno parlato, hanno definito la società dell'informazione o l'età della società dell'informazione come un parallelo, una nuova rivoluzione nel modo di produrre, scambiare beni e servizi, paragonabile appunto alla rivoluzione industriale. Esistono tuttavia alcune differenze notevoli tra la società dell'informazione, i cambiamenti in atto, la rivoluzione produttiva ed economica in atto nella società dell'informazione e la rivoluzione industriale così come l'abbiamo conosciuta nell'età moderna. \par
Il primo e più importante motivo, profilo di distinzione e differenza, riguarda appunto il concetto stesso di informazione o meglio nella società dell'informazione l'informazione è \textbf{la materia prima}, ciò su cui lavorano i soggetti della società dell'informazione è appunto l'informazione. 
La materia prima nella società dell'informazione, il bene, la risorsa che viene lavorata, raccolta, trasformata, scambiata non sono beni fisici, non è terra, sabbia o petrolio o carbone e quant'altro o altri artefatti, ma l'\textbf{informazione.} \par 
Quindi nell'età dell'informazione la materia prima che viene cercata, trasformata, scambiata, registrata, venduta, acquisita è l'informazione, è ancora più precisamente l'informazione digitalizzata, trasformata in modo che possa circolare su supporti informatici e telematici. \par

Una seconda caratteristica della società dell'informazione è \textbf{la pervasività, l'interconnessione e la convergenza delle nuove tecnologie}.\par
Allora, la società dell'informazione è caratterizzata dall'apparire di nuove tecnologie in particolare le tecnologie digitali, informatiche e telematiche, cioè le tecnologie che permettono di trasformare. Ad esempio la \textbf{digitalizzazione} che trasforma qualsiasi informazione in un codice digitale, un codice binario che può essere letto da alcune macchine o dalle macchine a ciò predisposte. 
\textbf{Informatizzazione e automazione}, vale a dire il fatto che queste informazioni digitalizzate possono essere trattate da alcune macchine che hanno un funzionamento in gran parte automatico, e \textbf{telematica}, cioè possibilità di trasmettere queste informazioni utilizzando particolari canali di comunicazione, che sono appunto quelli della telematica, le linee telefoniche in particolare, o anche le etere e così via. \par
Questo dà luogo alle caratteristiche che abbiamo menzionato, vale a dire la pervasività della nuova tecnologia, perché le tecnologie che servono a digitalizzare, informatizzare e trasmettere per via telematica le informazioni, sono pervasive, sono sensibilmente onnipresenti, le troviamo in qualsiasi ambito produttivo. Le troviamo in supermercati, negli aeroporti, nei sistemi di gestione del traffico, nella didattica in questo caso e così via. Quindi la diffusione di queste tecnologie è pervasiva, si trova nei settori produttivi, si trova nella vita quotidiana, si trova in qualsiasi ambito possiamo immaginare.\par
\textbf{Interconnessione}, interconnessione vuol dire che queste nuove tecnologie sono portate a dialogare tra loro, detto altrimenti i sistemi di registrazione, conservazione, acquisizione delle informazioni sono portati a interconnettersi, ad essere tra loro collegati e a dialogare tra loro. L'esempio principe di interconnessione è ovviamente la rete internet, la rete internet è la rete globale che connette tanti sistemi pluri-settoriali o connette singole macchine; quindi la rete internet è la rete globale che connette tanti sistemi più pluri-settoriali o connette singole macchine. Connettendosi alla rete internet si realizza questa interconnessione, vale a dire il fatto che i macchinari, le macchine, hardware e software che hanno la funzione di raccogliere, gestire, trattare, trasformare, acquisire informazioni dialogano tra loro , si interconnettono.\par
\textbf{Convergenza delle nuove tecnologie}, le tecnologie dell'era digitale, tecnologie informatiche, tecnologie telematiche tendono ad assorbire ruoli e funzioni che magari 20 o 30 anni fa venivano svolti da tecnologie di tipo diverso. Si pensi per esempio al passaggio alla televisione digitale, si pensi all'uso di videocamere o di macchine fotografiche digitali e così via. Quindi \hl{per convergenza intendiamo il fatto che tecnologie un tempo disparate o che avevano funzioni disparate o che avevano strutture, modi di funzionamento diversi, oggi tendono in tutto e in parte a convergere verso la tecnologia propria dell'età digitale, dell'età dell'informazione.} \par

Un'altra caratteristica della società dell'informazione è la \textbf{deterritorializzazione}.\par 
Per \hl{deterritorializzazione si indica la capacità che ha ciascun sistema, ciascuna macchina, ciascun utente che è connesso agli strumenti della società dell'informazione di interagire, di dialogare, di inviare e ricevere informazioni superando il vincolo dello spazio fisico.} 
Quindi per scambiare o ottenere informazioni con qualsiasi altro soggetto non ho necessità di essere fisicamente vicino a questo soggetto, posso scambiare in tempo reale informazioni con qualsiasi altro soggetto o sistema posto in qualsiasi altro punto del globo. Questa è la deterritorializzazione ed è una conseguenza della interconnessione dei sistemi della società dell'informazione.\par 
Un interessante dato sociologico è la riformulazione, la rimodulazione dell'idea di comunità. Tradizionalmente, in età premoderna, in età moderna, gli individui tendevano comunque a raggrupparsi, ad associarsi in vari tipi di comunità. La prima comunità è la famiglia, ovviamente, ma pensiamo ad associazioni e quant'altro. Tradizionalmente fino a tutta l'età moderna, tipicamente in età premoderna, ma fino a tutta l'età moderna e gran parte dell'età contemporanea, il meccanismo primario, principale, non esclusivo, di identificazione in una comunità era dato proprio dal territorio, dal fatto di trovarsi sullo stesso territorio. 

L'idea prettamente moderna di nazione riguarda il vincolo che si instaura tra soggetti che abitano allo stesso territorio e hanno pertanto vincoli di sangue, di cultura, di linguaggio, di religione e quant'altro. Nell'età dell'informazione le comunità tendono sempre di più a prescindere dal vincolo territoriale. Si formano comunità globali, comunità transnazionali, comunità deterritorializzate. Basta pensare al giro di amicizie che si può fare, per esempio, con un social network qualunque o così via. Questo porta ad un ulteriore profilo della società dell'informazione, assolutamente scontato, che è la globalizzazione. Nell'età dell'informazione le informazioni circolano su scala globale e le relazioni anche personali o commerciali possono svolgersi in maniera del tutto economica e in tempo reale su scala globale. 

Quindi società dell'informazione e globalizzazione sono due facce della stessa medaglia, sono strettamente interconnesse, sono strettamente legate.\par
Un'ulteriore caratteristica della società dell'informazione è la quantità di informazioni che vengono conservate, registrate e trasferite. Perché tutte queste operazioni, la conservazione, la registrazione e il trasferimento di informazioni, sono oggi, nella società dell'informazione, operazioni estremamente economiche e operazioni in gran parte automatizzate. Che cosa vuol dire? È ovvio che nel funzionamento delle società umane da sempre l'informazione ha un valore, da sempre l'informazione è un bene. Esistono registri del catastro, registri dei beni immobili che risalgono al Medioevo, esistono tavolette di argilla di vari secoli prima di Cristo che hanno, per esempio in Mesopotamia, che hanno la funzione di catalogare certi tipi di beni o soprattutto a fini fiscali. L'informazione è sempre stata un bene, nel senso di qualcosa che ha un valore economico, un valore nel funzionamento delle società umane. Tuttavia si pensi ai costi, in termini economici, in termini di risorse umane, in termini di tempo, ai costi della gestione di informazione, per esempio di un archivio esclusivamente cartaceo. 

Un archivio esclusivamente cartaceo in cui le informazioni sono stipate in faldoni, in fascicoli, custodite a loro volta in depositi, in archivi polverosi, magari non ordinatissimi, oppure queste informazioni potrebbero essere distribuite tra vari archivi diversi, distanti tra loro e che comunicano a fatica. Ecco, in un contesto di questo tipo la ricerca, la conservazione, il reperimento, lo scambio di informazioni è un'operazione costosa. Perché è necessario che qualcuno, un essere umano, dedichi molto tempo a cercare quelle informazioni, a reperirle, a lavorarle e eventualmente a trasmetterle ad un altro archivio che ha bisogno di quelle informazioni per interconnetterle con le informazioni in proprio possesso e così via. È evidente come tutte queste attività, operazioni, vengono assolutamente rivoluzionate nel contesto dell'età dell'informazione, nel contesto della digitalizzazione, dell'informatizzazione e automatizzazione. L'hard disk di un qualsiasi personal computer in vendita per poche centinaia di euro può contenere informazioni che equivalgono ad una pila di 30 metri di fogli di carta stampati su carta A4, quindi un supporto di dimensioni molto limitate oggi può contenere informazioni che fino a qualche decennio fa potevano essere contenute in vari depositi. Non solo, ma un qualsiasi personal computer o un sistema informatico di media complessità è in grado di reperire in maniera assolutamente veloce, agevole, tutte queste informazioni in tempi ristrettissimi. Di conseguenza, nell'età dell'informazione, le informazioni vengono conservate, registrate e trasferite in quantità impressionante e questo perché è molto economico e avviene attraverso procedimenti automatizzati. \par
Una caratteristica forse inquietante dell'assaltato dell'informazione è il cosiddetto \textbf{digital divide} o divario digitale. Il digital divide riguarda in generale la distanza che c'è tra chi può usare tutte o molte delle potenzialità le risorse offerte dalla società dell'informazione e chi è da queste risorse escluso o le fruisce in maniera esclusivamente passiva. Quindi potremmo articolare il problema del digital divide in due profili. 
Divide, cioè divario, distinzione, divaricazione, tra chi ha accesso alle risorse e opportunità della information society e chi ne è invece escluso. Quindi, fondamentalmente, tra chi ha le risorse economiche, in realtà abbastanza limitate, non sono ingenti quelle per procurarsi un accesso alle risorse informatiche e soprattutto chi ha le risorse culturali, quindi chi ha un sufficiente grado di alfabetizzazione informatica per accedere a queste risorse, e chi invece non ne ha. Oggi abbiamo una sorta di analfabetismo di ritorno perché per poter accedere alle risorse della società dell'informazione non è più sufficiente una scolarizzazione di base, ma occorre anche un'attitudine culturale e una serie di competenze che a dire il vero oggi si acquisiscono quasi in età infantile.

Per questo motivo il digital divide probabilmente colpisce in maniera più drastica giovani e vecchi, cioè distingue giovani e vecchi, persone anziane hanno più difficoltà ad accedere a queste risorse, persone più giovani crescono in un ambiente che include in maniera stabile queste risorse e le sanno usare in maniera quasi automatica.\par 
Un secondo profilo del digital divide è tra chi produce i contenuti della information society e chi ne fruisce passivamente, quindi tra chi attiva un blog per esempio, tra chi scrive articoli su un giornale online e quant'altro e chi invece semplicemente consulta queste risorse magari all'estremo senza avere le competenze culturali che gli permettono un vaglio critico su ciò che sta raccogliendo, su ciò che sta leggendo. Si pensi al funzionamento ad esempio di Wikipedia che è la più importante enciclopedia esistente online ed è un'enciclopedia sviluppata, alcuni dicono dal basso, cioè dagli utenti stessi. 

I contenuti, le voci sono inserite da utenti che si registrano. Per una voce redatta per Wikipedia possono succedere alcune cose, c'è un vaglio della redazione su alcune caratteristiche estrinseche della voce, per esempio il fatto che sia ben strutturata, non contenga insulti o opinioni troppo personali, abbia dei requisiti minimi. Davanti a una voce di Wikipedia però si può avere una reazione di due tipi o forse tre, una prima reazione è quella di ritenere insufficiente e prendere un'enciclopedia cartacea, una seconda reazione è quella di prendere acriticamente ciò che è scritto in quella enciclopedia, in quella voce, una terza reazione invece è quella di accreditarsi sul sito di Wikipedia e intervenire e modificarla se la si ritiene insoddisfacente, quindi il digital divide in questo caso distingue chi semplicemente assume acriticamente contenuti dalla rete e chi invece li immette o interviene eventualmente modificandoli. \par
Quali sono i principali problemi del diritto che riguardano il diritto nel contesto della società dell'informazione?\par
Un primo problema o un primo aspetto di novità è la destatualizzazione che in qualche modo è legato alla deteritorializzazione e globalizzazione che abbiamo visto essere due caratteristiche portanti, due profili quasi definitori e strutturali della società dell'informazione. La rete internet che è il veicolo principale della società dell'informazione è la rete globale per definizione, su internet ci si scambiano opinioni, idee, informazioni, beni e servizi su scala globale, gli attori, i soggetti giuridici che compiono transazioni giuridiche su internet, ma anche altre cose giuridicamente rilevanti, ad esempio compiono reati su internet, possono mettere a dura prova la caratteristica strutturale del diritto, la caratteristica portante del diritto, cioè la sua statualità. \par 
Sin dall'età moderna e tuttora il diritto positivo, così come lo conosciamo, è un diritto legato a doppio filo allo stato, al potere statale, questo in due modi quantomeno, in un primo senso perché lo stato è l'unico soggetto abilitato a produrre diritto per un certo territorio, quindi dato un certo territorio, per esempio l'Italia, l'unico soggetto in prima battuta abilitato a produrre diritto valido e applicabile in Italia è lo stato italiano. In un secondo senso il rapporto tra stato e diritto è che il territorio statale è non solo l'ambito spaziale del monopolio giuridico dello stato, ma anche il limite del monopolio del potere giuridico dello stato, vale a dire che le norme giuridiche dello stato, salvo alcune eccezioni, alcune norme penali per esempio, ma tendenzialmente il diritto prodotto dello stato si applica solo nel territorio dello stato, alcune eccezioni sono date da alcune norme penali per esempio o da alcune norme di diritto internazionale privato, ma anche se uno stato afferma che le sue norme o alcune sue norme possono essere applicate al di fuori dei suoi territori, al di fuori dei confini nazionali, dei confini statali, resta pur sempre il problema che lo stato può non avere gli strumenti per andare a applicare, per fare enforcement, per andare ad applicare le proprie norme al di fuori del proprio territorio.\par
Cosa voglio dire? Lo stato italiano potrebbe ben affermare che un certo reato, un certo fatto compiuto in Brasile è di competenza o è rilevante per il diritto italiano, è reato per la legge italiana. Tuttavia lo stato italiano potrebbe non avere il potere giuridico di perseguire quel reato in Brasile. Lo stato italiano avrebbe bisogno della collaborazione delle autorità, dello stato brasiliano affinché perseguono esse quel reato, ovvero trasferiscano l'autore del reato, facciano estradizione, effettuino un'estradizione, trasferiscano l'autore del reato nel territorio italiano, perché lo stato non può, lecitamente, se in mancanza di accordi con le autorità competenti, non può eseguire, attuare il proprio diritto al di fuori dei propri confini.\par  
Quindi il diritto nella società dell'informazione affronta alcune sfide. \par 
La prima è la destatualizzazione che ha due profili. Il primo profilo è quello che abbiamo appena visto, vale a dire la dimensione sovranazionale, una dimensione che travalica i confini nazionali, propria degli scambi che avvengono nella società dell'informazione, dove per scambio intendo scambio di informazioni in generale, siano esse opinioni, siano esse notizie, siano esse informazioni volta allo scambio di beni e servizi, siano esse informazioni volta a commettere un reato. In secondo luogo, e in un certo senso anche in connessione con questo, vi è l'aspetto dell'\textbf{autoregolamentazione}. \par
Una spinta crescente, sempre più forte nella società dell'informazione, è quella di lasciare che gli operatori si autoregolamentino, dettino da sé le regole che devono guidare e limitare le proprie condotte. Quindi il codice di autoregolamentazione per gli internet service providers, codice di autoregolamentazione per certi soggetti e così via. Questo perché? Per due ordini di motivi. Il primo motivo è dovuto al fatto che essendo che questi fenomeni spesso sovranazionali, il fatto di imporre regole giuridiche dall'alto in maniera verticale da parte dello Stato può risultare una chimera perché un provider italiano potrebbe installare i propri server, le proprie macchine in uno Stato straniero, semplicemente, per vanificare gran parte delle attività giuridiche dello Stato che possono riguardare quel server.\par
Quindi innanzitutto l'autoregolamentazione può essere utile per la dimensione sovranazionale della rete. E in secondo luogo perché vi è fondamentalmente un certo timore nei confronti dell'ignoranza del legislatore nei confronti dei fenomeni che devono essere regolamentati. C'è un certo timore verso la possibilità che il legislatore quando interviene intervenga in maniera inaccurata, creando danni allo sviluppo di queste tecnologie dell'informazione. Inoltre le tecnologie dell'informazione evolvono e le modalità in cui si scambiano le informazioni e così via, evolvono in maniera molto veloce. \par

Cambiare una legge può essere un processo molto lento, forse è più facile cambiare un codice di autoregolamentazione. Quindi la destatualizzazione del diritto nella società dell'informazione riguarda la dimensione sovranazionale, quindi il travalicare i confini nazionali, e la autoregolamentazione, vale a dire l'accantonamento, la proposta di accantonamento di meccanismi giuridici rigidi come la legislazione, e inseguire appunto forme di soft law, codici di condotta, best practices, lex mercatoria e così via. \par

Una seconda sfida per il diritto nella situazione dell'informazione è la dematerializzazione. Il diritto in gran parte ha a che fare tradizionalmente con cose fisicamente appprensibili, con cose che hanno la dimensione materiale, fisica, la proprietà, il possesso di cose, il furto di cose e quant'altro. Un aspetto importante, non un esclusivo, non l'unico, ma molto importante della regolamentazione giuridica è la regolamentazione di cose, di beni materiali, del loro uso, del loro scambio, della loro tutela e così via. Le cose e i beni che vengono scambiati nella società dell'informazione hanno una dimensione prettamente dematerializzata, smaterializzata. Si scaricano file musicali su internet, si scaricano file di film, circolano informazioni digitalizzate, circolano notizie e così via. Il diritto si trova a dover inseguire una realtà che è completamente diversa da quella che costituisce le caratteristiche tipiche della precomprensione(?) del giurista, avere a che fare con cose materiali. Si pensi alla smaterializzazione del documento, il documento informatico è un insieme di bit e non più cartaceo che viene conservato, fotocopiato, acquisito in duplice copia e quant'altro. Stessa cosa per la sottoscrizione, non viene apposta con una firma, ma per esempio apponendo il proprio nome in un documento e ponendo quel documento ad alcune procedure di validazione informatica. \par
Infine se i beni che vengono scambiati sono beni non materiali, allora nell'era dell'informazione il diritto giuridico principale, più importante o comunque un diritto che acquista sempre più importanza, non è più quello di proprietà, di impossessarsi di un bene ma di accedere a beni. Quindi non più quello di comprare una videocassetta per esempio, ma quello di avere l'accesso a un sito che proietta quel film, che manda in onda quel film che mi interessa.\par

Il rapporto tra diritto e nuove tecnologie, le tecnologie della società di informazione è un \textbf{rapporto dialettico}. Cosa vuol dire? Per un verso il diritto riflette e si adegua o dovrebbe adeguarsi o dovrebbe rispondere alle innovazioni della società di informazione. Per altro verso il diritto cerca di condizionare queste innovazioni tecnologiche. Quindi per un verso, l'abbiamo appena visto, il diritto ha la necessità di modificare alcune proprie categorie, alcuni istituti per renderli applicabili alle nuove categorie della società dei nuovi beni, della società di informazione. Per altro verso però è possibile che una certa regolamentazione giuridica indirizzi lo sviluppo delle risorse tecnologiche e delle innovazioni tecnologiche. Per esempio il diritto potrebbe richiedere per certi tipi di sistemi, per certi tipi di documenti, per certi tipi di attività un certo livello di sicurezza informatica, di sicurezza dei sistemi. Questo tipo di regolamentazione giuridica senz'altro incentiverà la ricerca e la corsa all'innovazione per esempio appunto sulla sicurezza dei sistemi. Questo rapporto chiaramente può anche essere molto conflittuale, un rapporto dialettico tra diritto e società dell'informazione. Si pensi a una vicenda avvenuta nel gennaio 2012 che ha riguardato la proposta negli Stati Uniti di approvare e promulgare due distinte leggi giudicate dagli operatori di settore molto limitative dello scambio di informazioni e della libertà su Internet. Si trattava di leggi, in particolare sulla tutela del diritto d'autore, tutela che veniva attuata in termini talmente ampi e con tecniche di tutela talmente rigide da permettere l'oscuramento di qualsiasi sito possibile su Internet. Contro questa legge si è attuato non solo un movimento di opinione, ma anche una vera e propria reazione quasi di guerriglia di molti hacker, di migliaia di attacco coordinato di decine di migliaia di hacker che hanno oscurato il sito del Congresso e il sito della FBI in America e così via. La reazione è stata sostanzialmente una ritirata del Congresso statunitense che ha deciso di postergare la discussione di questi progetti di legge a un futuro non meglio identificato. Quindi un rapporto che può anche essere molto conflittuale quello fra diritto e società di informazione.\par


\section{Esigenze giuridiche legate all'Information Society}, 
le vediamo molto velocemente perché su alcune di queste ci concentreremo nelle prossime lezioni. \par

\subsection{Il controllo sulle proprie informazioni personali}.
Se l'informazione è la materia prima e la merce di scambio nella società di informazione è ovvio che l'interessato, la persona a cui certe informazioni si riferiscono, ha un diritto o un interesse molto forte a controllare che queste informazioni circolino in maniera corretta, in maniera integra e così via, oppure che non circolino se è il caso di certe informazioni. 
\subsection{Sicurezza dei sistemi}
Il diritto dovrebbe garantire o dovrebbe imporre che i sistemi informatici per esempio quelli con cui un operatore commerciale registra i dati delle carte di credito dei suoi clienti abbiano livelli di sicurezza elevati. 
\subsection{Accesso ai dati pubblici}
La possibilità che i dati pubblici, i dati gestiti, le informazioni gestite e lavorate delle amministrazioni pubbliche siano accessibili a tutti, così come per esempio la legge sul procedimento amministrativo, la legge del 241 del 90 prevede per i documenti cartacei delle pubbliche amministrazioni. 
\subsection{Integrità dei documenti} Vale dire il fatto che il documento informatico possa essere considerato autentico e la sottoscrizione sia vera. 
\subsection{Protezione della proprietà intellettuale} 
Questo l'abbiamo visto appena poco fa, cioè il fatto che un prodotto dell'attività intellettuale, dell'attività dell'ingegno abbia su internet tutele le equivalenti, ma questo è discutibile e controverso, a quelle esistenti nel mondo reale, cioè il problema dell'estensione o meno, in quale misura, alla circolazione su internet in formato digitale delle opere tutelate nel mondo reale, libri, musica e quant'altro. 

\section{L'ambito dell'informatica giuridica}

Passiamo al secondo argomento di questa lezione, l'ambito dell'informatica giuridica. Ci occuperemo a livello introduttivo, a livello di panoramica, dei settori, delle problematiche di cui si occupa l'informatica giuridica. L'informatica giuridica ha una duplice faccia, ha una duplice rilevanza. Per un verso si parla di informatica giuridica in senso stretto, per l'altro verso si parla di diritto dell'informatica. 
\subsection{L'informatica giuridica in senso stretto} 
Riguarda l'utilizzo dell'informatica, delle tecnologie informatiche, delle tecnologie dell'età dell'informazione per la gestione o lo sviluppo di certe attività giuridiche. Quindi, potremmo dire, l'informatica giuridica in senso stretto è l'informatica applicata al diritto, è la traduzione informatica o il supporto, il sostegno informatico delle tecnologie legate all'informatica e anche della telematica, sostegno e supporto di queste tecnologie allo svolgimento di attività giuridiche. 
\subsection{Il diritto dell'informatica} 
Invece è quella branca del diritto che ha a proprio oggetto lo svolgimento di attività informatiche o vari profili della società dell'informazione. Quindi potremmo dire che nell'informatica giuridica in senso stretto, l'informatica è lo strumento, il supporto e l'infrastruttura di attività giuridiche. 
Nel diritto dell'informatica, l'informatica e la tecnologia dell'informazione sono invece l'oggetto della regolamentazione giuridica. 
\subsection{L'informatica giuridica in senso stretto} 

\begin{itemize}
    \item \textbf{informatica giuridica, documentaria o documentale}. Questo è il primo ambito di sviluppo di interesse, sia in senso cronologico che in senso quantitativo dell'informatica giuridica. L'informatica giuridica in senso stretto riguarda la documentazione giuridica e quindi la registrazione dei documenti giuridici, la loro trasformazione in supporto digitale, la loro gestione, nel senso l'accesso ai documenti giuridici, il così detto information retrieval e così via. 
    \item \textbf{informatica giuridica gestionale} L'uso dell'informatica per gestire vari aspetti di attività giuridiche. Vedremo degli esempi.
    \item \textbf{Attività giuridica decisionale} l'idea per alcuni versi utopica o utopistica, per alcuni versi in maniera sperimentale già esistente, già attuata, l'idea che gli strumenti informatici, il computer e così via, possono essere usati per raggiungere decisioni giuridicamente rilevanti. In maniera futuribile, utopistica, l'idea che un computer possa sostituire un giudice, non si sa se questa idea sia un'utopia bella o un incubo. In senso un po' più realistico, il fatto che alcune parti di alcuni procedimenti giuridici che potrebbero essere per esempio dei procedimenti amministrativi o svolti dalla certa burocrazia, alcune parti di questi procedimenti siano gestiti in maniera automatizzata e che quindi il computer possa svolgere un supporto ad un'attività che è comunque umana.
\end{itemize}

\subsection{Informatica giuridica documentaria}
\begin{itemize}
    \item riguarda per esempio la predisposizione di \textbf{banche dati legislative }, vedremo degli esempi in alcune elezioni successive.
    \item \textbf{banche dati giurisprudenziali} 
\end{itemize}

\subsection{ Informatica giuridica gestionale}
\begin{itemize}
    \item \textbf{Gestione dello studio legale}, può riguardare la automatizzazione dello studio legale, il fatto che per esempio un professionista organizzi il proprio studio con il supporto di strutture informatiche e telematiche, realizzi per esempio una rete locale nel proprio studio, una intranet nel proprio studio, sia dotato di software che gli permettono per esempio di archiviare le pratiche in un certo modo, di gestire le informazioni sui clienti in un certo modo.
    \item \textbf{Gestione dei servizi giudiziari come cancellerie, notifiche e quant'altro} in Italia esistono già in fase più che avanzata alcune attuazioni del processo civile telematico.  
        \item \textbf{Gestione della burocrazia} quindi uso dell'informatica per gestire aspetti delle attività amministrative burocratiche.
\end{itemize}

\subsection{Informatica giuridica decisionale}
\begin{itemize}
    \item \textbf{In ambito legislativo}, riguarda la possibilità che i sistemi informatici aiutino certi soggetti a prendere decisioni giuridiche, per esempio in ambito legislativo, alcune parti del processo legislativo potrebbero essere in maniera ipotetica, sperimentale, realizzate col supporto di strumenti informatici. Per esempio la gestione degli emendamenti potrebbe essere agevolata da alcuni strumenti informatici, dall'uso di certi software, o la gestione dei collegamenti tra la legge che si vuole discutere e approvare e il panorama legislativo esistente. Alcuni strumenti informatici potrebbero rendere più semplice collegare la disciplina che si vuole introdurre con le discipline settoriali già esistenti.  
    \item \textbf{In ambito giudiziario} abbiamo accennato poco fa l'utopia del giudice computer, probabilmente quanto sto dicendo verrà smentito in futuro ma sembra irrealizzabile o comunque indesiderabile, e altrettanto in ambito amministrativo.  
\end{itemize}

\subsection{Diritto dell'informatica} è il diritto che ha come proprio oggetto alcune attività che si svolgono con strumenti informatici o ha ad oggetto beni informatici, beni di tipo informatico. Le tradizionali partizioni del diritto si applicano anche al diritto dell'informatica, quindi il diritto dell'informatica è trasversale, abbiamo 
\begin{itemize}
    \item diritto privato dell'informatica
    \item un diritto pubblico dell'informatica 
    \item un diritto penale dell'informatica 
    \item diritto processuale dell'informatica 
\end{itemize}

\paragraph{diritto privato dell'informatica}
\begin{itemize}
    \item \textbf{contrattualistica}, per esempio il contratto che un'impresa fa con un internet service provider o con un soggetto che dovrebbe realizzare il sito web o la struttura per il commercio elettronico di quell'imprenditore, quindi la contrattualistica che ha ad oggetto lo scambio di servizi informatici o telematici.
    \item \textbf{commercio elettronico} quello che molti di noi fanno quasi quotidianamente su internet quando comprano un biglietto aereo su un sito web o quando acquistano a pagamento delle canzoni. 
    \item \textbf{diritto d'autore sul software} se il software goda della tutela della proprietà intellettuale, questi sono degli esempi di diritto privato dell'informatica.  
    \item \textbf{tutela dei contenuti delle pagine web} quando predispongo una pagina web che ha certi contenuti, questo è protetto dal diritto autore, nuovamente problematica di tipo privatistico-civilistico. 
    \item \textbf{la responsabilità civile applicata agli internet service provider} problema privatistico per eccellenza che verrà discusso in seguito.

\end{itemize}

\subsection{diritto pubblico dell'informatica}
\begin{itemize}
    \item \textbf{e-government} si intende la distribuzione, l'esecuzione di attività di servizio pubblico amministrativo tramite supporto informatico e telematico.
    \item \textbf{ e-democracy, cittadinanza elettronica} è la possibilità che alcune attività democratiche, attenzione alla democrazia, per esempio rivolgere petizioni al Parlamento e così via, o certe forme di democrazia diretta siano svolte tramite strumenti informatici. 
\end{itemize}

\subsection{diritto penale dell'informatica}
\begin{itemize}
    \item \textbf{reati comuni però commessi con strumenti informatici} quindi reati che possono essere commessi anche con attività non informatiche, per esempio una diffamazione, che però nel caso è stata commessa con strumenti informatici, per esempio la diffamazione online.
    \item \textbf{reati informatici in senso stretto}
\end{itemize}

\subsection{diritto processuale dell'informatica}
\begin{itemize}
    \item \textbf{il processo telematico}
    \item \textbf{le prove informatiche} vale a dire la possibilità che l'informatica sia utilizzata a fini probatori
\end{itemize}