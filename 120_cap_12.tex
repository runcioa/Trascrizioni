\chapter{Lezione 12: La rivoluzione digitale nella pubblica amministrazione}

Il tema di questa lezione è la rivoluzione digitale nella pubblica amministrazione. 
Gli argomenti di oggi sono:
\begin{itemize}
    \item L'amministrazione digitale
    \item La documentazione digitale
\end{itemize}

\section{L'amministrazione digitale}
Iniziamo dal primo argomento, l'amministrazione digitale. \par
Per amministrazione digitale si intende la digitalizzazione dell'amministrazione, processo disciplinato e incentivato dal \textbf{codice dell'amministrazione digitale}, di solito abbreviato in CAD, che è stato emanato con Decreto Regislativo 82/2005 e successivamente soggetto a numerose modifiche, alcune incisive; una modifica abbastanza importante nel 2010 e ulteriori modifiche nel 2011.\par 
Per amministrazione digitale si intende l'applicazione delle tecnologie dell'informazione, delle tecnologie informatiche e telematiche all'attività amministrativa e a tutto ciò che supporta lo svolgimento dell'attività amministrativa e altresia al prodotto, diciamo al pervedimento \footnote{pervedimento: Arrivare, giungere, sia in senso generico sia (oggi più spesso) quando il luogo o punto d'arrivo rappresenta il termine di un viaggio lungo, di un percorso faticoso o che ha comunque richiesto il passaggio attraverso gradi successivi o il superamento di qualche difficoltà}, dell'attività amministrativa.\par
Si è parlato di una rivoluzione culturale nella pubblica amministrazione. \par


Il modo in cui si raffigura e si pratica la pubblica amministrazione in Italia è tradizionalmente legata alla carta, ai timbri, al potere certificativo del pubblico ufficiale che appone una firma e un timbro, alla richiesta di documenti autentici, sottoscritti, firmati degli interessati, da chi avanza certe richieste o effettua certi adempimenti per la pubblica amministrazione, dalla richiesta per esempio di ricevere pagamenti di un certo tipo, in contanti o con bollettini postali e così via.\par
L'immagine tradizionale della pubblica amministrazione che si cerca faticosamente di superare in Italia  è quella di un pachiderma, di un macigno che ha procedure lente, farragginose, burocratiche, che comportano lo spostamento di carte polverose da un ufficio all'altro, che necessitano di firme su documenti stampati su carta, che attendono su scrivanie, eccetera eccetera.\par
È fatto anche di scarsa comunicazione sia con l'utenza, i cittadini, sia con le altre amministrazioni e così via.\par
Da alcuni decenni, in particolare a partire dalla legge 241 del 90 e poi anche per l'effetto di vari principi comunitari, si è cominciato a immaginare un ripensamento della cultura della pubblica amministrazione in Italia, non solo di singole strutture organizzative, ma della stessa cultura della pubblica amministrazione. Si è cercato di ripensare la cultura della pubblica amministrazione come un'attività statale dotata di imperio e di una certa posizione talvolta tacita, di supremazia nei confronti dei cittadini, quindi talvolta anche di scarso rispetto per le loro esigenze, da qui il prolungarsi abnorme, irragionevole, dei tempi di molte procedure amministrative, richieste dei cittadini, creando difficoltà.\par
Per esempio si pensi alle difficoltà che può avere un'attività commerciale nell'attendere il rilascio di tutte le licenze e autorizzazioni previste dalla legge.\par
Quindi si è cercato di immaginare una trasformazione culturale nella pubblica amministrazione da una visione statica della pubblica amministrazione quasi autoreferenziale e in una posizione di supremazia nei confronti dei privati ad una concezione sempre più di servizio della pubblica amministrazione.\par
L'amministrazione come servizio ai cittadini, l'amministrazione che al suo interno viene organizzata in maniera premiale in vista dei risultati.\par
Le burocrazie e gli uffici amministrativi devono agire sulla base dei risultati e perseguirli sulla base dei criteri, dei principi di economicità, efficacia, efficienza, trasparenza, pubblicità e legalità. \par
Questi sono i principi fondanti, i criteri base dell'attività amministrativa dalla fine del XX secolo in poi, quindi quantomeno dalla legge 241 del 90 in poi.\par
Legalità, il fatto che l'attività amministrativa deve svolgersi nell'ambito di una cornice di criteri, direttive e principi stabiliti dalla legge, altresì deve esserci una conformità tra i provvedimenti adottati dalla pubblica amministrazione e gli scopi e i presupposti indicati dalla legge, legalità, trasparenza e pubblicità, il fatto che il cittadino ha diritto di accedere e partecipare a vario titolo all'attività amministrativa, per esempio la legge sul procedimento amministrativo, appunto la già citata legge 241 del 90, che prevede che il cittadino abbia un diritto di accesso e di partecipazione, diritto per esempio di interlocuire, nell'ambito di procedimenti che lo riguardano, ha diritto a produrre documenti e così via, ha diritto che le sue istanze vengano tenute in considerazione nell'adozione del provvedimento finale. \par
Il cittadino ha diritto di partecipare ai procedimenti, diritto di accesso ai documenti che lo riguardano, diritto a consultare certi documenti e così via, quindi trasparenza, pubblicità, efficacia ed efficienza.
Per efficacia si intende l'esigenza che la pubblica amministrazione raggiunga i propri scopi in maniera concreta, quindi che la pubblica amministrazione raggiunga realmente gli scopi che si prefigge, gli scopi dell'azione amministrativa. \par
Per efficienza si intende l'esigenza che gli scopi perseguiti dall'autorità amministrativa siano perseguiti col minore dispendio possibile di risorse, quindi che siano usate alle risorse umane e economiche effettivamente necessarie allo raggiungimento dello scopo, senza inutili sprechi di risorse.\par
%07:30

E' evidente che in questo processo di ripensamento culturale della pubblica amministrazione, ispirato i criteri di trasparenza, pubblicità, efficacia, efficienza e così via, si inserisse in maniera senz'altro inevitabile le più generali procedure e risorse rese disponibili dalle tecnologie della società dell'informazione. 

Quindi a un certo punto è sembrato inevitabile che tutte le esigenze che abbiamo annunciato, la pubblicità, la trasparenza, i diritti partecipativi dei cittadini nei confronti dell'attività amministrativa, l'efficacia, l'efficienza, eccetera, tutti queste esigenze potessero essere soddisfatte o raggiunte o comunque potessero trovare un valido sostegno nelle tecnologie dell'informazione, nella digitalizzazione e nella informatizzazione delle attività amministrative, delle procedure e anche degli atti dei documenti amministrativi.

Questo spiega appunto la rivoluzione culturale che molti si attendono, da un'amministrazione autoreferenziale in posizione di supremazia ad un'amministrazione di servizio e di risultato.
Questa è la rivoluzione più generale. All'interno di questa rivoluzione più generale c'è una seconda rivoluzione che è quella della informatizzazione della pubblica amministrazione, della trasformazione di gran parte delle attività amministrative in attività informatizzate e digitalizzate. 

Questo dovrebbe consentire il raggiungimento di vari obiettivi, l'abbattimento dei costi, una maggiore trasparenza, forse anche una maggiore considerazione per l'ambiente, nella misura in cui sia meno uso e meno spreco di carta. 

Vediamo come possiamo tradurre l'incrocio di queste due rivoluzioni in base ai principi indicati dal codice dell'amministrazione digitale.

Quali sono gli obiettivi, o alcuni degli obiettivi, previsti dal codice dell'amministrazione digitale? 

Una rivoluzione culturale nella PA: maggiore efficienza e trasparenza: ‘dialogo’ telematico tra PA e con i privati, con pieno valore legale; alfabetizzazione informatica dei dipendenti
\par

O meglio per una maggiore efficienza e trasparenza, promuovere il dialogo telematico tra pubbliche amministrazioni e tra la pubblica amministrazione e i privati, con pieno valore legale, promuovere l'alfabetizzazione informatica dei dipendenti. 
Iniziamo da questi. 
%10:45
\subsection{promuovere maggiore efficienza e trasparenza}
E' evidente che il ricorso all'informatizzazione e anche alla telematica può incentivare questi processi. Per esempio tutte le pubbliche amministrazioni dovrebbero dotarsi di siti Internet enunciano chiaramente per esempio la struttura dell'ufficio, i contatti, le persone a cui riferirsi, gli orari di ricevimento del pubblico, la modulistica, una sintesi delle procedure che svolgono, indichino le loro funzioni e così via.

In questo modo l'amministrazione diventa trasparente rispetto al cittadino, perché il cittadino sa a quali amministrazione rivolgersi, sa a chi rivolgersi all'interno dell'amministrazione, produce già da casa o dall'ufficio la modulistica rilevante, eccetera eccetera.

\subsection{Dialogo telematico tra pubbliche amministrazioni e tra pubbliche amministrazioni e privati}

Spesso le pubbliche amministrazioni non comunicano contro loro. Le pubbliche amministrazioni raccogono grande quantità di dati personali dei cittadini per i loro scopi istituzionali ma spesso però i cittadini sono obbligati a ripresentare le stesse informazioni o modulistiche che contengono le stesse informazioni a diverse amministrazioni per ottenere determinati tipi di provvedimenti che a loro interessano o per svolgere certi adempimenti. 

L'idea è che facendo dialogare e interagire le banche dati delle varie amministrazioni, con certi limiti per problemi di riservatezza dei dati, si possono realizzare risultati in termini di efficienza e efficacia dell'azione amministrativa anche in termini di risparmio di tempo per i cittadini. 

Inoltre il dialogo telematico tra la pubblica amministrazione e i privati è un vero e proprio diritto istituito dal codice dell'amministrazione digitale di dialogare in maniera telematico con la pubblica amministrazione. 

Che cosa vuol dire? La possibilità si scaricare documenti in formato elettronico, produrre documenti in formato elettronico, quindi attribuire un certo valore legale al documento formato in maniera elettronica e poi trasmetterli alla pubblica amministrazione con strumenti elettronici, per esempio con la posta elettronica. 
%13:30
Ovviamente questo presuppone una massiccia opera di alfabetizzazione informatica dei dipendenti della pubblica amministrazione con corsi di formazione, presuppone un cambiamento di cultura, cioè il progressivo abbandono della carta e la coscienza nei pubblici dipendenti che ciò è necessario. 

Il codice dell'amministrazione digitale è di fatto il primo intervento sistematico in materia. A partire dagli anni 90, si erano susseguiti interventi su settori diversi della pubblica amministrazione, in particolare erano provvedimenti non benissimo collegati tra loro, ma comunque adottati in tempi diversi sul documento informatico, sulla firma digitale, sulla trasmissione attraverso la posta elettronica dei documenti, sulla tenuta dei archivi e così via. 

Il codice dell'amministrazione digitale si prefigge di essere un intervento organico che ridisegna in maniera complessiva il sistema e la dimensione digitale informatizzata della pubblica amministrazione. 
%14:56
Probabilmente uno dei limiti del codice della pubblica amministrazione è una certa enfasi declamatoria, forse un eccessivo ottimismo, perché contiene soprattutto enunciazioni di principi che ad alcuni sono sembrati fumosi e la cui attuazione richiede sforzi e risorse coordinate e un impegno serio e costante. 

Si enunciano diritti come il diritto del cittadino al dialogo digitale con la pubblica amministrazione e così via. 

Se si vuole che questi principi diventino realtà sono necessari, innanzitutto numerosi atti normativi di dettaglio e soprattutto un cambiamento culturale nella pubblica amministrazione. 

Un elemento in tal senso è presente nelle successive modifiche del codice della amministrazione digitale, soprattutto  quelle del 2010 e 2011, in cui si cerca di introdurre un'ottica sanzionatoria premiale nei confronti delle pubbliche amministrazioni, sanzionatoria per un verso, premiale per l'altro, che progressivamente si dotano di questi strumenti e traducono la loro attività amministrativa in termini informatici e telematici. 

Allora l'oggetto della disciplina, del codice dell'amministrazione digitale, in generale è l'utilizzo delle tecnologie informatiche e telematiche nella pubblica amministrazione, tramite tutta quella serie di principi che abbiamo menzionato poc'anzi. 

Il sistema pubblico di connettività, vale a dire un'integrazione sempre maggiore e tendenzialmente globale di tutte le pubbliche amministrazioni centrali, periferiche, amministrazioni locali e così via, in un sistema pubblico di connettività, in una rete che integri e che faccia comunicare nella maniera più efficiente ma anche nella maniera più sicura possibile le pubbliche amministrazioni. 

%17:40
La sicurezza è un presupposto indispensabile per la circolazione dei dati vista la quantità di informazioni personali anche sensibili, come dati sulla salute che vengono resi da un dipendente pubblico va in aspettativa, o dati sulla salute richiesti quando si accede a certe prestazioni previdenziali e assistenziali e così via.

Il sistema pubblico di connettività che dovrebbe succedere a vari esperimenti già attuati da alcuni decenni, per esempio la Rupa, la rete unitaria della pubblica amministrazione e così via. 
Quindi un'unica rete, un sistema di connessione che riguarda tutte le pubbliche amministrazioni. 

\subsection{Il valore giuridico del documento informatico e delle firme elettroniche}
La pubblica amministrazione produce documenti che hanno un certo valore legale e spesso la pubblica amministrazione acquisisce o richiede l'acquisizione da parte dei privati di documenti che hanno un certo valore legale. Se la produzione o l'acquisizione di documenti da parte della pubblica amministrazione viene fatta sul supporto informatico è necessario stabilire le condizioni alle quali questa documentazione informatica ha valore legale e di ciò si occupa per l'appunto il codice dell'amministrazione digitale.

L'efficacia legale di questi documenti va oltre quello amministrativo ma è ovvio che era necessario trattare di questo tipo di profilo nell'ambito di una normativa generale sulla informatizzazione della pubblica amministrazione. 

Lo stesso vale per la posta elettronica certificata che è un altro degli argomenti che ricadono nell'ambito di disciplina del codice o delle normative che esso delega. 
Il codice richiede l'adozione di documenti attuativi, decreti ministeriali che contengono regole tecniche per esempio sulla produzione di documenti digitali, sulla posta elettronica certificata e così via; regole per la trasmissione elettronica avente valore legale di atti tra pubblica amministrazione e tra le pubbliche amministrazioni e i privati e questo viene fatto con la normativa inclusa in questo corpus sulla posta elettronica certificata.


\subsection{Finalità del codice dell'amministrazione digitale}
Iniziamo a valutare le finalità sia di medio che di lungo periodo del codice dell' amministrazione digitale. Lo Stato, le Regioni e le Autonomie Locali assicurano la disponibilità, la gestione, la trasmissione, la conservazione e la fruibilità dell'informazione in modalità digitale e si organizzano ed agiscono a tal fine, utilizzando con le modalità più appropriate, le tecnologie dell'informazione e della comunicazione.

Qui abbiamo alcuni spunti interessanti. Innanzitutto il fatto che venga istatuito questo diritto del cittadino a fruire in maniera digitale e informatizzata dei servizi amministrativi. Ricevere prestazioni tramite supporto, per esempio comunicazioni da parte di uffici amministrativi, tramite email, tramite posta elettronica, inviare comunicazioni tramite email e quindi inviare documenti aventi valore legale, quindi formati sul supporto elettronico, anche questi inviarli tramite strumenti elettronici e telematici.

Più in generalela strutturazione dell'azione amministrativa con le opportunità rese possibili dalle tecnologie dell'informazione e delle telecomunicazioni. 

Il rinvio che viene fatto da questo corpo normativo alle tecnologie è ovviamente un rinvio mobile, vale a dire che le norme di principio contenute nel codice dell'amministrazione digitale non indicano in maniera precisa quali sono le tecnologie da utilizzare in quanto dovranno essere utilizzate le migliori o le più opportune tecnologie disponibili al momento e perché l'evoluzione tecnologica in sé che le consiglia, oppure perché, saranno previsti di volta in volta alcuni decreti ministeriali o decreti del Presidente del Consiglio dei Ministri per l'adozione di specifiche tecniche più opportune per certi fini. 

%23;27
La normativa di solito, per esempio i decreti ministeriali o decreti del Presidente del Consiglio e in certa misura anche il codice individuano criteri di base che devono essere eseguiti per la formazione di documenti e per trasmissioni di comunicazioni elettroniche;per esempio se sono sottisfatte queste regole di base, un certo documento prodotto avrà valore legale, efficacia probatoria, eccetera, eccetera. 

Il codice dell'amministrazione digitale si prefigge altresì obiettivi di lungo periodo. Sono poco più che declamazioni, ma probabilmente  è possibile per esempio abbiano una utile funzione culturale anche queste proclamazioni nel codice. 
Promuovere iniziative per l'alfabetizzazione informatica dei cittadini, lotta al digital divide, il divario digitale che è ciò che separa i cittadini, i soggetti che non hanno familiarità e possibilità anche fisica, economica, materiale, di accesso alle risorse informatiche. 
Per fisica si intende, per esempio, la predisposizione di strumenti informatici o modalità di fruizione degli strumenti informatici da parte dei soggetti affetti a disabilità, anche di questo si occupa il codice dell'amministrazione digitale. 

È ovvio che se l'amministrazione e i cittadini viaggiasero a due velocità nettamente diverse dal punto di vista dell'informatizzazione, questo produrrebbe forme di esclusione da diritti di cittadinanza di molti cittadini. Se la pubblica amministrazione viaggiasse a livelli di informatizzazione tali che molti cittadini non riuscissero a dialogare con tali risorse informatiche digitali della pubblica amministrazione, si creerebbero forme di esclusione sociale, cioè l'impossibilità di molti cittadini di accedere ai servizi digitali e ai servizi informatici della pubblica amministrazione. Di conseguenza il complemento culturale della digitalizzazione e informatizzazione nella pubblica amministrazione è ovviamente l'alfabetizzazione digitale e l'alfabetizzazione informatica dei cittadini e questo è uno degli obiettivi di lungo periodo che si propone il codice. 

Ulteriore obiettivo è favorire l'uso delle nuove tecnologie per una maggiore partecipazione dei cittadini al processo democratico e facilitare l'esercizio dei diritti politici e civili, ed è ciò che si chiama e-democracy digitale.
%26:42
Quindi le tecnologie dell'informazione sfruttate dalla pubblica amministrazione in una prospettiva di lungo periodo dovrebbero essere finalizzate non solo alla fruizione di servizi amministrativi ma anche alla partecipazione dei cittadini al processo democratico, in che modo non è chiarissimo oggi, sono promesse per il futuro, è la facilitazione dell'esercizio dei diritti civili e politici. 

Per esempio si potrebbe prevedere una estensione o una implementazione del voto elettronico che potrebbe essere svolto anche a distanza, in presenza di certi presupposti, per facilitare l'esercizio del voto elettronico anche da parte di persone che sono fisicamente impossibilitate a recarsi nei seggi elettorali.
Queste sono però promesse per un futuro in cui contorni ancora non si scorgono. 

\subsection{Diritto all'uso delle tecnologie}
I cittadini e le imprese hanno diritto a richiedere ad ottenere l'uso delle tecnologie telematiche nelle comunicazioni con le pubbliche amministrazioni; si pone in capo ai cittadini e alle imprese la possibilità, a cui si dà la veste di diritto giuridico, anche se tuttora non è chiaro quali siano gli strumenti per sanzionare l'inottemperanza di questo diritto, affinché i loro rapporti con le pubbliche amministrazioni si svolgano tramite l'utilizzo di tecnologie informatiche e telematiche. 

Per esempio per le imprese lo stesso codice prevede l'istituzione di uno sportello unico per le imprese con il quale le imprese possono dialogare e interagire anche in maniera telematica. Oppure altri esempi, alcuni dei quali già vigenti, come la possibilità di accedere a certi servizi come ad esempio fare visure sui registri delle imprese, dentro le camere di commercio, visure catastrali, fare queste visure in maniera telematica, per esempio da uno studio notarile o da uno studio legale. Quindi interagire da parte di tutti i cittadini, da parte di imprese, da parte di particolari soggetti qualificati, interagire con le pubbliche amministrazioni con strumenti informatici e telematici. 

\subsection{Participazione al procedimento amministrativo informatico}
Il procedimento amministrativo è una procedura svolta da una pubblica amministrazione che si articola in vari passaggi che è preordinata che ha il fine di adottare un certo atto o provvedimento amministrativo. I cittadini che sono interessati all'adozione di quel procedimento amministrativo, perché per esempio quel provvedimento inciderà sulla loro sfera giuridica, hanno diritto a partecipare al procedimento amministrativo e questo è un diritto già stabilito in via generale dal 1990 con la legge 241; ma in base ai principi del codice dell'amministrazione digitale hanno diritto di partecipare a questo procedimento e di accedere agli atti prodotti o utilizzati nell'ambito di questo procedimento tramite l'uso delle tecnologie informatiche. 

Quindi, per esempio, richiedere  la comunicazione via posta elettronica di certi atti di un procedimento. Normalmente in fase pre-digitale l'accesso si esercitava recandosi personalmente presso l'amministrazione che detiene quei documenti. Questo richiedeva dispendio di tempo, una volta anche di denaro, se l'amministrazione si trova in una città diversa. Quindi accedere fisicamente a questi documenti ha un costo in termini di tempo e di denaro; l'utilizzo delle nuove tecnologie potrebbe tagliare significativamente questi costi. 

Ogni atto e documento può essere trasmesso alle amministrazioni pubbliche con l'uso delle tecnologie e dell'informazione se è formato ed inviato nel rispetto dell'evigenza normativa. Il diritto al dialogo digitale con la pubblica amministrazione che dicevamo prima. È chiaro che per questo sono necessari dei presupposti, veda a dire che il documento e anche la sua trasmissione vengono formati secondo certi presupposti, certe regole tecniche indicate di volta in volta dalla normativa rilevante. 

\subsection{Compiti della pubblica amministrazione}
Compiti della pubblica amministrazione nell'implementare i principi, i criteri direttivi stabiliti dal codice dell'amministrazione digitale. 

\begin{itemize}
    \item Realizzare gli obiettivi di efficienza, efficacia, economicità, trasparenza, semplificazione e partecipazione. In buona misura abbiamo già visto cosa significhi ciascuno di questi obiettivi, sono i principi cardine della nuova pubblica amministrazione a partire dalla fine dagli anni 90. 



    \begin{itemize}
        \item Efficienza, efficacia, economicità vuol dire realizzare concretamente gli obiettivi dell'azione amministrativa, realizzarli senza sprechi di risorse, con un adeguato e proporzionato impiego di risorse
        \item Imparzialità significa non favorire ragionevolmente o in maniera illecita cittadini o categorie di cittadini a scapito di altri, quindi quando la pubblica amministrazione ha adotto un provvedimento, l'adozione di questo provvedimento deve essere imparziale, questo per esempio richiede che gli interessati siano ascoltati nell'ambito dell'adozione del procedimento
        \item  Semplificazione singifica ridurre al minimo le procedure necessarie, preordinate all'adozione di un certo atto
    \end{itemize}
\end{itemize}


\subsection{Adottare tecnologie adeguate nei rapporti interni tra diverse amministrazioni}

Adottare tecnologie adeguate nei rapporti interni tra diverse amministrazioni e tra queste e i privati.All'interno di una pubblica amministrazione o nell'interazione tra amministrazioni, ovvero nell'interazione tra amministrazioni e privati, la pubblica amministrazione deve adottare le tecnologie più adeguate  a seconda del tipo di procedimento da porre in essere. 

L'esempio tipico che abbiamo già visto è dotarsi di un sito internet funzionante e aggiornato in cui i cittadini possono trovare informazioni senza necessità di andare a consultare personalmente i dipendenti dell'ufficio, che può essere uno spreco di tempo sia per i cittadini sia per il personale che può svolgere altre manzioni e così via. 

%34:12
Assicurare l'uniformità e la graduale integrazione delle modalità di interazione degli utenti con i servizi informatici, quindi fare in modo che le modalità con cui gli utenti interagiscono con i servizi informatici delle pubbliche amministrazioni siano tendenzialmente omogenee e uniformi, così che il cittadino sa sempre cosa aspettarsi in termini di procedure tecnologiche quando interagisce con una pubblica amministrazione. 

Implementare i processi di informatizzazione relative all'erogazione di servizi a cittadini e imprese, questo è parte del diritto all'uso delle tecnologie che abbiamo menzionato precedentemente. 

L'implementazione, la razionalizzazione e la semplificazione dei procedimenti amministrativi, le modalità di accesso e di presentazione delle istanze da parte dei cittadini e delle imprese.

Per un verso la digitalizzazione e l'informatizzazione dovrebbero essere una spinta anche alla razionalizzazione e semplificazione. L'informatizzazione di per sé è solo uno strumento, però può essere un'opportunità anche per rivedere tutta una serie di procedure utilizzate dalla pubblica amministrazione. Nel momento in cui si riversano le procedure dal mondo fisico cartaceo al supporto digitale questa può essere una occasione per semplificare e razionalizzare le procedure esistenti. A tal fine è prevista anche una logica premiale che cerca di incentivare l'amministrazione su questa strada. 


\section{La documentazione digitale}
La documentazione digitale è il secondo argomento di questa lezione. È un argomento in realtà contenuto nel primo perché sono assolutamente interconnessi e abbiamo già avuto modo di vedere in che modo. 

L'amministrazione digitale può funzionare nella misura in cui l'amministrazione produce e riceve documenti, aventi in un certo valore legale in termini digitali. 

\subsection{Il documento informatico}
Il documento informatico è la rappresentazione informatica di atti, fatti o dati giuridicamente rilevanti. 
Quindi la nozione chiave nella digitalizzazione della documentazione in sede amministrativa ma anche oltre l'ambito amministrativo, della pubblica di informazione, è la nozione di documento informatico. 
Questo può venire registrato in teoria su qualsiasi supporto, un documento di Word, un documento PDF, ma anche un supporto di altro tipo che abbia carattere informatico; il documento informatico prescinde dal tipo di supporto, basta che sia strutturato in un modo informatico. 

Quali sono i problemi del documento informatico? Sono essenzialmente due, la riferibilità e l'integrità. 

\subsubsection{Riferibilità}
La riferibilità vuol dire il fatto che quel documento sia riconducibile alla persona che lo ha prodotto o che lo ha sottoscritto. Con la carta questo è abbastanza semplice e si risolve con la sottoscrizione, è possibile anche falsificare la sottoscrizione cartacea, questo è ovvio, però la sottoscrizione dà una presunzione di riferibilità del documento alla persona che l'ha prodotto. Talvolta gli uffici pubblici richiedono che la sottoscrizione sia apposta davanti a un pubblico ufficiale per assicurare questa riferibilità. 

\subsubsection{Integrità}
Il secondo problema è l'integrità, vale a dire il fatto che un documento informatico è più semplice da alterare rispetto a un documento cartaceo. Il problema della riferibilità ma anche quello dell'integrità vengono risolti con le firme elettroniche. 

Esistono vari tipi di firme elettroniche, la firma elettronica semplice garantisce solo la riferibilità, vale a dire il fatto che quel documento sia associabile a una certa persona, che sia stato prodotto da una certa persona. 

Le firme elettroniche avanzate, ne esistono tante, garantiscono invece la riferibilità e l'integrità; 
Un documento prodotto da un soggetto che ha strumenti di autenticazione che si presumono siano nella sua disponibilità esclusiva, per esempio perché sono protetti da password e così via, questi strumenti  garantiscono l'integrità perché la sottoscrizione con firme elettroniche avanzate tiene anche traccia delle eventuali modifiche al documento; in questo modo si garantisce che il documento prodotto sia integro. 

Tra le firme elettroniche avanzate troviamo la firma elettronica qualificata che garantisce la riferibilità e l'integrità perché viene usato un dispositivo di firma certificato e sicuro e la firma digitale che garantisce la riferibilità e l'integrità perché si utilizza un sistema crittografico, di solito una crittografia asimmetrica, chiave pubblica e chiave privata. 

%40:03
\subsubsection{Efficacia probatoria del documento informatico}
Al documento informatico viene riconosciuta una certa efficacia probatoria.
L'idoneitò del documento informatico a soddisfare il requisito della forma scritta e il suo valore probatorio sono liberamente valutabili in giudizio. Che cosa vuol dire? In un documento informatico senza sottoscrizioni particolari, senza firma elettronica o digitale sarà il giudice a valutare se quel documento assicura forma scritta e valore probatorio. 

Se nel documento informatico è apposta una firma elettronica anche qui forma scritta e valore probatorio sono valutati liberamente dal giudice. 

Se invece il documento informatico è sottoscritto con firma elettronica avanzata, qualificata o digitale, ha efficacia probatoria prevista dall'articolo 2702 del Codice Civile per la scrittura privata. Vale a dire, se la sottoscrizione è riconosciuta dal soggetto contro cui quel documento è prodotto, allora la provenienza del documento fa fede fino a querela di falso. 

Il documento con firma elettronica qualificata e firma elettronica digitale soddisfa la forma scritta ex articolo 1350 primo comma numeri da 1 a 12 del codice civile, vala a dire, che i documenti che hanno forma scritta ad substansiam e che sono idonei a trasferire certi diritti, per esempio la proprietà su beni immobili e così via, in tutti quei casi in cui ciò è richiesto dal Codice Civile a pena di nullità come previsto dall'articolo 1350, numeri da 1 a 12 del codice civile. 

La posta elettronica certificata è un tipo di posta elettronica che certifica le fasi di invio e di ricezione del messaggio. Il mittente riceve una ricevuta che è prova legale della avvenuta spedizione, riceve una ricevuta della avvenuta a consegna al destinatario con data e ora, anche se il messaggio non è stato letto dal destinatario. Il messaggio di posta elettronica così inviato ha valore giuridico equivalente alla notificazione a mezzo posta, ogni qualvolta ciò sia richiesto dalla normativa rilevante, e data e ora della trasmissione e la ricezione del documento sono opponib ili  ai terzi, sempre che tutto ciò si sia svolto in conformità alle regole tecniche previste dal Codice dell'amministrazione digitale e dalla normativa attuativa.