\chapter{Contratti informatici, prima parte}

\section{Introduzione ai contratti informatici}
Gli argomenti della lezione di oggi:

\begin{itemize}
    \item contratti a oggetto informatico
    \item contratti hardware
    \item contratti software
\end{itemize}

\subsection{Contratti a oggetto informatico}
Le novità che sono state introdotte negli ultimi 20 anni dall'utilizzo di sistemi informatici ha portato alla diffusione di contratti che riguardano il mondo dell'informatica. La dimensione informatica è certamente un luogo per fare degli affari, ma richiede una riflessione sulle regole generali. Il fatto che il mondo dell'informatica ha un'evoluzione molto rapida, fa sì che sia difficile fare riferimento a delle elaborazioni dottrinarie e giurisprudenziali consolidate. C'è una evoluzione della gestione dei contratti più rapida di quanto non sia l'evoluzione dottrinale e giurisprudenziale. 

Quando parliamo di contratti informatici ci riferiamo a due tipologie:

\begin{itemize}
    \item contratti relativi all'utilizzo di mezzi informatici
    \item contratti conclusi per via telematica 
\end{itemize}
L'espressione contratti informatici quindi ricomprende in linea generale tutti i contratti che fondano la loro funzione economico-sociale sull'informatica. Quindi fanno riferimento ai contratti informatici i contratti relativi all'utilizzo di strumenti dell'informatica hardware e software, ma anche i contratti che comportano l'acquisizione, l'elaborazione, la diffusione di dati a mezzo strumenti informatici. Ancora i contratti che si formano attraverso gli strumenti dell'informatica e in generale ogni attività giuridicamente rilevante che può essere compiuta adoperando l'elaboratore come uno strumento di formulazione o trasmissione di un atto. 

Rispetto a questi contratti occorre prima di tutto verificare quelle che sono le regole generali che restano applicabili, perché parliamo comunque di contratti.

\subsubsection{Stipula}
La stipula del contratto deve avere:
\begin{itemize}
    \item un accordo che comporta proposta, accettazione e conoscenza. L'accordo è notoriamente l'incontro della volontà delle parti su tutti gli aspetti del contratto e determina la nascita di un vincolo contrattuale. L'accordo è raggiunto quando vi sono una proposta, un'accettazione senza riserve e modifiche e la conoscenza dell'accettazione. Questo è il primo degli elementi essenziali del contratto.
    \item Il secondo è l'oggetto che deve essere possibile, lecito, determinato o determinabile. L'oggetto del contratto deve avere dei requisiti precisi altrimenti è nullo. Si parla di contratto impossibile, quando l'oggetto è impossibile. Deve essere l'oggetto lecito e poi deve essere determinato e determinabile. Queste sono caratteristiche tutte essenziali
    \item la forma che può essere di scrittura privata o atto pubblico. La forma è lo strumento attraverso cui le parti manifestano all'esterno la loro volontà negoziale. In generale la forma è libera, soltanto in alcuni casi è richiesta la forma scritta, in particolare per gli atti pubblici. In ogni caso la forma scritta è un elemento di prova della conclusione del contenuto del contratto. E è obbligatoria soltanto in alcuni casi.
    \item vi sono altri elementi, che sono elementi accidentali, condizione, termine e modo. Si tratta di elementi che possono essere o non essere presenti all'interno del contratto. La condizione è un avvenimento futuro incerto, da cui verificarsi le parti fanno dipendere l'inizio o la cessazione degli effetti del contratto. Il termine è un avvenimento futuro ma certo, da cui le parti fanno dipendere l'inizio o la cessazione del contratto. Il modo è un'obbligazione accessoria che può o meno essere apposta a negozi a titolo gratuito per limitarli. Questi elementi sono degli elementi accessori e non indispensabili
    \item il tempo e il luogo. Nel caso dei contratti informatici o meglio nei contratti telefomatici di cui si parlerà, il tempo e il luogo della conclusione del contratto sono particolarmente importanti perchè la distanza che vi è fra le persone che stipolano il contratto comporta che non necessariamente il tempo di conclusione del contratto e il momento in cui proposta e accettazione vengono formulate siano lo stesso momento, ma anche il luogo può essere del tutto diverso. È possibile stipolare un contratto a distanza in cui una delle parti è in un paese e l'altra parte è in un altro paese. Questa peculiarità, nei contratti telematici, è determinante perché può comportare delle difficoltà di applicazione della normativa.
\end{itemize}

\subsubsection{Esecuzione del contratto}
Una volta conclusa la fase della stipola si passa alla fase della esecuzione. L'esecuzione del contratto è:
  
\begin{itemize}
    \item l'adempimento degli obblighi (corretto). L'esecuzione del contratto quindi comporta prima di tutto un obbligo di adempimento da parte di entrambe le parti poiché il contratto fa nascere una serie di obblighi a carico di essi. Ognuna delle parti è tenuta ad adempire correttamente la propria parte, il proprio obbligo
    \item esaurimento degli effetti. Terminato l'adempimento del contratto, gli effetti terminano anche se chiaramente resta la modifica della realtà che è stata determinata e voluta dal contratto
\end{itemize}  
  
\subsubsection{Modalità particolari}
%7:18
\begin{itemize}
    \item offerta al pubblico. Nella offerta al pubblico che è di particolare importanza nei contratti vi è una proposta diretta ad un numero indeterminato di persone. Il caso tipico delle proposte che vengono introdotte su internet. E questa offerta contiene tutti gli elementi essenziali del contratto alla cui conclusione diretta. C'è un'effettiva efficacia di proposta quando essa diventa conoscibile e la conclusione del contratto dipende dalla accettazione del cliente, o meglio dalla conoscenza dell'accettazione da parte del proponente. Le tempistiche, le modalità di gestione dei contratti telematici fanno sì che vi sia una immediatezza tra la risposta e la conoscenza della stessa
    \item Moduli e formulari. Nel caso di un'offerta al pubblico, e non solo in quel caso, il proponente offre dei contratti standard che utilizzano dei moduli o formulari. Sono un classico esempio i contratti bancari o telefonici. L'altro contraente può accettare o meno, ma certamente non ha la possibilità di modificare le clausole proposte dall'offerente. Questa caratteristica è una caratteristica che distingue i contratti conclusi con moduli o formulari da tutti gli altri tipi di contratti. In particolare dai contratti nel quali, come detto all'inizio, c'è la realizzazione di un accordo fra la volontà di chi offre, la volontà di chi acquisisce, la volontà delle due parti, nelle quali è possibile negoziare, nelle quali è possibile stabilire le regole che dovranno essere eseguite. Nel caso di utilizzo di moduli e formulari, invece, questa possibilità manca. Uno dei due contraenti, quello che accetta, se vuole, la proposta, è un contraente che non ha possibilità di contrattare, ha solo la possibilità di approvare in maniera specifica delle clausole cosiddette vessatorie, che sono particolarmente pesanti per lui. Questa particolarità riguarda in maniera particolarmente forte la stipula di contratti su internet. Anche in questo caso vi è un'offerta con delle regole contrattuali predeterminate che il contraente può decidere o meno di accettare, ma in toto, non può derogare, non può modificare. Anche le modalità di stipula del contratto sono modalità particolari che spesso non consentono di sapere esattamente a quali rischi e problematiche si va incontro. Pensate che normalmente nel caso di contratti su internet le regole generali sono in una finestra a parte e devono essere esaminate appositamente.
\end{itemize} 

\subsubsection{Contratti complessi}
 Nel caso dell'informatica i contratti sono comunque complessi.
 
\begin{itemize}
    \item per la natura ibrida del software. Il software è costituito da un insieme di istruzioni in forma di algoritmi, cioè un processo immateriale, e da mezzi materiali su cui queste istruzioni vengono tradotte e incorporate. Si può parlare dischetti, di CD, di sistemi USB. In ogni caso la natura complessa del software comporta delle difficoltà di classificazione e si pone sempre un dubbio se considerare il software un prodotto o un servizio. Da questa diversa classificazione discende anche un diverso inquadramento contrattuale.
    \item  inscindibilità tra hardware e software. Si è detto che il software è spesso e volentemente incorporato su un supporto hardware, ma è anche vero che l'hardware, la macchina, senza programmi, non è assolutamente in grado di svolgere nessun compito; quindi un hardware non può essere utilizzato senza la presenza di un software. D'altra parte anche la relazione fra hardware e software si ritrova anche in quelle situazioni in cui alcuni programmi possono funzionare soltanto su determinate macchine
    \item la atipicità delle formule contrattuali. Le difficoltà di classificazione, di individuazione della gestione portano a una difficoltà di standardizzazione, di individuazione di clausole, di carattere generale e per questo si deve fare riferimento a dei contratti atipici, non disciplinati preventivamente dalla legge. Nel caso dei contratti informatici questa atipicità è sostanzialmente standardizzata. In sintesi si tratta della difficoltà di classificare in uno schema preciso contratti hardware e software, in generale contratti informatici
    \item  difficoltà di individuare una singola responsabilità contrattuale nel caso in cui vi sia una difficoltà fra i contraenti e uno dei due non adempia i propri obblighi. È assai difficile ricondurre la responsabilità dell'evento dannoso in capo ad uno specifico contraente e a quantificarne l'incidenza. Da questo deriva l'esigenza di contratti che disciplinino in maniera molto puntuale quelli che sono gli obblighi dei contraenti. Il tema a chi si attribuisce la responsabilità è un tema determinante che va valutato preliminarmente


\end{itemize}
 
\subsubsection{Interazione tra hardware, software e servizi}
Un altro aspetto di difficoltà è legato alla interazione tra:

\begin{itemize}
    \item contratti hardware
    \item contratti software
    \item contratti di servizi
\end{itemize}

Le macchine servono per automatizzare le attività e devono essere coordinate fra loro. Come detto, però, le macchine possono funzionare disponendo del necessario software di base. Ancora, per effettivamente operare un'automazione occorre anche avere delle risorse che consentano la manutenzione di hardware e software e quindi la fornitura di servizi. In sintesi, dunque, non si può scindere in maniera netta e definitiva fra contratti hardware, contratti software e contratti di servizi. Occorre sempre tenerli in considerazione tutti quanti. 

Spunto di riflessione per questa parte della lezione, quali sono le regole generali per i contratti informatici? Quando sono applicabili? 

\subsection{Contratti hardware}
I principali contratti hardware sono:

\begin{itemize}
    \item la vendita
    \item il leasing
    \item la locazione
    \item l'assistenza e la manutenzione
\end{itemize}

Si tratta dei principali tipi di contratti che possiamo trovare nell'informatica a con riferimento all'hardware. È evidente che c'è la possibilità di avere dei contratti misti, di svolgere attività diverse, però è opportuno puntare l'attenzione su quelle che sono le caratteristiche principali. 

Nel caso dei contratti hardware esistono delle clausole contrattuali e una disciplina dettagliata che si aggiunge alla disciplina negoziale del tipo scelto. 

Nel caso dei contratti informatici, pur tenendo conto delle peculiarità, dell'oggetto, degli strumenti a cui si fa riferimento, occorre tenere in considerazione quelle che sono le regole di carattere generale per il tipo di contratto scelto. 

Nel caso della vendita, del leasing, della locazione, dell'assistenza si deve fare riferimento a quelle che sono le regole abitualmente utilizzate per quel tipo di contratti in qualunque altro settore. 

Occorre poi disciplinare specificamente quello che è il tema dell'hardware:

\begin{itemize}
    \item la preparazione dei locali dove devono essere posizionati gli strumenti informatici
    \item il trasporto e la consegna. È molto frequente che la consegna degli hardware si perfezioni sul piano stradale, lasciando all'acquirente il trasporto al piano, e questo per distribuire la responsabilità tra venditore e acquirente
    \item L'installazione e il collaudo sono determinanti nel caso di strumenti informatici. Il collaudo dell'hardware prevede di svolgere una serie di test e il collaudo positivo porta alla sottoscrizione di un verbale di accettazione firmato sia dal cliente che dal collaudatore. Se invece il collaudo è negativo è possibile che ci sia il diritto alla sostituzione dello strumento o alla riduzione del prezzo. Questo perché? Perché l'hardware è una macchina ma una macchina deve funzionare e quindi la verifica del corretto funzionamento non può che essere determinante per il corretto adempimento del contratto da parte del venditore
    \item garanzia e responsabilità del fornitore. Il fornitore deve garantire il buon funzionamento della macchina, c'è però da precisare che non può essere garantito il fatto che la macchina funzioni senza alcuna interruzione. 
\end{itemize}


Queste in generale le caratteristiche più interessanti nel caso della vendita. 

\subsubsection{Contratti di locazione e leasing}

\begin{itemize}
    \item Nel caso di locazione e di leasing c'è prima di tutto il numero predeterminato di ore e mensili di funzionamento. Si tratta di un contratto che consente l'utilizzabilità degli strumenti senza doverli acquistare. E questo prevede quindi l'individuazione di un numero predeterminato di ore di funzionamento per poter quantificare quello che è il canone da dare al cedente
    \item la manutenzione ordinaria e straordinaria è generalmente a carico del cedente, il quale mantenendo la proprietà dei beni ha anche degli obblighi di manutenzione ordinaria e straordinaria perché il bene messo a disposizione deve essere costantemente utilizzabile da parte dell'utilizzatore 
    \item la locazione e leasing prevedono la restituzione o l'estensione o il riscatto del bene alla scadenza del contratto. Nel caso di hardware la possibilità di decidere alla fine del contratto se restituire il bene e chiudere il contratto oppure se estendere il contratto con lo stesso bene o sostituendo il bene con un bene nuovo, quindi con un hardware che sia adeguato all'evoluzione tecnologica, oppure possa decidere se riscattare il bene è estremamente importante proprio per i cambiamenti che rapidamente si pongono rispetto all'utilizzo e alle caratteristiche degli strumenti hardware di cui disponiamo.
\end{abstract}

\subsubsection{Assistenza e manutenzione}
Altro contratto tipico è il contratto di assistenza e manutenzione sull'hardware. 

\begin{itemize}
    \item L'assistenza e la manutenzione sull'hardware sono regolati da condizioni generali predisposte dal fornitore. Le condizioni generali di manutenzione predisposte dal fornitore rispondono a quelle caratteristiche di cui abbiamo parlato, di predisposizione preliminare della regolamentazione da parte di un contraente che non può che essere accettata dall'altra parte. 
    \item l'assistenza gratuita nel caso della locazione, per assistenza si intende la riparazione dei guasti segnalati dall'utente, come detto nel caso della locazione l'assistenza è gratuita e questo perché il locatore ha l'obbligo di conservare la cosa locata idonea all'uso per cui è destinata durante l'intera durata del rapporto. Quindi l'assistenza relativa alla riparazione dei guasti è dovuta e continua perché deve consentire di mantenere in funzione lo strumento. 
    \item La manutenzione accessoria invece è una previsione di interventi su chiamata in occasione di guasti e per interventi preventivi in occasione dei verifiche periodiche o puliture o eventuali inchiostrature delle macchine
\end{itemize}

La manutenzione quindi ha delle caratteristiche diverse rispetto all'assistenza perché può riguardare un intervento periodico. 

Spunto di riflessione di questa parte della lezione, quali sono le caratteristiche dei contratti hardware? Quando si tratta di regole di carattere generale che vengono prese in prestito dalla regolamentazione ordinaria e quando invece si tratta di regole specifiche elaborate con riferimento alla realizzazione di specifici contratti del tutto nuovi. 

\section{Contratti software}

I contratti software sono quelli che disciplinano l'utilizzo, la creazione o l'utilizzo del software. 

I contratti più frequenti: 
\begin{itemize}
    \item la licenza d'uso
    \item lo sviluppo software
    \item l'assistenza e la manutenzione
\end{itemize}

I contratti software hanno delle particolarità ulteriori rispetto ai contratti hardware perché il software ha delle caratteristiche particolari. Da un lato è un prodotto perché viene realizzato, dall'altro lato è un servizio perché svolge delle funzioni, è qualcosa di indispensabile per il funzionamento delle macchine ma è anche qualcosa che può avere una sua utilità autonoma rispetto al funzionamento delle macchine. 

Un elemento importante nell'affrontare i contratti software è il fatto che la disciplina dei contratti software deve tenere conto della tutela del diritto d'autore. Il legislatore italiano con decreto legislativo del 1992 ha equiparato tutte le opere dell'ingegno di carattere creativo che appartengono alla letteratura anche i programmi per gli elaboratori, cioè i software. 

I software, dunque, nell'ordinamento italiano sono tutelati con la normativa sul diritto d'autore. Per questo tutti i temi, le problematiche connessi ai contratti software di vendita, di fornitura, di sviluppo, di manutenzione sono delle problematiche che comunque tengono conto e che risentono dei temi della cessione dei diritti di utilizzazione delle opere di un autore prevista dalla legge sul diritto d'autore, legge 633/1941. 

I contratti software, nella prassi, sono stati elaborati tenendo conto di questi principi. 

\begin{itemize}
    \item Occorre anche tener conto del fatto che della distinzione fra software di base che permettono l'utilizzo dell'hardware, cioè il software di base o sistema operativo è l'insieme dei programmi che di fatto permettono di utilizzare il computer, quindi normalmente rientrano i contratti hardware
    \item software applicativo che permette l'esecuzione di operazioni. Il software applicativo è l'insieme di programmi che interagendo con il sistema operativo permettono all' elaboratore elettronico di eseguire delle specifiche applicazioni
    \item La differenza tra software standard e software personalizzato. Il software applicativo può essere standardizzato, prodotto cioè da software houses e commercializzato su vasta scala, oppure può essere personalizzato, cioè creato appositamente per un soggetto in modo da soddisfarne le specifiche esigenze.
\end{itemize}

\subsection{licenza d'uso}

 Tra i contratti software che abbiamo individuato vediamo prima di tutto la licenza d'uso. 
 
 La licenza d'uso comporta:
 
 \begin{itemize}
    \item la cessione del godimento del software. La licenza d'uso è il contratto più utilizzato per la commercializzazione di un software. In questo tipo di contratto il soggetto che detiene i diritti sul software per averlo elaborato direttamente o per averli acquistati da chi lo ha effettivamente elaborato, cede il godimento del software ad uno o più soggetti
    \item avviene attraverso un supporto magnetico oppure attraverso il download del programma
    \item Fino a qualche tempo fa la cessione del supporto contenente, il software, era indispensabile per poter usufruire del software. Ormai non è più così perché il software può essere tranquillamente scaricato da internet e quindi la sua utilizzazione prevede questo passaggio di download ma non richiede più l'utilizzo di un supporto specifico. Questo ha in gran parte eliminato tutta una serie di problemi legati all'utilizzo del software su più macchine nonostante se ne fosse acquistata la copia di un solo supporto. Quindi è in parte risolto quello che era un grosso problema di duplicazione abusiva dei software. Nel caso della licenza d'uso del software, gli obblighi sdel licenziante e del licenziatario sono da un lato di consentire l'utilizzo del software per il tempo stabilito e dall'altro il fatto di corrispondere un canone di acquisto periodico e l'obbligo di non divulgare il software a terzi,ma questo riguardava soprattutto il software acquisito con un supporto esterno che era l'obbligo di non consentire di fare un numero di copie superiori a quelle della licenza
    \item Termine e tempo indeterminato. La cessione della licenza può essere una cessione che ha un periodo definito oppure può essere una cessione a tempo indeterminato.
    \item gli aggiornamenti. Gli aggiornamenti possono essere e sono determinanti nel caso dell'acquisizione della licenza di utilizzo del software perché generalmente il software viene migliorato e modificato nel corso del tempo
 \end{itemize}
 
 

 %27:43
Le principali licenze sono:

\begin{itemize}
    \item la cosiddetta licenza strappo denominata così perché era la licenza era scritta sull'involucro che contiene il supporto sul quale è memorizzato il software e quindi l'accettazione delle regole previste nella licenza avviene nel momento in cui si strappa l'involucro per utilizzare il software
    \item Licenze freeware e licenze shareware sono due tipi di licenze che consentono un utilizzo per un periodo limitato di tempo del programma in maniera tale da poterne verificare le potenzialità e quindi in modo tale da poter decidere se c'è l'interesse ad acquisire la licenza a tempo indeterminato. Oppure sono quelle licenze che permettono l'utilizzo per così dire una parte del software, ad esempio un software di mera lettura ma non di formazione documenti e questo è un modo che porta il produttore a diffondere l'utilizzo di quel determinato software gratuitamente creando così un mercato generale che poi potrà portare all'acquisizione della licenza per l'utilizzo dell'altra parte del programma che è quello che ha maggiori funzionalità
    \item licenze open source. Si tratta di una tipologia di licenze che prevedono la conoscibilità del cosiddetto codice sorgente, che prevedono la diffusione senza particolari difficoltà e che consentono in molti casi anche un'integrazione dello sviluppo del software così come immaginato originariamente. Si tratta di una tipologia di licenze software che si sono sviluppate soprattutto negli Stati Uniti dove la normativa che tutela il diritto d'autore è una normativa diversa rispetto a quella europea ma che sono licenze diffuse in tutto il mondo e sulle quali c'è un grande sviluppo perché si ritengono uno strumento di grandissima utilità. 
\end{itemize}

\subsection{Sviluppo software}

\begin{itemize}
    \item La creazione e modifica del software. Il contratto di sviluppo software ha ad oggetto la creazione di un prodotto e la sua eventuale gestione pratica su incarico di un cliente e può comportare tutta una serie di attività e di servizi. Può essere un contratto che interviene per modificare un software esistente che è già in commercio, per adattarlo alle esigenze specifiche dell'utilizzatore. Può riguardare la creazione di un software del tutto nuovo e questo prevede progettazione e sviluppo. Può invece riguardare l'integrazione alla modifica di un software di base già esistente e utilizzabile. Questo comporta la verifica delle specifiche funzionali del cliente
    \item Il cliente è quello che dà l'indicazione delle specifiche funzionali che gli occorrono perché il software soddisfi le sue specifiche esigenze. Questo comporta quindi un rapporto di collaborazione stretto fra il cliente e lo sviluppatore, il quale appunto deve essere messo in condizioni di conoscere esattamente quali sono le esigenze del cliente e quindi può elaborare il software sulla base di queste
    \item Ciò comporta ancora che il contratto è un contratto assimilabile a un contratto di appalto o a un contratto di opera
\end{itemize}

Nel contratto di sviluppo software è determinante l'attività svolta dallo sviluppatore, oltre ai mezzi che lo sviluppatore mette in campo per portare a termine l'attività svolta. 
Nel concreto occorre:

\begin{itemize}
    \item preliminarmente l'analisi tecnica delle specifiche funzionali. Ciò significa esaminare insieme al cliente quali sono le sue esigenze e poi riportarle nella progettazione del software 
    \item Il contratto di sviluppo software richiede un collaudo finale, cioè la verifica dell'effettivo funzionamento del software sulla base di quelle specifiche funzionali che erano state indicate dal cliente con l'accettazione da parte del cliente stesso.
    \item l'addestramento e fornitura dei manuali. La gestione del contratto di sviluppo si conclude con l'addestramento del personale all'uso del software e con la consegna dei manuali che contengono le istruzioni d'uso al cliente. La mancata consegna dei materiali può essere in adempimento molto grave perché non ne consente l'utilizzo da parte completo del cliente
\end{itemize}

Anche nel caso dello sviluppo software, vengono in considerazione le problematiche legate al diritto d'autore perché è aperto il tema di chi sia il proprietario effettivo del software che è stato sviluppato. Questo è un aspetto che sarà oggetto di un specifico approfondimento in un'altra lezione. 

\subsection{Assistenza e manutenzione}
Può essere:

\begin{itemize}
    \item statica o correttiva
    \item dinamica o migliorativa
\end{itemize}

L'assistenza e la manutenzione del software hanno delle caratteristiche determinate dalle caratteristiche del software. La manutenzione statica e correttiva porta all'eliminazione di errori preesistenti nel programma, mentre la manutenzione dinamica o migliorativa porta alla modifica, al potenziamento o al miglioramento del programma stesso. L'esigenza di una manutenzione dinamica o migliorativa può derivare da delle cause esterne al software che interagiscono con il software. Modifiche legislative, miglioramenti del programma di base sulla base del quale è stato sviluppato il software o ancora delle esigenze nuove del committente che deve svolgere nuove attività. 

Nel caso dell'assistenza e la manutenzione del software c'è un soggetto che si impegna ad intervenire per la riparazione dei guasti o per l'evoluzione o per quel che è l'oggetto della manutenzione o in cambio di un canone periodico oppure del pagamento di ogni attività svolta, di ogni chiamata. 

Spesso e volentieri l'assistenza e la manutenzione sul software è collegata con l'assistenza e la manutenzione dell'hardware proprio per le peculiari caratteristiche di interazione che vi sono fra i due elementi. 

Nei contratti esistono o vengono introdotte delle clausole di esonero o di limitazione della responsabilità per i danni che possono essere causati e questo tipo di clausole di limitazione della responsabilità sono normalmente previste anche se il fornitore tende ovviamente a limitarle il più possibile. 

Tra le clausole specifiche vi può essere previsto il rilascio del codice sorgente. Il codice sorgente come detto è la conoscenza del codice sorgente è quello che consente di conoscere esattamente il funzionamento del software con la conseguenza che eventuali attività di manutenzione, di modifica e quant'altro potrebbero essere svolte da soggetti diversi rispetto a coloro che hanno sviluppato. Il rilascio del codice sorgente quindi dipende dalle caratteristiche del software, dalla gestione della tutela dei diritti. Certamente è indispensabile in tutti i casi in cui colui che offre la manutenzione non può più continuare a offrirla per inadempimenti, fallimenti o quant'altro. Quindi è un argomento da trattare con attenzione. 

Spunto di riflessione: quali sono le caratteristiche dei contratti software? In tutti i contratti informatici occorre verificare quali sono le caratteristiche contrattuali, nel caso dei contratti software occorre porre particolare attenzione per le peculiarità dell'oggetto di cui si parla.
