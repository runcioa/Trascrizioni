\chapter{Lezione 14: Contratti informatici - parte seconda}

 Gli argomenti della lezione di oggi:
 
 \begin{itemize}
    \item contratti di servizi
    \item contratti telematici
    \item contratti su internet 
 \end{itemize}
 
 
\section{Contratti di servizi} 
I contratti di servizi informatici hanno come obiettivo lo svolgimento di attività del tutto diverse. 


Il contenuto del contratto di servizi può andare dalla registrazione di dati alla progettazione di sistemi informativi. In altri termini, si tratta, l'oggetto di un contratto di servizi è il più disparato. Le attività sono le attività più varie.
Posso comprendere:

\begin{itemize}
    \item l'automazione delle attività. Se si vuole automatizzare la propria attività, si può affidare un servizio completo a soggetti specializzati che mettono a disposizione le macchine, i programmi e le persone. Questi sono tutti elementi che possono essere messi a disposizione nell'ambito di un contratto di servizi. 
    \item delega di attività. È possibile delegare a terzi determinate attività informatiche che non possono essere effettuate in proprio da parte di chi le delega o per mancanza di risorse o per impedimenti legali. Le attività delegate possono essere delle attività estremamente complesse o delle attività più semplificate. 
\end{itemize}   

  
 Nella gestione dei contratti di servizi, normalmente si distingue tra:
 
 \begin{itemize}
    \item normale prestazione. Quando si parla di normale prestazione si fa riferimento allo svolgimento di un'attività materiale o intellettuale di tipo tradizionale, la cui unica peculiarità è l'oggetto, cioè il sistema informatico. Nel caso di prestazioni normali, non ci sono delle complessità particolari per quanto riguarda la qualificazione giuridica e la disciplina applicabile. Si fa riferimento alle norme dettate dal Codice Civile per i contratti in generale
    \item prestazione computerizzata. Invece i contratti in cui la prestazione è computerizzata sono caratterizzati dalle stesse modalità pattuite per la prestazione. È tutto informatizzato. La natura particolare della prestazione automatizzata invece rende difficile il ricorso a categorie tradizionali di riferimento e richiede un esame approfondito delle regole della disciplina codicistica applicabili perché prevalgono o comunque sono estremamente importanti degli aspetti del tutto peculiari che rendono difficile il ricorso alle categorie tradizionali di riferimento
 \end{itemize}

\subsection{Contratto di appalto o contratto d'opera} 
Il fatto che vi siano una serie di attività miste da svolgere comporta in generale l'inquadramento dei contratti di prestazioni di servizi tra contratti di appalto o contratti d'opera. Va segnalato che il Codice Civile in linea generale distingue nettamente fra la prestazione di opera o la prestazione di un servizio e da delle discipline differenti, ma nel caso di una prestazione informatizzata non è sempre chiaro se la prestazione possa essere assimilata a un'opera oppure ad un servizio. Sotto questo profilo si ritiene spesso che la disciplina applicabile debba necessariamente essere una disciplina mista con dei principi, delle norme, prese da entrambi gli istituti. 

I principali contratti di servizi nell'informatica sono:

\begin{itemize}
    \item il contratto di outsourcing
    \item il contratto di integrazione di sistemi
    \item il contratto di disaster recovery
    \item il contratto di engineering
\end{itemize}

Questi sono i principali contratti che vengono individuati quando si fa riferimento ai contratti informatici di manutenzione. Si tratta di contratti che in parte sono peculiari, in parte sono contratti già utilizzati con riferimento anche ad altre attività. 

\subsubsection{Outsourcing}
Nell'outsourcing c'è la richiesta di un servizio informatico completo. Il contratto di outsourcing è il contratto più importante, prevede la gestione completa di una serie di servizi per conto di qualcun altro. È un contratto inquadrato nell'ambito del contratto di appalto caratterizzato dalla prestazione di beni e di servizi. Nel contratto di outsourcing possono essere previste diverse attività a partire dallo sviluppo dei programmi a finire con la fornitura dei beni. 

Attività informatiche (personale, infrastruttura e gestione). 

Nell'outsourcing delle attività informatiche può essere richiesta la prestazione di personale, le attività possono essere affidate alla gestione infrastrutturale dell'outsourcer e la gestione di tutto viene affidata all'outsourcer. Questo comporta una delega al fornitore di una serie di attività con evidenti e indubbi vantaggi di carattere economico e anche semplificazioni operative. Il fatto per il committente di spogliarsi della gestione di attività complesse mediante delega al fornitore del servizio indubbiamente consente e facilita lo svolgimento di qualunque attività. 

Da ciò deriva la responsabilità e la valutazione da fare in origine durante la gestione del contratto è a chi spetti la responsabilità per eventuali problemi, eventuali danni che si realizzano nello svolgimento della prestazione, nello svolgimento del contratto. E' opportuno che siano disciplinate nella gestione del contratto anche se in alcuni casi, con riferimento ad alcune specifiche attività, la responsabilità è stabilita per legge. Mi riferisco ad esempio alle responsabilità legate al trattamento dei dati personali nel caso in cui vi sia un contratto di outsourcing per la gestione dei sistemi informatici e la ripartizione delle responsabilità espressamente prevista dalla legge che qualifica il titolare come responsabile del trattamento. 

Perdita di controllo. 

Il fatto di affidare la gestione del sistema ad un soggetto esterno porta al rischio di non poter più controllare sia il proprio patrimonio, le attività delegate, sia anche il contenuto di determinate attività quali ancora una volta il trattamento dei dati. Questo rischio di perdita del controllo aumenta laddove si decida di ripristinare il proprio sistema, quindi riprenderlo in gestione, oppure di trasferirlo ad altro fornitore. In questo caso sarà necessario recuperare tutte quelle attività che sono state gestite all'esterno con eventualmente la necessità anche di aggiornarsi da un punto di vista tecnico. 

I contratti di outsourcing si dividono generalmente tra transfer outsourcing e simple outsourcing. 

Nel transfer outsourcing c'è un intero ramo d'azienda che viene trasferito con la gestione del sistema informativo. In questo caso quindi l'impresa trasferisce al fornitore del servizio la proprietà dell'intero ramo di azienda che si occupa della gestione del proprio sistema informativo. Se ne spoglia completamente l'impresa. Il ramo d'azienda può essere trasferito sia a una società mista in cui il cliente mantiene il controllo o mantiene la partecipazione, oppure può essere completamente trasferito all'esterno con la totale perdita del controllo da parte del cliente rispetto all'attività gestita. 

Il simple outsourcing comporta invece soltanto l'acquisizione di attività. In questo caso c'è semplicemente la cessazione dell'attività che è stata svolta fino ad un certo momento all'interno dell'azienda e la sua acquisizione sul mercato esterno sotto forma di servizio. 

\subsubsection{Integrazione di sistemi} 

Il contratto di integrazione di sistemi consente di realizzare un sistema informativo o un sottosistema informativo sulla base di specifiche esigenze e qualità dell'utente con funzionalità e prestazioni in comune. In comune si intende fra l'utente e il fornitore informatico. Ad esempio, un classico esempio di integrazione di sistemi è il controllo del traffico aereo di uno o più aeroporti. In questo caso dunque si integrano i sistemi del fornitore con i sistemi del cliente perché hanno delle peculiarità in comune. 

Occorre il software di integrazione. Nel momento in cui si devono integrare i sistemi vi è la necessità di software che consentano effettivamente il collegamento tra i sistemi stessi. Da ciò si comprende come il contratto di integrazione di sistemi sia un contratto particolarmente complesso da gestire e particolarmente complesso per la definizione di tutte le specifiche contrattuali. 

\subsubsection{Disaster recovery} 

Disaster recovery o piano di emergenza informatica. Si tratta di un contratto con cui si fornisce all'impresa di una certa dimensione dei servizi volti all'analisi delle inoperatività e dei problemi del sistema informatico e delle misure di recupero degli stessi con la messa a punto di un piano di emergenza informatica. Si prevede l'impiego temporaneo di sistemi alternativi. Il piano di emergenza è quello che viene elaborato per valutare come intervenire nel caso in cui si verificano le emergenze che vengono esaminate e può appunto prevedere delle procedure specifiche che consentano un impiego temporaneo di un centro di elaborazione alternativo o comunque l'utilizzo di macchine di soccorso in attesa della riattivazione dei sistemi principali. 

\subsubsection{Il contratto di engineering}

Il contratto di engineering riguarda attività varie, diversificate e articolate. Questo contratto che è molto utilizzato nei servizi informatici è in realtà un contratto che non nasce nel campo dell'informatica e fa riferimento ad attività estremamente diversificate che vanno dall'esecuzione e installazione di impianti industriali, all'elaborazione di progetti per costruzioni architettoniche o anche alla prestazione accessoria di semplice assistenza tecnica. 
Nel mondo dell'informatica si tratta di un contratto che è utilizzato volentieri proprio perché consente di gestire lo sviluppo e la realizzazione di attività che sono le attività più varie. 

Spunto di riflessione. Quali particolarità hanno i contratti di servizi informatici? 

\subsection{Contratti telematici} 

Si individua normalmente con questa dizione quei contratti che vengono stipulati in via telematica, cioè su internet. Si distinguono in:

\begin{itemize}
    \item contratti telematici in senso stretto. Sono quelli in cui la trasmissione della volontà avviene per via telematica, cioè la proposta e l'accettazione provengono direttamente ed esclusivamente dai contraenti e vengono trasmesse all'altro contraente per via telematica. In questo caso c'è un'analogia con altri contratti, ad esempio con i contratti telefonici. I contratti telematici in altri termini in senso stretto non costituiscono una vera e propria novità ma ricalcano le orme di altri contratti, quali quelli conclusi per telefono, e presentano delle caratteristiche simili, anche se chiaramente lo strumento internet ha delle caratteristiche diverse dallo strumento telefonico, delle caratteristiche delle quali occorre tener conto nella individuazione della disciplina contrattuale
    \item Sono diversi i contratti cibernetici, quelli nei quali la conclusione è automatica. Questo tipo di contratti sono conclusi automaticamente fra una persona e un computer, oppure fra due computer diversi considerati parti contrapposte. In questo caso la formazione della volontà contrattuale è ad opera di un computer, senza che vi sia un intervento umano successivo alla programmazione, perché chiaramente nella fase di programmazione vanno individuate le situazioni che derivano dalla stipula del contratto e quelle situazioni che portano alla stipula del contratto. Dopo la programmazione però non c'è più un intervento umano ma ci sono degli automatismi realizzati dai computer stessi. Si tratta di novità assoluta, questa tipologia di contratti è assolutamente nuova.
\end{itemize}

 Nei contratti telematici si prevede la prestazione di servizi telematici privati. Mancano in questo caso delle specifiche norme legislative e quindi questi contratti sono disciplinati da accordi fra utente e fornitore del servizio. Anche nel caso dei contratti telematici, come nel caso di molti contratti informatici, vi è lo squilibrio di forza contrattuale tra proponente e cliente, con conseguente necessità di tutela del contriente debole. Generalmente, infatti, uno dei contraenti, il proponente, è una multinazionale, una grande società, mentre il secondo, colui che accetta il contratto, è generalmente un piccolo operatore economico oppure un privato sprovvisto delle adeguate conoscenze informatiche e privo di particolari poteri contrattuali. In questi termini è necessario che l'ordinamento, così come in altre situazioni, tuteli il contrante debole, per cui sono previste delle regole specifiche per tutela del contrante debole, in particolare quando si tratti di un privato. Poiché il contratto tra operatore e privato ha comunque delle caratteristiche diverse rispetto al contratto stipulato fra due soggetti che operano nello svolgimento della propria attività economica.
 
 Nel caso di contratti telematici, le specifiche tecniche del servizio telematico sono parte integrante del regolamento contrattuale. Le specifiche tecniche infatti nel campo delle informatica costituiscono un elemento imprescindibile dalla stessa prestazione contrattuale e tale da giustificare il ricorso a mezzi di tutela giudiziali o stragiudiziali nel caso di mancanza dei requisiti tecnici che sono menzionati negli allegati. Nei contratti telematici dunque le specifiche tecniche acquisiscono una rilevanza determinante. 
 
 Spunto di riflessione per questa parte della lezione, quali particolarità hanno i contratti telematici? 
 
 \section{Contratti su internet}
 La maggior parte dei contratti telematici fanno sostanzialmente riferimento alla stipula su internet. Quando si parla di contratti su internet si fa riferimento a tutti quei contratti che vengono stipulati su internet, nei quali l'offerta della prestazione e l'accettazione avvengono su internet e ormai sono modalità di creazione dei rapporti contrattuali di grande diffusione. 
 Nei contratti su internet vi è la particolarità che intervengono più soggetti. 
 
 Il fornitore di connettività è colui che consente l'accesso a internet e occorre che sussista un rapporto contrattuale fra l'utente e il fornitore di connettività per avere la possibilità di accedere a internet. Anche il fornitore da parte sua ha l'esigenza di un rapporto con il fornitore di connettività per poter offrire il proprio prodotto su internet. Con il fornitore di altri servizi, dominio, sito e quant'altro, occorre anche instaurare un rapporto contrattuale specifico. L'utente, colui che stipula il contratto, deve quindi costituire un rapporto sia con il fornitore di connettività, sia con il fornitore dei servizi. Si tratta di rapporti diversi fra di loro, anche se spesso e volentieri il cattivo funzionamento della connettività e quindi dell'accesso a internet può avere delle conseguenze determinanti anche sulla possibilità di usufruire degli altri servizi che vengono offerti su internet. 
 
 Tra i fornitori di altri servizi, in questa sede si ricordano con particolare attenzione coloro che forniscono il nome di dominio, coloro che forniscono il sito, coloro che si occupano della struttura della pagina web e perché queste sono le attività indispensabili e preliminari per la prestazione della maggior parte delle attività su internet. L'esercizio di qualunque attività su internet non può prescindere dall'acquisizione di un nome di dominio, dalla creazione di un sito e dalla realizzazione di una pagina web. 
 
 La realizzazione della pagina web in particolare può essere predisposta dal titolare del sito oppure da un fornitore di servizi, da un webmaster. In questo caso vi è un contratto specifico di sviluppo del software che disciplina il rapporto tra il committente e il realizzatore. 
 
 \subsection{Tutela del diritto d'autore}
 Sia l'acquisizione di un nome di dominio, sia la realizzazione di un sito, sia la realizzazione della struttura della pagina web richiedono di considerare la tutela del diritto d'autore. Lo sviluppo di questi servizi, la prestazione di questi servizi attiene necessariamente a degli elementi che sono posti sotto la tutela del diritto d'autore. 
 
 Il nome di dominio è la denominazione, il nome che viene scelto per lo spazio sul quale si realizzeranno le attività. Questo nome deve essere un nome effettivamente utilizzabile da parte di chi lo utilizza e questa utilizzabilità deve comunque seguire delle regole di carattere generale estrane al mondo di internet, nonostante le regole specifiche di internet prevedano un principio di acquisizione del nome di dominio secondo un criterio semplicemente temporale in quanto il nome di dominio è considerato in realtà non un nome in senso stretto ma un indirizzo, l'indicazione di un certo luogo virtuale sul quale sono collocate le informazioni. 
 
 Per quanto riguarda il sito, la realizzazione di un sito comporta l'utilizzo di immagini, di grafica, di opere diciamo in senso lato che possono coinvolgere il diritto d'autore. Ad esempio l'utilizzo di fotografie o l'utilizzo di immagini o anche soltanto la realizzazione di qualche cosa di nuovo da pubblicare, si trattano di attività di carattere intellettuale tutelate. Lo stesso discorso può valere per la struttura della pagina web. Quindi la realizzazione di tutti questi aspetti richiede di tener conto della normativa che tutela il diritto di autore. 
 
 %24:00

\subsection{Prestazioni collegate ai contratti internet}
Le prestazioni collegate ai contratti internet quindi riguardano da un lato la creazione di tutte le opere che vanno su internet e dall'altro la manutenzione e l'aggiornamento. 
Nel momento in cui viene messo in piedi un sito, c'è un'attività preliminare di creazione che risponde a quelle caratteristiche di cui si è parlato, però non è sufficiente la realizzazione del sito, occorre anche mantenerlo. La manutenzione del sito può portare anche alla necessità di un aggiornamenti  e questo può essere oggetto di un contratto del tutto autonomo e diverso rispetto a quello con il quale il sito è stato realizzato. 
La possibilità di creare ed utilizzare un sito su internet richiede necessariamente che sia garantito l'accesso a internet. 

\subsection{Accesso a internet e responsabilità dei provider}
L'accesso a internet prevede la stipula di un accordo con un internet service provider (ISP) il quale deve fornire l'accesso, deve garantire la stabilità di tale accesso e svolge un ruolo autonomo. L'internet service provider può assumere diverse funzioni, può limitarsi a fornire l'accesso e in questo caso non c'è nessun intervento diretto su quelle che sono le attività successivamente svolte dal cliente, può intervenire nel senso di mettere a disposizione le macchine per lo svolgimento di determinate attività e in questo caso la regolamentazione del rapporto contratto tra l'internet service provider e l'utente è più complessa o può collaborare alla realizzazione e alla pubblicazione di contenuti che finiscono sul sito. In questo caso il coinvolgimento dell'internet service provider è certamente maggiore rispetto al coinvolgimento di un semplice access provider e vi può essere una individuazione di responsabilità in capo al provider anche con riferimento ai contenuti del sito. 

Sotto il profilo della responsabilità e della gestione di questi rapporti la normativa di riferimento è il decreto legislativo 70/2003 che sulla scorta delle indicazioni introdotte dalla normativa europea ha disciplinato questo tipo di attività. Ciò che vale segnalare rispetto alla responsabilità dei provider è che si esclude generalmente una responsabilità di carattere generale del provider per quelle che sono le attività svolte su internet da parte dell'utente. Questo per ragioni di natura tecnica, cioè la difficoltà di effettuare un controllo su ciò che viene realizzato, ma anche per ragioni di carattere giuridico. Non si vuole infatti che coinvolgendo sotto il profilo della responsabilità il provider questo comporti un controllo delle attività che vengono svolte da parte di coloro che inseriscono i contenuti.

Questa disciplina è sostanzialmente l'unica disciplina che ad oggi regola le relazioni tra utente e fornitore ed è un tipo di disciplina che, quando si verificano situazioni che vengono considerate come lesive dei diritti delle persone, si pone il problema se sia o meno necessario introdurre una forma di controllo generalizzato in capo agli internet provider. 
Forme di controllo generalizzato per il momento non ce ne sono, certamente l'internet provider può essere chiamato a rispondere nella misura in cui venga portato a conoscenza dell'esistenza della commissione di un reato su internet, perché in quel caso è tenuto comunque ad intervenire. 

Oltre a questo tipo di responsabilità che è una responsabilità che riguarda il contenuto, vi è la responsabilità di fondo, che è una responsabilità di carattere contrattuale nella relazione fra il service provider e l'utente che ha stipulato un contratto con lui che riguarda esclusivamente quelle che sono le prestazioni offerte dal service provider. 

Oltre alla relazione fra utente e service provider è determinante la relazione fra il service provider e il gestore delle telecomunicazioni ISP/TLC. Non è possibile fornire alcun servizio se non c'è un accordo tra il service provider e il gestore delle telecomunicazioni. Quindi a monte rispetto alla fornitura del servizio vi è un rapporto contrattuale fra il service provider e il gestore delle telecomunicazioni e anche questo tipo di rapporto ha delle peculiarità. 

\subsection{Tipologie di servizi internet service provider}

\begin{itemize}
    \item il servizio di hosting che è quello che consente soltanto l'accesso ai computer e allo spazio web
    \item l'housing che è un servizio che consente il semplice accesso alla rete
    \item il content provider che fornisce anche dei contenuti
\end{itemize}

La relazione che si crea fra l'utente e il fornitore è diversa nelle diverse situazioni così come sono diverse le responsabilità di entrambi o come sono diverse le responsabilità del fornitore. 

Il service provider è responsabile in quella che è la gestione del rapporto contrattuale con l'utente ma può essere chiamato ad una responsabilità per i contenuti di ciò che viene inserito su internet soltanto nella misura in cui in qualche maniera partecipi alla realizzazione di questi contenuti, cioè soltanto nella misura in cui intervenga in qualche modo per la realizzazione dei contenuti. 

Spunto di riflessione: quali sono gli aspetti innovativi dei contratti su internet. 
Dall'esame dei contratti informatici nel corso della lezione precedente e nel corso della presente lezione si è visto che vi sono dei contratti che riguardano esclusivamente la gestione delle macchine, la gestione dei software. Ma la maggior parte dei contratti passano su internet. La maggior parte dei contratti informatici oggi coinvolgono l'utilizzo di internet. 

L'utilizzo di internet nella gestione contrattuale, sia nell'acquisizione di beni o servizi tradizionali sia gestione contrattuale invece di beni e servizi informatici risente necessariamente dell'utilizzo di internet. Non si può prescindere. Ne risente perché nella gestione di un rapporto via internet non vi è la relazione diretta fra chi offre e chi acquisisce e quindi non vi è la possibilità di individuare quel rapporto di fiducia che può crearsi invece nella gestione di un rapporto contrattuale di visu. Nella gestione dei rapporti su internet vi è la possibilità di non palesarsi per quelli che si è effettivamente. Occorre garantire, sempre e comunque, che si possano eseguire le transazioni in maniera sicura e senza rischi, perché è l'unico strumento che consente la gestione dei rapporti su internet e il migliorare e l'aumentare di questi rapporti. 

Altra caratteristica che si è visto determinante nella gestione dei rapporti su internet è la modalità di stipula del contratto. Modalità di stipula che risente proprio del fatto che non c'è una relazione diretta e quindi sia quella che è la parte che è legata alla stipula del contratto stesso,  proposta e accettazione, ma anche quella che è legata al pagamento della prestazione da parte dell'utente, risente dell'utilizzo del sistema e prevede una disciplina particolare.

Tutto ciò che riguarda la gestione dei mezzi di pagamento è definita e disciplinata con estrema attenzione. Sulla gestione dei pagamenti su internet si sono purtroppo sviluppate delle modalità di frode rispetto all'utenza che sono del tutto peculiari e diverse rispetto a quanto mai accaduto in precedenza e rispetto a quanto può accadere al di fuori del mondo internet. 

I contratti che prevedono l'inserimento di particolari contenuti, tutto ciò che riguarda il contenuto delle informazioni che vengono fornite, tutto ciò che riguarda la gestione delle informazioni che vengono fornite, deve tener conto delle peculiarità della normativa, sia sul diritto d'autore ma anche la normativa posta a tutela dei dati personali poiché le informazioni che transitano sono informazioni che riguardano anche dati personali. 

In sintesi quindi la riflessione generale che si può fare sulla gestione dei contratti informatici è che si hanno come riferimento principale proprio la gestione del rapporto internet.