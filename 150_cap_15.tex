\chapter{Lezone 15: Responsabilità dell'Internet Service Provider e Processo Telematico}

In particolare gli argomenti di oggi sono:

\begin{itemize}
    \item la responsabilità dell'Internet Service Provider o ISP
    \item il processo telematico
\end{itemize}

\section{La responsabilità dell'Internet Service Provider}

Si tratta di un argomento abbastanza delicato e molto discusso soprattutto verso la fine degli anni 90 e tutti gli anni 2000 e riguarda il tema del tipo di responsabilità ed eventualmente dei presupposti della responsabilità da applicare all'Internet Service Provider per attività dannose verso terzi che siano state compiute attraverso i servizi della Società dell'Informazione forniti da un certo appunto Internet Service Provider. 

Si rendono opportune alcune distinzioni preliminari per capire lo stato dell'arte della legislazione italiana su questo argomento. 

Potremmo distingue tra Internet Service Provider e Content Providers. 

I Content Providers sono i fornitori di contenuto, cioè sono coloro che immettono in rete certi contenuti che possono essere i più diversi, possono essere articoli giornalistici, possono essere opinioni postate su un blog, possono essere opere dell'ingegno come film, come libri, saggi scientifici o pezzi musicali, tutte opere dell'ingegno prodotte da altri e non prodotte dalle persone che li mettono online. 

Dovrebbe essere evidente, che per questo tipo di attività, il porre personalmente online cose come manifestazioni del pensiero o opere protette da diritti di proprietà intellettuale eventuali illeciti attinenti a queste condotte vengono imputate direttamente all'autore della condotta stessa, vale a dire appunto al Content Provider, a chi effettivamente direttamente pone in essere questa condotta. 

Questo non rappresenta una grande differenza rispetto a quello che succede nel mondo al di fuori della rete. Il problema però è, se e a quali condizioni, qualche responsabilità giuridica possa essere imputata anche all'Internet Service Provider e non al Content Provider colui che immette in rete certi contenuti, ma al Service Provider colui che fornisce certi servizi strutturali, strumentali, tecnici. Perché si pone il problema di indagare questo tipo di responsabilità? Per tanti motivi. Innanzitutto perché spesso se si produce un danno derivante da attività svolta su Internet, quindi con la conseguente necessità di risarcire quel danno o di rimuovere l'atto dannoso, spesso l'autore effettivo della condotta dannosa può essere irreperibile, può essere non identificabile, può essere residente in un paese , in un luogo su cui lo stato non ha giurisdizione, può essere anonimo e così via. Di conseguenza spesso i soggetti danneggiati trovano più conveniente andare ad aggredire, per risarcimento di eventuali danni, l'Internet Service Provider, il quale è un soggetto normalmente strutturato in forma di impresa, è un soggetto identificabile, è un soggetto che può avere anche più capacità risarcitoria rispetto al singolo Content Provider che può essere un "Quisque de populo" \footnote{in latino significa "qualunque persona del popolo" ovvero un individuo comune, senza particolari prerogative o qualità.} non particolarmente solvibile. 

Si pone allora il problema di capire se e a quali condizioni possa sorgere una responsabilità in capo all'Internet Service Provider. Responsabilità civile o responsabilità penale? 

\subsection{la responsabilità penale}
Per la responsabilità penale c'è un primo ostacolo, per la praticabilità della responsabilità penale, che è quello derivante dal fatto che in Italia per i principi costituzionali e generali del diritto penale, la responsabilità per i reati è una responsabilità personale, quindi per imputare una responsabilità penale all'Internet Service Provider occorre che egli abbia in qualche modo concorso fattivamente e coscientemente alla condotta che ha causato un certo danno. 
Quindi la via della responsabilità penale sembra spesso poco praticabile e anche se vedremo nel seguito di questa lezione alcuni casi che portano la responsabilità dell'Internet Service Provider nella direzione delle responsabilità penale. 

\subsection{la responsabilità civile}
Oppure potrebbe fare ricorso alla responsabilità civile. La responsabilità civile non richiede necessariamente l'apporto personale dell'Internet Service Provider all'accagionamento del danno, sono previste alcune ipotesi di responsabilità oggettiva nell'ordinamento, oppure delle forme un po' più deboli rispetto alla responsabilità diretta. Alcuni per esempio hanno ipotizzato che l'Internet Service Provider possa essere considerato responsabile in sede civile per i danni causati da attività svolte attraverso i sui strumenti in quanto può essere considerato responsabile per culpa in vigilando, per ommesso controllo sui contenuti che transitano tramite i servizi che egli metta a disposizione. 

Altri hanno pensato di avvicinare la responsabilità dell'Internet Provider a quella prevista dall'articolo 2050 del Codice Civile, cioè la responsabilità per il servizio di attività pericolose che richiede un livello di dirigenza più elevato rispetto agli obblighi di controllo precauzionari previsti al tipo di attività. Vi è però un problema nell'attribuire un tipo di responsabilità anche risarcitoria all'Internet Service Provider. Il problema è, intanto evitare di ripiegare in forme di responsabilità oggettiva che sono sempre a rischio di tenuta rispetto ai principi generali e all'ordinamento dei principi costituzionali; in secondo luogo, attribuire la responsabilità all'Internet Service Provider per i contenuti che circolano sulle strutture che egli predispone, senza che l'Internet Service Provider abbia apportato alcun ruolo attivo nella predisposizione di questi materiali, per esempio file che circolano sulle strutture da egli apprestate, la responsabilità all'Internet Provider per questo può cagionare un effetto distorto che è quello di attribuire all'Internet Provider degli impropri poteri o compiti di censore della rete. Vale a dire che per evitare che il provider si senta a rischio di essere imputato di responsabilità civile, attui dei controlli preventivi o anche successivi, sui contenuti che circolano in rete.
Per alcuni tipi di contenuti si può agevolmente rendersi conto se il contenuto che circola è illecito, potremmo anche pensare a forme meno strutturate di immoralità o simili. Però per molti tipi di violazioni, di contenuti illeciti che circolano sulla rete, per esempio violazione dei diritti autore, controverse relative alla diffamazione o anche la violazione della privacy, per molti tipi di illeciti attuati tramite Internet può essere difficile per il Provider accertare egli in prima persona, al di là dell'intervento di un organo pubblico o giurisdizionale, accertare se il contenuto che egli ospita sia effettivamente illecito. Probabilmente nel dubbio molti contenuti perfettamente leciti varrebbero bloccati. 

Di fronte a questa esigenza di evitare forme di responsabilità oggettiva, di evitare di porre a carico dell'Internet Service Provider obblighi di controllo eccessivi e sproporzionati, che in realtà sarebbero difficilmente praticabili e potrebbe essere molto difficile per un Internet Provider controllare e verificare tutti i contenuti che circolano sulle su strutture, per evitare che il Provider si trasformi in una sorta di censore occulto, o censore non occulto ma senza nessun titolo per essere censore, a fronte di tutte queste esigenze è stata emanata prima una normativa comunitaria contenuta nella direttiva sul commercio elettronico e sui servizi della Società dell'Informazione e poi queste normative comunitarie sono state recepite nell'ordinamento italiano. 

La normativa italiana rilevante oggi è contenuta negli articoli da 14 a 17 del Decreto Legislativo 70/2003 (riproduce l'articolo 15 della Direttiva 2000/31/CE). Quali sono i principi a cui si ispira la normativa  comunitaria e italiana? Le normative sono pressoché pedisseque, collimano abbastanza tranne alcune piccole discrepanze. 
La normativa italiana e l'europea si ispira a tre criteri di base:

\begin{itemize}
    \item distinzione tra i tipi di servizi che possono essere resi da un internet provider. Si individuano attività diverse che possono essere perseguite da un soggetto fornitore di servizi della Società dell'Informazione; non si individuano soggetti diversi ma si individuano attività e a seconda del il tipo di attività, seguirà un certo regime di responsabilità o meno
    \item  Stesso regime di responsabilità per tutti gli illeciti. Qualsiasi tipo di illecito, violazione della proprietà intellettuale, violazione della privacy, diffamazione o altre cose, richiamano in causa lo stesso regime di responsabilità. Attenzione, qui stiamo parlando principalmente di responsabilità civile. Quindi lo stesso regime di responsabilità segue tutti i tipi di illeciti che si realizzano tramite le strutture, tramite i servizi, le reti e quant'altro, le macchine, i programmi predisposti dall'internet service provider. La responsabilità del provider non cambia anche se possono cambiare gli illeciti commessi tramite le sue strutture
    \item Infine un principio molto importante che ha però anche delle deroghe è l'assenza di un obbligo generale di controllo da parte dell'internet service provider sui contenuti messi in rete da terzi
\end{itemize}

Quindi si chiarisce a livello di principi, a livello di dichiarazioni esplicite che il provider non ha un obbligo generale di controllo sui contenuti che transitano sui suoi servizi. 
%13:43
Vediamo i tipi di servizi indicati dalla direttiva e dalla normativa italiana. 

Un primo tipo di servizio:

\begin{itemize}
    \item semplice trasmissione (mere conduit \footnote{La definizione di "mere conduit" (tradotto anche come "semplice trasporto" o "semplice trasmissione") si riferisce a un servizio che consiste nella trasmissione, su una rete di comunicazione, di informazioni fornite da un utente del servizio o nella fornitura dell'accesso a tale rete. In sostanza, il fornitore non seleziona, modifica o controlla le informazioni trasmesse, agendo come un semplice canale di comunicazione. }). Trasmissione di informazioni fornite da un utente oppure forniture di accesso ad una rete di comunicazione. Qui è come se l'internet provider fosse semplicemente colui che mette a disposizione un canale di comunicazione e semplicemente trasmette le informazioni che vengono immesse da altri. Non è tenuto a conoscerle, fornisce altresì l'accesso eventualmente alla rete di comunicazione. Per questo tipo di attività in generale il provider non è responsabile, non può essere civilmente responsabile a meno che non abbia dato inizio lui alla trasmissione e non sia intervenuto sui dati trasmessi 
    \item Un secondo tipo di attività è quello che si chiama caching, vale a dire la memorizzazione automatica intermedia e temporanea di informazioni effettuata al solo scopo di rendere più efficace il loro successivo inoltro ad altri destinatari. Quindi è la memorizzazione temporanea di informazioni. Per questo tipo di attività è previsto che non ci sia la responsabilità del provider se non è intervenuto sulle informazioni e se richiesto dalla autorità competente ha prontamente rimosso i contenuti illeciti. A queste condizioni anche il provider che fornisce attività di caching non incorre responsabilità o viceversa incorre nel caso opposto
    \item hosting, vale a dire memorizzazione di informazioni a richiesta di un destinatario del servizio, non si tratta più di una memorizzazione temporanea come nel caso del caching ma di una memorizzazione stabile delle informazioni, dei contenuti immessi da un terzo, dall'utente del servizio, sulle strutture, le macchine, le reti, i programmi, predisposti dall'internet service provider. Anche qui la responsabilità non sorge se il provider non è intervenuto sulle informazioni e se avendo avuto notizia dell'autorità competente che sono presenti contenuti illeciti non li ha rimossi prontamente
\end{itemize}


Vedremo che questa tipologia di servizi della società dell'informazione, la mera trasmissione,il caching e l'hosting avevano una loro attualità nel momento in cui sono stati pensati dal legislatore comunitario e poi attuati da legislatore italiano quindi fino agli 90, primi 2000; attualmente vi sono molte attività che si svolgono sulla rete, svolte da internet service provider che sono difficilmente riconducibili a queste tipologie, a questa tricotomia. Per esempio l'attività di un gestore di un social network, facebook o twitter, è difficilmente catalogabile in una sola di queste categorie e spesso include anche un intervento attivo del gestore nel tipo di informazioni che vengono collocate in rete. 

Il quadro immaginato e dipinto dal legislatore comunitario e recepito dal legislatore italiano presuppone un tipo di operatore della rete che può essere assolutamente neutrale come qualcuno che si limita a mettere a disposizione uno spazio virtuale e poi arriva a qualcun altro, il content provider, che mette contenuti da egli predisposti. La struttura di responsabilità prevista dalla normativa comunitaria e dalla legislazione italiana ha in mente uno sfondo di comunicazione internet di questo tipo. Questo quadro comincia un po' a incrinarsi nel momento in cui appaiono, sul mercato e sulla rete, operatori che fanno attività un po' ibride, come i gestori di social network, ma anche i motori di ricerca che come vedremo non è chiaro a quale di queste categorie siano riconducibili. 

\subsection{Obblighi del prestatore}

Come abbiamo detto il prestatore incorre in responsabilità se non svolge certe attività come nel caso in cui egli venga informato di contenuti illeciti nei servizi da egli predisposti. 

\begin{itemize}
    \item Un primo obbligo è informare l'autorità giudiziaria o amministrativa avente funzioni di vigilanza, per esempio in Italia l'AGCOM, l'autorità garante delle comunicazioni, qualora sia a conoscenza di presunta attività o informazioni illecite. Quindi supponiamo che l'internet provider sia venuto a conoscenza in qualsiasi modo di una attività o informazioni che possono essere illecite e gli non ha il dovere di rimuoverle, quindi di farsi censore in prima persona, ma ha il dovere di informare prontamente l'autorità giudiziaria o l'autorità amministrativa con funzioni di vigilanza
    \item Un secondo obbligo fornire a richiesta dell'autorità competenti le informazioni in suo possesso che consentano l'identificazione dei responsabili
    \item Infine laddove l'autorità giudiziario amministrativa competente richiede formalmente all'internet provider di rimuovere i contenuti o impedire l'accesso ai contenuti che vengono considerati dall'autorità giudiziaria o amministrativa competenti illeciti, laddove internet provider non sia prontamente intervenuto su richiesta dall'autorità giudiziaria o amministrativa per rimuovere questi contenuti o impedire l'accesso è soggetto a responsabilità civile

\end{itemize}

Il rovescio della medaglia del primo obbligo è che l'ISP può incorrere in responsabilità civile se avendo avuto conoscenza del carattere pregiudizievole illecito delle comunicazioni che circolano sulla rete tramite i servizi deglida egli predisposti non abbia provveduto a informarne l'autorità competente. Questo è il rovescio della medaglia del primo obbligo e laddove il primo obbligo non venga ottemperato, sorge la responsabilità civile a carico dell'internet provider. 

Esiste una casistica abbastanza ricorrente di controversie nelle quali si cerca di fare valere la responsabilità civile dell'internet service provider. I casi più ricorrenti che occupano molto spesso le aule del tribunale sono la diffamazione online e la pirateria online, laddove per pirateria online intendiamo qualsiasi sistema di comunicazione, trasmissione, diffusione di contenuti che sono opere dell'ingegno protette da diritto dell'autore. Per pirateria online potremmo intendere il file sharing o il download illegale di file musicali o di film e quant'altro sia protetto da diritto dell'autore. 

Come abbiamo visto le esigenze che qui emergono laddove si tratti di individuare un sistema più appropriato di attribuzione eventuale, di responsabilità all'internet provider, i problemi sono il rischio di attribuire al provider una qualche forma di responsabilità oggettiva tutte le volte in cui non si possa dimostrare che il provider abbia svolto qualche tipo di intervento attivo sui contenuti trasmessi, quindi ritenerlo responsabile per un'attività in cui egli non è veramente intervenuto sarebbe un tipo di responsabilità oggettiva, che non è di solito una buona soluzione, una soluzione che suscita vari problemi nell'ambito dei regimi di responsabilità. Inoltre, come dicevamo, bisogna evitare di attribuire all'internet provider i propri poteri di controllo e di censura. 

Un'impropria attribuzione o autoattribuzione di poteri censori all'internet provider sarebbero un immediato riflesso dell'imposizione sull'internet provider di un sistema troppo rigido di responsabilità per i contenuti che circolano tramite i servizi da egli approntati. 

\subsection{Diffamazione online}

Un primo filone di casistica relativa, presunta, rivendicata, responsabilità dell'internet provider riguarda la diffamazione online. Qui la giurisprudenza ha, fermo restando l'attribuzione di responsabilità all'autore materiale dell'informazione e del contenuto diffamatorio, ha coinvolto nella vicenda risarcitoria anche l'internet provider, a seconda che l'internet provider rilevante in quella vicenda avesse un ruolo più o meno attivo nei confronti dei contenuti immessi. 

L'esempio che ha interessato più di una volta la giurisprudenza italiana, specialmente verso la fine degli anni 90, è quello dei newsgroup moderati o non moderati o forum di discussione. Laddove questi siti abbiano un moderatore, si parla di newsgroup o forum moderati, e l'informazione o il dato diffamatorio era stato collocato nel sito e non rimosso dal moderatore, si è talvolta chiamata in causa a livello risercitorio, a livello di responsabilità civile anche il moderatore, come una sorta di concorso di responsabilità.

Viceversa, se l'internet provider appronta un newsgroup o forum di discussione non moderato, allora tendenzialmente si esclude la responsabilità dell'internet provider. 

Una tendenza abbastanza ricorrente che si ravvisa nella giurisprudenza è quella di assimilare le comunicazioni via internet, in generale, e i siti web in particolare alle pubblicazioni a stampa e quindi ad accostare la responsabilità per diffamazione a mezzo stampa alla responsabilità per diffamazione online.
Questo ha provocato alcune forzature perché ha talvolta condotto a un'estensione analogica del tutto impropria della responsabilità prevista dalla legge sulla stampa del direttore responsabile della testata sui contenuti del giornale ad una forma di responsabilità dell'internet provider o del gestore di un sito internet. Qui si tratta di una operazione interpretativa abbastanza rischiosa, abbastanza controversa, perché le norme sulla stampa sono talvolta presupposto di applicazione di norme penali e quindi non possono essere estese in via analogica. In più esistono alcune rilevanti differenze tra un giornale e un sito internet, tant'è però che per semplicità spesso questo regime esercitorio, le norme della legge sulla stampa sono state usate in maniera analogica per estendere la responsabilità dei gestori di siti internet rispettivamente alle informazioni o alle opinioni diffamatorie che circolavano su quei siti internet.

Abbiamo detto quando abbiamo commentato il decreto legislativo che ha introdotto la normativa sulla responsabilità dei provider e la direttiva comunitaria che essa aveva di fronte il quadro degli operatori e delle attività dei provider che oggi è abbastanza mutato. Proprio un caso affrontato al Tribunale di Milano nel 2010 fornisce un esempio di questo. Infatti in tale caso era successo che un soggetto aveva citato in giudizio il motore di ricerca Google perché col servizio del completamento delle ricerche su Google suggerite, al termine ricercato associava degli ulteriori termini suggeriti per estendere la ricerca o restringerla a seconda dei casi. Era successo che quando veniva digitato nel campo di ricerca il nominativo di questo soggetto che era un operatore finanziario per qualche motivo tra i suggerimenti di ricerca spuntavano le parole truffa e truffatore. L'interessato ha citato in giudizio il motore di ricerca, dapprima ha chiesto di rimuovere il collegamento con queste parole truffa e truffatore e non è stato fatto, dopodiché ha citato per risarcimento del danno da diffamazione; il Tribunale ha ritenuto che il motore di ricerca fosse responsabile della diffamazione perché effettuava una modifica delle informazioni, cioè associava al nominativo della persona interessata la qualifica di truffa e truffatore prendendola probabilmente da un accostamento casuale con alcuni siti. Quindi in questo caso il Tribunale ha giudicato il motore di ricerca responsabile per questo tipo di attività diffamatoria. 
%29:56

Un altro caso che ha visto la responsabilità degli Internet Provider, in questo caso responsabilità penale è il caso Google deciso del Tribunale di Milano nel 2010, Google contro Vividown. Si tratta di una vicenda che è stata alla ribalta degli organi stampa per molto tempo e ha portato alla condanna penale, ma con pena sospesa, di tre dirigenti di Google per un video postato su YouTube. 
Fondamentalmente era successo che in una scuola dei ragazzi avevano vessato, con atti di bullismo abbastanza sgradevoli, un loro compagno di classe affetto da sindrome di Down. Avevano ripreso il tutto con un telefonino e successivamente avevano postato il video su YouTube. Erano state fatte richieste sia dagli interessati sia dalla Polizia Postale di rimuovere il video, che è stato rimosso dopo circa due mesi, dopo essere stato soggetto a più di 5000 visualizzazioni e per un periodo il video risultava anche in cima alla classifica dei cosiddetti video divertenti.
Su questa vicenda si sono innestate tre cause giudiziarie. Una che riguardava i ragazzi autori della condotta, che sono stati condannati al Tribunale dei minori, un'altra che riguardava la scuola e il docente, per ommesso controllo e un'altra che riguardava i dirigenti di Google per non aver rimosso il video. In questo caso il Tribunale è arrivato a una decisione abbastanza strana. In primo luogo ha affermato che i dirigenti di Google non erano responsabili per diffamazione in quanto non avevano un potere di controllo sui contenuti messi in rete e di conseguenza la diffamazione potrebbe essere imputabile solo agli autori del filmato che insultavano la vittima del loro atto di bullismo. 
Il Tribunale ha ritenuto che la diffamazione non riguardasse i dirigenti di Google, ha ritenuto invece che i dirigenti di Google avessero effettuato un trattamento illecito dei dati sulla salute, dove il dato sulla salute è la condizione di disabilità della vittima di questa vicenda, che era un ragazzo affetto dai sindrome di Down. 
In realtà il percorso argomentativo è un po' complicato e anche non sempre solidissimo, ma la responsabilità fondamentalmente è stata ravvisata perché Google non avrebbe avvisato coloro che uploadano file sul sito sul fatto che esistono normative, per esempio la normativa sulla privacy, il trattamento dei dati personali e così via, quindi non avrebbe in un certo senso catechizzato adeguatamente i propri utenti, avrebbe tollerato per fini commerciali che sul sito venisse collocato qualsiasi tipo di filmato. 
Questo tipo di condotta è stato giudicato dal Tribunale un trattamento illecito dei dati personali. 

Infine ed è sempre un caso che riguarda motori di ricerca, è il caso About Ellie che ha riguardato il motore di ricerca Yahoo, era stato citato perché cercando questo film spuntavano dei link a siti da cui poter scaricare illegalmente il film. La decisione in questo caso è stata che se il provider riceve una segnalazione dalla casa di produzione del film per una situazione illecita, deve esercitare un controllo sui siti segnalati ed eventualmente rimuoverli dal motore di ricerca. 
Qui ritorniamo sempre al solito problema, cioè di attribuire all'Internet Service Provider un potere di controllo che può essere del tutto inadeguato, cioè il potere di controllo se chi sta diffondendo un video ha i diritti per farlo o meno. 

\section{Il processo telematico}
Passiamo adesso sinteticamente al secondo argomento di questa lezione che tratteremo in termini introduttivi sia perché non è ancora del tutto implementato nella realtà italiana, sia perché richiede un approfondimento tecnico del tutto superiore alla prospettiva introduttiva di questo corso.

Il processo telematico riguarda il processo civile e non è un nuovo tipo di processo, un processo speciale, ma è semplicemente il ricorso nell'ambito del processo civile a tutta una serie di strumenti di tecnologia delle informazioni e di telecomunicazioni. Sostanzialmente riguarda la possibilità che certi atti del processo e certi documenti processuali vengano scambiati tramite procedure informatiche o meglio ancora che i documenti vengano redatti con strumenti informatici e vengono poi scambiati a tutti gli effetti legali con strumenti informatici e telematici.  Per esempio laddove è necessario fare una notificazione o un deposito in cancelleria e quant'altro tutto questo venga fatto con strumenti informatici e telematici. 

La disciplina è stata introdotta per la prima volta con il DPR 123/2001 che è stato modificato successivamente e in maniera rilevante negli anni e abbiamo avuto recentemente nel 2011 un ultimo decreto integrativo attuativo che riguarda l'uso della posta elettronica certificata all'interno del processo. 

Le nozioni rilevanti questo deposito sono:

\subsection{Dominio giustizia}
Il dominio giustizia è l'insieme delle risorse hardware e software mediante il quale l'amministrazione della giustizia tratta in via informatica e telematica qualsiasi tipo di attività

\subsection{il sistema informatico civile o SICI}
Il sistema informatico civile o SICI che è il sottosieme delle risorse del dominio giustizia mediante il quale l'amministrazione della giustizia tratta del processo civile, quindi dedicate specificamente al processo civile. 

L'accreditamento o certificazione dell'avvocato, l'avvocato viene accreditato o certificato da un punto di accesso. 

Il punto di accesso è il soggetto che fornisce servizi di trasmissione dei documenti, la casella di posta elettronica certificata (PEC), per esempio può essere un Consiglio dell'Ordine o il Consiglio Nazionale Forense. Quindi il Consiglio dell'Ordine, il Consiglio Nazionale Forenze o anche altri soggetti possono attribuire agli avvocati l'accreditamento o certificazione, le strutture per l'accesso ai documenti e anche la casella di posta elettronica. 

È prevista la formazione di un fascicolo informatico che non sostituisce il fascicolo cartaceo, quindi quantomeno per ora si viaggia su due binari, informatico e cartaceo, il fascicolo informatico non è nient'altro che la versione informatizzata del fascicolo d'ufficio che è altresi cartaceo. 

Tramite il SICI che abbiamo visto prima, il difensore accede a tutti i fascicoli dei procedimenti in cui è costituito, quindi può accedere a distanza in remoto a tutti i fascicoli dei procedimenti di cui parte.

Anche gli ausiliari del giudice, come esperti, periti, consulenti tecnici, traduttori e quant'altro possono accedere al SICI nei limiti dell'incarico conferito, hanno un'autorizzazione limitata al tipo di incarico che hanno ricevuto. 

La trasmissione degli atti e dei documenti del processo da parte degli avvocati è effettuata tramite posta elettronica certificata, che ormai sappiamo essere il sistema di posta che assicura con efficacia legale la notizia dell'avvenuto invio della comunicazione, dell'avvenuta ricezione della comunicazione, anche se si tratta di una comunicazione non aperta e non letta però ricevuta, e che ha valore pari analogo alla notificazione a mezzo posta. 

La ricevuta di avvenuta consegna, nell'ambito del sistema di posta elettronica certificata, attesta l'avvenuto deposito dell'atto o del documento presso l'ufficio giudiziale competente. 

I documenti informatici prodotti nel processo sono sottoscritti con firma digitale. Abbiamo visto come funziona e cosa era la firma digitale a proposito del codice dell'amministrazione digitale. 

La riservatezza dei documenti scambiati viene assicurata al sistema di crittografia, sia nell'invio di atti al gestore sia nell'invio da parte di questi, da parte del gestore e le parti. 

Il cancelliere che di solito rilascia copie degli atti attesta la conformità delle copie agli originali sottoscrivendo anche gli con la propria firma digitale. 

La sentenza è redatta in formato elettronico e sottoscritta con firma digitale dall'estensore. Se una sentenza è emanata da un organo giudicante collegiale è sottoscritta sia dall'estensore sia dal Presidente.
