\chapter{Lezione 16 - Software, internet e diritto d'autore}

Gli argomenti della lezione di oggi:

\begin{itemize}
    \item la tutela del software
    \item i digital right management (DRM) e contenuti digitali
    \item le licenze d'uso
\end{itemize}


\section{La tutela del software}
Il legislatore italiano equipara il software alle opere dell'ingegno di carattere creativo con la legge sul diritto d'autore numero 633 del 1941. In Europa e in Italia si è scelto di proteggere giuridicamente il software con la stessa tutela che la legge accorda alle opere dell'ingegno di carattere creativo piuttosto che con altre forme di tutela più forti, ad esempio con brevetti per invenzioni industriali. 

Il software viene protetto nella sua forma di espressione (interfaccia grafica) e non nell'originalità della formula industriale (il software puro). Tutelare il software nella sua forma di espressione significa in concreto che il programma informatico viene protetto per l'originalità e la nobilità della sua forma di espressione così come fruibile, ad esempio il titolo, le rubriche, l'aspetto esterno e non nella originalità della formula industriale rappresentata;  l'idea del software, il software in quanto tale che dai giuristi viene comunemente definito come software puro è ciò che permette di trovare una soluzione tecnica a particolari problemi. 

Già si è accennato ad alcuni profili di tutela del software quando si è trattato di contratti informatici hardware e software. 

Il diritto d'autore tutela altre opere:

\begin{itemize}
    \item su internet i testi di qualunque tipo sono tutelati, anche le email, le pagine web nella loro parte testuale e i giornali online. A proposito dei giornali però vale la pena di ricordare che a questi si applica anche una disciplina specifica sull'editoria e quindi occorre in concreto anche distinguere tra quella che è una pubblicazione di testi priva di una struttura e quella che invece è una pubblicazione di testi a carattere giornalistico. Rispetto alla tutela dei testi, da segnalare anche che l'attività di copia e incolla di testi tratti da internet potrebbe in astratto a seconda delle situazioni costituire una violazione della normativa sul diritto d'autore
    \item le immagini, come immagini si intende le fotografie, le immagini artistiche, le opere cinematografiche, le opere musicali, tutte le immagini che vengono pubblicate su internet possono essere soggette alla tutela della normativa sul diritto d'autore, sempre in virtù dello stesso principio, cioè della realizzazione di un'opera di carattere creativo
    \item le banche dati. Per le banche dati la tutela fa riferimento alle situazioni nelle quali la banca dati è disposta secondo una certa scelta o c'è una certa disposizione del materiale e questa scelta e questa disposizione del materiale può costituire una creazione intellettuale dell'autore. Anche qui da notare che l'eventuale creazione di una banca dati contenente dei testi di altre persone non può poi prescindere dalla tutela accordata ai singoli testi. Un conto è la tutela della banca dati come serie di documenti o altre informazioni organizzati secondo un certo criterio e diverso è la tutela specifica accordata ad ogni singolo contenuto, ad ogni singolo testo
    \item le opere multimediali. In particolare all'interno delle opere multimediali sono tutelate le opere musicali; la musica ha comunque una tutela specifica, sono tutelate le opere audiovisive, sono tutelati i videogiochi
    \item le biblioteche online anche in questo caso sono da considerare che i testi raccolti all'interno della biblioteca hanno una loro tutela originaria di diritto d'autore
\end{itemize}

\subsection{Autore}

Ma chi è l'autore? L'autore è il creatore dell'opera quale particolare espressione del lavoro intellettuale. Questa è la definizione che si può ritrovare nella legge 633 del 1941 e a questa definizione occorre fare riferimento per valutare nelle diverse opere che possono essere sottoposte alla nostra attenzione chi sia effettivamente l'autore. In alcune situazioni può non essere semplice individuare l'autore di un'opera e quindi la legge soccorre dando dei criteri di riferimento. 

Nell'opera comune, che è quella creata con il contributo indistinguibile e inscindibile di più persone. In questo caso si parla di coautori e vi sono più coautori nella misura in cui essi hanno partecipato allo stesso modo alla creazione dell'opera; si tratta come detto di un'opera nella quale il contributo dei singoli non è individuabile. 

Se invece l'opera è costituita dalla riunione di parti di opere o di opere diverse in base ad una scelta di coordinamento per un fine letterario, una raccolta di testi di autori vari per esempio, si parla di opera collettiva e in questo caso è considerato autore dell'opera collettiva il costitutore, il coordinatore dell'opera; resta salva la tutela del diritto d'autore dei singoli testi che fa riferimento agli autori stessi. 

\subsection{I diritti dell'autore}

Detto questo vediamo quali sono i diritti dell'autore:

\begin{itemize}
    \item diritto morale inalienabile e senza limiti di tempo. Cosa significa? Significa che l'autore nel momento stesso in cui crea l'opera diviene autore e ha il diritto morale ad essere riconosciuto autore di quell'opera, cioè il diritto di paternità su quell'opera e questo diritto gli spetta senza limiti di tempo. Il diritto morale d'autore comporta il diritto di modificare l'opera in qualunque momento, il diritto dall'altra parte a che l'opera non sia modificata da altri, il diritto di divulgazione, il diritto di rivendica, il diritto di ritiro dell'opera dal mercato ove commercializzata
    \item diritto patrimoniale. Il diritto patrimoniale è temporaneo e cedibile. La temporaneità consiste nel fatto che questo tipo di diritto è attribuito per l'intera vita dell'autore e per un periodo dopo la morte. La peculiarità del diritto patrimoniale è che esso, a differenza del diritto morale, può essere cedibile. La cessione dei diritti patrimoniali comporta sostanzialmente la cessione della possibilità di sfruttamento economico dell'opera. E' ovvio che possa essere ceduto mentre il diritto morale non può essere ceduto. Nessuno potrà mai attribuire a se stesso la paternità di un'opera che non ha realizzato, ma sarà possibile sfruttare economicamente alcuni diritti ceduti dall'autore stesso. Tra i diritti patrimoniali, ad esempio il diritto di riproduzione, cioè della moltiplicazione in copie con qualsiasi mezzo e con qualsiasi forma, il diritto di adattamento o trasformazione o modifica, sempre senza violare quella che è la tutela dell'immagine o dell'autore e ancora il diritto di distribuzione al pubblico.
\end{itemize}

 Rispetto alla cessione di questi diritti o alla non cessione di questi diritti, l'autore ha delle forme di tutela. 
%08:53

\subsection{La tutela}

\begin{itemize}
    \item La prima forma di tutela di carattere preventivo sono le misure elettroniche di protezione. Le misure elettroniche di protezione sono delle misure tecnologiche che possono essere apposte sulle opere o sui materiali protetti e che sono destinati ad impedire o a limitare gli atti non autorizzati, nel caso del software di cui qui ci occupiamo, ad esempio la cifratura
    \item Oltre ad introdurre delle misure tecnologiche dirette ad evitare la possibilità di modifica, l'autore può introdurre delle informazioni elettroniche sul regime dei diritti. Cioè questo significa che l'autore può inserire nell'opera tutte le indicazioni che ritiene opportune rispetto ai suoi dati personali, rispetto al momento in cui l'opera è stata realizzata, rispetto a tutto ciò che può essere utile per identificare quell'opera e per far conoscere a chiunque il fatto che si tratta di un'opera riconducibile a quell'autore specifico
    \item la registrazione presso la SIAE, Società Italiana Autori Ed Editori. La registrazione presso la SIAE è relativa ed è possibile a qualunque tipo di opera protetta dal diritto d'autore; soltanto negli ultimi anni, dopo che nella legge 633 del 1941 è stata introdotta anche la tutela del software, la SIAE ha costituito un registro dedicato al software nel quale possono essere inseriti tutti i riferimenti dei software che vengono sottoposti alla tutela della SIAE
    \item Queste appena enunciate sono delle forme di tutela di cui l'autore si può avvalere per prevenire eventuali abusi rispetto all'utilizzo delle opere che da lui create. Laddove comunque vi siano delle violazioni dei diritti, l'autore potrà sempre rivolgersi all'autorità giudiziaria. Quindi il ricorso all'autorità giudiziaria sia civile sia penale è un'estrema razio con cui intervenire per correre ai ripari nel momento in cui il diritto d'autore viene violato. Questo si può fare ad esempio con riferimento allo sfruttamento non autorizzato delle opere, si può anche fare però con riferimento alla eventuale contraffazione, falsificazione delle opere da parte di altri
\end{itemize}

Spunto di riflessione su questa parte della lezione: quali sono i principi di tutela del software e in linea generale quali sono i principi di tutela del diritto d'autore e in particolare in che misura sono applicabili al software. 

\section{Digital right management e contenuti digitali}

La tecnologia digitale e internet facilitano la diffusione e la fruizione originale delle opere. Questo è un dato di fatto. Oggi è estremamente facile per chiunque abbia un minimo di conoscenza e di capacità di utilizzo dei sistemi informatici copiare dei testi oppure manipolarli, prenderne dei pezzi, arrangiare dei video o dei file musicali resi disponibili in rete, è possibile anche abbastanza facilmente accedere ai codici sorgenti dai software per modificarli. 

Si tratta di tutta una serie di attività possibili grazie alla tecnologia che possono incidere su delle opere realizzate da altri. È anche vero che attraverso la tecnologia è possibile condividere con altri le opere e ciò anche quando si tratti di persone distanti geograficamente ma con cui ad esempio si condividono interessi comuni. 

Tutto questo vuol dire che è possibile un conflitto con gli interessi morali e patrimoniali degli autori, interessi riconosciuti e tutelati dall'ordinamento. La facilità di accesso ad opere protette da parte degli utenti, facilità di utilizzo o manipolazione in autonomia anche in modo creativo rispetto alle opere ha portato effettivamente al rischio che queste attività, che sono attività difficilmente controllabili, entrino in conflitto con gli interessi di natura morale e patrimoniale degli autori che sono interessi riconosciuti e tutelati dagli ordinamenti che disciplinano la materia.

Generalmente si è assistito negli anni ad un ampio dibattito sul rapporto che deve esservi tra gli autori di un'opera e la comunità degli internauti, nel senso che vi è una scuola di pensiero secondo cui l'utilizzo, la pubblicazione su internet dovrebbe consentire una totale libertà di fruizione delle opere lì inserite. Dall'altra parte invece gli autori stessi in varie situazioni, quindi non solo gli autori tutelati dalle major ma anche giovani autori, singoli autori, desiderano proteggere il proprio lavoro pur avendo l'interesse a divulgarlo. 

\subsection{Digital Rights Management DRM}
Per questo sono nati i DRM, Digital Rights Management, che permettono di proteggere, identificare e tracciare le opere protette ed evitare usi non autorizzati. I DRM sono dei sistemi tecnologici mediante i quali titolari di diritto d'autore e di tutti i diritti connessi possono esercitare ed amministrare questi diritti nell'ambiente digitale. Si tratta di misure di sicurezza che possono essere incorporate nei computer, negli apparecchi elettronici e nei file digitali. 
%15:52
I DRM sono leciti sia in virtù e a partire dai WIPO Copyright Treaty.
La copertura giuridica internazionale dei DRM nasce dalla World Intellectual Property Organization che nel 1996 ha implementato un regolamento che è stato poi attuato in diversi Stati; innanzitutto la legge attuativa degli Stati Uniti e il Digital Millennium Copyright Act (DMCA), mentre in Europa il trattato è stato recepito da direttive 2009/24 e ed è stato anche ribadito con una sentenza relativamente recente della Corte di Giustizia CURIA C-128/11. 

In queste normative si attesta la licità dell'utilizzo dei Digital Right Management come strumento di tutela dei diritti d'autore su internet. La legge italiana 633 del 1941 di cui già abbiamo parlato consente agli autori di inserire delle misure tecnologiche di protezione dirette proprio ad evitare che qualcuno possa violare i diritti d'autore intervenendo illegittimamente sulle opere. Da notare che le misure tecnologiche di protezione introdotte devono essere, secondo la legge italiana, rimosse da chi le poste e soltanto in particolari casi stabiliti dalla legge, ad esempio per finalità di sicurezza pubblica o per assicurare il corretto svolgimento di un procedimento. 

Ci sono delle voci fuori dal coro che dicono che i DRM possono limitare la libertà dell'utente finale. In particolare questa è un'affermazione di Richard Stallman, un informatico di fama mondiale che per primo ha portato avanti l'idea del software libero e che ha voluto a più riprese sottolineare la possibile invasività di molte tecnologie DRM e addirittura ha reinterpretato questo acronimo DRM come digital restriction management invece che digital right management. Quello che ritiene Stallman e coloro che seguono la sua linea di pensiero è che le tecnologie DRM di fatto limitino pesantemente la libertà dell'utente finale e per questa ragione la Free Software Foundation di cui fra poco parleremo promuove un'iniziativa diretta alla rimozione dei DRM. 

Perché ci sono queste perplessità? 

Perché effettivamente attraverso la tecnologia DRM è possibile controllare la distribuzione di contenuti digitali e questo potrebbe potenzialmente implementare un sistema di censura su vasta scala. 

Spunto di riflessione: quali rischi possono derivare dai DRM? 

\section{Le licenze d'uso}

La risposta di moltissimi autori nel corso degli anni, sia giovani appassionati ma anche grandi multinazionali rispetto alle restrizioni poste dalla normativa e dalla tecnologia alla libera distribuzione e utilizzazione delle opere è stata la creazione e la diffusione di licenze che permettono una libertà molto ampia di utilizzo delle opere software e in ogni caso distribuite su internet.

\subsection{Software libero e open source}
\subsubsection{La Free Software Foundation e concetti morali (1980)}
Si tratta di due concetti molto importanti. La Free Software Foundation (FSF) del 1980 fa riferimento soprattutto a concetti morali. 
Il software libero a cui fa riferimento la Free Software Foundation fondata da Richard Stallman è un concetto che indica come libero il software nella cui distribuzione gli autori hanno deciso di indicare determinate caratteristiche. 

I concetti etici che sono:

\begin{itemize}
    \item la possibilità di studiare il software
    \item la possibilità di aiutare il prossimo
    \item la possibilità di favorire la comunità
\end{itemize}

In questa ottica sono state elaborate delle specifiche licenze d'uso che vengono utilizzate e applicate al software o alle opere distribuite. 
%20:49
Per poter applicare una licenza ad un'opera occorre averne il diritto, quindi l'esistenza di una licenza di software libero presuppone il riconoscimento del fatto che qualcuno ha un diritto d'autore da sfruttare.

\subsubsection{Open Source: definizione criteri (1990)}
Open Source invece prevede la definizione di alcuni criteri ed è stata elaborata nel 1990. Quando si parla di open source ci si riferisce alla open source definition derivata dalle Debian Free Software Guidelines e cioè a una serie di dieci punti pratici che definiscono quali criteri legali debba soddisfare una licenza per essere considerata effettivamente libera ovvero con il termine che si utilizza open source. 

Anche in questo caso come detto per il software libero si presuppone che colui che applica la licenza all'opera abbia diritto di farlo e quindi sia o l'autore dell'opera stessa o abbia ricevuto dall'autore il diritto di disporne a tutti gli effetti. 

Queste due tipologie free software e open source software partono da criteri giuridici applicati negli Stati Uniti nei quali il diritto d'autore ha delle caratteristiche leggermente diverse rispetto a quelle che sono le caratteristiche dei paesi di diritto europeo. In particolare l'ambito di applicazione del diritto morale d'autore negli Stati Uniti è ridotto rispetto a quello che è l'ambito di applicazione nei paesi di diritto europeo in particolare in Italia e la possibilità che ha l'autore di cedere i diritti è anche strutturata in maniera leggermente diversa. 
%23:00
In estrema sintesi nelle licenze free software o nelle licenze open source, l'autore indica nella licenza quali sono le attività che l'utilizzatore può fare e fra queste attività vi è generalmente la possibilità di modifica dell'opera senza alcuna limitazione purché  si mantenga l'indicazione di chi era l'autore originale dell'opera e si introduca l'indicazione di quali sono le modifiche apportate e da chi sono state apportate. 
Ne deriva sostanzialmente la possibilità di una sorta di catena di Sant'Antonio, un software può essere realizzato originariamente in un certo modo ma nel momento in cui viene distribuito con una licenza free o open source, chiunque può prendere quel software, utilizzarlo ma anche modificarlo e rimettere in circolazione il software così come modificato, purché mantenga le indicazioni di quali sono stati gli autori. 

Il primo obiettivo che si segue, in particolare con le free software, è quello di consentire alla comunità di studiare, di approfondire quelle che sono le innovazioni tecnologiche. Infatti il software libero è un concetto nato all'inizio degli anni 80, quando ancora la diffusione del software era certamente più limitata di quanto non sia oggi. 

L'evoluzione del concetto con l'open source è invece degli anni 90, quindi stiamo parlando di un periodo nel quale già le attività che venivano fatte e la capacità di comprendere l'importanza anche dal punto di vista economico dello sfruttamento del software era diversa. 

\subsubsection{Software proprietario}
Free software e open source si contrappongono al software proprietario per:
\begin{itemize}
    \item la conoscibilità del codice sorgente. Il software è composto di due parti fondamentali, un codice sorgente e un codice eseguibile. Il codice sorgente è intellegibile ed è adatto ad essere compreso da un programmatore, dall'uomo. L'eseguibile invece è un codice che non è intellegibile ma può essere eseguito dalla macchina. Nel software proprietario il codice sorgente generalmente non viene comunicato a nessuno, viene semplicemente reso conoscibile e utilizzabile il codice eseguibile in maniera tale che quelli che sono i segreti della realizzazione del software restano in capo a chi lo ha realizzato, a chi ha il diritto di sfruttamento del software stesso. Al contrario l'ideologia di base del software libero è che chiunque abbia la possibilità di studiare come è realizzato un software e quindi abbia accesso al codice sorgente e per questo nel free software viene sempre messo a disposizione il codice sorgente.
    \item nella possibilità di distribuzione libera. Deve esserci la possibilità di accedere al sorgente per sapere come è strutturato un software ma anche la possibiltà da parte di chi usufruisce del software di poter essere in grado di redistribuirlo anche modificato; questo tipo di libertà è una libertà che caratterizza il software libero
    \item possibilità di modifica, la possibilità di modifica è la terza caratteristica fondamentale che differenzia il software libero dal software proprietario
    \item Anche a fronte di queste libertà esistono comunque delle regole d'uso che gli autori che utilizzano delle licenze di software libero comunque impongono agli utilizzatori. In altri termini il fatto che vi sia una notevole libertà nell'approfondimento, nello studio, nella distribuzione, nella modifica del software non vuol dire che non ci sono delle regole da seguire. Anche se le regole sono delle regole relativamente limitate, sono delle regole abbastanza elastiche, comunque di regole si tratta e la loro violazione comporterebbe comunque un'attività illecita.
\end{itemize}

\subsection{Licenze d'uso}

Sono composte da:

\begin{itemize}
    \item condizioni di utilizzo del software che accompagnano il software e che specificano le modalità con cui l'utente può usare il prodotto garantendo diritti e imponendo obblighi. Il mancato rispetto delle condizioni di utilizzo è illecito
    \item Le licenze d'uso devono essere accettate, con riferimento al momento dell'uso, per l'installazione o addirittura preliminarmente. L'accettazione delle condizioni d'uso ha delle particolarità legate alle modalità di distribuzione del software. Un conto è l'acquisizione di un software su un supporto fisico, diverso è il download di un software da internet. Le modalità sono diverse e anche quindi le modalità di accettazione delle condizioni d'uso sono diverse. In queste modalità di accettazione si rivelano anche delle differenze che sono differenze fra ordinamenti. Non è questo il momento per approfondire gli aspetti legati ai contratti software, ma l'approfondimento di questi aspetti potete trovarlo nella lezione dedicata ai contratti.
\end{itemize}

\subsection{Copyright}

Il copyright statunitense ha una portata meno ampia del diritto d'autore di impostazione europea. Questo è un principio da tenere a mente quando si studiano le caratteristiche delle licenze di software libero. 

\subsection{Licenze free}

\begin{itemize}
    \item Tra le licenze più diffuse c'è la cosiddetta GNU-GPL, che è quella elaborata da Richard Stallman e che dà il cosiddetto permesso d'autore. Questa è anche chiamata una licenza di copyleft ed è quella forse più utilizzata anche al giorno d'oggi ed attribuisce la massima libertà di utilizzo
    \item creative commons che sono delle licenze che mantengono la riserva di alcuni diritti. Sono delle licenze più spesso utilizzate per i contenuti creativi piuttosto che per il software, quindi per la distribuzione di contenuti creativi su internet e che sono basate su una localizzazione, nel senso che la creative commons ha individuato localmente dei referenti per queste licenze che hanno elaborato e rivisto le licenze elaborate originariamente alla luce di quelli che sono i principi dell'ordinamento locale. E quindi è molto interessante anche confrontare le varie licenze creative commons elaborate nei diversi paesi. Per quanto riguarda l'Italia esistono un set di quattro clausole da cui derivano sei licenze e l'utilizzo di queste licenze rende molto semplice per il titolare dei diritti di segnalare con estrema chiarezza quali tipo di attività, riproduzione, diffusione, circolazione, modifica, commercializzazione e quant'altro sia permessa in modo esplicito.Le creative commons in particolare hanno dei simboli specifici dedicati a seconda della tipologia di licenza che viene utilizzata dall'autore che viene applicata all'opera
    \item la licenza Mozilla che nasce per tutelare i software del gruppo come Firefox, Thunderbird e così via. Anche questa è una licenza di software libero che ha ancora delle caratteristiche sue particolari
\end{itemize}


Il modello a codice aperto permette di concentrarsi sulla vendita di un pacchetto o di un servizio più che sulla vendita del software. Questo aspetto è estremamente interessante. Nel momento in cui io come autore acconsento al fatto che vi sia una estrema libertà nell'utilizzo dell'opera e quindi applico una licenza che dà un'estrema libertà nell'utilizzo dell'opera, mi interessa più fornire un servizio piuttosto che valutare il contenuto di quel software che viene distribuito. 

Questo approccio è stata una politica di business delle multinazionali ed è un approccio che viene seguito tuttora. Le grandi case che producono software e che hanno deciso di distribuirlo utilizzando delle licenze free si sono concentrate nello sviluppo dei servizi per il software e quindi hanno impostato la loro attività commerciale in un modo diverso rispetto a coloro che tradizionalmente producono software che continuano ad essere esclusivamente produttori di software. 

\subsection{I vantaggi del free software}
%33:48
\begin{itemize}
    \item Innanzitutto l'evoluzione. Il modello di sviluppo open source si adatta moltissimo ad un'impresa che ha interesse a creare nuovi prodotti e ha interesse a migliorare quelli esistenti
    \item La qualità. Si ritiene che la qualità di un software che viene sviluppato mano a mano da una comunità di utenti possa essere migliore di quella di un software sviluppato da un unico soggetto o comunque da un gruppo di soggetti che non ha altri raffronti con l'esterno
    \item manutenzione migliorativa del software. Anche in questo caso si ritiene che le proposte di modifica portate da un gruppo di utenti più vasto di quello che sono le persone che operano all'interno di una determinata società possa agevolare il miglioramento del software. Si dice a questo proposito che la comunità open source è una comunità ricca di individui che hanno grandi capacità e grande talento
\end{itemize}

Nel software open source però vi sono delle clausole di esonero, di limitazione della responsabilità per danni che possono essere estremamente significative. Il fatto che il software open source sia sviluppato da molti sviluppatori, cioè che molte persone contribuiscano alla realizzazione del software rende quasi inevitabile l'introduzione di clausole di esonero della responsabilità estremamente limitanti perché nel momento in cui il software è sviluppato da più persone che magari non hanno alcun rapporto fra di loro, nel caso in cui si verifichino dei danni può essere estremamente difficile risalire alle ragioni e chi è stato l'autore della modifica che ha comportato dei danni; quindi si comprende questo tipo di scelta anche se vi sono delle conseguenze negative rispetto all'utilizzatore finale. 

Il codice sorgente che non viene rilasciato al licenziatario nel caso del software proprietario viene invece rilasciato sempre e comunque visibile e accessibile nel software open source. 

Nel software open source l'apporto di contributi esterni è fondamentale sia per la riparazione di eventuali problemi sia per i miglioramenti del software. 

Esiste uno sviluppo di modelli di business che presuppongono l'uso del software open source. 
La diffusione che hanno avuto questo tipo di licenze nel corso degli ultimi 15 anni è tale che ormai anche le grandi multinazionali hanno deciso di adottare questo sistema come sistema da cui partire per fare business ed è qualcosa che possiamo verificare in qualunque momento in cui si acquisisce un software ma anche con riferimento a quelli che sono i contenuti distribuiti su internet perché quando si parla di licenze d'uso le licenze d'uso non fanno soltanto riferimento all'uso del software ma fanno riferimento e vengono utilizzate con riferimento alla diffusione di qualunque tipo di opera protetta dal diritto d'autore e quindi anche nella diffusione di contenuti digitali del carattere più vario a partire dalla musica per continuare con le fotografie e così via come abbiamo detto all'inizio di questa lezione. 

Quindi l'utilizzo di una licenza che consenta la distribuzione, la modifica e quant'altro è qualcosa che ha interesse per gli utenti sia per gli utenti che vogliono utilizzare un software ma anche per gli utenti che semplicemente vogliono acquisire e condividere ed eventualmente modificare laddove è possibile altre tipi di opere. 

Questi gli aspetti positivi, quali sono i rischi nell'utilizzo del software open source? 

Innanzitutto i costi, si parla di total cost of ownership (TCO). Quando si parla di costi infatti non si può soltanto prendere in considerazione quelli che sono i costi di acquisto, nel caso del free software, del software open source in linea di massima, può non esserci un costo di acquisto però poi vi sono le spese dei servizi di supporto e quindi spese che possono essere estremamente rilevanti anche quando si fa riferimento a software open source o free software. 

In secondo luogo l'inaffidabilità della comunità. L'efficienza del software nel caso dell'open source e del free software dipende strettamente dall'esistenza di una comunità che sia sufficientemente varia e sufficientemente ampia di programmatori. Se la partecipazione dei programmatori viene meno il progetto rischia di arenarsi. 

Ancora l'incompatibilità con applicazioni specifiche. E' possibile, se vi sono delle incompatibilità con applicazioni specifiche, che non si formi una massa critica utile per lo sviluppo in serie e questo ancora una volta può comportare la chiusura del progetto di sviluppo o può comportare un aumento considerevole dei costi. 

Da ultimo la limitazione della responsabilità di cui abbiamo già parlato e che effettivamente può essere un problema di cui tener conto nel caso in cui si creano delle difficoltà o vi sono dei danni. 

Lo spunto di riflessione: vi sono le ragioni per la riduzione delle limitazioni della tutela del diritto d'autore su internet o meno.

Il fatto che si parli di diffusione su internet deve portare, anche nel caso del software, a interrogarsi su quali siano le regole migliori. È corretto oggi continuare a portare avanti una tutela del diritto d'autore strutturata come lo era in passato a fronte di innovazioni come quelle a cui stiamo assistendo? Ha un senso ancora oggi voler continuare a avere l'idea che le opere potrebbero avere delle limitazioni nella loro modalità di distribuzione oppure bisogna seguire quelle che sono le indicazioni che molti del popolo di internet danno rispetto alla necessità dell'assoluta libertà?
