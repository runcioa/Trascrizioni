\chapter{Lezione 18 - Computer Crimes e Computer Forensics}

L'argomento della lezione di oggi è Computer Crimes e Computer Forensics.

\section{Computer Crimes}
Per Computer Crimes o diritto penale dell'informatica si intende in generale la rilevanza dell'informatica nel diritto penale. Questo può avvenire in due ordini di fatti specie. La prima tipologia di fatti specie di diritto penale dell'informatica sono i reati comuni commessi con strumenti informatici.Vale a dire, tutta una serie di reati comuni, reati che possono essere commessi nel mondo reale, con strumenti fisici o comunque con strumenti non informatici che  possono anche essere commessi con strumenti informatici. 

Questo accade per esempio con violazioni della privacy, effettuate con strumenti informatici, cosa che non è necessario avvenga con strumenti informatici. La privacy può essere violata in altre modalità. 

E' il caso della diffamazione online. La diffamazione è un reato che nasce nel mondo della stampa o con la comunicazione orale e così via. E' possibile ovviamente, siccome gli strumenti della socità dell'informazione diffondono informazioni, diffondono anche opinioni, idee, manifestazioni e pensieri, è possibile che la diffamazione avvenga online tramite internet.

La pedopornografia si è attuata con strumenti informatici. 
Quindi questi sono esempi di reati comuni che possono essere commessi con strumenti informatici. 

Altre altre famiglie di fattispecie riconducibili al diritto penale dell'informatica, invece, sono reati informatici in senso stretto. Vale a dire reati compiuti in danno di sistemi informatici. In questo tipo di reati il sistema informatico o l'informatica, il bene informatico, è il bene oggetto di protezione penale ed è ciò che la normativa penale intende proteggere.
Quindi in questa seconda famiglia di fattispecie, la norma penale intende proteggere un bene che ha qualità informatica, quindi ad esempio accesso abusivo a banche dati, accesso abusivo a sistema informatico, contraffazione di software e così via. Il bene giuridico che la norma penale protegge ha una natura informatica. A differenza, invece, della fattispecie del primo tipo in cui il bene giuridico è un bene giuridico normale, che fa parte della vita reale e la lesione al bene giuridico avviene tramite strumenti informatici. 

In Italia abbiamo avuto una normativa apposita sui reati informatici abbastanza recentemente. Mentre i primi provvedimenti negli Stati Uniti ci sono già dagli anni 80, il primo del 1984, poi un secondo importante provvimento è del 1986, solo successivamente il Consiglio di Europa ha emanato la raccomandazione 89 che raccomanda ai Paesi membri di adottare normative specifiche sulla protezione di beni aventi di tipo informatico o comunque sul diritto penale dell'informatica. 

In Italia è stata necessaria una nuova normativa che è arrivata nei primi anni 90. Era necessaria perché era difficile estendere in via interpretativa le disposizioni preesistenti, quelle previste dal codice penale o da altre disposizioni penalistiche fuori dal codice, ai beni di cui stiamo parlando per due motivi; in primo luogo perché il diritto penale italiano è retto dal principio di stretta legalità, vale a dire affinché una certa condotta sia sanzionata penalmente sia punibile con una pena, è necessario che quella condotta sia tipizzata in una fattispecie legislativa, quindi è necessario che il legislatore abbia preso esattamente in esame quel tipo di condotta. Questo per i molti beni come quelli di tipo informatico mancava. 

Vi erano delle lacune per esempio per la repressione penale dell'accesso abusivo ad una banca dati o a un sistema informatico. Inoltre, secondo aspetto, risultava talvolta difficile estendere in via interpretativa le normative esistenti su alcuni reati, nel momento in cui il reato stesso fosse stato commesso con strumenti informatici. 

Quindi si poneva il problema se la frode informatica potesse essere qualificata deplano, senza problemi di sorta, come un tipo di truffa; quindi se fosse legittima questa interpretazione estensiva o se invece stessimo versando in un'ipotesi di estensione analogica che per il diritto penale è vietata. 

Stesso dubbio poteva porsi, per esempio, sulla possibilità di estendere le tutele penali sul segreto della corrispondenza a forme di corrispondenza non cartacee, forme di corrispondenza diverse, che potrebbero essere quelle che si verificano su internet; quindi per evitare il rischio o il dubbio che certe estensioni di reati preesistenti a nuove fatti specie avvenissero con un'operazione interpretativa qualificabile come analogia e l'analogia nel diritto penale è vietata, allora il legislatore ha ritenuto opportuno, in talunni casi, introdurre specifiche fatti specie penali o estendere normative preesistenti. 

La normativa principale che disciplina il diritto penale dell'informatica è quella introdotta dalla legge 547 del 1993. Questa legge in gran parte inserisce nuovi articoli o modifica articoli esistenti nel codice penale, quindi sono introdotti nel codice penale o nel corpo del codice penale degli articoli dedicati espressamente ai reati aventi a che fare con l'informatica. Questo è stato fatto secondo le due direttrici che menzionavo poc'anzi, vale a dire o con l'introduzione di nuove fatti specie, l'esempio è l'accesso abusivo al sistema informatico, o con l'estensione di fatti specie già esistenti. Quindi in alcuni casi sono state create delle figure di reato del tutto nuove, perché non vi era nulla che potesse plausibilmente somigliare nel codice penale preesistente a quella fattispecie nuova che riguarda la protezione di un bene informatico. 
Oppure, seconda modalità, presa a una disposizione del codice penale che disciplina una certa condotta, una certa fattispecie, questa fattispecie è stata è stata integrata espressamente dal legislatore con il riferimento a certi beni informatici. L'esempio è quello della violenza sulle cose, la violenza sulle cose si può esercitare su cose fisiche, su cose materiali e il legislatore ha esteso questa possibilità anche alla violenza su sistemi informatici, reti e quant'altro. 

Vediamo alcuni esempi specifici di reati aventi a che fare con l'informatica, sono tanti, ne vedremo solo alcuni e più interessanti. 

\subsection{Esempi di reati informatici}
\subsubsection{Frode informatica}
Il primo esempio è la froda informatica disciplinata dall'articolo 640ter del codice penale è considerato un tipo di truffa e in effetti come collocazione all'interno del codice si trova molto vicino alla truffa. La truffa è un reato che consiste nell'impiegare artifici o raggiri per indurre qualcuno in errore affinché compia qualcosa che beneficia ingiustamente l'autore della truffa e danneggia ingiustamente il soggetto truffato. Potremmo dire che la frode informatica è una sorta di truffa ai danni di un calcolatore, ai danni di un sistema informatico o di un programma. 
Come si verifica? Si realizza tramite alterazione del funzionamento di un sistema informatico, tramite alterazione dei delle informazioni o del software, quindi è come se con una frode informatica il truffato fosse un computer o un sistema informatico, alterando il modo in cui funziona il suo software o uno dei suoi software, oppure alterando i dati, le informazioni in esse contenuti. Questi sarebbero gli artifici o raggiri applicati al mondo informatico. 

È necessario affinché il reato si realizzi che l'autore della frode informatica realizzi o intenda realizzare un ingiusto profitto per sé e al contempo realizzi o intenda realizzare un danno ingiusto ad altri. È pertanto richiesta per questa figura il dolo specifico, è una figura di reato esclusivamente dolosa. 
Il dolo consiste nell'attuare le alterazioni del sistema o dei dati al fine di realizzare un ingiusto profitto o realizzare un danno ingiusto per altri.

\subsubsection{Accesso abusivo a sistema informatico}
Un altro esempio di reato informatico è l'accesso abusivo a sistema informatico, disciplinato dall'articolo 615ter del codice penale. È un reato collocato tra i delitti contro l'inviolabilità del domicilio. Quindi si considera che il sistema informatico sia equiparato al domicilio di quel soggetto. Entrare abusivamente nel sistema informatico di un certo soggetto è come entrare nel suo domicilio, come entrare a casa sua o nella sede del suo domicilio professionale. Sappiamo inoltre che la tutela del domicilio è una tutela rafforzata perché è considerata inviolabile dalla Costituzione. 
La Costituzione considera il domicilio inviolabile, quindi si giustifica una protezione penale del domicilio, ancorché informatico. L'accesso abusivo al sistema informatico può realizzarsi in due tipi di fattispecie a e b, per comodità espositiva, non è un'articolazione fatta dal codice:

\begin{itemize}
    \item a) La prima fattispecie è introdursi indebitamente, illegitamente in un sistema munito di misure di sicurezza
    \item b) La seconda fattispecie è mantenersi indebitamente all'interno di un sistema informatico contro la volontà dell'avente diritto, volontà che può essere espressa o tacita
\end{itemize}

Entrambe le fattispecie, ma soprattutto la fattispecie a, e lo sottolineo perché su questa che si è verificato qualche dubbo interpretativo sono necessariamente effettuate con strumenti elettronici. La formulazione testuale dell'articolo infatti dice che si introduce in un sistema informatico. Si è infatti posto il problema se possa considerarsi accesso abusivo ad un sistema informatico per esempio entrare nella stanza in cui ha sede un server, in cui hanno sede le apparecchieture informatiche di un soggetto o di un'azienda e così via. Secondo l'interpretazione letterale che normalmente nel diritto penale dovrebbe essere preferita, si dovrebbe in realtà configurare il reato, o meglio la consumazione del reato, nel momento in cui si accede al sistema con strumenti informatici. 

La condotta a, introdursi abusivamente in un sistema informatico, è di solito prodromica alla condotta B, cioè al mantenersi in un sistema informatico. Quindi si accede abusivamente al sistema informatico per poi fare qualcosa all'interno di quel sistema informatico, captare dati riservati, hackerare il sistema affinché quella macchina svolga certe operazioni a distanza e così via. Tuttavia non è vero il contrario, vale a dire la fattispecie b, mantenersi all'interno di un sistema informatico contro la volontà dell'avente diritto, può realizzarsi senza la fattispecie a. Vale a dire, ci si può mantenere indebitamente all'interno di un sistema informatico contro la volontà dell'avente diritto, senza che l'accesso sia stato abusivo. L'accesso potrebbe essere avvenuto col consenso del titolare dell'avente diritto sul sistema informatico. 

Supponiamo ad esempio che io porti un computer a riparare presso un centro d'assistenza. Il soggetto che fa assistenza e riparazioni accede al sistema col mio consenso, perché l'ho utilizzato io a effettuare le operazioni di riparazione. Supponiamo poi che però quel soggetto introduca un software spia, un spyware, all'interno del mio sistema. In questo modo si realizza un mantenersi abusivamente all'interno del sistema anche se l'accesso originario era stato valido, era stato effettuato lecitamente. 

Altro esempio potrebbe essere il fatto della condotta del dipendente che utilizza certi sistemi informatici dell'azienda in cui lavora, per esempio accede a banche dati e così via, al di là dell'orario di lavoro. Supponiamo che il dipendente abbia l'autorizzazione di accedere ai sistemi esclusivamente all'interno dell'orario di lavoro e supponiamo che acceda al di fuori dell'orario di lavoro. In questo caso ha avuto accesso all'interno dell'orario di lavoro quando aveva titolo di farlo, ma si è mantenuto oltre e quindi la sua condotta è diventata quella della fattispecie b. 

La disposizione penalistica richiede l'esistenza di misure di sicurezza. Diciamo che in un certo senso l'esistenza di misure di sicurezza è funzionale a far presumere una volontà del titolare del sistema di selezionare gli accessi leciti. Quindi il fatto stesso che il sistema è protetto da misure di sicurezza fa presumere che l'accesso al sistema da parte dei soggetti che il sistema di sicurezza non riconosce, per esempio non sono in possesso della password, fa presumere un accesso abusivo, fa presumere che l'accesso si sia verificato senza la volontà del titolare dei diritti sul sistema. 
Il reato si consuma nel momento in cui si superano le misure di protezione. Quindi non è necessario che chi ha effettuato l'accesso abusivo al sistema poi raccolga dati o prenda cognizione di dati più o meno riservati contenuti all'interno del sistema, affinché il reato si consumi e scatti la responsabilità penale, è sufficiente superare le misure di protezione di cui il sistema è dotato. 

\subsubsection{Accesso abusivo a sistema informatico: reato di pericolo}
E' un reato di pericolo. Cosa vuol dire? Vuol dire che la tutela penale è anticipata rispetto al verificarsi di un reale o di un qualsiasi danno o ingiusto profitto e così via. 
La differenza di quanto abbiamo visto con l'esempio precedente della froda informatica. I reati di pericolo sono quei reati contrassegnati da un'anticipazione della tutela penale. 
La tutela penale o la responsabilità penale scatta in un momento anticipato rispetto a quello in cui si verifica un danno apprezzabile. Si presume la dannosità di un certo comportamento o la pericolosità ire ipsa \footnote{"In re ipsa" indica che la natura del fatto o della situazione è di per sé prova di un danno o di una responsabilità.} di un certo comportamento. 
Accedendo abusivamente ad un sistema informatico si compie la condotta penalmente rilevante anche se poi dentro quel sistema informatico non si è svolta nessuna operazione o anche se non si è in grado di provare quali siano le operazioni poi svolte all'interno del sistema informatico in cui si è praticato l'accesso abusivo.

È possibile che questo reato si presenti sotto la forma del tentativo, quindi reato consumato, reato tentato. Questo reato può verificarsi sotto la forma del tentativo. Come? Per esempio, se semplicemente si cerca di forzare le misure di protezione del sistema senza riuscirci, per esempio fare ingresso nei locali fisici in cui sono custoditi gli elaboratori protetti da misure informatiche di sicurezza. L'accedere ai locali dove sono custoditi elaboratori elettronici, sistemi informatici, può configurare, se il contesto consente questa interpretazione, quindi i mezzi idonei, un tentativo di accesso abusivo. 


\subsubsection{Diffusione di virus informatici}
Diffusione di virus informatici, articolo 615, quinques. 

Si intende per diffusione di virus informatici la diffusione, ovvero la comunicazione, la trasmissione, la registrazione di un programma informatico che ha per scopo o effetto il danneggiamento, l'interruzione, l'alterazione, o comunque l'interferenza con programmi informatici, con il loro funzionamento, con i dati che essi contengono, eccetera, eccetera. 
La fattispecie di diffusione di virus informatici richiede che il virus, quindi il programma destinato a interferire con altri programmi, sia effettivamente messo in circolo. Quindi sia comunicato individualmente a qualcuno, o sia diffuso con strumenti di un certo tipo. 
Affinché scatti la responsabilità penale non è necessario che un certo virus informatico sia stato prodotto o detenuto da un soggetto. Quindi, ammesso che io ne abbia la capacità tecnica, non commetterei un reato se progettassi un virus nel chiuso del mio studio, senza mettere in circolazione questo virus. D'altro canto, affinché scatti la responsabilità penale non è necessario che il virus sia stato prodotto da chi lo trasmette. 
Quindi è penalmente responsabile per diffusione di virus informatici chi trasmette, comunica, diffonde, i virus informatici anche se non li ha prodotti lui, anche se non sono stati prodotti in proprio. 

Un problema relativo a questa norma. La norma non richiede dolo specifico, ma solo dolo generico. Di conseguenza il dolo specifico sarebbe la volontarietà di trasmettere un virus, cioè la volontà di provocare un'interferenza, un danno e quant'altro, con un sistema informatico altrui o un danneggiamento dei dati che si contengono. La norma non richiede nulla di tutto ciò, non richiede questo dolo specifico. Di conseguenza c'è il rischio che un'interpreazione letterale di questa norma, di questa disposizione penalistica, dia luogo ad una eccessiva estensione dell'area della punibilità. A prima vista, infatti, potrebbe essere punito ai sensi della norma che stiamo commentando chi trasmette un virus ad una società informatica che si occupa di studiare virus, per progettare antivirus. Supponiamo che io mi accorga di avere un virus nel computer, lo trasferisco su un supporto, su un dischetto, su un CD, su un DVD  e trasmetto questo supporto che contiene il virus informatico ad una società specializzata in informatica che progetta sistemi antivirus dicendo, vi sto trasmettendo questo virus affinché lo analizziate. A prima vista questa condotta sembrerebbe ricadere nell'ambito della norma, quindi è necessario qualche tipo di interpretazione correttiva affinché la norma non punisca comportamenti che non hanno disvalore sociale. 

\subsubsection{Violazione della corrispondenza}
Esistono disposizioni penalistiche, in particolare l'articolo 616, che riguardano la tutela della corrispondenza. Ebbene questa disposizione è stata estesa a coprire anche la corrispondenza che si realizza con strumenti elettronici o informatici. 
Vediamo le fattispecie previste da questa norma. 

\begin{itemize}
    \item prendere cognizione del contenuto di una corrispondenza chiusa è reato ai sensi dell'articolo 616 del codice penale
    \item sottrarre o distrarre, al fine di prenderne conoscenza o farne prendere da altri cognizione, una corrispondenza chiusa o aperta. Quindi in questo caso l'oggetto della tutela è una corrispondenza o chiusa o aperta, che viene sottratta o distratta per prenderne cognizione o per farne prendere cognizione ad altri
    \item Terza fattispecie, distruggere o sopprimere in tutto o in parte una corrispondenza chiusa o aperta
\end{itemize}

Ebbene, in seguito ad una estensione operata espressamente da legislatore la disciplina penale di queste fattispecie si applica a qualunque forma di comunicazione a distanza, anche elettronica, per esempio email, chat, Skype, messaggeria su cellulari e quant'altro. Quindi la riservatezza di queste comunicazioni è tutelata dal codice penale nelle modalità che abbiamo visto. 

Abbiamo visto che le norme penali rilevanti fanno riferimento alla distinzione tra corrispondenza chiusa e aperta. La cassazione ha utilizzato, ai fini di distinguere tra le due fattispecie, cioè distinguere se versiamo in una fattispecie di corrispondenza chiusa o di corrispondenza aperta, ha utilizzato la circostanza del legittimo possesso della password. Quindi, se un soggetto prende cognizione di un messaggio di posta elettronica a lui non indirizzato, forzando la password, ovvero perché quella posta elettronica si trovava su un computer aperto disponibile, in questo caso commette l'illecito penale che abbiamo visto. 
Se invece la comunicazione elettronica viene conosciuta da un soggetto al quale la comunicazione non è indirizzata, ma quel soggetto dispone della password che permette di accedere a quella comunicazione elettronica, in tal modo non si realizza il reato che stiamo esaminando. 

Per esempio, questa fattispecie si verifica molto spesso sui loghi di lavoro nell'ipotesi del dipendente in ferie, ed è necessario accedere alla casella di posta elettronica di quel dipendente per lo svolgimento di certe pratiche o di certi affari precedentemente trattati da quel dipendente. Ora, il datore di lavoro o un suo incaricato che accede alla casella personale di posta elettronica del dipendente assente sta commettendo il reato di cui ci stiamo occupando? La soluzione è quella indicata dalla Cassazione. Se il datore di lavoro o l'incaricato detiene legittimamente la password per accedere a quella casella di posta elettronica, cioè l'interessato è stato debitamente informato del fatto che qualcun altro ha la password e potrà accedere, allora non si commette il reato di violazione della corrispondenza applicato alle caselle di posta elettronica. 

\section{Computer Forensics}

Per computer forensics o informatica forense si intende la ricerca di prove utilizzabili in giudizio attraverso il reperimento e l'analisi di sistemi informatici, reti, dati e quant'altro. Quindi è l'utilizzo di mezzi di ricerca della prova, che possono essere utilizzate in un giudizio penale. In realtà l'uso in giudizio penale è quello più frequente e anche quello oggetto di specifica normazione in Italia, ma nulla vieta che informazioni, dati di questo tipo, possono essere impiegati anche in un processo civile. 

Concentriamoci sul caso paradigmatico che è quello dell'uso nel processo penale. 
Quindi, la ricerca di prove utilizzabili in giudizio penale, quindi prove relative al compimento di un reato o a traccia di un reato, prove relative alla colpevolezza, prove relative alla complicità e quant'altro, attraverso il reperimento e l'analisi di sistemi informatici, di reti, di insieme di dati e così via. 

Questo tipo di attività è disciplinato adesso dalla legge 48 del 2008 che recepisce in Italia la Convenzione Europea sul Cybercrime e che ha determinato anche una serie di interventi, di modifiche al codice di procedura penale per tenere conto del ricorso a questo tipo di tecnologia nella ricerca della prova. 
%32:27
L'informatica Forense o Computer Forensics entra in gioco solo dopo che il reato è stato commesso, quindi non ha a che fare con misure di sicurezza o con finalità preventive o di anticipazione della commissione del reato. E' l'insieme delle misure, degli accorgimenti, degli strumenti che sono finalizzati all'accertamento e alla ricostruzione di un reato e della relativa responsabilità penale. 
Come si attua in pratica questo tipo di attività? 
Ad esempio, analisi di memoria di massa come hard disk, supporti rimovibili, CD, DVD e quant'altro. Oppure analisi di flussi di comunicazione di dati, quindi andando a ricostruire i rapporti, i flussi di informazione tra client e server, come sono transitati sulla rete e quant'altro. Si tratta quindi di un'attività di ricostruzione o su supporti hardware o su infrastrutture di rete. 

\subsection{Tecnicismo e digital divide processuale.}

Il problema con l'uso di queste tecniche è che sono tecniche spesso molto specialistiche, spesso sono sperimentali e controverse, la cui attendibilità, non è del tutto pacifica nella comunità degli utenti; quindi da un primo punto di vista l'uso di queste metodologie può essere controversa in giudizio e dare luogo a contestazioni dibattimentali. 

Da un secondo punto di vista può dare luogo a una sorta di digital divide processuale. Vale a dire, può esserci troppa distanza tra il tecnico, tra il modo tecnico in cui viene raccolta, viene formata la prova, in cui è stato predisposto il mezzo di ricerca della prova e le cognizioni delle parti e soprattutto del giudice. Quindi il giudice potrebbe essere inabilitato a vagliare in maniera attendibile il tipo di attività che è stato svolto per ricostruire la prova. Al limite il giudice potrebbe essere ostaggio del tecnicismo delle parti o dei loro consulenti, dei loro periti. 

Un problema che si è posto è se l'accertamento informatico, nelle varie modalità in cui si può svolgere, sia un accertamento ripetibile o non ripetibile. Un accertamento ripetibile è come dice la parola stessa un accertamento che può essere ripetuto indifferentemente tante volte. Un accertamento non ripetibile invece è un accertamento che o per la sua natura modifica le cose  oggetto dell'accertamento, oppure è un accertamento che si svolge su cose deperibili e così via. 

La differenza è rilevante perché nel caso di un accertamento non ripetibile sono necessarie tutta una serie di garanzie procedimentali che riguardano soprattutto il coinvolgimento del difensore, della parte che ha diritto a partecipare all'attività di accertamento, proprio perché non è più ripetibile. La Cassazione ha talvolta affermato che l'accertamento informatico non determina un'alterazione dello stato di cose perché si tratta di prendere conoscenza del contenuto per esempio dell'hard disk di un computer. 
Si tratta però di una ricostruzione errata perché è quasi inevitabile che i dati contenuti nella memoria di un computer, una volta che vi si acceda per fare un accertamento di questo tipo, vengono in qualche modo alterati. L'accertamento informatico di questo tipo quasi inevitabilmente determina un'alterazione dei dati contenuti. Affinché non determini questa alterazione è necessario prendere tutta una serie di accorgimenti specifici che non è scontato vengano adottati per esempio dalla Polizia Postale o dagli organi preposti. 

Esistono problemi specifici per quanto riguarda la conservazione, l'analisi e la presentazione dell'evidenza digitale. Si tratta esattamente di ciò che ho appena detto. È possibile che nell'acquisire i dati e nel conservarli i dati vengano alterati. Un'alterazione, una modifica dei dati può essere conseguente anche alle procedure di analisi, al modo in cui vengono analizzate. È possibile che accedendo nei file di log, accedendo comunque nella memoria del computer, accedendo ai file da analizzare, questi dati vengono in qualche modo alterati. 

Infine, può essere necessario introdurre cautelle specifiche nella modalità di presentazione in giudizio dell'evidenza digitale. Qui infatti sarà necessario coniugare le esigenze di affidabilità scientifica e tecnica della procedura eseguita con le specifiche garanzie processuali dettate dal codice di procedura per il processo penale. 

I mezzi di ricerca della prova più direttamente incidenti nell'ambito della computer forensics sono:

\begin{itemize}
    \item l'ispezione informatica. L'ispezione informatica si svolge su strumenti fisici e riguarda il prendere osservazione, il prendere atto del loro stato. Nel fare questo si potrà anche prendere cognizione dei dati che si contengono. La perquisizione di ispezione chiaramente riguarda strumenti che possono avere attinenza col reato che si è verificato.
    \item la perquisizione informatica. La perquisizione informatica invece riguarda l'accesso, la presa di cognizione di strumenti informatici che sono stati funzionali alla realizzazione di un reato o che sono nella disponibilità dell'autore del reato o di i suoi complici, che per esempio potrebbero nascondere tracce del reato. Un tipo di perquisizione informatica un po' controverso è la perquisizione a distanza o su internet. Vole a dire che in remoto un agente di polizia si infiltra nel computer di un soggetto, ne accerta il contenuto, vede cosa contiene, i siti visitati, vede anche in tempo reale cosa sta facendo e registra tutto questo eventualmente anche registrando ciò che viene digitato sulla tastiera. Si tratta di una pratica seguita in alcuni paesi. In Italia si deve ritenere al momento non ammessa perché sarebbe un mezzo di prova tipico, perché fonde la perquisizione con il pedinamento e l'intercettazione, quindi fonde diversi mezzi di ricerca e la prova ed è altamente l'intrusivo dell'inviolabilità del domicilio e della riservatezza della corrispondenza dell'interessato.
    \item il sequestro informatico. il sequestro informatico riguarda il sequestro per esempio di dati da utilizzare a fini probatori. Avviene in due modalità di solito, o sequestrando fisicamente l'hard disk, oppure copiando il contenuto dell'hard disk e lasciandolo nella disponibilità dell'interessato. La giurispovenza spesso eccede nel sequestrare hard disk nella loro totalità e talvolta la Cassazione dovuta intervenire, gli inquirenti talvolta eccedono, la Cassazione talvolta è intervenuta dicendo che questi sequesti sono eccessivi perché sequestrano un complesso eccessivo di dati rispetto alle finalità probatorie.
\end{itemize}

Si conclude qui questa lezione di informatica giuridica, per ulteriori approfondimenti vi invito a visitare il sito web.
