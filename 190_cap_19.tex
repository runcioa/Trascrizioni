\chapter{Lezione 19 - Privacy}

Il tema di questa lezione è la privacy. 
\section{La protezione}

\subsection{L'evoluzione del concetto e della normativa di protezione dei dati personali}

Il concetto chiave nella normativa sulla protezione dei dati personali è esattamente, tautologicamente, il diritto alla protezione dei dati personali, diritto attualmente codificato nella normativa italiana dal cosiddetto codice della privacy, che è il decreto legislativo 1996/2003. 

Il diritto alla protezione dei dati personali può essere considerato come la somma di due distinti diritti, il diritto alla riservatezza, detta privacy o privacy in senso stretto e poi il diritto all'identità personale. 
Diritto alla riservatezza e diritto all'identità personale sono due diritti opposti in un certo senso complementari. 
Che cosa vuol dire? 

Per diritto alla riservatezza o privacy in senso stretto si intende il diritto a che certe informazioni personali non siano conosciute da terzi, quantomeno senza l'autorizzazione, il consenso dell'interessato, della persona a cui quelle informazioni si riferiscono.  

In senso stretto per riservatezza si intende il diritto di escludere gli altri dalla conoscenza di certe informazioni personali. 

Il diritto all'identità personale invece è il diritto di ciascun individuo di controllare la correttezza, la veridicità delle informazioni personali che lo riguardano e che circolano, per esempio nell'ambito informativo, nella stampa o nell'ambito burocratico, se per esempio un'amministrazione detiene informazioni personali al fine di erogare certi servizi e così via. 

Quindi tra diritto alla riservatezza e diritto all'identità personale si dà un rapporto quasi simbiontico di opposizione e complementarietà, perché mentre la riservatezza mira a escludere dalla conoscenza di certe informazioni, il diritto all'identità personale mira ad assicurare che le informazioni personali circolino in maniera corretta, in maniera veritera e aggiornata. 
 
Come avremo modo di vedere, il diritto alla protezione dei dati personali comprende entrambi questi profili. 

Un diritto di escludere e un diritto di controllare, di escludere dalla conoscenza di certe informazioni, e un diritto di controllare certe informazioni. 

Il diritto alla riservatezza in senso stretto ha delle origini relativamente recenti per quanto riguarda il suo ingresso nel diritto, la sua rilevanza giuridica. Si suole collocare l'esordio del diritto alla riservatezza nel mondo giuridico con un articolo di dottrina pubblicato su Harvard Law Review del 1890, quindi più di un secolo fa, da parte di due giuristi americani, uno era un avvocato, uno era un giudice. 
Questo articolo mirava sostanzialmente a criticare, a stigmatizzare certi usi, certi costumi degli organi di informazione della stampa. Si può dire che uno dei due autori dell'articolo, Warren e Brandeis, erano i due autori, uno dei due autori fosse interessato per fatto personale. 

Di fatto era successo che la moglie dell'avvocato, autore dell'articolo, fosse una donna molto inserita negli eventi sociali e avesse organizzato un ricevimento a casa propria e il giorno dopo, una giornale locale aveva pubblicato un ricco, dettagliato e forse a tratti irriverente resoconto di quella serata illustrando il lusso della dimora e così via. Il marito, che era un avvocato e un suo amico giudice, reagirono pubblicando un articolo, appunto sul Harvard Law Review, una delle più prestigiosi riviste giuridiche americane, in cui rivendicavano l'esistenza  e la rilevanza del diritto alla privacy inteso come right to be let alone, cioè diritto di essere lasciato in pace. 

Nella sua dimensione originaria il diritto alla privacy o alla riservatezza ha una qualificazione spaziale, più o meno metaforica. Cosa vuol dire? Che nella sua qualificazione originaria il diritto alla privacy o alla riservatezza viene legato a un ambito spaziale all'interno del quale l'individuo è sovrano di fare tutto ciò che vuole senza essere guardato da altri. 
Quindi, per esempio, tipicamente nello spazio della propria dimora, nella propria casa, all'interno delle mura di casa non può entrare neanche il sovrano, diceva un adagio di Common Law. Quindi la privacy nasce con una qualificazione spaziale e un legame diretto con la proprietà. All'interno della mia proprietà, all'interno della mia casa, posso fare ciò che voglio, all'interno della mia casa, della mia proprietà si colloca la mia vita privata e nessuno, né poteri pubblici, né organi di informazione, di stampa, sono autorizzati a spiare, a guardare cosa succede dentro il perimetro della proprietà privata. 

In tal modo il diritto alla privacy nasce come un diritto, diciamo così, borghese ed elitario, il diritto della borghese è il diritto dei proprietari, il diritto di una società in cui si sposano vizi privati, pubblica e virtù. Il proprietario, il borghese deve poter contare, vuole poter contare su uno spazio quasi fisico di intangibilità della sua vita personale, della sua vita privata. 
In tal modo il diritto alla privacy o alla riservatezza si traduce come un diritto al segreto, cioè un diritto a che certe cose, certe informazioni personali non vengano divulgate, non vengano conosciute, siano mantenute per l'appunto riservate. 

Le origini del diritto alla privacy le abbiamo collocate negli Stati Uniti. Cosa succedeva in Italia nel corso del seconda metà del Novecento? Il codice civile del 42 non diceva nulla a riguardo, non ci sono disposizioni che esplicitamente nel codice civile si occupino del diritto alla riservatezza. Vero è che per esempio nelle disposizioni sul diritto d'autore o al diritto all'immagine qualche spunto si poteva trarre. Stava il fatto che il codice civile italiano nella sua formulazione originaria e neanche adesso contiene una codificazione esplicita del diritto alla riservatezza o diritto alla privacy. 
Tuttavia, di fronte all'inerzia o al silenzio del codice civile la società evolveva, la società cambiava. Succedeva che alcuni mezzi di comunicazione, mass media, cominciavano ad avere sempre maggiore diffusione. Cinema, stampa, fotografia. La diffusione di questi mezzi di comunicazione di massa comincia a creare problemi per la tutela della riservatezza personale. 
%09:02
Abbiamo una casistica anche abbastanza eclatante, un primo caso negli anni 50 è il caso Caruso, un secondo caso negli anni 60 è il caso Petacci. 

Nel caso del tenore Caruso, era stato fatto un film che si chiamava Caruso leggenda di una voce e in questo film il giovane Caruso veniva raffigurato come proveniente da un ambiente socialmente molto umile, non molto raffinato, dedito all'alcol e ad altre intemperanze. Al momento della pubblicazione di questo film Caruso ormai era defunto e i suoi eredi protestarono perché questo film denunciava o poneva in piazza vicende sulla cui veridicità si poteva anche discutere ma comunque di rilievo esclusivamente personale e riservato. 
La Corte di Casazione nega che esista un diritto alla riservatezza di cui gli eredi di Caruso si sarebbero potuti avvalere. 

Simile decisione, non tutta identica veniva adottata poco dopo negli anni 60 nel caso Petacci. Erano state pubblicate dei diari e delle lettere di Claretta Petacci, l'amante di Mussolini, in cui venivano riportati passaggi abbastanza intimi e personali di Claretta Petacci, anche della vita affettiva e personale di Claretta Petacci. Gli eredi di Claretta Petacci agirono in giudizio per tutelare la memoria di Claretta Petacci ma la Corte di Casazione nuovamente disse che non esiste un diritto alla riservatezza specificamente codificato nell'ordinamento italiano ma aprì però una via d'uscita,  di clausola di sicurezza dicendo che esiste però nell'ordinamento italiano un diritto generale di salvaguardia della personalità e condotte di questo tipo possono incidere su tale diritto; pur non essendoci un diritto specifico alla riservatezza, condotte di questo tipo come la divulgazione di materiale estremamente riservato, personale, attinente alla vita affettiva e sessuale possono incidere sul diritto assoluto alla personalità. 

Il quadro cambia poco dopo negli anni 70, siamo nel 75 in particolare la Corte di Casazione si trova a giudicare di un caso di una vicenda in cui la parte interessata è la principessa Imperatrice Soraya o ex Imperatrice Soraya. 
Soraya Sfandiari si lamentava che fossero stati ripresi con un teleobiettivo degli atteggiamenti intimi e personali che ella aveva intrattenuto con un suo amico o amante all'interno della propria dimora. La Soraya Sfandiari instaura un processo giudiziario che arriva fino alla Cassazione; la Cassazione afferma che nell'ordinamento italiano esiste un diritto alla riservatezza ricavabile per analogia da varie disposizioni settoriali e riconosciuto anche nella Costituzione, laddove parla dell'inviolabilità del domicilio, dell'inviolabilità della corrispondenza e delle comunicazioni. 

Quindi prende forma in Italia in via giurisprudenziale, nell'inerzia quasi totale del legislatore, un diritto alla riservatezza nella sua forma originaria, plasmata poco a poco da singole pronunce giurisprudenziali che si vanno arricchendo nel tempo. Nella sua formulazione originaria riguarda prettamente i rapporti con i mass media, quindi la possibilità che vicende personali, informazioni personali, vengano divulgate da mezzi di comunicazione di massa come giornali, film e quant'altro. Successivamente si evolve con il concetto di privacy.

\subsubsection{L'evoluzione del concetto di privacy}

\begin{itemize}
    \item Dal segreto al controllo (accesso, rettifica,...)
    \item Diffusione e informatizzazione delle banche dati
    \item Circolazione massiccia delle informazioni personali su personal computer, internet e così via
\end{itemize}

Cosa vuol dire evoluzione dal segreto al controllo? In relazione ai fattori che ho appena manzionato, diffusione e informatizzazione delle banche dati, articolazione dei supporti informatici su PC e loro interconnessione su su internet e così via, si realizza una sempre maggiore archiviazione, registrazione, conservazione e possibilità di accesso a informazioni personali. 
Si realizzano grandi banche dati, inizialmente detenute in mano pubblica da Stati, Pubbliche Amministrazioni, poi sempre più polverizzate e dalle grandi banche dati passiamo alle infinite banche dati che ciascuno di noi può costruirsi per esempio o che un giornale può costruire e così via, diffuse e polverizzate tra innumerevoli soggetti. 
Quindi l'evoluzione tecnologica, informatica, la digitalizzazione, la rivoluzione telematica, cioè il fatto che l'informazione passa attraverso le reti telefoniche, gli strumenti telematici, determina il fatto che l'informazione personale si diffonde, viaggia, viene trasferita con costi contenutissimi e praticamente in tempo reale. In questo quadro di mutato contesto sociale e tecnologico l'esigenza non è più solamente quella del segreto ma è anche quella del controllo, cioè il controllo delle proprie informazioni personali che viaggiano in quantità enorme e in tempo reale sulla rete. 

Potere di controllo cosa vuol dire? Che l'interessato, la persona a cui si riferiscono le informazioni personali deve poter rivendicare un diritto di accesso alle informazioni personali, vale a dire la possibilità di sapere quali informazioni personali qualcuno stia detenendo e per quali fini, in quali modalità le tratta, eccetera eccetera. 

La rettifica avviene una volta preso atto che qualcuno tratta i miei dati personali e che magari quei dati personali sono inesatti, sono non veritieri o obsoleti, la possibilità di correggerli, rettificarli, eccetera eccetera. 

Infine, e qui si ritorna in realtà al profilo del segreto, la possibilità di impedire l'ulteriore circolazione di queste informazioni personali se si verificano certi presupposti. 

L'evoluzione del concetto di privacy si arricchisce di un'ulteriore dimensione che alla fine in realtà ingloba la precedente, dal segreto al controllo. Il controllo include anche il potere di blocco, quindi di rivendicare il diritto al segreto. Controllo come potere di seguire queste informazioni nel loro cristallizzarsi in banche date e nel loro circolare nella rete. 

Abbiamo visto il substrato tecnologico che rende questo attuale, quindi la circolazione massiccia delle informazioni personali su personal computer, internet, eccetera eccetera. 

\section{La protezione dei dati personali nell'ordinamento italiano}
\subsection{Principi generali}

A fronte dell'evoluzione tecnologica e sociale che abbiamo sinteticamente indicato precedentemente, come si poneva la normativa italiana relativamente al contesto tecnologico che abbiamo visto? Sostanzialmente nulla fino al 1996. Si tenga conto che abbiamo parlato di un processo che inizia a fine 800 e che la giurisprudenza italiana ha cominciato a segnalare gli anni 50, 60 e 70. Fino al 1996 non si muove nulla, salvo una legge di settore. Vale a dire che era stato istituito presso il Ministero dell'Interno un centro di elaborazione dati che aveva finalità di prevenzione di reati, ordine pubblico, conteneva informazioni finalizzate sostanzialmente all'ordine pubblico. La normativa che istituisce il centro di elaborazione dati presso il Ministero dell'Interno, che è la legge 121 dell'81, prevede una serie di diritti degli interessati, cioè dei soggetti di cui informazioni personali vengono riversate in questo centro di elaborazione dati, che sono diritti  attinenti alla protezione dei dati personali. 

Però è una normativa di settore che prevede diritti abbastanza limitati. Tuttavia il legislatore italiano non si muoveva nel vuoto, a parte le evoluzioni giurispronenziali che abbiamo menzionato. Il legislatore italiano aveva un quadro sovranazionale e comparatistico che si andava arricchendo. Comparatistico cosa vuol dire? Vuol dire che cominciavano a proliferare le legislazioni nazionali in vari Stati Europei. In Francia già negli anni 70 c'era stata una modifica del Code civil per introdurre principi di protezione dei dati personali, in Germania vari lenders erano dotati di normative apposite, i paesi scandinavi altrettanto e così via, ma ci sono anche principi e normative riconosciuti a livello sovranazionale da parte di organismi sovranazionali da cui l'Italia deriva e quindi l'Italia aveva l'obbligo di tenere conto di questi principi e ratificarli, principi rilevanti per la protezione dei dati personali e per la privacy. 

l'articolo 8 della Convenzione Europea dei Diritti dell'uomo che prevede il diritto al rispetto della propria vita privata e familiare

la Convenzione del Consiglio d'Europa numero 108 dell'81 che si propone di implementare i principi dell'articolo 8 della CEDU Convenzione Europea dei Diritti dell'uomo di implementarli a livello di informazioni personali e trattamenti dei dati personali. 

Avevamo nel 1995 la direttiva CEE 46 del 95, il sistema comunitario di denominazione degli atti, quindi nel 1995 c'era una direttiva comunitaria abbastanza dettagliata in realtà, perché essendo una direttiva richiamava la necessità di attuazione da parte dei singoli stati membri e abbastanza articolata sulla protezione dei dati personali. 
L'Italia si trovava nella necessità di adottare una normativa privacy per la protezione dei dati personali per ratificare il trattato di Schengen. Vale a dire l'Italia aveva aderita al trattato di Schengen sulla libera circolazione delle persone nel territorio degli Stati che aderiscono alla Convenzione Schengen, quindi in una cosiddetta area Schengen. Ai fini della ratifica di tale accordo la Convenzione prevedeva la necessità che i singoli stati contraenti si dotassero di una normativa sulle banche dati e sul trattamento dei dati personali; di conseguenza l'Italia si trovava nella situazione di dover ratificare entro il 1996 il trattato Schengen e dotarsi di una normativa sui dati personali. 
Esisteva in quel momento una normativa recentissima  sui dati personali che era quella europea che abbiamo menzionato precedentemente, la direttiva 95/46, ecco che allora l'Italia coglie due piccioni con una fava e ratificando Schengen si dota anche di una normativa sui dati personali che codifica il trattato Schengen, e che recepisce quasi interamente la normativa europea, così come introdotta dalla direttiva che abbiamo precedentemente menzionato. 

%23:12
Questo avviene con la legge 675 del 1996, legge del 31 dicembre del 1996, il legislatore si era dotato di una legge nell'ultimo giorno utile per ottemperare i propri obblighi sovranazionali.
Il legislatore sembrava cosciente che questo lavoro di attuazione di una normativa sul trattamento dei dati personali, era stato fatto in maniera forse un po' affrettata per alcuni versi e allora la legge 675 del 1996 viene accompagnata, caso abbastanza singolare, da una legge gemella, che era una legge delega,la legge 676 del 1996, che autorizza il governo a emanare una serie di decreti che possono integrare e modificare la legge 675. 

Il Parlamento italiano per un verso approva la legge sulla protezione dei dati personali e nello stesso istante con la legge 676, delega il governo a emanare tutta una serie di provvedimenti integrativi, correttivi, eccetera, della legge 675. Negli anni seguenti il governo ha emanato numerosi decreti delegati che arricchiscono, integro e modificano la legge originaria e la conseguenza è che nel giro di 4-5 anni il panorama normativo italiano sulla privacy era diventato abbastanza caotico, perché vi era la legge madre la 675, e  una pletora di decreti settoriali delegati emanati di lì a poco. In più era venuta fuori una nuova fonte atipica del diritto, le cosiddette autorizzazioni generali del garante per la privacy. 

Nel giro di pochi anni si avvertiva già l'esigenza di rimettere ordine al panorama normativo sulla protezione dei dati personali e questo è stato fatto col decreto legislativo 196 del 2003. Si tratta di un decreto delegato, quindi anch'esso adottato in attuazione di una delega del Parlamento, che ha assunto la denominazione di codice della privacy o codice in materia di protezione dei dati personali. Si tratta di un testo che ricomprende e cerca di sistematizzare, in maniera coerente tutta la normativa, anche con alcuni innovativi principi. La legge 675 del 1996 alla fine prevedeva una cinquantina di articoli, il decreto legislativo codice della privacy viaggia quasi verso i 200 ed è stato anch'esso ulteriormente integrato da successivi provvedimenti, tra cui uno recentissimo datato dicembre 2011. 

E' importante notare che la legge si applica a tutti i dati personali e a tutte le operazioni di trattamento di dati. Questa è una precisazione importante perché nella precomprensione comune, anche degli organi di stampa, specialmente nei primi tempi di vigenza di questa legge, si faceva passare l'interpretazione secondo cui la 675 prima e poi il codice privacy, fossero leggi sulla privacy informatica o sulle banche dati. In realtà la legge sulla privacy o legislazione sul trattamento e protezione dei dati personali, si applica a tutti i dati personali e a tutte le operazioni di trattamento di dati, quindi non soltanto ai dati personali e ai tipi di trattamento di dati svolti con strumenti informatici, né soltanto ai tipi di trattamenti e i dati svolti su banche dati o alle informazioni personali riversate in banche dati. È vero che i trattamenti di dati personali che si svolgono con strumenti informatici sono molto importanti e hanno talvolta un regime specifico, vero è che l'ipotesi del dato personale riversato in una banca dati è una delle ipotesi di cui si occupa questa normativa, quindi sì, la privacy informatica è uno degli oggetti di questa legge, sì le banche dati sono uno degli oggetti di questa legge, tuttavia non sono gli oggetti esclusivi. Questa normativa italiana copre anche trattamenti svolti in maniera non informatizzata e anche trattamenti di dati personali svolti al di fuori di banche dati. 

\subsection{Concetti chiave della normativa italiana sul trattamento dei dati personali}

\begin{itemize}
    \item Dato personale
    \item Trattamento
    \item Autorità indipendente
\end{itemize}

Il primo concetto è quello di dato personale, un secondo concetto essenziale, un concetto chiave è il trattamento e infine l'autorità indipendente. 

\subsubsection{L'autorità indipendente}

La legge italiana, così come richiesto dalla direttiva comunitaria, richiede che venga istituito un soggetto appositamente competente  per controllare l'implementazione della legge, anche con poteri autonomi di ispezione, di verifica, di controllo, e competente anche a ricevere segnalazioni, istanze, reclami e anche ricorsi da parte di privati cittadini o di soggetti che ritengono che vi sia stata una violazione della normativa sui dati personali. 
Quindi la legge italiana ha istituito un organismo che è un'autorità amministrativa indipendente che si chiama Garante per la protezione dei dati personali, detta anche Garante per la privacy in maniera ufficiosa, che ha indipendenza sia dal governo, dall'esecutivo, sia dal Parlamento e che può svolgere funzioni para giurisdizionali, simili a quelle giurisdizionali, perché è possibile che chi ritenga che i propri diritti attinenti alla sfera della protezione dei dati personali siano stati lesi, può attivare uno strumento di tutela davanti al Garante, oppure in alternativa può attivare tutela davanti al giudice ordinario. 
Quindi si introduce una dimensione di tutela, una dimensione di protezione dei diritti nuova e ulteriore e anche in un certo senso inedita nella sua modalità nell'ordinamento rispetto alle tradizioni di forme di tutela giudiziaria. 

\subsubsection{Il dato personale}
Cominciamo con la nozione di dato personale. Qui ripercorriamo per sommi capi il modo in cui la legge stessa definisce questi concetti rilevanti. Dato personale è qualunque informazione relativa a persona fisica, la persona fisica a cui si riferisce questa informazione si chiamerà interessato, persona fisica identificata o identificabile anche indirettamente. Quindi un dato personale è qualsiasi informazione che rende un soggetto identificato o identificabile. 
È un'informazione che mi consente di identificare qualcuno o di renderlo identificabile anche indirettamente. 
Che cosa vuol dire? Che magari la singola informazione in sé non identifica direttamente un soggetto, ma mettendo quella informazione in un contesto di altre informazioni agevolmente ricavabili, facilmente acquisibili, ciò rende facilmente, ancorché indirettamente, identificabile il soggetto interessato. Quindi dato personale è qualunque informazione relativa a persona fisica, cioè l'interessato identificato o identificabile anche indirettamente. 

Fino al dicembre 2011, interessato poteva anche essere una persona giuridica, ente, associazione, per esempio un'impresa, una società per azioni, un partito politico, un sindacato, una confessione religiosa e così via. In una manovra normativa di semplificazione, questo riferimento è stato tolto. Quindi dal dicembre 2011 in poi, l'informazione personale o meglio l'interessato a cui si riferisce l'informazione personale è solo persona fisica. Questo è considerato una semplificazione dell'attività specialmente commerciale, professionale, imprenditoriale. Vero è che spesso è comunque difficile distinguere se un dato personale si riferisce a un soggetto persona fisica o a un'ente o a una persona giuridica. Per esempio il caso di una persona giuridica, una società per azioni, questa persona giuridica ha determinati organi, il presidente del Consiglio d'amministrazione, l'amministratore delegato. Può essere abbastanza difficile, se non assolutamente fittizio, distinguere una informazione personale che si riferisce alla persona giuridica da un'informazione personale che si riferisce al legale al rappresentante, comunque presidente di una società per azioni per esempio. Si pensi altresì al caso di un'impresa gestita da un imprenditore singolo, l'impresa può essere considerata come un'ente, un'entità che ha una sua soggettività separata o relativamente separata dalla persona dell'imprenditore e tuttavia può essere difficile distinguere un dato personale che si riferisce all'azienda, all'impresa e un dato personale che si riferisce all'imprenditore. 

Esempiti dati personali, cioè informazioni personali che rendono un certo soggetto attualmente solo a persona fisica identificato o identificabile. Può essere una fotografia, può essere un numero di telefono, un indirizzo email, un codice fiscale, quindi anche un mero codice numerico, un codice di identificazione o un codice fiscale o un numero di telefono sono ovviamente dati personali. 

Si possono distinguere i dati personali in vari tipi. 

Muovendosi in questa tipologia che stiamo introducendo di solito cambia il regime giuridico di questi dati, il livello di tutela, i presupposti di illicità del trattamento, ecc. 

\begin{itemize}
    \item dati identificativi sono i dati che permettono l'identificazione diretta dell'interessato.
    \item  Dati anonimi sono dati che in origine o a seguito di trattamento non possono essere associati ad un interessato identificato o identificabile. Quindi un dato anonimo è un dato che sia in sé per sé oppure perché è stato successivamente trattato in un certo modo, assogettato a certe operazioni di trattamento, non rende identificabile alcun interessato. Per esempio una foto in cui siano state oscurate le fattezze del viso della persona ritratta, questo diventa un dato originariamente personale, perchè identificava qualcuno grazie a sui tratti somatici, successivamente anonimizato oppure la voce che può essere dato personale ma se trattata con strumenti informatici che la rendono non riconoscibile diventa data anonimo.
    \item I dati sensibili sono quelli idonei a rivelare l'origine razziale ed etnica, le convinzioni religiose, filosofiche o di altro genere, le opinioni politiche, l'adesione a partiti, sindacati, associazioni, organizzazioni a carattere religioso, filosofico, politico o sindacale e altri dati personali idonei a rivelare lo stato di salute e la vita sessuale. I dati sensibili hanno un regime di tutela differenziato, di solito più rigoroso e all'interno della categoria dei dati sensibili quelli idonei a rivelare lo stato di salute e la vita sessuale, a loro volta hanno spesso un livello di tutela ancora maggiorato, ancora più rigoroso.
    \item dati giudiziari, anche i dati giudiziari di solito hanno un livello di tutela più elevato, più rigido rispetto agli altri tipi di dati personali. Sono dati giudiziari, i dati personali riguardanti ai provvedimenti penali o la qualità di imputato o indagato in senso degli articoli 60 e 61 del codice di procedura penale. Un dato giudiziare è tipicamente quello  di essere stato rinvito a giudizio.
\end{itemize}


\subsection{Il trattamento}
L'altro concetto chiave dell'architettura della normativa è il trattamento. Per trattamento intendiamo qualunque operazione che ha ad oggetto dati personali. 

Esempi di questo tipo di trattamento. 

Non sono operazioni necessariamente automatizzate o informatizzate, un trattamento può svolgersi in maniera cartacea o manuale, altresì, il trattamento può non essere strutturato in una banca dati, può essere un trattamento occasionale che non avviene in un insieme strutturato di dati. 

Alcuni esempi di trattamento. 

\begin{itemize}
    \item raccolta 
    \item registrazione
    \item organizzazione
    \item conservazione
    \item consultazione
    \item elaborazione
    \item modificazione
    \item estrazione
    \item comunicazione
    \item diffusione
    \item cancellazione
    \item blocco
    \item interconnessione
    \item distruzione
    \item raffronto
    \item utilizzo
\end{itemize}

Due tipi specifici di trattamento che richiedono tutele specifiche, regimi talvolta più rigorosi sono la comunicazione, vale a dire il dare conoscenza dei dati personali a soggetti determinati in qualunque forma, anche mediante la loro messa a disposizione o consultazione, vale a dire per esempio in un ufficio pubblico il rendere un certo registro cartaceo o informatico accessibile a chiunque venga a fare certe richieste, certi tipo di indagini, sempre che costui sia determinato, vale a dire per esempio se vado all'ufficio dell'anagrafe e richiedo un certificato non mio e per fare questo l'ufficio dell'anagrafe mi richiede un documento, cioè mi identifica, divento una persona determinata. 

%40:46
\subsection{La comunicazione e la diffusione}

Un altro tipo di trattamento su cui dobbiamo concentrare l'attenzione è la diffusione ed è quando la conoscenza dei dati personali viene data a una platea indeterminata di soggetti in qualunque forma anche mediante la loro messa a disposizione o consultazione.

Anche se in astratto la normativa non rileva la forma della comunicazione e diffusione che possono avvenire in qualsiasi forma, tipicamente la comunicazione è inviare una lettera o inviare una mail o fare una telefonata, la diffusione è per esempio pubblicare un post in un sito internet. 

Anche se non rileva nello specifico il modo in cui queste cose si fanno, il garante ha individuato cautele differenti per il caso di dati online, per esempio ha individuato livelli diversi di accessibilità per certi tipi di dati o la richiesta di non accessibilità su motore di ricerca. 

I soggetti del trattamento sono:

\begin{itemize}
    \item il titolare, il soggetto a cui competono le decisioni in ordine della finalità del trattamento
    \item il responsabile, figura eventuale e contingente che può essere indicata dal titolare affinché curi certi passaggi, certe operazioni di trattamento
    \item l'incaricato, l'incaricato è un soggetto, persona fisica che è autorizzato dal titolare o dal responsabile a svolgere certi tipi di trattamento
\end{itemize}

