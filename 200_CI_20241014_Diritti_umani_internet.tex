\chapter{Diritti umani e internet}
\section{Registrazione della classe interattiva del 14 ottobre 2024}


L'informatica giuridica che cos'è?
L'informatica giuridica è una disciplina che sostanzialmente utilizza delle tecnologie di supporto che vanno comunque ad ausiliare l'operatore giuridico in determinate attività quindi questo ausilio può avvenire nell'ambito della giustizia quindi pensiamo al processo telematico ma può avvenire anche nell'ambito della professione di avvocato pensiamo comunque a dei motori di ricerca o comunque banche dati che aiutano l'avvocato nella ricerca magari delle sentenze.

L'informatica giuridica differisce dal dritto dell'informatica perchè quando parliamo di diritto parliamo di diritto oggettivo, un insieme di leggi e di norme che hanno ad oggetto l'informatica prendiamo ad esempio le norme,  le leggi che riguardano i reati informatici; in quel caso  parlando dei reati informatici avremo per esempio la norma sull'accesso abusivo al sistema informatico, la norma sulla truffa informatica, tutte queste norme che hanno ad oggetto l'informatica. 

Parlando di reati ci sarà un bene tutelato dalla norma e per ogni reato vedremo quale è. Poi ci sono le norme che riguardano i contratti informatici quindi anche in questo caso parliamo di diritto dell'informatica.

La società dell'informazione che è caratterizzata da una serie di requisiti 
Cerchiamo di definire meglio i concetti, quali sono le caratteristiche della società dell'informazione?
Intanto cos'è la società dell'informazione?
La società dell'informazione è quella società che ha ad oggetto l'informazione come bene primario l'informazione che non è veicolata con uno strumento cartaceo ma con un strumento digitale, su un supporto digitale e questa è la prima cosa inoltre quali sono le caratteristiche della società dell'informazione?
La deterritorializzazione, la destatualizzazione e la globalizzazione e la pervasività. Queste sono le caratteristiche della società dell'informazione.

La globalizzazione e la pervasività sono concetti collegati e l'informatica e le tecnologie dell'informazione sono oramai pervasive perchè pervadono ogni campo della società e vanno comunque anche ad perorare lo stesso concetto di identità tant'è che parliamo di identità digitale.
%10:30
La globalizzazione ha un rapporto biunivoco  con la digitalizzazione diciamo che la globalizzazione accresce la pervasività delle nuove tecnologie e le nuove tecnologie senza globalizzazione non si sarebbero diffuse così tanto.
Diciamo che la globalizzazione è un fenomeno che connota l'apertura delle economie, delle frontiere, risulta sostanziale alla crescita degli scambi commerciali su scala globale.

Andiamo agli altri concetti che invece sono legati alla categoria dello stato che nel diritto negli studi giuridici è studiata nel diritto costituzionale. Quando andiamo a definire lo stato lo definiamo in base a tre categorie:  il territorio, il popolo e la sovranità.

Andando ad affermare che nella società digitale si verifica la deterritorializzazione andiamo ad affermare che questa società non è circoscritta sui confini e le frontiere del proprio dello stato o nazione perché vengono scardinate queste frontiere in quanto la digitalizzazione va ben oltre i confini di uno stato.
Collegata alla destatualizzazione ci sono i concetti di destatualizzazione che è un concetto che ha a che fare con la crisi della sovranità che è un potere assoluto, perpetuo che fu teorizzato da boden nel 600 e che è riferibile allo stato. 
Affermando che la società digitale è caratterizzata da destatualizzazione oltre ad affermare che c'è una perdita del connotato territoriale dello stato perché la digitalizzazione va ad operare dento e fuori dai confini dello stato, la destatualizzazione  va comunque a colpire anche la perdita di sovranità.
Quello che voglio dire è collegato anche al concetto di giurisdizione, se avviene una violazione di dati in un determinato stato però quei dati sono riferibili ad una persona che comunque risiede in un altro stato di chi è la giurisdizione? Di quale dei due stati? 
La regolamentazione a livello europeo e anche la regolamentazione a livello internazionale tendono a risolvere questi conflitti. 
Per quanto riguarda ad esempio la disciplina privacy, sapete che c'è un regolamento europeo 679/2016 che è andato a rendere omogenee tutte le normative che avevano oggetto la produzione dei dati personali ma che erano proprie di ogni  nazione, questo ovviamente all'interno dell'unione europea.

\section{La nascita di internet e i diritti fondamentali}
In questa lezione tratto dell'accesso ad internet come diritto fondamentale. 
Dobbiamo partire dalla fine di un percorso teorico nel luglio del 2015 quando è stato redatto il documento della dichiarazione dei diritti in internet. Questo documento che è stato redatto da una commissione presieduta dal professore Stefano Rodotà e voluta dall'allora presidente della Camera Laura Boldrini.
Questa dichiarazione è importante perchè oltre a definire una serie di concetti fondativi nell'era della società digitale perché sostanzialmente definisce internet come il fenomeno che ha contribuito in maniera decisiva a ridefinire lo spazio pubblico e privato.
Internet ha anche ristrutturato e costituito nuovi rapporti tra le persone, tra le persone e le istituzioni e anche tra le istituzioni.
Vediamo prima una questione più tecnica, questa dichiarazione dei diritti in internet che valore ha?
E' un atto, una dichiarazione vincolante? Se io contravengo a qualche direttiva che è espressa in questa dichiarazione ci saranno delle sanzioni a mio carico? Questa dichiarazione dei diritti in internet ha il carattere proprio della legge quindi la vincolatività e anche la cogenza (concetto intercorrelato con la costrizione).
La risposta è no questa dichiarazione dei diritti in internet ha un valore precettivo; è un insieme di direttive e di principi e criteri direttivi che nell'ambito dell'Unione Europea hanno creato le condizioni per l'adozione di atti normativi. 
Detto ciò un argomento importante è l'accesso ad internet che in questa dichiarazione è definito come diritto fondamentale della persona e condizione per il pieno sviluppo individuale e sociale.

Partiamo dalla fine e ripercorriamo il percorso; in realtà il presupposto dell'accesso ad internet e quindi il requisito dell'accesso ad internet come definito in questa dichiarazione quindi il fatto che il diritto di accesso ad internet sia un diritto fondamentale è stato oggetto di un ampio dibattito in dottrina.
Possiamo parlare di due filoni interpretativi; il primo che considerava l'accesso ad internet come un mero strumento di accesso alla rete e non un diritto fondamentale ma attraverso quest'accesso io posso esercitare dei diritti fondamentali come per esempio la manifestazione del mio pensiero.
Il secondo filone interpretativo, supportato dal professor Stefan Rodotà, sosteneva che il diritto di accesso ad internet è un diritto fondamentale e non è un mero strumento per l'esercizio di altri diritti; perché il diritto di accesso ad internet per sé stesso plasma la personalità dell'individuo ed è un diritto diverso da quello formalizzato all'articolo 21 della costituzione (la libertà di manifestare il proprio pensiero) tant'è che la commissione presiduta da Rodotà aveva ipotizzato di inserire all'interno della costituzione italiana un articolo 21 bis in cui fosse proprio formalizzato anche il diritto di accesso ad internet.
Questo dibattito teorico che poi ha condotto invece all'affermazione di questa dichiarazione di diritti di internet all'articolo 2  vedete che il diritto di accesso ad internet è definito come un diritto fondamentale e a all'esercizio questo diritto è collegata la realizzazione della personalità dell'individuo che è un collegamento con l'articolo 2 della costituzione.

Ogni persona quindi ha il diritto di accedere ad internet in condizioni di parità e con modalità tecnologicamente adeguate che significa il diritto di accesso ad internet deve anche essere assicurato nei suoi presupposti sostanziali e non solo come possibilità di collegamento alla rete.

Questo articolo ricalca in qualche modo anche l'articolo 3 della costituzione italiana in cui viene formalizzato il diritto all'uguaglianza, ugualianza formale e ugualianza sostanziale.
Le istituzioni pubbliche devono garantire i necessari interventi per il superamento di ogni forma di divario digitale e come all'articolo 3 della costituzione è detto che la repubblica rimuove ogni ostacolo alla ugualianza personale quindi al fatto che ogni cittadino al pari dignità sociale ed uguale davanti alla legge quindi è compito della repubblica rimuovere gli ostacoli di ordine economico e sociale che limitano di fatto la libertà e l'uguaglianza dei cittadini.

Quali sono gli altri spunti interessanti da cogliere dalla lettura di questa dichiarazione?

Un altro concetto è quello della neutralità della rete cioè il diritto che dati trasmessi e ricevuti in internet non subiscano discriminazioni, restrizioni, interferenze in relazione al criterio del mittente, del ricevente, il tipo di contenuto, del dispositivo utilizzato; quindi il diritto ad un accesso neutrale ad internet è una condizione necessaria per l'effettività dei diritti fondamentali delle persone.

In questa dichiarazione troviamo anche un articolo che si riferisce alla tutela dei dati personali; considerate che poi nel 2017 abbiamo avuto l'adozione del regolamento europeo.

Altro concetto importante è il diritto all'autodeterminazione informativa; ogni persona ha diritto di accedere ai propri dati quale che sia il soggetto che li detiene, il luogo dove sono conservati e può richiederne (questo articolo è collegato  alla tutela della privacy) l'integrazione, la rettifica, la cancellazione secondo le modalità previste dalla legge.

Troverete inoltre un altro articolo l'articolo 8 che ha oggetto i trattamenti automatizzati e viene stabilito che nessun atto, provvedimento giudiziario, amministrativo o comunque qualsiasi decisione che può incidere in maniera significativa nella sfera delle persone può essere fondato unicamente sul trattamento automatizzato dei dati volto a definire il profilo o la personalità dell'interessato.

Il regolamento 679/2016 ha una disposizione che ricalca questo principio sulla protezione dei dati personali, ha una norma che ricalca questo criterio direttivo ed è un'idea che viene dalla dichiarazione dei diritti fondamentali in internet.

In questa dichiarazione viene anche definita l'identità digitale e viene definita come la rappresentazione integrale e aggiornata delle identità del soggetto in rete. Questa nozione è importante e suscita molte riflessioni. 

L'articolo titolato diritto all'identità afferma che l'uso di algoritmi e tecniche probabilistiche deve essere portato a conoscenza delle persone interessate che in ogni caso possono opporsi alla costruzione e alla diffusione di profili che le riguardano.

Voglio parlare dell'anonimato quindi della protezione dell'anonimato, base fondativa anche di internet. 
Ogni persona può accedere alla rete e comunicare utilizzando strumenti di natura tecnica che però tutelino l'anonimato e quindi evitino la raccolta dei dati personali.

Vi inviterei a leggere questa dichiarazione dei diritti internet, il materiale che avete sulla protezione dei dati personali sul regolamento europeo 796/2016.
