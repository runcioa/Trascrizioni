\chapter{Lezione 2 - Evoluzione dell'informatica giuridica.}
Gli argomenti di questa lezione saranno:
\begin{itemize}
    \item una breve storia dell'informatica giuridica 
    \item il rapporto tra informatica giuridica e conoscenza del diritto 
\end{itemize}

\section{Breve storia dell'informatica giuridica}
L'informatica giuridica intesa come, in senso stretto, applicazione dell'informatica, per gestire alcuni aspetti del diritto, aspetti della produzione del diritto, aspetti dell'applicazione del diritto, aspetti dell'accesso e conservazione e fruibilità delle norme giuridiche. 
La storia dell'informatica giuridica intesa in questo senso la possiamo fare iniziare più o meno verso gli anni 50, in particolare una ricerca sperimentale che si svolse presso lo Health Law Center di Pittsburgh. Cos'era successo? Lo Stato della Pennsylvania aveva deciso di riformare, riformulare anche la propria legislazione sanitaria in modo da purgarla, da emendarla da certi usi lessicali, da certi usi terminologici, considerati poco politicamente corretti. In particolare in alcuni testi legislativi statali era presente la dizione, la formulazione bambino ritardato. Era una legislazione che riguardava evidentemente assistenza alle famiglie, bambini ritardati, eccetera. Ora negli anni 50 il legislatore della Pennsylvania si pose l'obiettivo di riformulare i testi di legge in materia sanitaria in modo da individuare i testi che contenevano quella dizione e sostituire tale dizione con bambino eccezionale, bambino non comune o cose di questo tipo. Una dizione che sarebbe in quel momento risultata più politicamente corretta. Allora un istituto giuridico di un'università, l'Health Law Center di Pittsburgh, immaginò, e mise anche in opera, un programma per calcolatore che immagazzinava tutte la normativa statale rilevante ed era in grado di trovare nella normativa la parola bambino ritardato. Segnalando la presenza di questa parola in un certo testo era poi possibile modificare quel testo e emendarlo dalla formulazione giudicata linguisticamente infelice. Questo, come possiamo vedere, è anche un tipico e il primo esempio di uso dell'informatica a fini giuridici. In primo luogo, perché l'informazione giuridica, il dato giuridico, il dato dei testi di legge, vengono immagazzinati con uno strumento informatico. In secondo luogo, perché l'informatica viene utilizzata non solo per mera conservazione, ma anche per il recupero di informazioni, vuole dire la ricerca di testi di legge che contengono certe parole. Sostanzialmente in tal modo il gioco era già pronto per ideare un sistema di informatica giuridica documentaria e di banca dati legislativa. Infatti una volta immagazzinata tutta la legge statale in un database si poteva cercare qualsiasi parola presente, in qualunque delle leggi immagazzinata in quel database. Quindi si erano gettate le premesse o si era svolto gran parte del lavoro per la progettazione di un database legislativo, di una banca dati legislativa. È praticamente scontato. Una volta che questa operazione poteva essere fatta con quella singola parola, è ovvio che poteva essere fatta con qualsiasi altra parola. Si sono così gettate le basi per la ricerca del materiale legislativo sul database, su banche di dati che contengono, su cui sono stati immagazzinati testi di legge. In questo periodo però l'informatica era in uno stato che è completamente diverso da quello in cui è oggi, quello in cui diventerà negli anni 2000. I computer erano degli enormi calcolatori che avevano bisogno di alcune decine di metri quadrati di locali per essere custoditi, calcolatori enormi che lavorano con schede perforate e che si vedono tutt'ora in qualche film e questi calcolatori non dialogano tra di loro, non comunicano tra di loro. Ogni calcolatore fa il proprio lavoro, gestisce le proprie informazioni, custodisce le proprie informazioni e fa l'operazione richiesta, per esempio trovare una certa informazione tra quelle custodite, ma non dialoga con altri calcolatori. I calcolatori non sono interconnessi da una rete telematica, c'è una distinzione, una separazione fisica tra i calcolatori e le telecomunicazioni. Negli anni 60 e 70 questi spunti che abbiamo appena visto vengono sviluppati in maniera sempre maggiore. Abbiamo alcune applicazioni nell'editoria giuridica, nascono società commerciali, specialmente negli Stati Uniti, che si prefigurano come obiettivo quello di rendere questo tipo di servizio, la registrazione, la conservazione di materiale normativi, leggi, precedenti, anche dottrine, e collocare questi prodotti editoriali sul mercato, quindi proporre a pubblica amministrazione, a corti giudiziari, a studi privati l'accesso a queste risorse immagazzinate, risorse normative o giurisprudenziali o dottrinarie immagazzinate su calcolatore. Quindi nascono le prime applicazioni che sono di tipo privato commerciale negli Stati Uniti, dell'informatica applicata all'editoria giuridica. 

Nascono alcune riviste specializzate, specialmente negli Stati Uniti, in cui cosa che denota il prendere piede di un dibattito sempre più ricco e specialistico e con proposte tal volta avveniristiche, molte delle quali non sono realizzate, di applicazione dell'informatica a diritto. In Italia abbiamo un'idea che viene attuata e che è tuttora in piedi, del tutto innovativa che fa da battistrada in tutti i paesi europei, è stata fatta esattamente in Italia, è il sistema Italgiure predisposto nell'ambito del CED, del Centro Elaborazione Dati della Cassazione. Ad opera di alcuni magistrati, in particolare Vincenzo Borruso e altri magistrati della Cassazione, si comincia a sperimentare e successivamente ad attuare la possibilità di immagazzinare dati giurisprudenziali e in particolare massime della Cassazione, perché la Cassazione è in Italia l'unico ufficio giudiziario che dispone ufficialmente di un sistema di massimazione. Esiste, presso la Cassazione, un ufficio del massimario che si cura di estrapolare dalla decisione della Cassazione, dalla sentenza della Cassazione, la massima o principio di diritto. Questo è collegato alla funzione nomofilattica che ha la Cassazione nell'ordinamento italiano, vale a dire la funzione di assicurare l'uniforme interpretazione del diritto. Poiché le decisioni della Cassazione sono estrapolate in massime, l'idea era quella di raccogliere in un database informatico, in una banca dati, le massime della Cassazione. Questa sistema venne ideato e poi venne attuato e si chiama sistema italgiure. Questo sistema venne successivamente esteso anche ad altre decisioni giudiziarie e anche a dati normativi. Diremo qualcosa sul sistema italgiure un po' più avanti. Per il momento continuiamo con questa breve rassegna storica sull'evoluzione dell'informatica giuridica. \par 
Sempre negli anni 60 e 70 inizia il dibattito su banche dati e privacy ad opera di giuristi, Stefano Rodotà in Italia, Spiros Simitis in Germania, di giuristi particolarmente sensibili al tema della riservatezza informatica. Che cosa succede? Come vedremo vanno creandosi in questo periodo sia a livello di possibilità tecnica, sia a livello di realtà effettuale, vanno creandosi banche dati di grandi dimensioni, banche dati, sovente in mano pubblica, sovente gestite dagli Stati, dalle pubbliche amministrazioni, che raccolgono numerose informazioni personali sui cittadini. Sorge immediatamente il problema del controllo su queste informazioni, vale a dire sul fatto che queste informazioni siano raccolte in maniera esatta e sul fatto che queste informazioni non siano utilizzate per fini distorti, per esempio a fini discriminatori. Come potrebbe succedere se per esempio un certo datore di lavoro volesse con un'operazione che l'informatica rende molto semplice conoscere le opinioni politiche di certi possibili lavoratori o di persone che si candidano a un posto di lavoro. Quindi in questo contesto inizia in Italia soprattutto, ma non solo, anche negli Stati Uniti e in Europa, il dibattito sul rapporto tra privacy e banche dati. Un libro molto importante di Rodotà di questo periodo si chiama Elaboratori elettronici e controllo sociale ed evoca esattamente quest'idea. \par
Il background tecnologico è esattamente quello che abbiamo detto, il formarsi di grandi banche dati, banche dati di grandi dimensioni, che raccolgono tantissimi dati e informazioni personali e che non dialogano, non sempre dialogano tra loro, ma sono accessibili a distanza, cioè possono essere collegati con dei terminali e quindi per esempio una banca dati di grandi dimensioni detenuta da un'amministrazione centrale, per esempio a Roma, a tale banca dati può accedere l'amministrazione periferica situata a Palermo, a Milano o ad Aosta. 

Negli anni 80 e 90 abbiamo una presa di coscienza sempre maggiore da parte del legislatore innanzitutto comunitario e poi da parte di alcuni legislatori nazionali in Europa, in Italia si arriverà a questo con un po' di ritardo, una sempre maggiore presa di coscienza sia delle opportunità, sia dei rischi che offrono la formazione di banche dati.\par
Le opportunità offerte dalle banche dati sono oggetto di normative comunitarie sulle banche dati, vale a dire normative comunitarie che intendo, e poi nazionali, che intendono tutelare la proprietà intellettuale sulla banca dati. Che cosa vuol dire? Una banca dati può essere considerata un'opera dell'ingegno, una creazione intellettuale, perché è frutto di ingegno, è frutto di inventiva il modo in cui i dati sono catalogati, sono organizzati, il modo in cui si effettua la ricerca e così via. Quindi sorge la necessità commerciale di proteggere il valore di queste invenzioni.\par 
D'altra parte però si capisce, si prende coscienza che le banche dati come punta dell'iceberg possono dare luogo anche a rischi, i rischi che dicevamo prima sul controllo delle informazioni, integrità delle informazioni, la circolazione delle informazioni personali contenute nella banca dati. Per questo motivo si sviluppano in questo periodo normative  sia nazionali che comunitarie sulla privacy anche in rapporto agli elaboratori elettronici e anche in rapporto alle banche dati. Cioè queste normative in realtà alla fine travalicano l'ambito informatico e l'ambito delle banche dati, ma è pur vero che trovano nel fenomeno dell'informatica e delle banche dati loro nocciolo duro. \par
Lo sfondo tecnologico, il background tecnologico delle normative anni 80 e 90 è un cambiamento nel modo in cui si diffonde la tecnologia informatica. Se precedentemente avevamo il grande enorme calcolatore di uso anche abbastanza tecnico, specialistico, che deve essere governato da un operatore ultra specializzato che indossa un camice bianco in una sorta di laboratorio e così via e inserisce schede magnetiche, schede perforate o nastri magnetici e quant'altro, negli anni 80 e 90 si diffonde il personal computer, il PC, vale a dire l'informatica diventa alla portata di tutti o quasi, i computer diventano di piccole dimensioni e con un processo che è stato descritto da alcuni tecnici, da alcuni esperti del settore, un processo considerato ineluttabile, vale a dire sempre minori dimensioni dell'apparato e sempre maggiore capacità di calcolo e di memoria. Cioè un personal computer degli anni 80 o 90 che si trova sul tavolo di qualsiasi anche studente di scuola o universitario, ha una potenza di calcolo e di memoria incomparabilmente superiore al macchinario, al grande calcolatore enorme che viene gestito dal tecnico in camice bianco di qui sopra. \par
Quindi diffusione di personal computer e interconnessione di questi computer con il World Wide Web, il WWWW. Gli inizi nel corso dei primi anni 90, a livello tecnologico esisteva già, ma comincia diventare alla portata di tutti l'accesso alla rete globale, all'internet e quindi l'interconnessione di tutti questi personal computer tra loro e ad altri computer, a banche dati eventualmente e così via. Quindi si passa dal modello quasi della monada, il singolo calcolatore enorme, difficile da utilizzare, costoso e che prende molto spazio, si passa al computer piccolo che ha grandissima potenza di calcolo e di memoria e che è interconnesso, potenzialmente, con qualsiasi altro computer tramite la rete globale.\par 
Infine dagli anni 2000 in avanti tramite la rete globale, tramite internet, vi è una sempre maggiore interconnessione delle fonti disponibili, delle banche dati per esempio. Ad esempio banche dati come italgiuri della Cassazione inizialmente nascono come banche dati residenti e utilizzabili solo negli uffici della Cassazione, dopodiché l'accesso viene esteso ad alcuni uffici giudiziari, poi a tutti gli uffici giudiziari, poi anche ad alcune pubbliche amministrazioni, poi viene permesso l'accesso ai privati, a pagamento e così via.\par 
Quindi tramite le nuove tecnologie l'accessibilità delle fonti, dei database per esempio giuridici aumenta sempre di più. Aumenta in maniera esponenziale come fenomeno praticamente quotidiano il ricorso al commercio elettronico, l'entità di scambi di beni e servizi che si svolgono su Internet. Prende piede, a dir il vero con una certa fatica, ma è un fenomeno ormai anche questo irreversibile, l'informatizzazione della pubblica amministrazione, vale a dire che la burocrazia, gli uffici trasferiscono gradualmente sul supporto elettronico le proprie informazioni, i propri archivi e interconnettono anche i propri archivi, cioè possono dialogare in tempo reale scambiandosi le informazioni, là dove ciò sia previsto dai loro compiti istituzionali ovviamente.\par 
C'è un background tecnologico e Internet diventa sempre più diffuso e viaggia anche su supporti diversi, il Wifi, il satellite e così via, non più solo sulla rete telefonica. E' l'avvio di fenomeni che sono tuttora in evoluzione sotto i nostri occhi come ad esempio il cloud, la nuvola, cioè vale a dire la distribuzione sia della memoria, sia della potenza di calcolo dei computer su una rete. Ciascun computer che aderisce a un cloud, che si inserisce in un cloud, mette in comune e prende da altri computer sia la propria memoria, sia la propria capacità di calcolo.\par 
Questo pone alcuni problemi giuridici per esempio sulla riservatezza e la sicurezza delle informazioni, perché si verifica una condivisione ma anche una delocalizzazione sempre maggiore delle informazioni che prima magari risiedevano nel computer dell'utente.\par 
Conclusa questa velocissima panoramica sulla storia dell'informatica giuridica che ci ha permesso di aprire alcuni problemi che vedremo in dettaglio in questa lezione e nelle prossime, passiamo al successivo argomento.

\section{L'informatica e la conoscenza del diritto}

La conoscenza del diritto può essere intesa in senso ampio o in senso stretto. Il rapporto tra informatica e conoscenza o conoscibilità del diritto è evidente laddove ripensiamo a quello che abbiamo detto precedentemente, cioè che l'aspetto documentale o documentario è uno dei primi in ordine cronologico e tutt'ora uno dei primi in senso quantitativo e di importanza dei ruoli che ci si aspetta sia svolto dall'informatica nel diritto. 
Vale a dire il ruolo della conservazione della documentazione giuridica e la sua accessibilità semplificata nelle procedure, nei costi, nei tempi, tramite le risorse dell'informatica.
\subsection{Conoscenza del diritto in senso ampio}
Conoscenza del diritto in senso ampio è un concetto che ricopre la possibilità, l'opportunità di accedere a tutte le informazioni rilevanti per fruire di un servizio statale, quindi conoscenza del diritto in senso ampio riguarda la disponibilità delle informazioni, in questo caso disponibilità di  informazioni con strumenti informatici sulla rete internet, informazioni che riguardano l'azione dei poteri statali e la possibilità del soggetto di fruire delle opportunità legate a servizi statali. Ad esempio il fatto che un'amministrazione pubblichi un certo bando di concorso, che pubblicizzi la disponibilità di finanziamenti per certi tipi di attività e così via. 
\subsection{Conoscenza del diritto in senso stretto}
In senso stretto è il senso di cui ci occuperemo più specificamente nel seguito di questa lezione.
La conoscenza del diritto o conoscibilità riguarda la possibilità di accesso a norme giuridiche, quindi non a qualsiasi atto di una pubblica amministrazione che offra opportunità, informazioni, eccetera, ma a norme giuridiche, esempio tipico a leggi, a norme legislative.\par 
\subsubsection{differenza conoscenza e conoscibilità}
Una piccola premessa, una piccola distinzione che può essere utile, conoscenza e conoscibilità non sono la stessa cosa, anche se in maniera un po' imprecisa ci capiterà di usarle come sinonimi.\par 
\hl{Conoscenza} del diritto riguarda l'effettiva conoscenza, apprensione cognitiva che un soggetto ha del contenuto di una norma giuridica o dell'esistenza di una norma giuridica. \par
\hl{Conoscibilità} invece riguarda la possibilità, l'opportunità, l'accessibilità alle norme giuridiche. Riguarda la possibilità astratta, l'esistenza di presupposti idonei affinché le norme giuridiche, il diritto sia conoscibile, affinché le norme giuridiche diventino oggetto di conoscenza. Quindi è ovvio che i due concetti non sono coestensivi \textit{(hanno la stessa estensione ovvero si applicano agli stessi oggetti/eventi/fenomeni)}, posso avere idonee, plausibili, condizioni di conoscibilità di norme giuridiche, perché esistono tutta una serie di circostanze che sono idonee a rendere il diritto conosciuto e allo stesso tempo qualche cittadino può restare all'oscuro delle norme giuridiche e non realizzare la conoscenza del diritto.\par
Tenendo in mente questa distinzione ci occuperemo in realtà non tanto di conoscenza ma di conoscibilità, vale a dire ci occuperemo delle condizioni affinché il diritto sia conoscibile, anche se contingentamente può restare vero il caso che qualcuno non conosca di fatto il diritto. \par
Vediamo un po', conoscenza e conoscibilità del diritto, l'approccio tradizionale. L'approccio tradizionale tuttora in auge in gran parte nel nostro sistema giuridico è una presunzione di conoscenza. La presunzione di conoscenza del diritto è data dalla pubblicazione delle norme giuridiche, in particolare delle norme legislative ma non solo, anche di atti aventi forza di legge e alcune categorie di regolamenti, pubblicazione di questi atti nella gazzetta ufficiale. 
Vale a dire lo Stato, tramite un ente pubblico che è il Poligrafico della Zecca dello Stato, predispone un servizio di pubblicazione degli atti aventi forza di legge, una volta promulgati dalle competenti autorità politiche, politico-giuridiche questi atti vanno pubblicati in un servizio ufficiale che è la Gazzetta Ufficiale dello Stato.\par 
Una volta che quell'atto è stato pubblicato nella Gazzetta Ufficiale si presume che i cittadini lo conoscano e questo è l'approccio tradizionale. Questo approccio tradizionale si ripercuote su alcune discipline, per esempio sull'articolo 5 del codice penale su cui è intervenuta una sentenza della Corte Costituzionale che menzioneremo tra breve, ma che nella sua formulazione originaria codificava il cosiddetto principio dell'ignorantia legis non excusat, vale a dire che non si può addurre come scusa del fatto che non è stata seguita una legge penale, una norma penale, il fatto che non se ne conosceva l'esistenza. 
Questo si ricollega alla presunzione di conoscenza che ho detto prima, al fatto che quella norma penale è stata pubblicata in Gazzetta Ufficiale. \par
Questo approccio, che è l'approccio tradizionale, ha alcuni limiti. In primo luogo si tratta di una concezione palesemente collegata alla concezione liberale dello Stato minimo e al principio di stretta legalità del diritto penale.


Che cosa vuol dire? È come se dietro questa concezione della pubblicazione in Gazzetta Ufficiale come presunzione di conoscenza delle norme e dietro il principio dell'ignorantia legis non excusat, ci fosse dietro una realtà istituzionale, un sistema di Stato che non è più quello che abbiamo noi oggi. Era  il modello di Stato liberale ottocentesco, cioè lo Stato minimo, lo Stato che adotta poche leggi chiare che restano in vigore per molto tempo, non vengono cambiate e modificate continuamente, di facile lettura e così via.\par
In questo modello che è comunque un modello semplificato, ideale, si potrebbe anche presumere, anche qui con un buono sforzo di volontà, che gli atti normativi importanti per il cittadino siano pubblicati in una pubblicazione periodica alla Gazzetta Ufficiale che il cittadino, con una semplice operazione di consultazione, magari andando in una biblioteca, può apprendere, può conoscere. \par
Chiaramente questo modello si sposa con una concezione, anch'essa abbastanza superata, che è quella secondo cui lo Stato, che è Stato minimo, interviene soprattutto per vietare, è uno Stato guardiano notturno, lo Stato che lascia libertà ai soggetti di svolgere le loro attività commerciali, nelle loro condotte di vita, intervenendo però con pochi precetti penali di divieto per proteggere alcuni beni. Chiaramente questo, a sua volta, si lega all'esigenza della certezza del diritto penale cioè i cittadini devono essere avvisati in anticipo di quali siano le condotte di vita e questo preavviso si trova nella pubblicazione in Gazzetta Ufficiale di certi atti normativi.\par
Questo modello, come si vede, è oggi superato, e la pubblicazione in Gazzetta Ufficiale come strumento di presunzione, iuris et de iure (\textit{una presunzione legale che non ammette prova contraria. In pratica, se qualcosa è presunto "iuris et de iure", allora è automaticamente considerato vero dalla legge, e non puoi dimostrarne il contrario, nemmeno se hai prove. La legge presume iuris et de iure che il marito sia il padre del figlio nato durante il matrimonio.}), di conoscenza del diritto, è ormai superata ed inadeguato per il tipo di Stato che abbiamo oggi in Italia e che hanno quasi tutte le democrazie di tipo occidentale.\par
Si è passati dallo Stato minimo, lo Stato che produce poche norme chiare, stabili nel tempo, secondo un ideale illuministico, ad uno Stato sociale e regolatore, cioè uno Stato che assume tantissimi compiti, interviene in tantissime attività sociali, promuove certe attività, si fa carico di certi interessi dei cittadini come ad esempio l'istruzione, la sanità e quant'altro, interviene a regolare tantissime attività private che non sono più lasciate a libera contrattazione delle parti ma sono regolate dall'alto dallo Stato e quindi in questo panorama dello Stato sociale e regolatore la produzione normativa non è più riconducibile a poche norme chiare, stabili nel tempo e così via, ma diventa una produzione normativa, massiccia, caotica, perché lo Stato interviene in tantissimi ambiti della vita sociale e della vita dei cittadini e quindi produce tanto diritto. E' anche una produzione caotica perché è una produzione normativa soggetta a obsolescenza, è facile che le norme giuridiche invecchino, sia necessario sostituirle perché emergono nuove esigenze, emergono innovazioni tecnologiche o innovazioni sociali che possono evolversi in maniera anche abbastanza rapida e così via. \par
In più la quantità spesso è anche nemica della qualità, si può produrre molto diritto e male, nel senso che le varie norme giuridiche possono essere poco coordinate tra di loro, possono essere oscure, può non essere chiaro quale sia il collegamento tra la norma giuridica promulgata in un certo momento e il panorama giuridico preesistente, se si sia verificata una abrogazione tacita o si possono armonizzare queste norme e così via. \par
Un ulteriore fattore di complicazione del panorama normativo attuale è il fatto che lo Stato oltre a intervenire su tanti ambiti della vita sociale e individuale, non è più lo è stato monadico (\textit{monadico si riferisce a ciò che ha natura di monade, cioè qualcosa di unitario, indivisibile, autosufficiente}), isolato, non più nella sua sovranità, ma integrato in sistemi sovranazionali come l'Unione Europea, nel caso dell'Italia, la Convenzione Europea dei Diretti dell'Uomo, la World Trade Organization e così via. \par
Vele a dire che norme giuridiche vengono fuori da fonti diverse, anche queste che spesso si accavallano, sono caotiche, vengono prodotte in quantità massiccia, nuovamente in questo quadro l'ideale illuministico della pubblicazione di poche leggi chiare in un documento, la Gazzetta ufficiale, che basta sfogliare per avere cognizione del diritto, diventa sempre più lontano nel tempo e appartenente a un passato ormai irrecuperabile.\par 

Nel contesto dello Stato sociale regolatore, il contesto che abbiamo appena visto, la fonte cartacea come fonte di conservazione e organizzazione dei documenti normativi, diventa insufficiente o inadeguata perché la fonte cartacea è fissa, si pubblica un volume cartaceo, ma se gli atti normativi si susseguono nel tempo in maniera anche disordinata e anche veloce, la fonte cartacea diventa obsoleta perché il singolo volume su cui è stato pubblicato un atto normativo del 2003 diventa inservibile, diventa obsoleto sei mesi dopo, un anno dopo, due anni dopo, quindi la fonte cartacea diventa quasi uno spreco, diventa inservibile, diventa obsoleta.
\par
In più, poiché la produzione normativa è massiccia e caotica, la fonte cartacea ha il limite di non rendere chiari o di rendere difficoltosi i collegamenti sistematici, poiché le norme giuridiche sono destinate a interagire con un tessuto spesso molto complesso e spesso disordinato di ulteriori collegamenti normativi nazionali ma anche sovranazionali e così via. Pensiamo ad una legge di attuazione di una direttiva comunitaria, allora può essere necessario svolgere dei collegamenti sistematici tra l'atto normativo che ci interessa e ulteriori atti normativi nazionali, sovranazionali e così via. Questi collegamenti sistematici non sono impossibili ma sono difficili con la fonte cartacea perché evidentemente la singola legge dovrebbe tirarsi dietro come riferimenti, per esempio in nota, decine e decine di altri atti normativi e quindi la fonte cartacea diventerebbe enorme, si dilata la quantità di carta che è necessaria per far fronte a queste esigenze con i relativi costi e con la relativa difficoltà anche di consultazione.\par 

Quindi la fonte cartacea è insufficiente, costosa e ingombrante per i motivi che abbiamo detto. Abbiamo detto anche prima che un personal computer degli anni 90 o 2000 è in grado di immagazzinare informazioni che trasferite sul supporto cartaceo, occuperebbero una pila di carta di fogli A4 alta 30 metri. Ecco, la fonte cartacea è ingombrante, è costosa, deve essere trasportata nelle biblioteche per la sua diffusione.\par 

\section{Prospettiva contrattualistica} 35.00
Prospettiva contrattualistica, che cosa vuol dire? Il modello della presunzione iuris et de iure, della conoscenza assoluta del diritto, una volta che il diritto sia pubblicato in Gazzetta Ufficiale si rivela insufficiente e inadeguato anche in una prospettiva contrattualistica che dovrebbe governare rapporti tra Stato e cittadino. I\par
In un contesto in cui la conoscenza o meglio la conoscibilità del diritto diventa sempre più difficile nel modo tradizionale, con la pubblicazione in Gazzetta Ufficiale, eccetera, eccetera, è come se lo Stato venisse meno ad una parte del proprio patto sociale, del proprio contratto sociale, ad una parte degli obblighi che gli derivano dal contratto sociale. \par
La prospettiva contrattualistica, il contratto sociale è come se dicesse la seguente cosa: "il cittadino è tenuto all'obbedienza alle leggi, il contratto sociale, il riunirci in società richiede che ciascuno obbedisca alle leggi che governano la nostra società, viceversa la società si disgregherebbe, quindi il contratto sociale richiede un certo livello di obbedienza allo Stato, un certo livello di obbedienza alle leggi, ma d'altra parte lo Stato deve fare in modo che le proprie leggi siano conoscibili da parte dei cittadini."\par 
L'obbligo di obbedire al diritto è controbilanciato in una prospettiva filosofico-politica, in una prospettiva contrattualistica, in una prospettiva di contratto sociale, dal diritto dei cittadini alla conoscibilità del diritto stesso, a che il diritto sia reso conoscibile dallo Stato, a che le norme giuridiche prodotte dallo Stato siano conoscibili da parte dei cittadini e quindi all'uso di tutte quelle tecnologie, incluse le tecnologie informatiche e telematiche, che possono agevolare tale conoscibilità e il superamento delle tecnologie antiquate, per esempio la pubblicazione cartacea che non può più assicurare la conoscibilità.

\subsection{Conoscibilità del diritto è l'articolo 3 comma 2 della Costituzione}
Questo è riferimento alla sentenza della Corte Costituzionale che ho citato prima. La sentenza 364 del 1988 è intervenuta sull'articolo 5 del Codice Penale che abbiamo indicato in precedenza. Nella sua vecchia formulazione dell'articolo 5 del Codice Penale diceva che riguardo alle leggi penali non si può invocare a scusa dell'aver commesso un reato la mancata conoscenza delle leggi penali. La Corte Costituzionale in questa occasione ha detto che questo articolo deve essere integrato, riformulato, nel senso di dire che talvolta l'ignoranza della legge può essere scusabile. Vale a dire che possono esserci certe circostanze come l'oscurità del dato normativo, la difficoltà di reperirlo, la contraddittorietà della giurisprudenza e delle interpretazioni e sono circostanze che rendono il dato normativo difficilmente conoscibile. Date queste circostanze si può scusare anche l'avere commesso un illecito penale.\par
Questo si riporta all'articolo 3 comma 2 della Costituzione dove ci dice che è compito della Repubblica rimuovere gli ostacoli di ordine economico e sociale alla piena partecipazione di tutti i cittadini alla vita sociale, economica, politica del Paese. Evidentemente produrre diritto e renderlo poco conoscibile rappresenta da parte dello Stato la realizzazione di un ostacolo a questa piena partecipazione dei cittadini. Quindi anche i principi costituzionali richiedono una piena conoscibilità, sempre maggiore del diritto e non solo una presunzione assoluta di conoscenza.\par  38:47
\subsection{Problemi per banca dati legislativa}
Quando si vuole costruire una banca dati legislativa quali sono i problemi che troviamo? 
\begin{itemize}
    \item L'accessibilità, cioè renderla accessibile a quanti più soggetti possibili
    \item  la necessità di coordinare testi vigenti e testi storici, vale a dire il tenere traccia delle eventuali modifiche normative verificatesi
    \item Avere criteri di ricerca del materiale normativo legato ai contenuti, quindi a ciò che una legge dice, alla materia di cui si occupa e non ai soli dati estrinseci, vale a dire il nomine uiris, la data di promulgazione, il numero del documento, queste informazioni estrinseche che non rendono facilmente individuabili molti documenti normativi che hanno contenuti eterogenei
    \item  possibili inadeguatezza della ricerca testuale, questo perché i termini legislativi possono essere talvolta ambigui, indeterminati o usati in maniera diversa
    \item contenere solo dati recenti, perché sono i dati che quando sono stati prodotti potevano già essere riversati in un supporto informatico, i dati disponibili solo sul supporto cartaceo sono più costosi per trasferirli su supporto informatico
\end{itemize}

Esempi in Italia sono:
\begin{itemize}
    \item il CED della Cassazione
    \item  siti web della Camera e del Senato che contengono legislazione
    \item sito web della Corte Costituzionale che contiene le sentenze della Corte Costituzionale
    \item il GIURITEL del Poligrafico
\end{itemize}
Un esperimento abbastanza utile, svolto in Italia e tuttora in corso di implementazione è un esperimento pubblico prima noto come norme in rete, poi modificato sia nel nome sia nel modo di funzionamento che Normattiva. 
Norme in rete era un portale che consentiva un punto di accesso unificato alla legislazione tramite una federazione di pubbliche amministrazioni che mettevano in comune i propri dati legislativi cercando anche di uniformare i riferimenti utilizzati.
\subsubsection{Normattiva}
Normattiva è subentrata nel 2010 ed è una banca dati pubblica gratuita disponibile online, ipertestuale vale a dire che contiene i rimandi ipertestuali delle modifiche, degli emendamenti e quant'altro, contiene testi in multivigenza vale a dire che tiene traccia delle modifiche normative verificate nel tempo e contiene solo atti numerati come leggi, decreti legislativi, DPR e decreti ministeriali numerati, il che dovrebbe dare luogo anche ad uno scoraggiamento del proliferare delle fonti atipiche. 
Anche Normattiva ha alcuni limiti, ma che sono in realtà marginali, 
\begin{itemize}
    \item non contengono norme regionali
    \item non contiene un'organizzazione per materie
    \item la grafica del sito è un po' rigida 
\end{itemize}
Si chiude qui la lezione odierna di informatica giuridica per ulteriori approfondimenti vi invito a consultare il sito web.