\chapter{Mercato unico digitale}
\section{Introduzione al mercato unico digitale}
L'argomento di oggi è il mercato unico digitale e riguarda l'approccio europeo alla regolamentazione dei mercati e dei servizi digitali.

La Commissione Europea ha avviato già nel 2015 una strategia per il mercato unico digitale, ci sono stati due importanti regolamenti europei che hanno regolato la materia.

\subsection{Definire il mercato digitale}

Il mercato digitale ha le caratteristiche tipiche di un mercato oligopolistico ed ha un altro grado di concentrazione, ma pochi operatori.
Inoltre, il mercato unico digitale ha un alto grado di autonomia della piattaforma dominante.
C'è l'esistenza di sistemi verticalmente integrati in grado di porre efficaci barriere all'ingresso del mercato e c'è anche un basso grado di contendibilità del mercato, che significa che è un mercato senza barriere all'ingresso, c'è anche un'alta probabilità di fusione di pratiche sleali e scorrette.
Mentre in astratto la rete è libera, nella realtà essa è sottoposta all'esercizio del potere di pochi soggetti.
Quindi, l'idealizzazione della rete come luogo di espressione e della libertà individuale non centralizzata e non gerarchizzata si è presto dimostrata solo come un ideale astratto. In realtà, il peso delle poche piattaforme globali ha prodotto una struttura allo stesso tempo fortemente centralizzata e priva di regole e di controlli.
I tentativi di autoregolamentazione hanno dimostrato la necessità di disciplinare il potere privato anche in vista della tutela dei consumatori.
Quindi, il legislatore europeo ha scelto di introdurre accanto agli ordinari rimedi attinenti alla disciplina antitrust, alcuni strumenti tipici della regolamentazione di settore, incluso la regolamentazione asimmetrica, ai quali si accompagna un sistema di vigilanza, controllo e sanzione fortemente centralizzato.
Le proposte di regolamento europeo sui mercati digitali e sui servizi sono state approvate prima dal Parlamento Europeo e poi del Consiglio Europeo nel 2022.
A partire dal 2023, secondo tappe cadenzate, si è sviluppata l'applicazione di questa normativa.
Il regolamento relativo alla disciplina dei mercati, o meglio il Digital Market Act, riguarda strettamente i rapporti tra le piattaforme e i fornitori di servizi, mentre il regolamento in materia di servizi, il Digital Services Act, ha per oggetto prevalentemente i rapporti tra gli erogatori di servizi e gli utenti.

\subsection{Digital Market Act}
Noi oggi esamineremo specificamente il Digital Market Act.
Quindi la disciplina del mercato e l'ordinamento europeo si è sempre sviluppata lungo tre direttrici principali:

\begin{itemize}
    \item la tutela della concorrenza volta a garantire l'apertura dei mercati e la competizione fra più operatori
    \item la regolamentazione dell'accesso ai mercati e delle attività imprenditoriali
    \item la costruzione di un mercato unico europeo all'interno del quale l'attività economica non sia soggetta a limiti o barriere nazionali mediante i quali gli Stati potrebbero difendere e proteggere le proprie imprese rispetto alle imprese degli Stati membri dell'Unione Europea  
\end{itemize}

Quindi sostanzialmente nel mercato digitale opera quest'ultima direttrice che non è rilevante sia perché i grandi operatori non sono europei, sia perché la regolamentazione nazionale non sarebbe in grado di porre efficaci barriere.
Quindi la tutela della concorrenza è stata utilizzata invece in una prima fase per cercare di limitare e controllare le condotte oligopolistiche e monopolistiche delle piattaforme, con effetti però circoscritti alle transazioni economiche e commerciali, non potendo estendersi a profili molto rilevanti della struttura e della configurazione del mercato, delle regole di accesso e del regime di responsabilità degli operatori.


La disciplina di questi profili richiede invece una complessa e nuova regolazione che tenga conto delle caratteristiche peculiari e inedite del mercato digitale.
Quindi la disciplina del mercato digitale, del ruolo delle piattaforme online nel mercato unico è stata già approfondita in una comunicazione della Commissione europea nel 2017, una comunicazione relativa alla revisione di medio periodo della strategia europea in materia.

A questo primo passo è seguito nel 2021 la proposta di regolamento denominata per appunto Digital Markets Act, approvato nel 2022.

La disciplina di fondo è improntata a una configurazione delle piattaforme rispetto ai mercati che è riassunta nella considerazione per cui a differenza delle imprese in altri settori queste piattaforme non competono sul mercato o per il mercato ma queste piattaforme sono il mercato; quindi ciascuna piattaforma cerca di costruire un mercato nel quale attirare il maggior numero possibile dei clienti rendendo poi difficile sia l'uscita dal mercato dei propri clienti sia l'entrata sul mercato di operatori concorrenti che vengono spesso acquistati prima che possano mettere in atto una dinamica concorrenziale.
Facciamo un esempio: l'acquisto di WhatsApp e Instagram da parte di Facebook.
Per far fronte a questa specifica configurazione del mercato digitale, il legislatore europeo ha ritenuto che non fosse sufficiente l'applicazione dell'ordinaria disciplina antitrust basata sul collegamento tra determinati obblighi e determinati comportamenti degli operatori sui mercati accertati e esposti dall'autorità competente.
Accanto e in aggiunta alle regole antitrust, utilizzabili anche per le piattaforme, il regolamento sui mercati digitali introduce un quadro di obblighi exante (a partire da prima), indipendentemente dalla preventiva analisi e determinazione del mercato rilevante, che in genere invece è il presupposto per l'applicazione della disciplina antitrust.
Quindi dobbiamo affermare che il presupposto di applicazione degli obblighi sta invece per il digital market act non in specifici comportamenti, attenzioni da accertare, ma in elementi oggettivi come la natura e la dimensione dell'operatore.
A questi elementi è stata data nel regolamento una specifica qualificazione giuridica mediante l'introduzione di una nozione di gatekeeper.

\subsection{Gatekeepers}
Gatekeepers sono tutte le imprese che forniscono dei servizi di piattaforma di base, includendo fra i servizi tutti quelli di intermediazione online, quindi motori di ricerca online, servizi di social network online, servizi di piattaforma per la condivisione di video, servizi di comunicazione interpersonale indipendente dal numero, i servizi operativi, i browser web, gli assistenti virtuali, i servizi di cloud computing e servizi pubblicitari.
Quindi quanto alle dimensioni, la qualificazione di gatekeeper si applica alle imprese che nel fornire un servizio di piattaforma di base come punto di accesso, il gateway, utilizzato dai fornitori per raggiungere gli utenti finali, abbiano un impatto significativo sul mercato interno e detengono una posizione consolidata e duratura o siano in un futuro prossimo in grado di acquisire questo tipo di posizione.

Attenzione, si presume che un'impresa sia qualificata come gatekeeper se raggiunge determinate soglie.
Un'impresa è qualificata come gatekeeper se raggiunge un fatturato annuo nella unione pari o superiori a 7,5 miliardi di euro in ciascuno degli ultimi tre esercizi finanziari oppure una capitalizzazione di mercato media non inferiore a 75 miliardi di euro nell'ultimo esercizio finanziario relativamente ad un servizio di piattaforma di base fornito in almeno 3 stati membri o ancora se fornisce un servizio di piattaforma di base nell'ultimo esercizio finanziario ad almeno 45 milioni di utenti finali attivi su base mensile stabiliti o situati nell'unione ed almeno 10.000 utenti commerciali attivi su base annua stabiliti nell'unione.

La sussistenza di queste condizioni opera su piano giuditico come una presunzione, cioè a queste condizioni è connesso un obbligo di notifica della commissione in capo alle imprese intercettate e interessate in presenza di questi indicatori; quindi la piattaforma dovrà attivare un procedimento di notifica alla commissione e la commissione provvederà alla designazione della piattaforma quale gatekeeper entro 45 giorni dal ricevimento delle relative informazioni.
Si tratta di presunzioni confutabili dal soggetto interessato, al quale grava però l'onere di dimostrare che pur raggiungendo le soglie indicate dal regolamento non sussistono gli elementi per la designazione come gatekeeper, cioè gli obblighi previsti dal regolamento possono essere inoltre applicati o essere ulteriormente specificati sulla base anche di un dialogo regolatorio per l'appunto tra il gatekeeper e la commissione.

Questa è la garanzia della flessibilità del sistema.
La commissione può peraltro procedere alla designazione di una piattaforma come gatekeeper anche in assenza di una notifica, quindi può procedere nell'esercizio dei propri poteri in materia di indagine di mercato, di ispezione esaminati successivamente.
L'ambito di applicazione degli obblighi incombenti sul gatekeeper è peraltro definito con riferimento agli utenti stabiliti o titolati nella Unione Europea a prescindere sia dal luogo di stabilimento o residenza della piattaforma sia dalla normativa altrimenti applicabile alla fornitura del servizio.

\subsection{Obblighi previsti per i Gatekeepers}
Quali sono gli obblighi previsti con il digital market act?
Gli obblighi imposti ai gatekeeper sono numerosi e riguardano principalmente l'uso e l'accesso ai dati, ai rapporti con i fornitori, agli utenti finali, l'interoperabilità e sistemi di controllo e di monitoraggio.
Quanto ai rapporti con i fornitori e gli utenti finali, occorre consentire agli fornitori terzi di offrire sulla piattaforma i medesimi servizi offerti dal gatekeeper anche a prezzi e condizioni diversi così come si deve consentire il contatto fra utenti commerciali e utenti finali anche mediante applicazioni di software diverse da quelle proprie della piattaforma.
Il regolamento impone anche rilevanti obblighi quanto all'interoperabilità definita come la capacità di scambiare informazioni e di fare un uso reciproco delle informazioni scambiate tramite interfacce o altre soluzioni in modo che tutti gli elementi hardware e software funzionino con gli altri elementi hardware e software con gli utenti in tutti i modi destinati a funzionare.
Quali altri obblighi sono previsti?
L'obbligo di consentire agli utenti finali di disinstallare con facilità qualsiasi applicazione software presente nel sistema operativo del gatekeeper fatte salve le necessità tecniche connesse al funzionamento del sistema operativo e il divieto di impedire che gli utenti finali possano passare ad abbonarsi a servizi e applicazioni di software di terzi accessibili tramite servizi di piattaforma di base del gatekeeper.
Il gatekeeper non può infine assicurare un trattamento di favore ai propri prodotti e servizi quanto al loro posizionamento e la loro indicizzazione rispetto a servizi o prodotti analoghi di terzi e deve invece applicare delle condizioni trasparenti, eque e non discriminate.
Per quanto riguarda i poteri di vigianza e controllo dobbiamo specificare che la commissione dispone di poteri di indagine, di esecuzione, di monitoraggio e sanzionatori ivi compreso il potere di avviare indagini di mercato per valutare e definire i parametri qualitativi e quantitativi utili alla qualificazione delle imprese.
Se la commissione accerta all'inosservanza degli obblighi regolamentari, può sia irrogare delle sanzioni patrimoniali fino al 10\% di fatturato anno, sia nei casi più gravi adottare delle misure cautelari, accettare impegni ed indicare ai Gatekeepers  misure specifiche da tenere per conformarsi agli obblighi regolatori se gli impegni dell'impresa non siano sufficienti.
Possono anche essere imposti dei rimedi comportamentali o strutturali nel rispetto il principio di proporzionalità, il caso in cui la commissione accerti a seguito di un indagio di mercato che il gatekeeper abbia violato sistematicamente gli obblighi e abbia così ulteriormente rafforzato e ampliato la sua posizione di mercato.
Il regolamento detta anche delle specifiche previsioni per quanto riguarda il rapporto tra la commissione e i giudici nazionali.
Ad esempio questa previsione è diretta a garantire la coerenza complessiva del sistema di controllo e di sindacato sulle condotte dei gatekeeper.
I giudici nazionali possono, quando si trovano a giudicare su un caso per il quale si applica il regolamento, chiedere alla commissione di trasmettere loro le informazioni in possesso e anche di rilasciare un parere sulle questioni applicative rilevanti.
Quindi cosa possiamo dire in conclusione?
Il regolamento sui mercati digitali è un assetto regolatorio articolato complesso che va ben oltre i rimedi previsti dalla disciplina antitrust e avvicina la disciplina delle piattaforme alle regole da tempo previste per i mercati finanziari o per i servizi pubblici come le energie, le telecomunicazioni per arrivare all'applicazione di questo impianto regolatorio che è entrato in funzione solo a partire dal 2024.
