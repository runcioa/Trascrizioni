\chapter{I soggetti e i ruoli della protezione dei dati personali}

\section{Regolamento europeo 679/2016}
Non avete una videolezione sul regolamento europeo 679/2016 il regolamento che ha innovato la disciplina in materia di trattamento dei dati personali.
Quindi voi partirete solo dal concetto di privacy che è spiegato benissimo in quella videolezione, ho associato a quella videolezione un estratto di un libro che esamina il regolamento europeo 679/2016.
Oggi parleremo dei soggetti e i ruoli della protezione dei dati personali.

\subsection{Chi è l'interessato?}

L'interessato è la persona fisica a cui appartengono i dati personali oggetto del trattamento quindi una persona fisica e non persona giuridica, perché c'è stata anche una modifica in merito.
Il GDPR garantisce all'interessato una serie di diritti, il diritto di accesso ai propri dati personali, il diritto alla rettifica, il diritto alla cancellazione dei dati (il diritto all'oblio), il diritto alla limitazione del trattamento, il diritto alla portabilità dei dati, il diritto all'opposizione al trattamento e anche un altro diritto che trovate in un articolo specifico del GDPR, l'articolo 22 che è il diritto a non essere sottoposti a profilazione.

\subsection{Chi sono gli altri soggetti?}
Chi sono gli altri soggetti?
Il titolare del trattamento, ad esempio.
Il soggetto al quale compete di definire le finalità, le modalità dei trattamenti che egli stesso intende operare, compreso anche quelle che sono le modifiche tecniche organizzative tese a garantire la sicurezza dei trattamenti.
Inoltre, un'altra funzione che ricopre il titolare del trattamento è anche conformare la sua attività alle norme nazionali e internazionali in materia di trattamento di dati personali e anche in altri ambiti, perché una normativa non è un corpo a se stante, quando vado ad applicare questa normativa, ma questo anche in un ambito giudiziario, devo effettuare un'interpretazione sistematica, cioè che fa riferimento a un sistema di più leggi.
Quindi questo vale anche e soprattutto per le interpretazioni di una norma, io giudice come faccio ad interpretare una norma senza tenere in considerazione tutte le altre norme del sistema giuridico?

\subsection{Il responsabile}
Il responsabile è un soggetto diverso dal titolare, quindi esterno, terzo, che si occupa di elaborare i dati per conto del titolare e in base alle sue istruzioni che vanno contraattualizzate anche ai fini della documentazione della conformità.
Generalmente è responsabile della gestione e della conservazione delle banche dati spesso su cloud e quindi della messa in sicurezza dei dati.
Quali sono le specifiche responsabilità di questa figura?
Assistere il titolare nell'adozione delle misure di sicurezza in materia di protezione di dati personali con riguardo alla propria area di competenza, assumere in autonomia le decisioni nell'ambito delle attribuzioni conferitegli e anche il dover coordinarsi col titolare, ma anche con il DPO.
Il DPO è una nuova figura che ha avuto il suo ingresso nello scenario nel contesto della protezione dei dati personali proprio con il regolamento europeo 679-2016, è una figura che se è nominata deve coordinarsi con il responsabile.
Il DPO è nominato per garantire il corretto trattamento dei dati personali comunicati dal titolare.
Altra funzione del responsabile è eseguire quando necessario, con le sue competenze, una valutazione di impatto e astenirsi ad adottare autonomie decisionali in materia di trattamenti dei dati o meglio decisioni diverse da quelle previste dal contratto stipulato con il titolare.
Deve controllare periodicamente l'efficacia delle misure di sicurezza adottate e la loro conformità alle norme e nonché alle istruzioni date dal titolare, adottare le misure idonee a consentire agli interessati l'effettivo esercizio dei propri diritti, in particolare agevolare senza ritardi l'accesso ai dati da parte degli stessi.
Ci deve essere una collaborazione con il titolare e il DPO se è nominato, ma anche una collaborazione specifica per l'esecuzione dell'attività di verifica dell'adeguatezza e della conformità dei trattamenti posti in essere.
Il responsabile mette a disposizione del titolare anche le informazioni rilevanti per dimostrare rispetto degli impegni assunti.
%11:50

\subsection{Soggetto incaricato al trattamento dei dati}
Abbiamo poi il soggetto incaricato al trattamento dei dati, una persona fisica appositamente istruita dal titolare al fine di eseguire dei compiti materiali in nome e per conto del titolare.

\subsection{Data Protection Officer DPO}
Andiamo a questa figura nuova, il Data Protection Officer DPO.
E' una figura professionale che ha il compito di coadiuvare il titolare del trattamento nell'adempimento degli obblighi derivanti dal GDPR.

Viene nominato obbligatoriamente quando il trattamento include dati sensibili su larga scala o quando il trattamento presenta un rischio elevato per i diritti e libertà degli interessati.
Questa figura è importante perché svolge anche una funzione di mediazione tra l'interessato e l'autorità di controllo.

\subsection{Autorità di controllo}
L'autorità di controllo è il garante 

Quali sono gli altri formanti di questa normativa?
Innanzitutto deve essere specificata la la finalità e questo è importantissimo. Inoltre deve essere garantito un diritto all'informativa.

C'è il diritto all'obblio che è collegato alla deindicizzazione.

Il diritto all'obblio che cos'è?
Un diritto ad essere dimenticati. Stefano Rodotà, quando trattava questo tema lo definiva come il diritto di essere dimenticati però, voi avete anche una normativa europea e quindi lo potete trattare analizzando quell'articolo specifico del regolamento europeo al 79/2016.
Se parlate di diritto all'obblio, c'è anche il diritto alla cancellazione dei propri dati personali, ma c'è anche un'altra articolazione, la deindicizzazione, che è un'altra cosa.
Perché non si cancella nulla.
Cos'è che succede deindicizzando? Non è accessibile.
Vi dovete rendere conto e fare cose con le parole.
È un sistema sociale diverso tra tecnica.
Quindi sono due sistemi sociali differenti.
Il diritto è la tecnica, però abbiamo anche detto il diritto è tecnica.
Ci dobbiamo chiedere una tecnica diversa, ovviamente, dalla tecnica intesa come nuove tecnologie.
O il progresso tecnologico, certamente.
Ma mi raccomando, noi facciamo cose con parole.
Bisogna essere precisi quando analizziamo questi concetti.

Cos'è il diritto all'oblio e quale significato assume oggi in questo contesto di continuo mutamento e trasformazione digitale?
Voi sapete che anche un tema molto caldo in questo periodo è la digitalizzazione e il diritto del lavoro.
Quindi, voglio dire, tutto viene trasformato.
Innanzitutto il diritto all'oblio risponde a questa esigenza che ognuno di noi ha di poter far dimenticare a tutti delle informazioni che non sono più attuali, oppure sono inesatte o meglio, non mi rappresentano più.
Quindi, il diritto all'oblio si esplica nel nostro ordinamento nel diritto proprio di essere dimenticati ed ha a che fare con l'identità personale.
Oltre a questo diritto all'oblio, prima vi parlavo delle interpretazioni sistematiche, ma nel diritto c'è anche un'altra attività interpretativa, il bilanciamento.
Il diritto all'oblio viene bilanciato con il diritto alla memoria.
Gli esseri umani vivono da sempre una contraddizione rispetto a ciò che vogliono che sia ricordato, quindi che tutti sappiano di loro stessi e ciò invece che preferiscono che venga dimenticato.
Da un lato aspirano in qualche modo a ciò che la memoria dovrebbe tendere, quindi all'immortalità e sapendo di non poterla avere, cercano di lasciare il più tempo possibile il ricordo di sé, la memoria di sé, come un modo per prolungare la vita.
All'opposto, ogni persona ha anche la preoccupazione che un comportamento, un atto, un'azione negativa conpiuta nel corso della propria esistenza possa essere ricordato per sempre.

Il diritto all'oblio è un diritto più giovane rispetto al diritto alla memoria. Sicuramente il diritto all'oblio va bilanciato con il diritto alla memoria. Inoltre c'è un altro bilanciamento che può essere effettuato con il diritto di cronaca.

In questo caso il diritto all'oblio si scontra con il diritto all'informazione; il diritto all'oblio ne costituisce un diritto contrapposto con la sua accezione attiva il diritto di informare ma anche nella sua accezione passiva il diritto di essere informati.
Quindi informare su fatti relativi ad un individuo fornendo al contempo informazioni personali sullo stesso nel rispetto però di tre fondamentali parametri.

Quindi quando è che il diritto di cronaca prevale sul diritto all'oblio?

\begin{itemize}
    \item Per un interesse pubblico
    \item Quando c'è un contesto di attualità, cioè io devo informre quando un fatto si verifica
    \item La veridicità. Devo fornire informazioni corrispondenti al vero
\end{itemize}


Ci può essere una contrapposizione tra diritto all'oblio e libertà di manifestazione del pensiero.
Quindi ogni individuo gode del diritto alla libertà di opinione, di espressione, incluso il diritto di ricevere informazioni e idee attraverso ogni mezzo.
Quindi libertà che gli strumenti della tecnica, quindi della società dell'informazione consentono a chiunque, quindi attenzione non solo ai giornali, di poter rendere pubblica delle informazioni degli individui.
Voi pensate al caso delle foto, video privati, anche a sfondo sessuale che vengono trasmessi.

Quindi abbiamo fatto una panoramica dei diritti che si pongono in contrasto, comunque in una posizione oppositiva con il diritto all'oblio.

Ma come lo dobbiamo definire il diritto all'oblio?
Storicamente, certamente, lo vediamo nascere con il concetto di privacy e diritto alla riservatezza.
Senza dubbio è emerso anche nell'era precedente all'avvento del web, quando si faceva riferimento a dei dati che erano considerati oscuri per l'impossibilità di accedervi.
Quando per esempio alcuni giornalisti entrarono in posseso di alcuni dati posseduti dalla FBI, quindi ci si scontrava con questo diritto anche anteriormente all'avvento della società digitale.
Poi il diritto all'oblio viene comunque riconosciuto nei tribunali, quindi prima nella prassi giurisprudenziale e poi nella normativa.

Quindi è interessante anche parlare di questa evoluzione del concetto di diritto all'oblio.
Oggi trovate una positivizzazione nel Regolamento Europeo 679/2016.

Dobbiamo far riferimento anche alla deindicizzazione, che è un'articolazione del diritto all'oblio e che è fondamentale per comprendere la portata di questo diritto.
La deindicizzazione consente un'operazione sostanzialmente differente dalla cancellazione.
Non elimina il contenuto, ma lo rende non direttamente accessibile tramite i motori di ricerca esterni all'archivio in cui quel contenuto si trova. Il web non dimentica.
La deindicizzazione è uno dei compromessi per garantire una tutela attraverso quei contenuti che rimarranno li dove sono.

Articolo 17 GDPR della legge 679/2016.
L'interessato ha il diritto di ottenere dal titolare del trattamento la cancellazione dei dati personali che lo riguardano.
Quindi, che cos'è il diritto all'oblio?
È il diritto di ottenere dal titolare la cancellazione dei dati personali.
In che modo?
Senza ingiustificato ritardo. Quindi, è specificato anche il modo.
Il titolare del trattamento ha l'obligo di cancellare senza ingiustificato ritardo i dati personali.
Se sussiste uno dei motivi seguenti.

\begin{itemize}
    \item A. I dati personali non sono più necessari rispetto alle finalità per i quali sono stati raccolti
    \item B. L'interessato revoca il consenso a cui si basa il trattamento
    \item C. L'interessato si oppone al trattamento e non sussiste alcun motivo legittimo prevalente per procedere al trattamento
    \item D. I dati sono stati trattati illecitamente
    \item E. I dati devono essere cancellati per adempiere a un obbligo giuridico previsto dal diritto dell'Unione Europea o dello stato membro cui è soggetto il titolare del trattamento 
    \item F. I dati sono stati raccolti relativamente all'offerta dei servizi e la società delle informazioni di cui al paragrafo 8.
\end{itemize}

Quindi qui trovate la casistica, l'ipotesi in cui questi dati devono essere cancellati.
Tuttavia la norma dice ai paragrafi 1 e 2 che non si applicano nela misura in cui il trattamento deve essere necessario.
Quindi attenzione, ci sono delle eccezioni.
Quando:
\begin{itemize}
    \item A. il trattamento è necessario per l'esercizio del diritto e la libertà di espressione di informazioni 
    \item B. Per l'adempimento di un obbligo giuridico che richiede il trattamento previsto dal dritto dell'Unione Europea o di uno stato membro per motivi di interesse pubblico anche nella sanità, ai fini di archiviazione del pubblico interesse, della ricerca scientifica, storica o fini statistici e per l'accertamento dell'esercizio e la difesa di un diritto in sede giudiziaria.
\end{itemize} 
Quindi mi sembra che la norma sia molto chiara.
Voi avete l'enunciazione sia del diritto alla cancellazione dei dati sia delle circostanze in cui l'interessato ha diritto alla cancellazione dei dati ma avete anche specificate l'insieme di casistiche in cui i miei dati non possono essere cancellati e quanto appena detto non si applica.

E la prossima volta parleremo dell'articolo 22 del GDPR.
E se volete rifletterci studiate anche la parte relativa nel libro che vi ho assegnato.


Perché voi ne dovete consegnare solo due.
Anche da un punto di vista tecnico.
Io comunque li trovo anche per mail.
Quindi il termine dell'iscrizione all'esame coincide con il termine per la consegna degli esercizi.

Mi raccomando il 13 novembre. Ci sarà la professoressa Flick. Che vi tratterà Data governance e direttiva open data.
E vi presenterà anche le slide.
