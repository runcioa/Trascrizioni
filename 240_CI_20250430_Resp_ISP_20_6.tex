\chapter{Responsabilità ISP}

Iniziamo a parlare della responsabilità dell'internet service provider.
Cosa avete capito da questa videolezione?
La responsabilità dell'internet service provider è relativa a una participazione alla stessura dei contenuti.
Quindi la regola generale qual è in materia di responsabilità dell'internet service provider?
Che è obbligato a intervenire su segnalazione da parte dell'autorità in quanto non può di sua iniziativa censurare eventuali testi, eventuali cose scritte sul web perché la responsabilità di giudicare è degli organi competenti e non sua, lui deve agire su direttiva.




Quali sono le novità in materia di responsabilità dell'internet service provider?
Sostanzialmente a febbraio del 2024 è divenuto applicabile il Digital Service Act che introduce delle regole che vanno ad armonizzare in tutta l'Unione Europea la regolamentazione del settore dei servizi digitali e delle piattaforme online con l'obiettivo di garantire un ambiente online sicuro e affidabile per gli utenti, e quindi di contrastare la diffusione di contenuti illegali.
Il Digital Services Act introduce delle norme disciplinanti il mercato unico dei servizi digitali, rafforzando gli obblighi degli operatori commerciali a tutela degli utenti finali.
Tale regolamento si inserisce nella strategia per il mercato unico digitale in Europa e sostituisce la precedente direttiva del 2000, la direttiva 2031, sul commercio elettronico.

Il Digital Services Act è in vigore dal 16 novembre del 2022, ma le sue norme sono divenute pienamente applicabili per tutti gli operatori dal 17 febbraio 2024, mentre per le piattaforme e motori di ricerca di grandi dimensioni sono applicabili già a partire da agosto 2023.

Il regolamento prevede norme specifiche per le piattaforme online e motori di ricerca, il cui numero medio mensile di destinatari del servizio dell'Unione Europea supera i 45 milioni.
Dicevo che il Data Services Act si applica a tutti gli intermediari online.
Quindi, cosa sono questi intermediari online?
Ci sono gli Internet Service Provider che forniscono i propri servizi all'interno dell'Unione, a prescindere dal fatto che il fornitore sia situato in uno stato membro o meno.
Ciò che è rilevante è che il suo servizio abbia come destinatari un numero significativo di utenti europei.
Esempi di servizi digitali a cui si applica il regolamento sono per esempio i Marketplace, i Social Network, gli Apple Store, i servizi di cloud, i motori di ricerca, le piattaforme di noleggio e i servizi di accesso ad Internet.
Andiamo un po' più nel dettaglio.
Il Digital Services Act si applica ai fornitori di tre tipologie di servizi.
Servizi di semplice trasporto, "mere conduit", ossia quei servizi che forniscono accesso ad una rete di comunicazione e consentono tramite essa la trasmissione di informazioni fornite da un utente.
Servizi di memorizzazione temporanea, servizi di caching, ossia servizi attraverso i quali le informazioni vengono automaticamente memorizzate in via intermedia e temporanea con il solo scopo di facilitare la trasmissione ad altri destinatari.
Poi ci sono invece i servizi di memorizzazione di informazioni, i cosiddetti servizi di hosting, che consentono la memorizzazione delle informazioni fornite dall'utente dietro sulla richiesta, nonchè la condivisione di informazioni e contenuti online.
Questa suddivisione delle tre tipologie di fornitori è importante perchè è presente nelle vostre video lezioni.
Il Data Service Act introduce in primo luogo obblighi di carattere generale che sono applicabili a tutti gli Internet Service Provider.
Nello specifico i prestatori di servizi intermediari online devono istituire degli appositi punti di contatto che facilitino i rapporti del fornitore con i destinatari del servizio e con le autorità nazionali ed europee.
Inoltre, cosa devono fare?
Se hanno la sede fuori dall'Unione Europea devono nominare un rappresentante legale dell'Unione Europea e devono predisporre poi delle condizioni di utilizzo del proprio servizio e pubblicare annualmente delle relazioni di trasparenza.
Notate bene il riferimento al discorso sulla trasparenza.
Altre disposizioni del Digital Services Act invece vanno a regolamentare i servizi di hosting, i quali devono predisporre un meccanismo di segnalazione dei contenuti online attraverso un sistema, cosiddetto di Notice and Take Down, ossia l'hosting provider deve informare il destinatario di quali contenuti non possono essere pubblicati e deve informare il destinatario di quali siano le conseguenze di tale comportamento.
Qualora vengono riscontrati dei comportamenti che contravengono a queste disposizioni devono essere adottati dei provvedimenti tempestivi e motivati ad esempio restrizioni della visibilità, limitazioni dell'account responsabile e blocco dei pagamenti.
Per quanto riguarda gli obblighi aggiuntivi applicabili ai platform online i fornitori devono predisporre dei sistemi interni di gestione dei reclami dando priorità alle segnalazioni dei cosiddetti trusted flogger.
Chi sono questi trusted flogger?
Sono i segnalatori attendibili, ossia sono enti di natura pubblica o privata che sono dotati di particolari requisiti di competenze e di indipendenza rispetto agli internet service provider e in caso di controversia deve essere garantito a destinatari del servizio il diritto a rivolgersi ad organi giurisdizionali o ad organismi di risoluzione stragiudiziale.
Inoltre, assume particolare importanza il divieto di introdurre delle interfacce o percorsi di navigazione ingannevoli volte a manipolare, ad influenzare le scelte dei fruitori del servizio, nonché il divieto di uso dei dati particolari per la profilazione a fini pubblicitari.
Per quanto riguarda i minori è fatto divieto assoluto di profilazione, usando anche solo i dati comuni.
Andiamo a vedere quali sono gli obblichi applicabili ai marketplace.
Il prestatore del servizio deve tutelare i consumatori garantendo la sicurezza e la trasparenza della piattaforma e richiedendo particolari requisiti di tracciabilità agli operatori commerciali che intendano offrire i propri prodotti e i propri servizi sul mercato online.
Per quanto concerne gli obblichi supplementari ci sono delle specifiche valutazioni che devono essere effettuate in relazione ai rischi derivanti dalla progettazione, dal funzionamento e dall'uso del servizio stesso e quindi devono essere anche introdotti dei meccanismi di attenuazione di quei rischi.
Tra i rischi principali vi è quello di diffusione di contenuti illegali e contenuti lesivi della salute fisica e mentale, discriminatori che riguardano la violenza di genere e di stampo terroristico.

Contenuti lesivi dei diritti fondamentali quali la libertà di espressione, di informazione e anche contenuti che possono ledere i minori.
Inoltre è necessario sottolineare che il Digital Services Act prevede per ciascuno stato membro la designazione di uno specifico coordinatore dei servizi digitali che assume un ruolo di vigilanza e di verifica della corretta applicazione del regolamento. In Italia questo ruolo è stato assegnato all'AGCOM.
Cioè l'ente regolatore delle comunicazioni elettroniche.
Più che onente regolatore è un'autorità.
Attenzione a l'aspetto, perché qui c'è una questione giuridica.
Tra l'altro di diritto amministrativo.
Siamo di fronte a un'autorità indipendente.
Voi sapete che se andate a studiare il diritto costituzionale, non tanto il diritto amministrativo, esaminate i vari poteri dello Stato.
Magistratura, governo, parlamento.
Ci sono le autorità indipendenti che si pongono fuori dagli organi necessari che rappresentano il potere, che sono espressioni del potere.
L'autorità per le garanzie nelle comunicazioni è un'autorità amministrativa indipendente italiana di regolazione con sede a Napoli e con una sede secondaria a Roma.
Allora io ho caricato nei materiali comuni a tutte le lezioni, non ho caricato tutti i materiali che devo caricare, perché dovevo caricare anche la direttiva open data, ma ho caricato il data governance.
Voi sapete che il 13 ci sarà la professoressa Flick che vi parlerà di questo nuovo tema.
Quindi per favore prendete visione del regolamento, il data governance act e poi caricherò anche la direttiva.
