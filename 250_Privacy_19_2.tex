\chapter{Privacy}

Per quanto riguarda la privacy il codice della privacy vigente è stato riformato dal regolamento europeo 679/2016 e vi avevo detto di guardare anche sul sito del garante della privacy.
Sul sito del garante c'è uno schema con i cambiamenti che sono stati apportati dal regolamento europeo ricepito da un decreto legge di agosto 2017.
Quali sono i cambiamenti?
Il sito del garante illustra quali sono i fondamenti della disciplina che è praticamente orientata al raggiungimento di alcuni obiettivi e ha il suo fondamento nel principio di legalità che va a performare tutta la disciplina.
Innanzitutto cosa viene detto?
Prima nel codice previgente si parlava anche di un principio di necessità, cioè che praticamente i dati dovevano essere trattati fino a quando non fosse stato raggiunto l'obiettivo del trattamento.
Non a caso nell'articolo 17, ora il diritto all'oblio, uno dei requisiti per esercitarlo è che non sia più necessario trattare i dati, quindi che il trattamento dei dati abbia esaurito il suo fine.
Poi, altra questione sulla quale volevo farvi riflettere sono le definizioni.
Qual'è la definizione di trattamento?
Il trattamento è una serie molto lunga di operazioni che vengono fatte dall'acquisizione del dato fino alla sua cancellazione. C'è anche la conservazione, l'archiviazione, il trasferimento.
La definizione di interessato. Chi è l'interessato?
È la persona fisica a cui si riferiscono i dati.
Ora, andiamo alla parte sul titolare.


Perché sul titolare il responsabile e anche la nuova figura del responsabile del data officer presenta anche delle novità.
Non so se avete visto sul sito del garante, se non ricordo se il link lo avevo messo nei materiali, però credo di sì.
Sul sito del garante viene sottolineato che ci sono delle novità.
Innanzitutto, come diceva il vostro collega, tutto è basato sul principio della responsabilizzazione.
Per quanto riguarda la definizione di titolare responsabile e il data officer, diamo delle definizioni.
Innanzitutto, chi è il titolare?
Il titolare è una persona fisica o giuridica, una autorità che stabilisce quali sono le finalità e i mezzi per il trattamento del data di personale.
Il titolare è colui che decide le sorti del trattamento.
Invece, il responsabile è chi è?
Il responsabile è colui che opera per conto del titolare.
Non viene più data nel regolamento la definizione di incaricato, tuttavia, adesso ci si riferisce all'interno della definizione di terzo, come una persona autorizzata al trattamento dei dati sotto l'autorità diretta del titolare o del responsabile.
Avete visto il sito del garante, sul sito del garante è sottolineato la novità, quindi, in poche parole, cercate di spiegarlo.
Cosa significa sub responsabile?
Il sub responsabile è il personale che ha il più responsabile, il più responsabile è il personale che ha il più responsabile, il superDAY, il wipes can per il responsabile.
La sua responsabilità potrebbe essere articolata.
La responsabilità di chi?
Del responsabile del trattamento dice?
Sì.
Ok.
E su questa parte, sul principio invece della responsabilizzazione, in che modo incide, soprattutto su quello che nella precedente legislazione italiana, europea, era il concetto delle misure di sicurezza ad applicare i dati o delle misure di protezione, perché attraverso il principio di responsabilizzazione in effetti non si dettano più delle vere e proprie misure di sicurezza, ma si lascia a fare le misure di sicurezza che sono le prime.
E' un'idea che si fa in modo che la responsabilità di chi?
Quindi, in qualche modo è una sua responsabilizzazione sulla protezione dei dati personali.
E poi la nuova figura invece del data protection on, che è la di cui si tratta?
La di cui si tratta?
La di cui si tratta?
La di cui si tratta?
La di cui si tratta?
La di cui si tratta?
La di cui si tratta?
La di cui si tratta?
La di cui si tratta?
La di cui si tratta?
La di cui si tratta?
La di cui si tratta?
La di cui si tratta?
La di cui si tratta?
La di cui si tratta?
La di cui si tratta?
La di cui si tratta?
La di cui si tratta?
La di cui si tratta?
La di cui si tratta?
La di cui si tratta?
La di cui si tratta?
La di cui si tratta?
La di cui si tratta?
La di cui si tratta?
La di cui si tratta?
La di cui si tratta?
La di cui si tratta?
La di cui si tratta?
La di cui si tratta?
La di cui si tratta?
La di cui si tratta?
La di cui si tratta?
La di cui si tratta?
La di cui si tratta?
La di cui si tratta?
La di cui si tratta?
La di cui si tratta?
La di cui si tratta?
La di cui si tratta?
La di cui si tratta?
La di cui si tratta?
La di cui si tratta?
La di cui si tratta?
E' un'altra figura che fa da tre d'union fra l'autorità garante e praticamente il mondo dell'organizzazione.
Lui vigila anche sull'osservanza della normativa che è stata introdotta con il regolamento europeo in oggetto.
Ora invece andiamo alla definizione di dato che qualcuno di voi prima aveva fornito.
Sì.
Allora innanzitutto il dato personale, qual è la differenza con la disciplina previgente?
Sono stati anche previsti dei dati che hanno un particolare regime.
E sono stati, cioè dal dato sensibile, sono stati differentiati, hanno trovato una loro differenziazione, anche dati sanitari, i dati biometrici e i dati genetici.
Genetici.
Allora il dato personale è qualsiasi informazione che concerne una persona fisica identificata o identificabile.
Si considera identificabile la persona che può essere identificata direttamente o indirettamente con particolare riferimento ad un codice identificativo, per esempio o il nome oppure un numero di dati relativi all'applicazione, ecc.
I dati personali, scusate, i dati genetici sono quei dati che sono relativi a caratteristiche genetiche ereditarie o acquisite di una persona fisica che forniscono quindi informazioni uniche sulla fisiologia o sulla salute di questa persona fisica.
Il dato biometrico.
Anche questa è una novità che non era prevista dal codice della privacy.
Sono dati personali ottenuti da un trattamento tecnico specifico relativo a caratteristiche fisiche, fisiologiche o comportamentali di una persona fisica che ne consentono o ne confermano un'ivoca.
Per esempio, immagine facciale o dei dati datiloscopici.
I dati relativi alla salute.
Allora, quali sono questi dati relativi alla salute?
Voi vi ricordate la disciplina previgente nei dati sensibili?
C'era una differenzazione inerente non alla definizione di dato sensibile, ma alla disciplina.
C'è i dati relativi allo stato di salute e anche alla sessualità dell'individuo.
Avevano un trattamento differenziato, perché il trattamento era soggetto oltre che al consenso dell'interessato, ma anche ad un assenso del garante della privacy che doveva dare una autorizzazione al loro trattamento.
Ora, i dati relativi alla salute.
Sono quei dati relativi alla salute fisica o mentale di una persona fisica, compresa anche la prestazione dei servizi di assistenza sanitaria che rivelano informazioni sul suo stato di salute.
Inoltre, quali sono i dati che sono trattati, ma vengono trattati in forza di?
Innanzitutto, il consenso, giusto?
Un contratto.
Poi, se la legge lo prevede, come un adempimento previsto dalla legge.
E la novità è anche per un interesse vitale di un soggetto.
Vengono trattati a garanzia di un soggetto che non può provvedere autonomamente alla tutela di un suo interesse vitale.
Ci siamo?
Sì.
Queste cose le avete trovate?
Sì, sono all'interno dell'articolato.
Poi, abbiamo detto che, per iniziante, li vedete anche sul sito del garante.
Qui abbiamo, il regolamento prevede, una serie di presupposti di ricettà del trattamento.
E abbiamo detto il consenso, il contratto, un obbligo legale, la salvagoria di interessi vitali e poi c'è anche compiti di interesse pubblico connessi all'esercizio dei pubblici poteri e anche il legittimo interesse del titolare.
Ok?
Ci siamo?
Sì.
Cosa cambia nel consenso?
Intanto che, diciamo, ora non ricordo esattamente cosa dice il testo, però che deve essere comprensibile, esplicito, può essere anche orale, non è detto che debba essere per forza di aver scritto, e appunto, diciamo, i garanti...
Ma che significa esplicito?
Cioè, nel senso che deve essere in grado di mostrare che l'interessato ha prestato il suo consenso ad uno specifico trattamento.
Ok?
Non cambia invece il fatto che il consenso deve essere libero, specifico, informato e inequivocabile e non cambia che non è ammesso il consenso tacito o presunto, quindi no a caselle prespuntate su un modo.
Inoltre, poi abbiamo parato del contratto.
Della salvaguardia di interessi vitali.
Quindi, in questo caso, la priorità della salvaguardia di un bene supremo mette, tra parentesi, la prerogativa di non acconsentire all'utilizzo di dati personali.
Mi sentite?
Sì, adesso sì.
Ah, sì.
Inoltre, abbiamo detto il legittimo interesse del titolare del trattamento.
Questa categoria era conosciuta anche nel codice della privacy.
Dell'obbligo legale, immagino che non ci siano.
Per quanto riguarda i minori, il trattamento dei dati personali di minori al di sotto dei 16 anni, non è, diciamo che all'interno del regolamento c'è una norma specifica che riguarda il consenso nel caso di offerta diretta di servizi a minori.
Quindi, il trattamento dei dati personali di minori al di sotto dei 16 anni, o se previsto dagli stati membri un'età inferiore ma non al di sotto dei 13 anni, infatti per quanto è legito, se nel caso di consenso, il trattamento dei dati personali di minori al di sotto dei 14 anni, è legito solo se nella misura in cui il consenso è autorizzato dal titolare della responsabilità genitoriale.
Per quanto riguarda invece il diritto all'obbligo, che è praticamente disciplinato all'articolo 17 del regolamento, cosa possiamo dire sul diritto all'obbligo?
Beh, diciamo che è stata sicuramente una delle introduzioni principali, soprattutto dovuto al fatto che ovviamente oggi i dati sono soprattutto in rete e quindi ovviamente fermangono all'interno dei motori di ricerca anche vita naturale durante e quindi diciamo che i garanti europei hanno ritenuto comunque di dare la possibilità all'interessato di ottenere dal titolare la cancellazione di tutti i dati personali che lo riguardano senza ritardo, ovviamente con una serie di motivazioni in più, quindi la non necessità per esempio rispetto alle finalità per le quali sono state raccolte, la revoca del consenso o diciamo un'opposizione al trattamento degli stessi e il trattamento illecido.
Benissimo, io questo articolo 17 consiglierai di leggervelo, comunque di vedere il codice italiano, quindi l'interessato ha il diritto di ottenere dal titolare del trattamento la cancellazione dei dati personali senza ingiustificato ritardo e il titolare del trattamento all'obbligo di cancellare senza ingiustificato ritardo i dati personali se susiste uno dei seguenti motivi, quindi abbiamo detto, ha detto anche lei, i dati non sono più necessari rispetto alle finalità per le quali sono stati raccolti, l'interessato revoca il suo consenso che abbiamo detto il consenso è uno dei presupposti di lecità del trattamento, no?
Certo.
Oppure l'interessato si oppone al trattamento dei dati personali e non susiste alcun motivo legittimo prevalente per procedere al trattamento.
I dati sono stati trattati illecitamente, cosa succede se i dati sono trattati illecitamente?
I dati sono inutilizzabili.
Inoltre i dati devono essere cancellati per adempiere un obbligo legale previsto dal diritto dell'Unione Europea degli Stati membri.
Oppure i dati sono stati raccolti relativamente all'offerta dei servizi della società di informazione a minori di età.
Ricordatevi che ogni stato membro può scegliere il limite di età per il trattamento dai, quindi abbiamo detto dai 16 ai 13, scusate, ai 16.
Inoltre ci sono delle eccezioni, cioè l'interessato invece non può essere esercito a diritto all'obbligo per fatti che lo riguardano.
Se ricorra una delle seguenti ipotesi, se il trattamento dei dati personali è effettuato per l'esercizio del diritto alla libertà di espressione e di informazione, l'interessato non può esercitarlo.
Se il trattamento dei dati personali è effettuato per l'esercizio del diritto alla libertà di espressione e di informazione, l'interessato non può esercitarlo.
Oppure nel caso in cui il trattamento si è effettuato per l'adempimento di un obbligo legale previsto dal diritto dell'Unione Europea.
Inoltre, per motivi di interesse pubblico nel settore della sanità pubblica, quindi avete capito quel è il punto di contrasto, cioè io voglio esercitare il diritto all'obbligo però c'è un interesse pubblico a quel trattamento dei dati.
Ok?
Ai fini di archiviazione del pubblico interesse di ricerca scientifica o storica o a fini statistici, in questo caso il diritto all'obbligo rende impossibile o pregiudica questi fini.
Per l'accertamento o l'esercizio o la difesa di un diritto in sede giudiziaria.
Ora, nel campo del diritto all'obbligo c'è un sotto insieme che è la deindicizzazione.
Il diritto all'obbligo può consistere anche nel deindicizzare delle informazioni che sono associabili al mio nome dopo che si è effettuato una ricerca sul motore di ricerca.
Ok?
Sì.
Sulla deindicizzazione volete dire qualcosa su questa tematica?
Può gentilmente ripetere.
Sì, certo, che all'interno della categoria del diritto all'obbligo, cioè come se il diritto all'obbligo fosse in realtà una nozione ampia, più ampia e all'interno è praticamente come un sotto insieme, potremo annoverare anche la deindicizzazione.
Ci fu una famosa sentenza della Corte di Giustizia Europea chiamata proprio Google Spain.
In questa pronuncia è stato affermato il diritto dell'interessato di chiedere al motore generale di ricerca di cancellare i collegamenti risultanti dalla ricerca del proprio nome sul motore di ricerca.
La Corte di Giustizia ha rilevato che l'attività del motore di ricerca ha una sua rilevanza autonoma perché localizza le informazioni pubblicate o messe in rete da terzi e le indicizza in maniera automatica, le memorizza temporaneamente e le mette a disposizione di tutti gli utenti di internet secondo un determinato ordine di preferenza, cioè se io vado a cercare su internet il nome favorito a barra, usciranno fuori determinate informazioni.
Ok?
Tuttavia, se io nel 1999 sono stata in un'attività di ricerca di cancellare le informazioni pubblicate e non ho mai avuto la possibilità di fare un recente di ricerca, e non voglio più che appaia sul motore di ricerca che io sono stata condannata in via definitiva per un reato X, posso esercitare questo diritto, posso chiedere a Google, questa era una causa contro Google, di deindicizzare, cioè fare in modo che quella determinata informazione non sia più visibile, voglio dire quando vado a fare una ricerca su internet che è relativa al mio nome.
Ok?
Sì, sì.
Sì, perché non è più visibile, perché non è più visibile che è il mio nome.
Allora, quindi, per effetto di questa sentenza, cosa succedeva?
Che le autorità di controllo potevano ordinare al gestore del motore di ricerca di sopprimere dall'elenco dei risultati dei link verso pagine web pubblicate da terzi e contenenti informazioni relative a una determinata persona.
Tuttavia, le informazioni vengono legittimamente mantenute sul sito web di riferimento, cioè qui si sta parlando non di una cancellazione, ma di una deindicizzazione.
Ok?
Ok.
Ci siete, tutti e tre?
Sì.
Sì, sì.
Ok.
Ok.
E tuttavia c'era anche la possibilità di agire, insomma, per ottenere dagli editori dei siti web la cancellazione delle informazioni che riguardavano una determinata persona, ma tuttavia questa non è la condizione per ottenere la cancellazione dei link da parte dei motori di ricerca.
E quindi, questa è il riferimento alla deindicizzazione.
Ora, per quanto riguarda gli aspetti della...
Perché io comunque ho fatto riferimento magari ad aspetti che comunque sono paralleli alle vostre videolezioni, per quanto riguarda ciò che avete appreso nella videolezione, volete riferire qualcosa che vi è rimasto impresso?
Marcelli.
Sì.
Allora, se volessi uscire questa...
Una domanda sulla riforma, no?
Sul trattamento dei dati personali.
Ovviamente tra le cose che non abbiamo parlato ora perché altrimenti sarebbe una ripetizione.
Certo, certo.
Forse la prima cosa da osservare in generale è che è lo strumento con il quale è stata effettuata la riforma.
Cioè il fatto che comunque sia un regolamento mentre in precedenza fino a quel momento la normativa si deriva introdotta negli Stati membri attraverso una direttiva e comunque insomma è una cosa importante perché il regolamento come sappiamo è direttamente applicabile in tutti gli Stati membri dal momento della sua vigenza.
Nel caso specifico appunto il regolamento era del 2016 ma la sua vigenza partiva dal 25 maggio se non ricordo male del 2018.
Esatto.
Questo è stato fatto anche proprio per uniformare le difese delle differenti scusi discipline che vigevano in materia di trattamento dei dati personali in tutti i paesi dell'Unione Europea.
Quindi diciamo che questo intervento è stato anche un intervento di armonizzazione delle discipline.
Esatto, poi diciamo sicuramente per quello che riguarda l'Italia, per quello che riguarda l'Italia, per quello che riguarda l'Italia diciamo c'è stata poi la scelta di non abrogare direttamente il codice privacy ma di previgente, diciamo ma di emendarlo rispetto a quegli articoli che potevano risultare diciamo parzialmente o totalmente incompatibili con il nuovo regolamento.
E questo se non sbaglio è stato fatto in concetto con il decreto legislativo 101 del 2017.
Poi vediamo che mi viene in mente appunto l'introduzione di queste sicuramente diciamo che il regolamento ha come spirito quello soprattutto di in qualche misura di andare un po' a rinnovare soprattutto gli aspetti più tecnologici, più tecnologici, più tecnologici, più tecnologici, più tecnologici, più tecnologici, in misura di andare un po' a rinnovare soprattutto gli aspetti più tecnologici legati al mondo della grossa diffusione dei dati, quindi dell'utilizzo di big data, dei social media, del cloud e in generale di internet che chiaramente essendo la precedente legislazione derivante da una direttiva del 95 non era ancora del tutto diciamo concepita quindi ovviamente gli strumenti poi di trattamento dei dati e il trattamento dei dati stesso è diventato molto più intensivo e massivo.
Per questo appunto come dicevamo sono stati introdotti intanto appunto questo principio di responsabilizzazione di cui abbiamo già parlato la figura del data protection officer che comunque è responsabile per protezione dei dati che è comunque questa figura assolutamente intermedia poi ci sono stati introdotti di cui forse non abbiamo parlato due principi abbastanza importanti che riguardano anche gli aspetti un po' di sicurezza ovvero il privacy by design e il privacy by default.
Il primo il privacy by design tende sostanzialmente a specificare il fatto che nel momento in cui anzi nel momento forse ancora antecedente all'inizio di un trattamento andrebbero diciamo a una misura di tipo preventivo quindi il titolare dovrebbe già mettere appunto tutta una serie di misure di protezione delle informazioni appunto per far sì che il trattamento già nasca diciamo strutturato appunto by design con tutti i sistemi di protezione delle informazioni stesse.
La privacy by default invece sta a sottendere il fatto che ovviamente i trattamenti devono essere fatti minimizzando appunto diciamo il trattamento e la conservazione dei dati stessi ovvero applicare appunto per default un concetto di privacy non andare magari a conservare informazioni ultrone rispetto a quelle previste dal trattamento piuttosto che informazioni che non occorrono ai fini del trattamento.
Perfetto ok allora ci sono ci sono altre cose per quanto riguarda non ovviamente non il vostro collega che ha intervenuto ma Fontani e la signora Maria Cristina volete dire qualcosa su questo argomento?
Sì intanto appunto il cambiamento anche di impostazione forse abbiamo già parlato per la valutazione di impatto un altro di quegli appunto elementi fondamentali che da appunto il principio dell'accountability si identifica principalmente appunto con questa valutazione di impatto e che sostituisce se non sbaglio la notifica al garante prima c'era la notifica al garante dell'inizio del trattamento adesso viene rilasciato al titolare quindi quindi la facoltà anzi l'obbligo diciamo di valutare i rischi delle libertà dovute al trattamento ed eventualmente consultare il garante qualora tale valutazioni non fossero così totalmente sicure per la protezione dei dati e poi ha legato a questo anche registro dei trattamenti a determinate categorie di società per pubbliche amministrazioni e alcune categorie di società che sicuramente parla di aziende con almeno 250 dipendenti però fa anche riferimento a trattamenti su larga scala o la profilazione quindi tutte queste condizioni c'è l'obbligo appunto di tenere questo registro dei trattamenti da poter poi presentare alla storia che qualora tendessero necessario insomma la consultazione.
Ok, allora lei e signora Maria Cristina vuole dire qualcosa?
No, no, nel senso che ho affrontato un po' queste lezioni purtroppo non sono riuscita da approfondire quello che era nel sito del garante quindi mi sono trovata un po' per problemi personali insomma non sono riuscita però lo farò sicuramente questa settimana.
Va bene poi magari datemi, si scusate, conferma se il materiale c'è.
Ma scusi professore, chiedo, ma dov'è che ha pubblicato il...
Dovrebbe essere accanto alla lezione aspettatura, prendo il calendario.
Dovrebbe essere, questa era la lezione 7 mi sembra.
La 7 e la 8 se non ricordo.
La 7 e la 8 dovrebbe essere all'interno di queste aspettate, ora clicco sopra.
Ah no, sono anche uscita al sistema quindi devo rentrarmi.
Va beh comunque fatto se dovrebbe stare nel materiale didattico della 7 e della 8.
Però se voi non l'avete trovato...
Allora, nella 7 non c'è.
Guardi, io francamente non l'ho trovato nella 7 e nella 8 però...
Ah perfetto, allora vi carico anche un saggio insomma un materiale aggiuntivo.
Allora, per quanto riguarda la prossima volta, farete tutte le videolezioni che riguardano l'accesso, quindi l'accesso civico e l'accesso civico generalizzato.
Ok?
Ok.
Quindi sono le videolezioni.
Allora, 3, trasparenza e accesso agli atti delle pubbliche amministrazioni.
4.
10.
9.
10.
Non le avevate già fatte, vero?
No.
No.
Ok.
Le notifiche delle sanzioni del codice della strada però lo dovete fare per oggi?
Quella era per oggi, sì.
Per oggi.
Quindi queste qui che vi ho detto, quindi abbiamo detto 3.
4 e 10.
4 e 10 e poi parleremo delle sanzioni per quanto riguarda le violazioni del codice della strada.
Ok?
Ok.
Benissimo.
Allora, buona serata e buon lavoro.
Grazie.
Buona sera a lei.
Arrivederci, grazie.