\chapter{Lezione 4 - La Governance di Internet I parte}

In questa lezione inizieremo a parlare di un argomento molto complesso in continua evoluzione, la governance di Internet. Vediamo gli argomenti di oggi:

\begin{itemize}
    \item La governance di Internet
    \item il World Summit on Information Society (WSIS)
\end{itemize}

\section{Governance di Internet}

\subsection{Definizione:}

\textbf{L'Internet Governance è il concetto che include tutte le attività che determinano la direzione dell'uso e dello sviluppo della rete Internet nei suoi vari aspetti, aspetti tecnici, infrastrutturali, aspetti economici e legali, aspetti sociali e politici.}

\subsubsection{Il fondamento della internet governance:}

\textbf{Il principio fondamentale della internet governance è che nessun soggetto può gestire autonomamente Internet, cioè la rete deve restare globalmente distribuita e priva di un organo di controllo centrale.} 

Questa necessità di un controllo diffuso e la volontà di evitare che ci sia un controllo centralizzato è da un lato necessaria per la struttura della rete e per l'immensità (La rete coinvolge e riguarda attori in tutto il mondo e quindi sarebbe estremamente complesso prenderne il controllo) dall'altro il punto è che si vuole mantenere la rete come strumento di libertà e di possibilità di circolazione libera delle informazioni e di sviluppo per tutta la società.

\subsubsection{Le componenti della internet governance}

La internet governance si articola su tre livelli:

\begin{itemize}
    \item Un livello fisico costituito dall'infrastruttura. Le infrastrutture della rete sono fondamentali per il passaggio delle informazioni.
    \item Un livello logico, il codice con il quale si scrivono le informazioni che poi diventano fruibili per la collettività.
    \item i contenuti, le informazioni che sono immesse non soltanto dai professionisti ma anche dagli utenti e che sono quelle che circolano ovunque, che sono intelligibili ovunque.
\end{itemize}

I protagonisti:

\begin{itemize}
    \item Gli stakeholders della rete sono innanzitutto i governi,
    \item le istituzioni nazionali e internazionali,
    \item le imprese private,
    \item la società civile.
\end{itemize}

Come potete vedere c'è una situazione multilivello. La partecipazione possibile sulla rete è una partecipazione veramente di tutti gli attori, di tutti i diversi livelli di soggetti che interagiscono in modo diverso e che possono contribuire allo sviluppo della società dell'informazione in modo diverso.

Attività e obiettivi:

\begin{itemize}
    \item la condivisione
    \item le regole
\end{itemize}

Le regole in particolare si concretizzano nello sviluppo e nell'applicazione di principi e di procedure decisionali e programmi condivisi.

La necessità di condivisione è sentita a tutti i livelli però dall'altra parte si pone la necessità di regole che permettano da un lato di consentire a tutti di usufruire allo stesso modo delle risorse e usufruirne sia come fruitori finali ma anche come artefici e come attori di quanto viene distribuito e sia per cercare di evitare abusi della rete che sono comunque sempre possibili.

Le scelte che riguardano la governance di internet hanno delle ricadute importanti sui governi a livello politico e sulla società tutta anche a livello economico. Sono diverse le iniziative che sono state assunte nel corso degli ultimi 20 anni quindi a partire dai primi anni 2000 soprattutto per gestire la governance di internet, per individuare il modo migliore di creare una governance su internet. Le iniziative che sono state assunte sono iniziative assunte soprattutto a livello internazionale proprio in considerazione del fatto che la rete è qualcosa di globale. Attualmente tra i temi importanti che sono affrontati nell'ambito della internet governance ci sono:
 
\begin{itemize}
    \item la cyber security
    \item il cyberbullismo
    \item la lotta all'illegalità intesa ad ampio spettro
    \item i temi legati alla violazione del copyright
\end{itemize}

La internet governance quindi è qualcosa che coinvolge la collettività tutta a livello internazionale e che appunto deve essere affrontata con degli strumenti di carattere internazionale.

Bisogna distinguere tra la internet governance e la e-governance.\par

La i-governance non  va  confusa con la e-governance che indica le iniziative di un governo nazionale per la fornitura di servizi via internet a cittadini, imprese e altri governi ed è legata allo sviluppo delle infrastrutture informatiche e di telecomunicazione.

A questo punto un nostro primo spunto di riflessione. Qual è il principio fondamentale della internet governance?

\section{World Summit on Information Society}
Il World Summit on Information Society, il vertice internazionale della società dell'informazione, è stato lanciato con la risoluzione 56/183 dell'Assemblea Generale delle Nazioni Unite il 21 dicembre 2001. In questa occasione l'Assemblea Generale ha approvato lo svolgimento del vertice mondiale della società dell'informazione, WSIS, iniziando e avviando due fasi. La prima fase si è svolta a Ginevra dal 10 al 20 di dicembre 2003, la seconda fase si è svolta a Tunisi dal 16 al 18 novembre 2005.

Quindi, la risoluzione 56/183 dell'Assemblea Generale delle Nazioni Unite è il punto di partenza del World Summit on Information Society e ha come presupposto il fatto che il cambiamento tecnologico trasforma l'ambiente in cui si è sviluppata la società dell'informazione.

Quali azioni si intendevano portare avanti? Innanzitutto, ci si è resi conto che il World Summit doveva essere una piattaforma in evoluzione continua, perché è la stessa società dell'informazione in evoluzione continua proprio in considerazione dell'evoluzione e dello sviluppo della tecnologia. E quindi la promozione di qualunque tipo di policy, qualunque tipo di attività in questo ambito deve coinvolgere i livelli nazionali, regionali e interregionali, ma deve avere un luogo, un punto di riferimento in cui effettivamente sia possibile scambiarsi delle idee e in cui sia possibile costruire qualcosa di nuovo.

La struttura iniziale data al World Summit dell'Information Society quella del 2003-2005, aveva l'obiettivo di tener conto dell'evoluzione per individuare i successivi sviluppi.

Anche successivamente, dopo il 2005, il World Summit si è continuato a tenere periodicamente e ha portato delle grandi e importanti novità.

Vediamo la prima fase:

\subsection{nel 2003, dichiarazione di Ginevra}

L'obiettivo individuato nell'ambito del World Summit di Ginevra era l'individuazione di una volontà politica di stabilire le basi per una società dell'informazione per tutti. Società dell'informazione per tutti vedremo che comporta una serie di iniziative. Tutti a qualunque età, e quindi un tema di digital divide legato alle generazioni. Tutti vuol dire tutti i luoghi e tutti i paesi del mondo e quindi questo significa portare l'infrastruttura tecnologica che consente di partecipare al mondo internet anche nei luoghi più sperduti del mondo. Tutti vuol dire questioni economiche, quindi evitare di avere un divario tra coloro che hanno possibilità economiche di interagire con determinati strumenti tecnologici e coloro che non ce le hanno. Tutti vuol dire formazione e informazione sufficiente per comprendere quali sono i temi fondamentali nella società dell'informazione.

Il \textbf{piano d'azione per lo sviluppo sostenibile} è stato elaborato all'interno della dichiarazione di Ginevra. \par

I protagonisti individuati dal summit di Ginevra:
%----------------------------------------------------------------------
\begin{itemize}
    \item i governi. I governi hanno un ruolo guida e un ruolo di sviluppo nell'attuazione delle strategie della società dell'informazione nazionali e globali, strategie che devono essere lungimiranti e sostenibili da tutti i punti di vista e tali da permettere di creare e portare avanti un ambiente favorevole e competitivo per gli investimenti necessari all'infrastruttura della società dell'informazione e per lo sviluppo di nuovi servizi.
    
    \item Il settore privato e la società civile in un dialogo con i governi possono svolgere un ruolo consultivo importante nell'elaborazione delle strategie elettroniche nazionali.
    
    \item La società civile è direttamente impegnata in quello che è il mondo della società dell'informazione e della comunicazione. Soltanto la partecipazione di tutti gli attori della società civile, siano essi persone individui, persone fisiche, siano essi soggetti che rappresentano interessi che vengono dalla società civile è essenziale per dare delle indicazioni su quello che è lo sviluppo sostenibile, lo sviluppo utile della società dell'informazione.
    
    \item le istituzioni internazionali e regionali.
\end{itemize}

L'impegno del settore privato accanto a quello dei governi è importantissimo per l'evoluzione della società dell'informazione, delle tecnologie, dell'informazione e della comunicazione, sia per la realizzazione delle infrastrutture, sia per la creazione di informazioni e l'immissione di contenuti e la gestione anche dei contenuti, e sia per lo sviluppo delle applicazioni che consentono di offrire servizi al grande pubblico. Il settore privato quindi non è soltanto un attore del mercato ma svolge un ruolo in un contesto di sviluppo sostenibile il più ampio possibile.

Le istituzioni pubbliche internazionali e regionali, quando si parla di istituzioni regionali non si intende ovviamente alle regioni di un singolo paese ma a macro regioni, soggetti che comprendono diversi paesi nelle diverse aree geografiche.

Le istituzioni internazionali hanno il ruolo determinante perché in un contesto nel quale il collegamento è un collegamento ampio all'interno del mondo, all'interno di una società globale, non è pensabile che il ruolo di gestione, organizzazione, sviluppo e quant'altro sia limitato ad un ambito di carattere nazionale.

Le tematiche sono complessive, sono globali. La rete ha come particolarità proprio il fatto di essere accessibile e fruibile in tutto il mondo, senza limitazioni e confini di carattere geografico e questa è la ragione per la quale qualunque tipo di valutazione, qualunque tipo di gestione deve essere condiviso a tutti i livelli.

Oltre ad individuare gli stakeholders cioè i componenti del mondo della società dell'informazione, la dichiarazione di Ginevra si è posta anche degli obiettivi. \par

\subsubsection{2003 - Obiettivi}

\textbf{L' obiettivo primario è la costruzione di una società dell'informazione inclusiva nella quale le tecnologie dell'informazione e della comunicazione sono al servizio dello sviluppo.}

L'obiettivo di carattere generale è quello di promuovere il più possibile l'inclusione a tutti i livelli.

\begin{itemize}
    \item comunità cioè trovare il modo di collegare tutti i luoghi del mondo, anche i più piccoli villaggi, alla rete per includerli nello scambio di informazioni, stabilire quindi dei punti di accesso per tutte le comunità.
    
    \item Istruzione e ricerca: poter collegare università, college e scuole consente di condividere le informazioni e la cultura a tutti i livelli.
    
    \item gli uffici pubblici, il tema delle attività della e-democracy, quindi delle attività che collegano le amministrazioni pubbliche ai cittadini, rendere accessibile tutto ciò che è la vita pubblica di un cittadino all'interno del proprio paese.
    
    \item accessibilità vuol dire adattare i programmi e gli strumenti alla portata di tutti e quindi ancora una volta dai bambini agli anziani senza che vi siano limitazioni legate al livello diverso di preparazione culturale.
    
    \item i contenuti, occorre incoraggiare lo sviluppo di contenuti e mettere in atto le condizioni tecniche per facilitare la presenza e l'uso di tutte le lingue del mondo su internet, quindi non soltanto le lingue più diffuse ma veramente tener conto della ricchezza linguistica dei paesi del mondo e consentire a tutti un accesso. Questo significa ad esempio implementare degli strumenti di traduzione dei contenuti, implementare l'utilizzo anche di scritture e caratteri che non sono comunemente utilizzati e anche questo è uno strumento per far sì che tutti nell'ambito del mondo abbiano accesso alle tecnologie dell'informazione e della comunicazione.
\end{itemize}

Per raggiungere questi obiettivi occorre lavorare sulla formazione e sull'informazione a partire dai livelli scolastici dei più piccoli, quindi a partire dai primi passi dei bambini e per andare avanti fino al livello universitario e poi ancora continuare a mantenere il collegamento fra le biblioteche, fra i musei, fra tutte quelle istituzioni che diffondono e raccolgono la cultura.
Occorre far sì che tutta la popolazione abbia la possibilità di accedere a servizi della società dell'informazione dai più piccoli villaggi ma anche in tutti gli ambiti sociali e culturali dei paesi.

\subsubsection{2003 - Linee di azione}

\begin{itemize}
    \item  infrastrutture, per infrastrutture si intende la banda larga, si intende il satellite e tutte le infrastrutture che consentono la trasmissione dei dati. L'infrastruttura come detto più volte è essenziale per raggiungere l'obiettivo dell'inclusione digitale consentendo a tutti un accesso universale, sostenibile, ubiquo e accessibile, tenuto conto di quelle che sono le soluzioni via via sviluppate nei diversi paesi e che possono richiedere delle fasi di transizione.
    
    \item accesso all'informazione e alla conoscenza con tutti gli strumenti possibili
    
    \item sviluppo delle competenze, ognuno dovrebbe poter avere le competenze necessarie per beneficiare pienamente della società dell'informazione, pertanto la creazione di capacità e l'alfabetizzazione informatica sono essenziali. Le tecnologie dell'informazione e della comunicazione possono contribuire a raggiungere un'istruzione universale, ad ottenere e ad implementare l'istruzione universale in tutto il mondo, occorre creare gli strumenti per erogare formazione anche agli insegnanti i quali a loro volta potranno poi aiutare i propri allievi a raggiungere determinati obiettivi di competenza. Occorre creare strumenti di formazione permanente e comprendere, all'interno di questi strumenti, anche persone che hanno maggiore difficoltà ad accedere agli strumenti di formazione e di informazione migliorandole anche le competenze professionali.
    
    \item fiducia e sicurezza, per poter stare su Internet occorre avere fiducia nell'uso delle tecnologie, delle informazioni e della comunicazione e quindi occorre promuovere la cooperazione tra i governi anche per proteggere l'integrità dei dati, per proteggere l'integrità della rete, per contrastare e prevenire la diffusione di comportamenti criminali. Quindi nell'obiettivo fiducia e sicurezza c'è la prevenzione del cybercrime, la prevenzione della sicurezza delle informazioni e la tutela della privacy, della riservatezza nella rete.
    
    \item ambiente sicuro, occorre sviluppare tecnologie e strumenti che consentano di mantenere la sicurezza nell'ambito della rete.
    
    \item applicazione delle ICT in diversi settori, lo sviluppo sostenibile della tecnologia, dell'informazione e della comunicazione deve riguardare tutti i settori, dalla pubblica amministrazione alle imprese, dall'istruzione alla formazione, dalla sanità all'occupazione, all'ambiente, all'agricoltura e alla scienza.
    
    \item questo obiettivo di tener conto dell'applicazione della tecnologia nell'ambito dell'applicazione della tecnologia in tutti i settori consente ancora una volta di avere la massima inclusione.
    
    \item sviluppo dell'identità e della diversità culturale e linguistica attraverso l'uso delle lingue che consente di includere la cultura e quindi di tener conto e di sviluppare l'identità culturale, tener conto delle tradizioni, tener conto delle religioni senza escludere nessuno in un concetto di massima inclusione.
    
    \item i media hanno un ruolo fondamentale in questo contesto per la creazione di contenuti corretti, per la ricerca e la diffusione di un'informazione corretta.
    
    \item Dimensione etica, l'etica nella società dell'informazione dovrebbe essere soggetta a valori universalmente condivisi. Trovare valori universalmente condivisi non è semplice data la diversità presente nel mondo.
    
    \item cooperazione internazionale e regionale come detto è fondamentale il collegamento e la cooperazione fra soggetti nell'ambito del mondo intero per riuscire a trovare dei punti di riferimento comuni.
\end{itemize}

L'agenda di Ginevra del 2003 dava uno spunto, un ponte verso il futuro.

\subsection{2005 - agenda di Tunisi}

Nel 2005 l'agenda di Tunisi ribadisce che l'attuazione del World Summit Information Society deve tener conto delle linee di azione stabilite a Ginevra. Nell'agenda di Tunisi si è ribadito e si è chiarito il ruolo delle agenzie e delle Nazioni Unite in questo sviluppo. 

Con l'agenda di Tunisi ci si è resi conto della necessità di avere e creare nuove risorse finanziarie per stare al passo con l'evoluzione della tecnologia e ci si è resi conto e si è sottolineata l'importanza della sostenibilità per le piccole e medie imprese. Non solo le grandi compagnie ma anche le piccole e medie imprese devono poter essere incluse a tutti i livelli all'interno dello sviluppo della società dell'informazione.

Ci si è resi conto, inoltre, della sempre più crescente importanza all'interno di questo contesto dell'apporto della società civile, delle persone, dei cittadini, a fianco ai grandi stakeholders quali governi e le imprese di grandi dimensioni.

Nello sviluppo delle tematiche le organizzazioni internazionali, le agenzie delle Nazioni Unite possono svolgere un ruolo fondamentale perché hanno degli osservatori privilegiati rispetto a quelle che sono le grandi tematiche che coinvolgono il mondo intero. E quindi nella definizione della internet governance si è ribadita la necessità di avere un punto di riferimento comune per tutti e per questa ragione si è deciso di affrontare le nuove sfide con dei punti di riferimento più precisi.

\subsubsection{2005 - Internet governance}

\begin{itemize}
    \item Nell'internet governance, così come indicato nel summit di Tunisi del 2005, si tiene conto dei nomi a dominio, dei numeri di protocollo e degli indirizzi IP come strumenti di contatto e collegamento con individuazione univoca. Parleremo in una delle prossime lezioni del significato dei nomi a dominio, dei numeri di protocollo e degli indirizzi IP e di qual è il loro ruolo nella società dell'informazione e della comunicazione.

    \item è necessario coinvolgere in misura sempre maggiore i diversi stakeholders,
    
    \item è necessario coinvolgere i paesi in via di sviluppo,
    
    \item è necessario individuare delle priorità.
    
    \item nelle priorità individuate nel 2005 c'era il multilinguismo, i contenuti locali, lo spam e la cybersecurity. Inoltre si ribadì l'importanza dell'individuazione di tecnologie standard comuni perché l'utilizzo di diversi tipi di tecnologie poteva creare un digital divide fra diversi paesi in base alla scelta di una tecnologia piuttosto che un'altra.
    
    \item come luogo per esprimere le varie potenzialità e le varie idee è stato istituito l'\textbf{Internet Governance Forum, IGF}, di cui parleremo nella prossima lezione
\end{itemize}

Negli anni successivi al 2005, ogni anno si è tenuto un summit dell'informazione e della comunicazione ma arriviamo al summit al 2015.

\subsection{2015 - ICT e sviluppo sostenibile}

In questo summit abbiamo l'individuazione, la creazione e l'avvio dell'Agenda per sviluppo Sostenibile attuata con la risoluzione dell'Assemblea generale delle Nazioni Unite 70/1 e si è varata l'Agenda 2030 per lo Sviluppo Sostenibile. 

L'Agenda 2030 riconosce e promuove il ruolo delle tecnologie, delle informazioni e della comunicazione (ICT).

L'agenda 2030 per lo Sviluppo Sostenibile riguarda diverse aree dello sviluppo della collettività, le Nazioni Unite hanno ribadito ancora una volta l'importanza delle tecnologie dell'informazione e della comunicazione per lo sviluppo generale di tutti gli ambiti e di tutti i contesti. Quindi non parliamo soltanto di tecnologie per l'informazione ma parliamo di tecnologie per qualunque ambito di interesse del mondo. E quindi anche l'Agenda 2030 ribadisce la necessità di includere in maniera complessiva nell'educazione e nella promozione dell'apprendimento tutte le fasce sociali, tutti i paesi del mondo e tutte le età.

L'Agenda 2030 per lo Sostenibile tra i suoi obiettivi riconosce la necessità di promuovere l'uguaglianza a vario livello e in particolare di coinvolgere il genere femminile, all'epoca ancora non adeguatamente rappresentato e non adeguatamente presente nel mondo della tecnologia.

Ancora per lo sviluppo sostenibile si sottolinea l'importanza dell'evoluzione e della possibilità di accesso alla tecnologia attraverso lo sviluppo di strumenti nuovi. Si riconosce l'importanza di sviluppare delle tecnologie abilitanti che consentano di accedere fin dall'inizio alle informazioni e alla rete. 

Sempre nel 2015 il World Summit Information Society indica le nuove sfide tenendo in considerazione l'Agenda per il 2030.

\subsection{2015 - WSIS verso nuove sfide}

\begin{itemize}
    \item L'uomo è al centro.  Il World Summit riafferma l'impegno per costruire una società dell'informazione e della comunicazione incentrata sulle persone che sia inclusiva e orientata allo sviluppo in cui tutti possano creare contenuti, accedere, utilizzare e condividere informazioni e conoscenze, consentendo agli individui, alle comunità e ai popoli di raggiungere il loro pieno potenziale di sviluppo e migliorare la qualità della propria vita.
    
    \item ICT e sviluppo sostenibile. Il potenziale delle tecnologie dell'informazione e della comunicazione può essere sfruttato come già indicato dall'Agenda per lo sviluppo sostenibile 2030 per raggiungere proprio quegli obiettivi. E quindi sono invitati tutti i governi, la società civile, il settore privato a lasciarsi coinvolgere e a collaborare nella promozione della società dell'informazione.
    
    \item individua alcuni temi nuovi, nuovi rischi sociali e per la salute. L'incremento dell'utilizzo delle tecnologie ha degli aspetti positivissimi come si è visto ma comincia a manifestare anche delle problematiche di cui occorre tener conto e che occorre affrontare nei successivi sviluppi. Da un lato un tema sociale, l'utilizzo delle tecnologie e dell'informazione come strumento oramai quasi principale di gestione delle relazioni, comincia a evidenziare alcuni problemi rispetto ai quali occorre porsi degli interrogativi e che occorre iniziare ad affrontare. Ad esempio, ne parleremo in una delle nostre prossime elezioni, il tema del linguaggio come strumento non di inclusione ma di esclusione, la discriminazione o l'odio espressi attraverso un linguaggio all'interno di un ambiente sociale del quale fa parte una larghissima parte della comunità. Allo stesso tempo cominciano a manifestarsi dei fenomeni nuovi che possono avere un impatto alla salute legati ad esempio ad una dipendenza eccessiva dalle tecnologie. Sono elementi di cui occorre tener conto per vedere non soltanto gli aspetti positivi della tecnologia ma anche vederne gli aspetti negativi, affrontarli e trovare delle soluzioni. Questo tema verrà sviluppato negli anni successivi al 2015.
    
    \item Altro tema che comincia a diventare importante di cui occorre tenere conto è il finanziamento. Lo sviluppo delle tecnologie dell'informazione e della comunicazione richiede di mettere in campo delle risorse economiche estremamente ingenti. Risorse economiche che possono essere reperite all'interno del settore privato in quanto a volte è più difficile reperirle all'interno del settore pubblico, all'interno dei governi con quello che può conseguirne in termini anche di controllo. Il potere economico nella gestione della società dell'informazione consente il controllo. D'altra parte lo sviluppo della tecnologia richiede ingenti finanziamenti. Occorre quindi trovare delle modalità corrette, delle modalità sostenibili per continuare a sviluppare le tecnologie nella maniera migliore e più aperta. Occorre trovare delle forme di investimento che mantengano la libertà nello sviluppo e nell'attività di ricerca.
    
    \item libertà di espressione privacy. Nella società dell'informazione e della telecomunicazione è strumento di diffusione della conoscenza, è strumento attraverso il quale è possibile informarsi, è strumento attraverso il quale è possibile informare. A fronte di questo bisogna tener conto e occorre trovare degli strumenti per ottemperare questo aspetto, il diritto all'informazione, con i diritti della persona e in particolare il diritto alla propria riservatezza, alla propria privacy, al controllo delle informazioni che girano sulla rete e che vengono diffuse. Di questo come vedremo si sono occupate le istituzioni internazionali e in particolare l'Unione Europea con provvedimenti che sono stati assunti successivamente al 2015 e di cui parleremo in una delle prossime lezioni.
    
    \item la sicurezza delle informazioni. In un mondo, in un contesto nel quale le informazioni sono per così dire il pane quotidiano, nel quale le informazioni diventano un bene primario, nel quale le informazioni sono anche merci di scambio, occorre individuare, sviluppare tutto ciò che riguarda la sicurezza delle informazioni stesse e quando si parla di sicurezza si intendono diversi aspetti.
    
    \item  Sicurezza delle informazioni vuol dire evitare che le informazioni possano essere rubate, possono essere acquisite illecitamente. Sicurezza delle informazioni vuol dire evitare che le informazioni possano essere compromesse e non corrispondere più al vero. Sicurezza delle informazioni vuol dire mantenere la riservatezza quando si tratta di informazioni che non possono essere condivise con tutti. Mantenere e sviluppare la sicurezza delle informazioni vuol dire anche incrementare la fiducia delle persone rispetto all'uso delle proprie informazioni sulla rete. E quindi questo tema è in continuo sviluppo e deve essere oggetto di ricerca e promozione.
\end{itemize}


Arrivati a questo punto alcune sono le domande a cui potete rispondere. Gli spunti di riflessione.
\begin{itemize}
    \item Quali sono gli stakeholders coinvolti nella internet governance?
    \item Quali sono le sfide indicate dal World Summit for the Information Society per il futuro della società dell'informazione?
    \item Qual è il principio fondamentale della internet governance?
    \item
\end{itemize}
