\chapter{Lezione 5 - La Governance di Internet II parte}

Gli argomenti di oggi:

\begin{itemize}
    \item Internet Governance Forum
    \item il ruolo dell'Europa
\end{itemize}

\section{Internet Governance Forum IGF}

L'Internet Governance Forum è stato promosso dalle Nazioni Unite nel World Summit for Information Society che si è tenuto a Tunisi nel novembre del 2005. Il forum è stato convocato per la prima volta nei mesi di ottobre e novembre del 2006 e successivamente ogni anno si sono svolti degli incontri di questo forum nel quale si sono incontrati e scambiati opinioni tutti gli stakeholders che sono interessati alla governance di Internet. Ricorderete che tra gli stakeholders per la governance di Internet vi sono i governi, le imprese private e la società civile. Si parla quindi di un contesto nel quale si cerca di rappresentare, di ascoltare la voce di tutti coloro che ad ogni titolo sono parte del mondo Internet e quindi appunto sono interessati alla sua governance.

Cosa si tratta nell'Internet Governance Forum?

Si affrontano tutte le tematiche che possono interessare la governance di Internet e dobbiamo sempre ricordare che in questo contesto vi sono degli aspetti di carattere politico, vi sono degli aspetti di carattere tecnico e vi sono le questioni complesse legate anche al rapporto politico fra i diversi Stati e fra i diversi attori. L'Internet Governance Forum non ha dei poteri decisionali ma è un luogo, una struttura, un'organizzazione che nonostante non abbia poteri decisionali è diventata un punto centrale di riferimento per tutti i dibattiti che riguardano il mondo Internet.

Internet Governance Forum che cosa è?

\begin{itemize}
    \item Un luogo di incontro multilaterale aperto a tutti
    \item nel quale discutere della governance di Internet
    \item e affrontare regole, procedure, infrastrutture e programmi che ne determinano il funzionamento e l'evoluzione
\end{itemize}

L'elemento fondamentale che caratterizza l'Internet Governance Forum è la partecipazione di tutti, una multiparticipazione nella quale l'espressione delle valutazioni e delle idee risente della provenienza dei diversi Stackholders e la partecipazione è caratterizzata dalla democrazia e dalla trasparenza. \par

Mandato e temi dell'Internet Governance Forum:

\begin{itemize}
    \item L'obiettivo è facilitare la discussione e il dialogo tra diversi organismi sui temi della governance di Internet.
\end{itemize}

Il mandato dell'Internet Governance Forum è quello di discutere delle questioni di politica pubblica relative agli elementi chiave per la governance di Internet e promuovere la sostenibilità, la solidità, la sicurezza e la stabilità e lo sviluppo di Internet.

Non è semplice il dialogo tra i diversi attori ed è per questo che l'Internet Governance Forum deve trovare il modo di facilitarlo e deve aiutare e facilitare lo scambio di buone pratiche coinvolgendo per quanto possibile le competenze accademiche, scientifiche e tecniche dei diversi paesi che partecipano. Attraverso queste attività nel corso degli anni si sono raggiunti dei risultati interessanti.

Quali i temi principali?

\begin{itemize}
    \item sostenibilità
    \item sicurezza
    \item stabilità e sviluppo
\end{itemize}

Come funziona l'Internet Governance Forum?

Abbiamo visto che la sua istituzione è partita direttamente dall'Assemblea Generale delle Nazioni Unite attraverso il World Summit for Information Society.

Come è stato strutturato l'Internet Governance Forum? Il funzionamento prevede una partecipazione multilaterale attraverso degli incontri virtuali, gruppi di discussione tematici e riunioni periodiche. 

Il codice di condotta dell'Internet Forum prevede delle regole semplici:

\begin{itemize}
    \item rispetto e non discriminazione
    \item focus su temi specifici
    \item partecipazione informata di tutti gli attori
    \item trasparenza
    \item etica e integrità
\end{itemize}

Il forum ha una sua localizzazione presso l'Assemblea Generale delle Nazioni Unite a Ginevra ed è sostanzialmente un segretariato del quale fanno parte un gruppo di persone che si occupano delle attività pratiche. Il forum è affiancato da un gruppo di lavoro, il Multi-Stakeholder Advisory Group (MAG) che è composto di 50 membri nominati dal Segretario Generale delle Nazioni Unite individuati dai governi, dal settore privato, dalla società civile e dalla comunità tecnica. Il Multi-Stakeholder Advisory Group è stato costituito sempre nel 2006 e supporta il forum nella organizzazione e nella programmazione delle attività annuali.

I componenti del MAG collaborano con il forum per:
\begin{itemize}
    \item individuazione dei temi di discussione
    \item pianificazione degli incontri
    \item selezionare le relazioni
\end{itemize}

\subsection{Gli Internet Governance Forum locali}

L'Internet Governance Forum è il punto di riferimento centrale, è la struttura, chiamiamola così, centrale, ma a fianco ad essi sono nati nel corso degli anni una serie di strutture diversificate a livello locale, inteso come a livello nazionale ma anche a livello di macroaree, macroregioni o anche per aree tematiche. Gli Internet Governance Forum locali sono collocati in diversi paesi del mondo:

\begin{itemize}
    \item per l'Italia  (https://www.isoc.it/igfitalia)
    \item Forum regionali (Europa, Africa, Asia, eccetera)
    \item Forum di area di interesse (Forum dei Giovani)
\end{itemize}

L'individuazione delle aree di riferimento degli Internet Governance Forum è basata su dei criteri di carattere geografico, aree geografiche, criteri di carattere culturale e criteri linguistici. Tutti gli Internet Governance Forum locali aderiscono a Internet Governance Forum e partecipano alle conferenze annuali dell'Internet Governance e portano avanti anche delle conferenze a livello locale e regionale. L'obiettivo di questa struttura così organizzata è quello di consentire e agevolare la massima partecipazione possibile a questo Forum che come detto è un Forum che deve essere multipartecipato e quindi poter ascoltare le esigenze che provengono da tutti i livelli di coloro che accedono e partecipano ad Internet.

Per quanto riguarda l'Internet Governance Forum italiano, è coordinato dalla Internet Society Italia in collaborazione con il CNR, il Consiglio Nazionale delle Ricerche, ed è iniziato e partito nel 2008, quindi ha ormai 10 anni di storia, ed è partito con la partecipazione dei rappresentanti del governo, del settore privato e della società civile.

Spunti di riflessione. Quali obiettivi ha Internet Governance Forum? \par
Passiamo adesso alla seconda parte della nostra lezione.

\section{Il ruolo dell'Europa}
L'Unione Europea contribuisce da oltre 15 anni al sostegno e allo sviluppo di Internet e ha svolto e continua a svolgere una funzione essenziale nella vita di tutti. L'Unione Europea non può restare al di fuori di quella che è la governance di Internet e quindi ha operato in maniera organizzata e strutturata in modo da portare le istanze dei paesi europei e dare le proprie indicazioni, le proprie direttive.

Obiettivo ancora una volta è la promozione dell'innovazione e della crescita, la promozione del commercio e della democrazia e dei diritti umani. Nel 2014 in particolare l'Unione Europea ha preso una posizione importante con una comunicazione dedicata al proprio ruolo nella governance di Internet.

\subsection{La comunicazione COM(2014) 72 sul ruolo dell'Europa nella governance di Internet.}
 
L'Unione Europea ha in quest'occasione preso atto di una situazione di carattere generale che è evoluta nel corso del tempo. In particolare ha preso atto di divergenze di opinioni sul futuro di Internet e per questo suggerisce una evoluzione verso piattaforme di scambio per gli stakeholders. In sostanza nell'ambito dei 20 anni nei quali Internet è nato e cresciuto ci sono state delle evoluzioni significative. Accanto alle potenzialità e alle possibilità che si sono sviluppate sono anche nate delle nuove criticità che nei primi anni non erano nemmeno prevedibili.

Le nuove  criticità riguardano in primo luogo divergenze di opinioni sul futuro e su come rafforzare la governance multi stakeholder in modo sostenibile. Ricordiamoci che Internet è di tutti e di nessuno e quindi una governance partecipata al massimo livello è ciò che può consentire uno sviluppo sostenibile e valido per tutti. Al contrario,  negli ultimi anni c'è stata una perdita di fiducia per sorveglianza su larga scala e criminalità informatica. 

Di cyber crime, di criminalità informatica si è sempre parlato fin dai primi anni ma con la partecipazione di tutti al mondo Internet ovviamente anche fenomeni di carattere criminale possono essere maggiormente diffusi e  suscitano un allarme maggiore. Quando parliamo di fenomeni criminali parliamo di fenomeni di vario tipo che vanno dal furto di identità al linguaggio d'odio, alla diffamazione ma anche all'utilizzo della rete da parte di gruppi organizzati criminali che sfruttano la rete per fare i propri traffici.

Altro aspetto che ha suscitato preoccupazione nell'opinione pubblica è la preoccupazione di essere osservati, di essere spiati. Rispetto alla comunicazione del 2014 di cui stiamo parlando vi sono state delle evoluzioni successive. All'epoca si parlava e ci si preoccupava del controllo che i governi potevano mettere in piedi rispetto alle attività svolte dai comuni cittadini su Internet. Negli anni successivi ci si è resi conto che in effetti questo controllo è un controllo che può anche essere effettuato da parte di soggetti privati e quindi mentre alcuni anni fa ci si preoccupava delle rivelazioni di Snowden \footnote{Snowden nel 2013 è stato assunto da un'azienda di tecnologia informatica consulente della NSA, la Booz Allen Hamilton. Nello stesso anno S. ha rivelato migliaia di documenti segretati della NSA ai giornalisti del Guardian, che svelavano l’esistenza di un programma di intelligence di sorveglianza di massa in tutto il mondo, denunciando così violazioni della privacy, della libertà di informazione e reti di spionaggio.} oggi ci si preoccupa di fenomeni come quelli di Cambridge Analytica \footnote{Lo scandalo dei dati Facebook-Cambridge Analytica è stato uno dei maggiori scandali politici avvenuti all'inizio del 2018, quando fu rivelato che Cambridge Analytica aveva raccolto i dati personali di 87 milioni di account Facebook senza il loro consenso e li aveva usati per scopi di propaganda politica.} e quindi l'attenzione dei partecipanti è comunque massima e così anche l'attenzione di coloro che sviluppano e portano avanti la governance di Internet. 

Quali sono i rischi di queste situazioni? 

Innanzitutto un freno per l'innovazione e la crescita delle imprese nel settore di Internet e in secondo luogo la creazione di strutture di governance regionali e nazionali che non dialogano tra di loro e che quindi escono da quello che è il dialogo diffuso della governance auspicata e da questo il rischio ulteriore è quello di una frammentazione della rete con ciò che ne consegue in termini di sicurezza delle comunicazioni e anche di efficacia.

In altri termini la continua perdita di fiducia nei confronti della rete potrebbe disaffezionare, potrebbe allontanare dalla rete coloro che sono interessati a farne un uso di carattere lavorativo mentre lasciare invece spazio all'anarchia o lasciare spazio a dei fenomeni che sono essi stessi in contrasto con quelli che sono i diritti delle persone.

Qual'è la visione comune che porta avanti l'Unione Europea? 

Per la governance di Internet l'Europa punta alla difesa dei diritti fondamentali e dei valori democratici attraverso regole chiare che rispettino tali valori. L'Europa punta alla promozione di una rete unica con le stesse leggi che si applicano in altri settori della vita quotidiana, in sostanza Internet non può e non deve essere un luogo nel quale i diritti non possono essere tutelati.

Infine occorre stimolare un modello realmente multiparticipativo attraverso tutti gli strumenti che la tecnologia consente di utilizzare.

L'idea in sostanza dell'Unione Europea è quella di portare avanti e rafforzare i principi che già sono stati oggetto di discussione, di dibattito e che sono portati avanti dal World Summit for Information Society a livello globale e sotto l'egida delle Nazioni Unite. 

L'acronimo Compact è quello che caratterizza la politica dell'Unione Europea e vuol dire:
\begin{itemize}
    \item Responsabilità \textbf{C}iviche
    \item organizzazione, quindi un sistema \textbf{O}rganizzato
    \item un sistema \textbf{M}ultiparticipativo
    \item un sistema per \textbf{P}romuovere la democrazia e i diritti umani
    \item una rete basata su un'\textbf{A}rchitettura tecnologica
    \item per \textbf{C}onquistare la fiducia degli utenti
    \item per agevolare una governance \textbf{T}rasparente.
\end{itemize}
\par
Lo spazio Compact è quindi uno spazio organizzato nel quale possono operare tutte le componenti della società civile. Compact si fonda sull'agenda di Tunisi del 2005, l'agenda di Tunisi del World Summit for Information Society. Da quel momento in poi c'è stata una proliferazione dei principi della governance internet in diverse sedi, ma nella maggior parte dei casi questi principi sono stati sostenuti da gruppi di interesse limitati in un ambito geografico limitato. Ad avviso dell'Unione Europea per portare avanti una reale intesa su questi principi è necessario sostenerli maggiormente da parte delle istituzioni internazionali. 

Su questa premessa l'Unione Europea ha iniziato a dare una serie di indicazioni che riguardano aspetti politici, tecnici, giuridici nell'ottica della collaborazione e partecipazione globale.

In primo luogo partecipazione democratica attraverso:
\begin{itemize}
    \item discussioni intergovernative in contesti multiparticipativi
    
    \item  azioni di buon governo effettuate con (trasparenza, rendicontazione e inclusione). Il buon governo necessariamente passa attraverso questi tre elementi. La trasparenza è quella che consente di sapere come stanno agendo le istituzioni, la rendicontazione è ciò che permette di seguire il percorso che è stato seguito, l'inclusione consente a tutti gli stakeholders di avere consapevolezza di ciò che sta accadendo.
    
    \item Collaborazione con l'Internet Governance Forum. Come detto, la governance di internet deve essere partecipata al massimo livello. Questo significa che anche a livello di organizzazioni, a livello di istituzioni occorre una collaborazione ed è questo il motivo per cui la stessa Unione Europea ha istituito un punto di riferimento che collabora con l'Internet Governance Forum.
    
    \item collaborazione con ICANN e IANA, i due soggetti che a livello internazionale si occupano dell'assegnazione dei nomi di dominio. Parliamo dell'ICANN, Internet Corporation for Assigned Name and Numbers, e di IANA, Internet Assigned Numbers Authorities. In una diversa lezione affronteremo meglio il discorso su questi due soggetti importanti.
    
    \item occorre tenere in considerazione gli strumenti giuridici internazionali esistenti. Uno dei temi che rende complessa la gestione del mondo internet è il fatto che nei rapporti fra gli Stati vi sono delle relazioni che passano attraverso strumenti giuridici peculiari. È vero che il mondo si sta orientando verso forme di regolamentazione pattizia tra diversi Stati, diverse rispetto agli strumenti utilizzati in passato. Tuttavia, è ancora necessario tener conto degli strumenti esistenti. L’obiettivo dell’Unione Europea non è quello di superarli, ma di includerli e utilizzarli per lo sviluppo della rete.
\end{itemize}


\subsection{Apertura e accessibilità}

Altro tema è l'apertura e accessibilità.

L'architettura deve essere aperta e distribuita senza barriere all'ingresso:
\begin{itemize}
    \item l'ubicazione per l'accesso a internet deve essere irrilevante (modalità e accessibilità), deve quindi essere possibile accedere con diverse modalità e da qualunque posto alle stesse risorse.
    
    \item No alla censura, dal punto di vista tecnico questo significa evitare blocchi, rallentamenti della rete e discriminazione di qualsiasi tipo rispetto all'accesso alla rete
    
    \item Internet exchange points, la creazione di internet exchange points è quello che consente anche tecnicamente di facilitare e velocizzare l'accesso alle risorse della rete. 
\end{itemize}

Occorre partecipazione democratica, apertura e accessibilità,  occorre sviluppare delle linee di azione per una governance cooperativa:

\begin{itemize}
    \item  L'Europa si pone come obiettivo il miglioramento dei risultati dell'internet governance forum, ovvero si pone come obiettivo quello di avere una maggiore influenza.
    
    \item Dialoghi tematici anziché nuovi organismi; il fatto di non creare organismi diversi ma dialoghi tematici consente di sviluppare un colloquio, di sviluppare degli approfondimenti in maniera meno rigida di ciò che accade quando si creano dei nuovi organismi.
    
    \item Ciò però comporta una problematica nella definizione del ruolo degli attori, chi fa che cosa, nella valutazione, nella decisione di come deve essere strutturata la governance di internet. I diversi attori che fanno parte della comunità hanno ruoli, obiettivi e possibilità diverse. Occorre definire quale ruolo hanno i diversi attori e come possono interagire fra di loro arrivando in conclusione comunque a prendere delle decisioni, a fare delle valutazioni, a risolvere i problemi che di volta in volta si possono porre.
    
    \item L'obiettivo però di tutto questo ancora una volta è non calare dall'alto le scelte ma sollecitare quanto più possibile la responsabilizzazione dei singoli. Quindi responsabilizzazione anche attraverso auto-valutazioni e valutazioni fra pari. Obiettivo quindi generale è quello ancora una volta di ampliare il campo, di raccogliere tutto ciò che è possibile e di andare verso una partecipazione massima e una globalizzazione.
\end{itemize}

%22:50
\subsection{Globalizzazione delle decisioni fondamentali (verso il 2016)}

La globalizzazione delle decisioni fondamentali auspicata nella comunicazione del 2014 verso il 2016 prevedeva;

\begin{itemize}
    \item un ampliamento della collaborazione di ICANN con la platea internazionale. Ricordiamo che dal punto di vista tecnico Internet è nato negli Stati Uniti con delle caratteristiche molto precise ed era fortemente legato alla realtà territoriale ed è per questo che gli organismi che si sono per primi occupati dell'organizzazione e gestione della rete sono organismi che sono nati negli Stati Uniti con quel tipo di impostazione. ICANN, in particolare, l'autorità che si occupa dell'assegnazione dei nomi a dominio che sono determinanti per raggiungere le risorse e i soggetti su internet era un'organizzazione statunitense. L'Unione Europea ha spinto molto per ottenere un ampliamento del panel dei partecipanti a questa organizzazione per consentire una globalizzazione effettiva della rete.
    
    \item la globalizzazione di IANA è un obiettivo di sicurezza e stabilità dell'intero sistema dei nomi a dominio. E' importante su internet poter raggiungere tutte le diverse risorse e i soggetti che partecipano da parte di tutti i paesi del mondo e quindi è necessario poterli raggiungere tutti quanti con delle modalità semplici e possibilmente comuni.
    
    \item L'Europa auspicava e auspica anche la definizione di un calendario per la globalizzazione.
\end{itemize}

\subsubsection{Processo multipartecipativo}

Il processo è un processo multipartecipativo nel quale è necessaria

\begin{itemize}
    \item la trasparenza
    \item l'inclusività  e l'equilibrio
    \item occorre una rendicontazione costante delle attività
\end{itemize}

Da allora si è andati verso un quadro della normativa dell'Unione Europea diverso e più approfondito.

\subsubsection{Verso un quadro normativo UE }
I temi:

\begin{itemize}
    \item protezione dei dati personali o data protection
    \item criminalità informatica (cybercrime),
    \item sicurezza della rete (cyber security)
    \item tutela dei minori sia come vittime di adulti che come vittime di cyberbullismo,
    \item linguaggio d'odio. Il linguaggio d'odio negli anni ha preso un grandissimo spazio tanto da arrivare a preoccupare fortemente cittadini e governi.
    \item Fake news è un tema che è entrato all'ordine del giorno a partire dal 2017 in maniera dirompente, tanto che nel 2017 il Collins Dictionary ha definito che fake news era la parola dell'anno.
    \item Libertà di espressione, la libertà di espressione sulla rete è ciò che accompagna tutte le altre politiche di verifica, disciplina e gestione. La rete è uno strumento formidabile di comunicazione e informazione per tutti. Tutte le attività che avvengono sulla rete sono di capacità e libertà di espressione e questa deve essere tutelata al massimo grado.
\end{itemize}

Vale la pena segnalare che nel corso dell'Internet Governance Forum italiano del 2018 i temi appena illustrati sono stati individuati come temi da affrontare e risolvere. Oltre a questi nell'Internet Governance Forum italiano del 2018 si è parlato di etica e quindi come obiettivo di carattere generale, la tutela di interessi e diritti; si è parlato di cittadinanza digitale, segno che Internet e il mondo digitale non sono ancora entrati completamente nell'ambito delle attività dei cittadini e delle persone; si è parlato di innovazione e tecnologia come strumenti utili per lo sviluppo dell'economia e si è parlato di sovranità dei dati e di data sharing, condivisione dei dati. Infine si è parlato dell'importanza di Internet e del lavoro anche attraverso Internet in una sharing economy che è quella che caratterizza il mondo di questi anni.

Rimangono i temi legati al diritto.

\subsubsection{Conflitti fra giurisdizioni e leggi}

Conflitti fra giurisdizioni e leggi sono sempre possibili considerando che c'è:
\begin{itemize}
    \item l'applicazione extra-territoriale di norme nazionali.
    \item Allo stesso tempo spesso vi sono degli accordi contrattuali tra imprese e utenti, accordi che superano e che prescindono le norme nazionali perché spesso riguardano imprese e utenti collocati in paesi diversi. Basta pensare al tema del rapporto fra gli utenti e i fornitori di servizi Internet, Internet Service Provider, ma anche basta pensare all'utilizzo di licenze open source per la diffusione di contenuti o ancora basta pensare al rapporto che si crea tra i singoli utenti e i social network a cui questi sono iscritti.
    \item Obbligazioni extra-contrattuali degli intermediari, gli Internet Service Provider hanno delle caratteristiche particolari, hanno degli obblighi contrattuali ma hanno anche degli obblighi extra-contrattuali nei confronti dei diversi soggetti che accedono attraverso di loro alle risorse.
    \item Individuare la giurisdizione competente e la legge applicabile in caso di discussioni o in caso di problemi può non essere semplice. Non è detto che le indicazioni che sono fornite sui disclaimer pubblicati dai diversi fornitori di servizi siano sufficienti per radicare la competenza e la giurisdizione in uno stato piuttosto che un altro quando il rapporto è legato a soggetti con poteri e possibilità diverse, appunto fornitori di servizi multinazionali, individui soggetti singoli, soggetti privati.
    \item La casistica è variegata in ogni caso. Il punto è che al pari di tutte le altre attività trasfrontaliere internet pone una serie di sfide relative all'applicazione delle leggi. Non sempre si tratta di sfide che riguardano internet di per sé, si può trattare di sfide che riguardano determinati tipi di contenuti indipendentemente dal fatto che siano su internet con la peculiarità che su internet avviene ad esempio una transazione, avviene un rapporto.\par
\end{itemize}

Le decisioni giuridiche che sono state prese nel corso degli anni in diversi tribunali nell'ambito di diversi paesi europei e non solo, a volte hanno portato a conclusioni diverse e divergenti fra di loro. Il risultato di una situazione di questo tipo è da un lato il cosiddetto forum shopping e quindi l'interesse per determinati soggetti, compagnie, società a posizionarsi all'interno di paesi nei quali sia più semplice gestire determinate problematiche. Dall'altro queste decisioni diverse hanno portato anche a conclusioni diverse rispetto ai cittadini che tale decisione avevano promosso. Questo è un tema, come anticipato prima, che può portare a una grande incertezza rispetto alla rete e questa incertezza può essere fonte di difficoltà e di problemi. Questa difficoltà comunque è la ragione per cui molte attività sono oggi regolamentate da rapporti di carattere contrattuale fra privati, anche se bisogna dire che il potere contrattuale e quindi decisionale dei singoli cittadini certamente non è mai lo stesso di chi fornisce i servizi. Occorre in questo contesto sviluppare quindi delle nuove sinergie, delle nuove possibilità di accesso.

\subsubsection{Obiettivi dell'Unione Europea per Internet}

Quali sono gli obiettivi dell'Unione Europea per Internet?

\begin{itemize}
    \item Una rete di reti unica, aperta, libera e non frammentata
    \item soggetta alle stesse leggi e alle stesse norme che si applicano alla vita quotidiana
    \item con il rispetto dei diritti umani, delle libertà fondamentali e dei valori democratici
    \item il rispetto delle diversità linguistiche e culturali e con attenzione alle persone vulnerabili.
\end{itemize}

Come raggiungere questi obiettivi? 

Che cosa fare per renderli concreti e non soltanto tenerli sulla carta? 

Innanzitutto con la verifica di ciò che effettivamente accade ed è per questa ragione che l'Unione Europea ha istituito un osservatorio.

\subsubsection{Osservatorio europeo}
L'osservatorio europeo è il Global Internet Policy Observatory (GIPO) come risorsa per lo sviluppo di politiche di governance per la comunità internazionale (http://giponet.org/en). Potete andare a verificare e controllare quello di cui il GIPO parla. Quando si tratta di attività su internet vale sempre la pena di andare a verificare ciò che dicono i siti, quello che contengono, perché le piattaforme multi-stakeholder, attraverso le quali realizzare gli obiettivi del WSIS, dell'Internet Governance Forum e della stessa Unione Europea passano attraverso una partecipazione alle piattaforme su internet, che è la sede più propria.
Oltre al GIPO abbiamo anche l'European Dialogue on Internet Governance (EuroDIG) che è una piattaforma informale per gli stakeholders (https://www.eurodig.org/index.ph?id=74). La European Dialogue on Internet Governance è una piattaforma sulla quale è possibile dare suggerimenti, è possibile raccontare esperienze e attraverso questa piattaforma l'Unione Europea partecipa ai forum dell'Internet Governance Forum.

\subsubsection{GIPO - chi e cosa}
Vediamo in poche parole come è strutturato il GIPO.

\begin{itemize}
    \item composto di 12 membri indipendenti, di cui 8 indicati dall'Unione Europea.
    \item Sviluppa strumenti per aiutare la comunità globale a sviluppare a loro volta processi decisionali su internet
    \item raccoglie, analizza e condivide informazioni. La raccolta, la condivisione e l'analisi delle informazioni è il punto di partenza attraverso il quale rendersi conto di come la rete funziona, come si sta sviluppando e quali sono le problematiche principali da affrontare per il futuro.
\end{itemize}

I nostri spunti di riflessione. Quali sono gli ambiti principali di armonizzazione normativa auspicata dalla Unione Europea? Che cos'è GIPO?


Adesso rivediamo gli spunti di riflessione di questa lezione. Quali obiettivi ha Internet Governance Forum? Quali sono gli ambiti principali di armonizzazione normativa auspicata dall'Unione Europea? Che cos'è GIPO?