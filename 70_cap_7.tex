\chapter{Lezione 7 - Internet e diritti umani - I parte}

Gli argomenti della lezione di oggi sono:
\begin{itemize}
    \item i diritti umani fondamentali  
    \item Internet e rivoluzione della comunicazione 
    \item Internet e violazione di diritti
\end{itemize}

\section{I diritti umani fondamentali}
I diritti umani fondamentali sono stati individuati all'indomani della Seconda Guerra Mondiale in diversi trattati e atti internazionali e nella Costituzione italiana. 
\begin{itemize}
    \item La Dichiarazione universale dei diritti umani (ONU 1948) 
    \item la Convenzione europea dei diritti umani (CEDU 1950) 
    \item la Carta dei diritti fondamentali dell'Unione europea del (UE 2000)
    \item la Costituzione italiana del (1948)
\end{itemize}

Si tratta di atti che rivestono un'importanza fondamentale per quelli che sono i principi generali a sostegno del nostro ordinamento. \par

\subsection{Libertà di espressione}
Libertà di espressione sancita dall'Art. 19 della Dichiarazione ONU, dall'Art. 10 CEDU, dall'Art. 11 della Carta UE e dell'Art. 21 della Costituzione italiana. \par
Fra i diritti fondamentali c'è ne uno in particolare che ci interessa per quello che riguarda Internet oggi ed è il diritto alla libertà di espressione e di comunicazione. Questo diritto è sancito dall'Art. 19 della Dichiarazione dell'ONU che dice che ogni individuo ha diritto alla libertà di opinione e di espressione incluso il diritto di non essere molestato per la propria opinione e quello di cercare, ricevere e diffondere informazioni e idee attraverso ogni mezzo e senza riguarda ad alcuna frontiera.\par
Negli stessi termini si esprime la Convenzione europea dei diritti dell'uomo che ribadisce il diritto alla libertà di espressione di ognuno, diritto che comprende la libertà di opinione e la libertà di ricevere e comunicare informazioni e idee senza ingerenze da parte delle autorità pubbliche e senza limiti di frontiera. \par
La Convenzione dei diritti dell'uomo però aggiunge che l'esercizio di queste libertà poiché comporta doveri e responsabilità può essere sottoposto a formalità, condizioni, restrizioni e sanzioni che sono previste dalla legge e che costituiscono misure necessarie in una società democratica alla sicurezza nazionale, all'integrità territoriale, alla pubblica sicurezza, alla difesa dell'ordine, la prevenzione dei reati, alla protezione della salute e della morale, alla protezione della reputazione dei diritti altrui per impedire la divulgazione di informazioni riservate o per garantire l'autorità e l'imparzialità del potere giudiziario.
\par
Ecco quindi che la Convenzione europea introduce un contemperamento fra il diritto di libera espressione e altri diritti. \par
La Carta dei diritti dell'Unione europea ribadisce il diritto di ogni persona alla libertà di espressione, ribadendo che questa libertà include la libertà di opinione ma anche la libertà di ricevere informazioni. \par
Negli stessi termini l'articolo 21 della Costituzione italiana prevede tra le altre cose che soltanto in casi eccezionali sia possibile disporre il sequestro di stampati.\par
Ovviamente questi diritti sono stati individuati in un momento storico nel quale il principale mezzo di espressione era la carta stampata e quindi non si poteva all'epoca tenere in conto di quelle che sono le caratteristiche che oggi abbiamo con Internet.

Libertà di opinione e di ricevere e comunicare informazioni e idee senza ingerenza di autorità pubbliche e senza limiti di frontiera. \par

Abbiamo visto che la libertà di espressione ha due aspetti. Il primo aspetto riguarda il diritto ad essere informati, il diritto a ricevere informazioni. Questo diritto è strettamente connesso con il diritto di cronaca, diritto dovere di cronaca, anzi quello che hanno i giornalisti nel momento in cui informano il pubblico di notizie che abbiano interesse pubblico e che per questo nell'esercizio di questo diritto di cronaca non possono essere limitati in alcun modo, non possono essere censurati. L'altro aspetto è il diritto di esprimere liberamente le proprie opinioni attraverso qualunque mezzo e questo aspetto del diritto della libertà di espressione è un aspetto che interessa particolarmente quando affrontiamo Internet.\par
Internet infatti è uno strumento che ha cambiato completamente il modo di comunicare perché consente a qualunque persona di esprimere la propria opinione senza la necessità di avvalersi di strumenti. \par
%05:48
Il diritto di esprimere liberamente il proprio pensiero deve essere contemperato con altri diritti: (la privacy, la reputazione, la pubblica sicurezza, la prevenzione dei reati, la tutela dei segreti).\par
La convenzione europea dei diritti dell'uomo ricorda che vi sono anche altri diritti fondamentali che devono essere rispettati nello stesso modo rispetto alla di libertà di espressione. Questo significa che anche nel momento in cui chiunque decida di comunicare le proprie opinioni attraverso internet dovrà comunque tener conto del fatto che vi sono altri tipi di diritti che si contrappongono alla libertà di espressione del proprio pensiero e che vanno comunque tenuti in considerazione. Immaginiamo ad esempio il fatto di raccontare al pubblico delle notizie riservate che riguardano terzi, ad esempio un segreto di stato, immaginiamo di utilizzare delle espressioni nel comunicare con il pubblico attraverso internet che sono diffamatorie perché sono lesive dell'onore e della reputazione di altri.\par
Spunto di riflessione di questa parte della lezione è chiedersi con quali altri diritti può interferire la libertà di espressione e comunicazione.

\subsection{internet e della rivoluzione della comunicazione}
Dopo 65 anni dalla Dichiarazione Universale dei Diritti Umani esiste un nuovo contesto. Internet rappresenta uno dei grandi strumenti di cambiamento. Come abbiamo detto nel momento in cui sono state promulgate le carte dei diritti fondamentali, il mondo era un mondo nel quale l'espressione delle proprie idee poteva essere veicolata attraverso strumenti che nella gran parte dei casi erano strumenti sottoposti ad una gestione. Immagino i giornali in particolare, ma lo stesso discorso vale anche con riferimento allo strumento televisivo. Si tratta di strumenti di comunicazione di massa a cui non tutti possono accedere per comunicare le proprie opinioni. Nel momento in cui si accede a questi strumenti di comunicazione di massa per comunicare qualche cosa, esiste una forma di valutazione preventiva effettuata da chi, l'editore, il giornalista, pubblica l'informazione e che quindi necessariamente filtra quelle che sono le informazioni che passano al pubblico.\par
Con l'utilizzo di internet è cambiato tutto. Internet è accessibile da parte di chiunque o meglio è in concreto utilizzato da una grandissima parte della popolazione mondiale e non ha alcun tipo di valutazione preventiva rispetto al contenuto, all'informazione, alla comunicazione che viene inserita. Questo significa che dall'altra parte chi riceve, chi si informa attraverso internet non ha alcuna possibilità di una valutazione preventiva rispetto alla qualità dell'informazione fornita, rispetto alla veridicità dell'informazione fornita, rispetto al fatto che l'informazione fornita sia un'informazione che non lede alcun altro diritto. Questo è un aspetto che va valutato con estrema attenzione.\par 
Più di un terzo della popolazione mondiale è connessa a internet, si parla di più di 600 milioni di famiglie (informazione da International Telecomunication Union, un rapporto del 2012 che probabilmente oggi è ancora diverso).\par
Il fatto che un numero così ingente di persone sia collegato ad internet è un segnale che lo strumento oggi riveste un ruolo importante per la comunicazione di massa. Quello che emerge dagli ulteriori approfondimenti che sono stati fatti è che nella maggior parte dei casi le persone utilizzano internet come strumento di carattere personale, per informazioni personali, per parlare della propria vita e soltanto in parte lo strumento è utilizzato per ragioni di carattere professionale o lavorativo.\par
In particolare le tecnologie dell'information and communication sono un fattore di crescita economica, di esportazione di investimenti in bene e servizi, specie nei paesi in via di sviluppo. Nei paesi nei quali le comunicazioni sono sempre state estremamente difficili, l'avvento di internet ha comportato dei cambiamenti notevoli anche nella gestione delle attività commerciali.\par
Esistono paesi del terzo mondo, in particolare ho in mente l'Africa, nei quali mancano moltissime cose ma esiste internet e questo consente anche ai piccoli produttori di far conoscere i propri prodotti ed eventualmente di mettere in moto un meccanismo di esportazione che può essere un fattore di crescita economica notevole. \par
Internet, anche attraverso l'utilizzo di email e social networks, è uno degli strumenti più utilizzati per comunicare. L'utilizzo dei social networks si è diffuso sostanzialmente negli ultimi 10 anni ed ha cambiato ancora di più la modalità di gestire la comunicazione di massa. Nei social network in particolare si proiettano gli interessi, le comunicazioni, i desideri della popolazione molto spesso su questioni di carattere squisitamente personale. Internet è  anche uno strumento di esposizione della persona che ha radicalmente stravolto le modalità di relazionarsi fra le persone come precedentemente la conoscevamo. I giovani che oggi si affacciano a internet sono nati all'interno del mondo digitale e questo non può non essere considerato anche in prospettiva futura.\par 
Spunto di riflessione di questa parte della lezione. Quale ruolo ha internet nel mondo moderno? Questa domanda è una domanda a cui occorre rispondere tenendo conto di tutti gli aspetti di cui si è parlato. \par

\subsection{internet e violazione di diritti}

Ora parliamo di internet e della possibilità che il suo utilizzo comporti la violazione di altri diritti. \par
Internet può trasformarsi in uno strumento di aggressione di altri diritti. Vi sono diverse situazioni concrete che si sono verificate negli ultimi anni e che ci fanno comprendere come internet possa essere uno strumento di aggressione di altri diritti, possa essere uno strumento che mette in difficoltà, che crea problemi e questo va gestito.\par
Molto spesso quando si parla di introdurre delle limitazioni, delle regole in internet, il popolo degli internauti si solleva contestando che vi possa essere un bavaglio al diritto di informazione, che vi possa essere un bavaglio alla libertà di espressione. Questo tipo di valutazioni, che sono valutazioni correnti, dovrebbero essere fatte con estrema attenzione anche perché spesso nel momento in cui si protesta rispetto alla possibilità di una limitazione del diritto di informazione, non ci si rende conto che chiunque può essere leso nei propri diritti proprio da questa stessa libertà.\par
Vi sono alcuni casi concreti di cui adesso parleremo che inducono certamente a riflettere.\par 
\subsubsection{Il caso Data Gate}
Nel caso Data Gate, così chiamato, la National Security Agency degli Stati Uniti ha raccolto un'ingente quantità di intercettazioni e dati geolocalizzati in tutto il mondo. Questo episodio del cosiddetto Data Gate è arrivato sulle cronache di tutti i giornali, di tutti i media nello scorso autunno. È capitato che la National Security Agency abbia raccolto un'ingentissima quantità di dati, di informazioni e intercettato comunicazioni relative alle attività più diverse, ma anche relative ad alcuni capi di Stato.\par
Su richiesta di informazioni, la NSA ha evidenziato che questa attività era stata svolta per la lotta contro il terrorismo. Quindi le ragioni per questa raccolta di dati erano delle ragioni estremamente rilevanti di sicurezza pubblica. Non c'è dubbio che la sicurezza pubblica è qualche cosa che deve essere tutelato alla pari della libertà di informazione. Tuttavia il problema che si pone è quali sono i limiti che questo modo di raccogliere informazioni debba avere.\par
L'ONU è intervenuta sul tema a tutela della privacy, cioè a tutela della riservatezza delle comunicazioni. L'impulso all'intervento è stato dato da alcuni Stati, in particolare il Brasile, che non hanno gradito un'interferenza da parte degli Stati Uniti nei propri affari di Stato.\par
Il tema importante in questo caso va oltre quelle che sono le relazioni tra singole persone, ma giunge proprio alla relazione fra gli Stati, relazione che per il momento è una delle relazioni più importanti per mantenere l'autonomia e la libertà di ogni singolo Stato. \par
\textbf{La raccolta di dati ha un aspetto legato al divieto di interferenze arbitrarie nella vita privata e nella corrispondenza che è un altro diritto fondamentale sancito dall'articolo 12 della Carta ONU, dall'articolo 8 della Convenzione Europea dei Diritti dell'Uomo, dagli articoli 7 e 8 della Carta dei Diritti dell'Unione Europea e per dall'articolo 15 della Costituzione Italiana.}\par
Abbiamo detto all'inizio di questa lezione che il diritto alla libertà di espressione, è uno dei diritti fondamentali enunciati dalle carte internazionali e dalla Costituzione Italiana. Ve ne sono altri che essendo diritti fondamentali sono alla pari della libertà di espressione. \par
Tra questi vi è proprio il diritto alla segretezza della corrispondenza e, con riferimento alle più recenti carte internazionali, particolare alla carta di Nizza, il diritto a tutelare la propria vita privata, la cosiddetta privacy, rispetto a qualunque tipo di interferenza esterna.\par
Nel caso dei mezzi di comunicazione di massa, la tutela della propria riservatezza rispetto a ogni interferenza esterna passa attraverso la possibilità di chiedere una limitazione delle informazioni che riguardano la persona stessa.\par
\subsubsection{il furto di identità}
L'utilizzo di dati personali può comportare la commissione di alcuni fatti che costituiscono reato per quanto riguarda la legge italiana. Innanzitutto il furto di identità che è colui si finge un'altra persona per ottenere vantaggi, ad esempio con l'accesso a conti correnti bancari, oppure per danneggiare la reputazione altrui, ad esempio con l'apertura di account di posta elettronica e con l'uso dei dati, ad esempio, su siti pornografici.\par
Questi fatti costituiscono reato. Il furto di identità costituisce reato nell'ordinamento italiano. Dal punto di vista concreto il furto di identità può comportare un depauperamento patrimoniale delle persone. Mi riferisco a quei casi nei quali vengono rubate le informazioni relative ad esempio all'accesso ai conti correnti bancari e vengono carpite queste informazioni attraverso diverse tecniche, chiamate fishing o in altri modi, tecniche le quali inducono in errore la persona che è titolare di quei dati, sono tecniche che fanno sì che la persona ingannata comunichi le proprie credenziali e poi queste credenziali vengono utilizzate da chi le ha raccolte per accedere al conto corrente bancario e svuotarlo.\par
Un altro esempio concreto estremamente diffuso è quello nel quale si utilizzano i dati, le informazioni relative ad una persona per danneggiarne la reputazione. Si tratta di casi che si sono verificati di frequente quando c'è un motivo di rabbia, di odio nei confronti di una persona, si utilizzano i suoi dati personali per aprire un account di posta elettronica e attraverso questo si fanno comunicazioni che mettono in cattiva luce la persona di cui i dati sono stati sottratti.\par
Uno dei reati che si possono contestare con riferimento al furto di identità è:


\begin{itemize}
    \item \textbf{reato di sostituzione di persona} prevista e punita dall'articolo 494 del nostro codice penale. 
    \item \textbf{reato di truffa} che è prevista e punita dagli articoli 640 e seguenti del codice penale;
    \item \textbf{reato di trattamento illecito di dati personali} previsto e punito dall'articolo 167 del Decreto Legislativo 196 del 2003 detto anche codice della privacy
\end{itemize}

\par

In che cosa consistono questi reati?\par
La sostituzione di persona è il fatto di chi assuma in qualunque modo l'identità di un altro per ottenere un vantaggio o per danneggiare qualcun altro. Si tratta di un reato che è previsto dal codice penale emanato nel 1930 e quindi stiamo parlando di un reato che teneva in conto delle modalità di realizzazione ben diverse da quelle che si utilizzano oggi su internet. Il caso poteva essere quello ad esempio del farsi rilasciare una carta di identità a nome di altri o anche semplicemente il fatto di farsi credere una persona diversa da quello che si è effettivamente con riferimento a titoli di qualunque genere qualifiche professionali, proprietà e quant'altro con l'obiettivo o di ottenere dei vantaggi oppure di danneggiare qualcun altro. L'obiettivo in questo senso non è rilevante, ciò che conta è il fatto di essersi sostituiti ad un'altra persona. Nel mondo di internet questo reato è assolutamente ipotizzabile anche se viene realizzato con modalità del tutto diverse. Come abbiamo detto il fatto di sostituirsi ad un altro su internet può avvenire attraverso l'utilizzo di credenziali altrui oppure attraverso l'apertura di account di qualunque genere a nome di altri utilizzando i dati di altri. Anche su internet questo reato può essere compiuto a proprio vantaggio per ottenere dei vantaggi oppure per danneggiare qualcun altro. A questo proposito occorre fare attenzione perché il fatto di danneggiare altri non è collegato necessariamente a un danno di carattere patrimoniale, può anche essere collegato ad un danno di carattere morale, ad una lesione di carattere morale.\par
Per quanto riguarda la truffa, il reato base è previsto e punito dall'articolo 640 del codice penale. Si tratta del fatto di chi utilizzando artifici e raggiri induce una persona in errore e lo induce a fare delle attività di disposizione di carattere patrimoniale a vantaggio proprio di chi commette il fatto o a vantaggio di un terzo. In questo caso quindi la truffa ha un collegamento diretto con un vantaggio di carattere patrimoniale. Gli artifici e i raggiri possono essere di qualunque genere purché abbiano come effetto di indurre in errore la persona che viene truffata.\par
Con riferimento ad internet il caso tipico è quello dell'utilizzo di credenziali altrui, anzi la raccolta di credenziali altrui attraverso degli artifici quali ad esempio il mandare un messaggio di posta elettronica apparentemente riconducibili all'Istituto Bancario presso il quale si ha il conto corrente. Questo artificio, questo inganno, l'invio della mail che apparentemente è riconducibile all'Istituto Bancario è un artificio che ha lo scopo di ingannare la persona a cui viene inviato e far sì che comunichi le proprie credenziali di accesso al conto corrente e questo consente di svuotare il conto corrente, quindi di sottrarre denaro al soggetto a cui fanno riferimento le credenziali.\par
Oltre a questa fattispecie di truffa il legislatore ne ha introdotta un'altra che è la cosiddetta truffa informatica prevista dall'articolo 640 ter del codice penale e introdotta negli anni 90 che prevede come reato il fatto di trarre in inganno, in questo caso il termine non è corretto, il computer. Nel caso è evidente che non si può ingannare il computer ma che l'utilizzo di questa formula significa che vengono utilizzati dei meccanismi per bypassare quelle che sono ad esempio le misure di sicurezza che proteggono un computer e ottenere in questo modo l'accesso al sistema.\par
L'ultimo reato che abbiamo introdotto è il trattamento illecito di dati personali previsto e punito dall'articolo 167 del decreto legislativo 196/2003 detto anche codice privacy. In questo caso si fa riferimento all'utilizzo di dati personali altrui in violazione di una o più norme previste dal codice stesso della privacy.\par
Il codice della privacy prevede in linea generale che ogni persona abbia diritto alla tutela dei dati che lo riguardano rispetto al trattamento da parte di altri. Come trattamento si intende qualunque tipo di attività venga svolta con questi dati a partire dalla raccolta, all'utilizzo, alla registrazione all'interno di una banca dati oppure all'utilizzo per qualunque tipo di attività. Il codice della privacy prevede che il trattamento di dati personali di terzi sia subordinato al consenso dell'interessato, cioè al consenso della persona a cui i dati fanno riferimento e che i dati raccolti siano utilizzati per scopi determinati e leciti. \par
Vi sono poi una serie di norme nel codice della privacy che disciplinano in maniera estremamente precisa e concreta le modalità di trattamento corretto dei dati personali. Nel caso in cui le modalità di trattamento corretto dei dati personali vengano violate e in particolare alcune delle norme previste dal codice non vengano rispettate, il codice prevede un reato che è quello del trattamento illecito. C'è però una condizione perché questo reato si possa effettivamente configurare e la condizione è che occorre che vi sia un trattamento effettuato a danno della persona a cui i dati fanno riferimento oppure che questo trattamento consista nella diffusione o nella comunicazione.\par
Questo secondo aspetto, cioè il trattamento illecito effettuato mediante la comunicazione e la diffusione, è quello che più interessa con riferimento ad internet. Diffusione infatti di informazioni significa comunicarle ad un numero indeterminato di persone. La pubblicazione di dati su internet che non sia una pubblicazione ristretta ad un gruppo chiuso e quindi accessibile da parte di soggetti non iscritti o che non fanno parte di questo gruppo, è una diffusione. Per fare un esempio concreto, la pubblicazione di dati, il postare dati su social network quali Facebook o altri, comporta una diffusione del dato e quindi se questa diffusione viene effettuata in violazione di quelle che sono le norme del codice della privacy, si ha trattamento illecito sanzionabile penalmente.\par
Da ricordare, con riferimento ai tre reati di cui abbiamo parlato, che le sanzioni per i reati sono sanzioni di carattere penale espressamente previste dall'ordinamento e quindi nel momento in cui l'autorità giudiziaria viene portata a conoscenza o vengono denunciati questi reati, l'autorità giudiziaria ha l'obbligo di procedere e quindi l'utilizzo dei dati su internet deve essere, sotto questo profilo, valutato con estrema attenzione. \par
Spunto di riflessione di questa parte della lezione riguarda quali sono i reati che possono essere commessi su internet. \par
Facciamo adesso un breve riepilogo degli argomenti di questa lezione.\par 
Abbiamo parlato dei diritti fondamentali, dei diritti umani fondamentali così come individuati dalle carte fondamentali internazionali e dalla Costituzione italiana. Tra questi diritti fondamentali vi è il diritto alla libertà di espressione e comunicazione che è un diritto che è sancito per quanto riguarda il nostro ordinamento dalla Costituzione italiana all'articolo 21, per quanto riguarda le carte internazionali è sancito dalla Dichiarazione universale dei diritti umani del 1948, dalla Convenzione europea dei diritti dell'uomo del 1950 e dalla carta dei diritti fondamentali dell'Unione europea del 2000.\par
Tutte queste carte internazionali e la Costituzione italiana affermano che è diritto fondamentale di ognuno quello di comunicare liberamente le proprie opinioni e quello di informarsi. Questo diritto è un diritto che va sempre con gli altri diritti fondamentali che sono sanciti ugualmente dalle carte dei diritti fondamentali e dalla Costituzione italiana. Questo significa che nell'esercitare il diritto alla libertà di espressione occorre tener conto degli altri diritti.\par
Il diritto alla libertà di espressione è un diritto che ha due aspetti, un aspetto passivo e un aspetto attivo. L'aspetto passivo è il diritto ad essere informati, di essere tenuti al corrente di fatti di rilevanza pubblica. Dall'altra parte c'è il diritto di esprimersi, il diritto di comunicare ad altri le proprie opinioni che è un diritto che ha acquisito una rilevanza del tutto nuova con l'avvento di internet. Internet infatti è uno strumento che consente a chiunque di esprimere la propria opinione di farla conoscere a chiunque altro, teoricamente in tutto il mondo.\par 
La comunicazione di informazioni, di notizie attraverso internet ha una caratteristica particolare che è quella di non avere limiti di tempo e di spazio, cioè nel senso che le informazioni pubblicate su internet risiedono senza nessun tipo di limitazione. Questo anche quando l'informazione stessa sia cancellata dal sito, dal luogo dove è pubblicata. Esiste infatti la possibilità di richiamare le informazioni in altro modo ed esiste la possibilità che esse vengano ridiffuse da terzi e in questo modo sostanzialmente non ci sono limiti alla comunicazione.\par 
Questo è un'innovazione che ha cambiato completamente i rapporti. Internet però può essere anche uno strumento proprio per le caratteristiche che ha di consentire la comunicazione e diffusione di informazioni senza limiti di tempo e di spazio. Internet può anche essere uno strumento che viola altri diritti. La comunicazione di informazioni senza alcuna autocensura, chiamiamola in questo modo, può portare alla violazione di altri diritti. In particolare possono essere commessi alcuni reati quali il furto di identità, nel caso in cui i dati di una persona vengono utilizzati su internet per parlarne male, per danneggiarla o per avvantaggiarsene. È possibile che vi sia la sostituzione di persona che è proprio il fatto di sostituirsi ad altri. È possibile che vi sia un trattamento illecito di dati personali quando si violano alcune specifiche norme del codice della privacy. Occorre quindi verificare con attenzione ciò che si fa.