\chapter{Lezione 8 -  Internet e diritti umani - II parte}

In questa lezione continueremo a parlare di Internet e diritti umani. Gli argomenti della lezione di oggi:
\begin{itemize}
    \item Internet e violazione di diritti
    \item la tutela dell'accesso a Internet
    \item la domanda se Internet sia o no un diritto fondamentale
\end{itemize}

\section{Internet e violazione di diritti}

Internet può trasformarsi in uno strumento di aggressione di altri diritti. Nella lezione precedente abbiamo iniziato a parlare del fatto che Internet è uno strumento di grande potenza, di grande utilità, ma può anche trasformarsi in uno strumento di violazione di altri diritti, che possono essere anche diritti fondamentali.\par

\subsection{Diffamazione}

La diffamazione è il comportamento che si commette quando si offende l'onore o la reputazione di qualcuno, in particolare in rete. Si tratta di un reato che è sanzionato dall'articolo 595 del codice penale e che ha come caratteristica quello di poter essere commesso in qualunque situazione. È sufficiente offendere l'onore o la reputazione di taluno parlando o scrivendo ad un numero indeterminato di persone in assenza del diretto interessato. Se questo reato viene commesso attraverso uno strumento di comunicazione di massa, nell'idea del legislatore del codice penale del 1930 erano i giornali, il reato è aggravato.

Internet indubbiamente è uno strumento di comunicazione di massa, anzi possiamo dire che è lo strumento principe di comunicazione di massa, perché è uno strumento che consente di arrivare ad un numero immenso di persone senza particolari difficoltà. Il fatto di offendere l'onore o la reputazione di taluno può a volte essere giustificato. Si tratta di quelle situazioni, in particolare che riguardano le notizie, le informazioni di carattere giornalistico, nelle quali l'offesa all'onore o la reputazione di taluno avviene raccontando fatti effettivamente accaduti, situazioni effettivamente esistenti che oggettivamente hanno come conseguenza quella di offendere l'onore o la reputazione.

\subsubsection{Diritto di cronaca}
In questi casi si parla di diritto di cronaca. Il diritto di cronaca riguarda specificamente i giornalisti e deve rispondere a dei principi, in particolare verità, continenza formale e continenza sostanziale.

Che cosa significa? Significa che fa parte della libertà di informazione, il fatto di raccontare delle notizie vere di interesse pubblico però anche facendo ciò che è necessario è che il modo di raccontare queste notizie corrisponda a dei criteri ben precisi che sono, oltre alla verità, anche una continenza formale, cioè una modalità espressiva, un linguaggio utilizzato che di per sé non ecceda, non trascenda rispetto a quelli che sono gli obiettivi di comunicazione, e una continenza sostanziale, cioè l'indicare delle notizie, delle informazioni, effettivamente necessarie per trasmettere la notizia. E quindi non condire le informazioni che si danno con delle informazioni non necessarie, con dei particolari non necessari, i quali, anche se potrebbero aumentare l'interesse della notizia, in realtà non sono essenziali per la trasmissione dell'informazione.

Vi sono dei termini, delle forme di espressione, che non aggiungono nulla, in particolare sono gli insulti e le oscenità che sono di scarsa utilità sociale e sono del tutto inutili per la verità. Già dagli anni 40 negli Stati Uniti si è ritenuto che ci sono alcune categorie di discorso ben definite e limitate la cui prevenzione e punizione non ha mai sollevato alcun tipo di problema nel rapporto fra diritto all'informazione e divieto di utilizzo di certe frasi. Si tratta proprio delle volgarità e delle oscenità, delle calunnie, degli insulti, di tutte quelle parole di scontro che per la loro stessa espressione comportano un danno, tendono a provocare la violazione dell'ordine pubblico e non hanno parte in alcun modo nell'esposizione di idee e sono di scarsa utilità sociale ai fini della verità che qualsiasi beneficio che ne potrebbe derivare è ampiamente superato da un interesse sociale più grande nell'ordine e nella moralità.

Queste parole erano state affermate dalla Corte Suprema degli Stati Uniti in una nota sentenza degli anni 40, ma i principi sono principi che sono assolutamente condivisibili e applicabili anche nel nostro ordinamento. Il fatto di avere come linea guida nelle modalità di espressione i principi del diritto di cronaca di cui si è parlato, cioè verità della notizia, continenza sostanziale e continenza formale consente di evitare di trascendere nelle comunicazioni che vengono fatte all'esterno.

C'è da dire che la pubblicazione di informazioni su internet ha degli effetti moltiplicatori del danno e quindi per questa ragione la cautela deve essere ancora maggiore. In particolare la pubblicazione di una notizia su internet \textbf{resta senza limiti di tempo}, non è sostanzialmente possibile cancellare le informazioni che vengono pubblicate su internet se non a prezzo di grandi difficoltà. Questo perché le informazioni pubblicate oltre ad essere contenute sul sito sul quale vengono pubblicate possono essere ritrasmesse e ripubblicate attraverso tutti i collegamenti che a quel sito possono essere fatti. E anche quando le informazioni vengono cancellate dal sito dove originariamente erano state pubblicate può accadere che siano in ogni caso reperibili su altre pagine, su altri siti.

Questo comporta \textbf{un'amplificazione del danno}, questo aspetto specifico che riguarda internet ha come conseguenza che la lesione dell'onore della reputazione eventualmente fatta attraverso internet può danneggiare l'interessato molto più a lungo di quanto non avvenga attraverso altri strumenti quali la carta stampata.
Un giornale viene pubblicato e poi viene dimenticato. Come abbiamo detto la pubblicazione di un'informazione su internet non ha questa caratteristica.

Un altro tema di particolare complessità collegato alla pubblicazione su internet è il \textbf{sistema della indicizzazione dei contenuti sui motori di ricerca}. Questo è un'altra specificità dello strumento che fa sì che anche una notizia vecchia e ormai superata nel tempo possa ritornare a galla per così dire attraverso le modalità di indicizzazione dei siti. Ed è accaduto in diverse situazioni che ad esempio la notizia di un procedimento penale a carico di una persona che ha come effetto quello di portare una lesione dell'onore della reputazione ancorché su fatti reali, è successo che notizie di questo genere abbiano continuato ad essere reperibili in modo semplice anche se nel frattempo il procedimento si era concluso con una assoluzione e quindi quella notizia relativa all'accusa in corso, non era più una notizia attuale ma anzi era una notizia superata da una notizia diversa.

Quindi ecco che la ragione per cui dover verificare con attenzione ciò che si pubblica su internet diventa essenziale.

\subsubsection{Responsabilità dei provider, dei blogger}

C'è da chiedersi a questo punto se vi sia una responsabilità dei provider, dei blogger, ovvero di tutti coloro che intervengono nella pubblicazione della notizia. L'autore, chi pubblica, non è l'unico soggetto che opera per la pubblicazione delle informazioni. La caratteristica di internet è che chiunque può trasmettere le informazioni ma per far questo comunque occorre lo strumento che è lo strumento gestito da provider, blogger e quant'altro. Quindi la domanda che ha suscita sempre un dibattito estremamente rilevante, i soggetti che concorrono nella trasmissione delle informazioni concorrono anche eventualmente nel reato di diffamazione unitamente a coloro che pubblicano?

C'è da dire che non esiste una normativa specifica su questo tema, esiste soltanto la \textbf{legge 70/2003} che è una legge che in recepimento di direttiva comunitaria prevede una responsabilità degli internet service provider soltanto nella misura in cui partecipano attivamente alla pubblicazione di determinati contenuti. Non c'è però alcun obbligo di controllo dei gestori, dei blogger, degli internet providers, assimilabile all'obbligo di controllo che invece la legge italiana prevede con riferimento alla carta stampata.

La normativa sull'editoria infatti prevede uno specifico controllo del direttore responsabile del giornale sugli articoli che vengono pubblicati. Questo non esiste su internet e da questa differenza di normativa sono nati una serie di dibattiti dottrinari e in ambito giurisprudenziale proprio per valutare in che termini e in che misura si possa invocare una responsabilità dei gestori dei blog e degli internet provider.

In linea di massima si può dire che il gestore del blog può essere ritenuto responsabile quando ha un effettivo potere di intervento rispetto alla pubblicazione. Ad esempio quando si è di fronte a un blog moderato. Ma in linea di massima si cerca di evitare un coinvolgimento del prestatore del servizio rispetto a quella che è l'attività che fa il singolo.

\subsubsection{Diritto all'oblio}

L'ultimo tema del profilo della diffamazione è il diritto all'oblio. Cos'è il diritto all'oblio? Il diritto all'oblio è il diritto ad essere dimenticati, è quel diritto che riguarda tutta la normativa in tema di protezione dei dati personali, in tema di riservatezza, ed è il diritto ad ottenere che le informazioni pubblicate, le informazioni conosciute che ci riguardano, siano informazioni effettivamente attuali.

Nel momento in cui, ed è il caso di cui si parlava prima, della pubblicazione di informazioni relative a procedimenti penali non più attuali, oppure la pubblicazione di notizie relative ad attività non più effettive, oppure la pubblicazione di notizie su persone che hanno cambiato la propria attività e che non desiderano più essere ricordati. Ecco, il diritto all'oblio è il diritto a far sì che ciò che transita sulla rete, ciò che transita in generale sui mezzi di informazione, siano notizie effettivamente attuali, effettivamente ancora di qualche interesse.

\subsubsection{Hate Speech}

Un altro tema che preoccupa con riferimento alle pubblicazioni di informazioni su internet è il cosiddetto hate speech.
Con Hate Speech si intende l'incitamento all'odio, all'intolleranza, alla discriminazione (razza, etnia, religione, genere, orientamento sessuale).

Hate speech è una espressione che è stata elaborata negli anni dalla giurisprudenza americana e che viene tradotta in italiano con la formula d'incitamento all'odio e indica quel genere di parole e di discorsi che hanno l'unica funzione di esprimere odio o intolleranza nei confronti di una persona o di un gruppo e che rischiano però di provocare delle reazioni violente contro quel gruppo o da parte di quel gruppo.

Nel linguaggio ordinario questa espressione indica un genere di offesa fondata su qualunque tipo di discriminazione ai danni di un certo gruppo. La condanna dello hate speech è condivisa anche se si pone, anche in questo caso, un tema di rapporto fra il diritto alla libertà di espressione e la tutela di altri interessi. Il tema è di grandissima attualità perché negli ultimi anni questa modalità di espressione si sta diffondendo sul web e quindi si sta verificando effettivamente un abuso del web per questo tipo di obiettivo.

Non esiste una normativa generalizzata a tutela di coloro che vengono colpiti dallo hate speech. Esistono azioni di vario tipo che vengono svolte sulla rete. 

Web e social networks in linea di massima hanno elaborato delle linee guida che sono diverse a seconda dei vari social networks. In particolare le grandi aziende come Google e Facebook hanno affidato la compilazione delle norme di utilizzo dei servizi a un gruppo di lavoro specifico, da questo gruppo di lavoro sono state elaborate le linee guida che sono consultabili agevolmente sui suddetti social network.

YouTube vieta esplicitamente lo hate speech inteso secondo una definizione generale di linguaggio offensivo di tipo discriminatorio. Facebook ha in linea di massima lo stesso principio però ammette che sia possibile pubblicare dei messaggi che abbiano dei chiari fini umoristici o satirici. E cioè si può trattare di un'offesa, si può trattare di un messaggio offensivo e aggressivo ma se viene fatto con un chiaro intento umoristico satirico non c'è più la necessità di rimuoverlo, è legittimato. Twitter non vieta esplicitamente lo hate speech e non lo cita nemmeno tranne in una nota sugli annunci pubblicitari in cui si specifica però che le campagne politiche contro un candidato generalmente non sono considerate hate speech.

Vale la pena di evidenziare che i grandi social network di cui stiamo parlando non sono radicati di base in Italia pur avendo generalmente delle sedi localizzate in Italia e quindi hanno come criteri ispiratori dei principi più che altro legati alla cultura giuridica e non solo degli Stati Uniti.

C'è da dire che non è semplice in concreto valutare tra i contenuti che vengono pubblicati quali sono semplicemente offensivi in quanto critici e quali invece hanno come possibile effetto quello di suscitare delle reazioni violente. E questo è un motivo per cui non è semplice effettivamente fare delle valutazioni.

In Italia si può consultare il sito nohatespeechmovement.org che è un sito promosso dalla comunità europea per sollevare il problema e sensibilizzare. E ancora in Italia proprio il sito nohatespeech.it che è un sito diretto alla sensibilizzazione. Questo nell'idea che la formazione, la sensibilizzazione e la cultura siano gli strumenti preventivi migliori rispetto a questo tipo di fenomeno.

Esiste dal punto di vista normativo la \textbf{legge 205/93} per la repressione dei crimini d'odio che prevede che vengano puniti tutti i reati commessi con finalità di discriminazione o di odio etnico, nazionale, razziale, religioso, ovvero per agevolare le attività di organizzazioni, associazioni o movimenti o gruppi che hanno tra i loro scopi le stesse finalità.

Questo principio, la finalità di odio ed discriminazione, costituisce un aggravante per i reati generali e costituisce un vero e proprio reato a parte nel caso di istigazione.

La \textbf{legge 38/2001} ha esteso alle minoranze linguistiche gli stessi principi già previsti dalla legge 205/93.

%20;50
\subsubsection{adescamento di minori}

Altro tema importante oggi su internet è quello che viene chiamato adescamento di minori. Consiste nell'\textbf{avvicinamento di bambini adolescenti a mezzo internet}. Questo avvicinamento avviene per scopi sessuali e avviene con un sistema di conquista della fiducia del minore che fa dei danni rilevantissimi. Il fenomeno è in aumento negli ultimi anni e questo aumento è agevolato dal fatto che anche i minori, anche i bambini molto piccoli, hanno la disponibilità degli strumenti di connessione alla rete. E' di tale portata che sono state costituite delle task force dedicate alla prevenzione e repressione di questo fenomeno.

In Italia questo condotta è considerata un crimine ed è prevista una specifica tutela per i minori dei 16 anni da parte \textbf{ dell'articolo 609 undiecis del codice penale}. Si tratta di una norma che è stata inserita a seguito della ratifica della convenzione di Lanzarote sulla tutela dei minori e che appunto prevede la punibilità di chiunque attraverso internet contatti un minore per gli scopi che sono stati detti.

La cosa interessante è che è sufficiente il tentativo, significa che non è necessario che ci sia effettivamente un contatto a seguito di questo adescamento ma già il solo fatto di raggiungere il minore per gli scopi che abbiamo detto attraverso lo strumento di internet è un reato.

L'obiettivo è quello di offrire una tutela penale ai minori di 16 anni che siano vittime di questi comportamenti seduttivi perpetrati attraverso mezzi di comunicazione a distanza. Il limite di età dei 16 anni è stato individuato tenendo conto della particolare fragilità dei ragazzi minori di quell'età. È evidente che se poi il contatto avviene e avvengono altri fatti esistono altre norme del codice penale per reprimerli.

Lo spunto di riflessione di questa prima parte della lezione è: quali cautele per pubblicare su internet?

\section{La tutela dell'accesso a internet}

Passiamo adesso al secondo argomento, la tutela dell'accesso a internet. Si è parlato a lungo in questa lezione e nella lezione precedente delle possibilità che internet offre, dei rischi in cui si incorre su internet, nel senso che internet può essere uno strumento di violazione di altri diritti. Occorre però chiedersi, e il mondo si chiede, le organizzazioni internazionali si chiedono in che termini e fino a che punto l'accesso a internet debba essere tutelato.

Nel rapporto ONU del 2011 si è evidenziato come internet sia un mezzo chiave per la libertà di espressione. Il relatore del rapporto ha chiarito con grande fermezza che poiché internet è diventato uno strumento indispensabile per realizzare una serie di diritti umani, la lotta contro la disuguaglianza e per accelerare lo sviluppo e il progresso umano, garantire l'accesso universale a internet dovrebbe essere una priorità per tutti gli Stati.

Questo si legge nel testo redatto da Frank Larue, il relatore speciale delle Nazioni Unite, che ha scritto un documento sulla promozione e la tutela del diritto alla libertà di opinione e di espressione. Quindi l'accesso a internet deve essere considerato una priorità per gli Stati.

Nel rapporto si richiama come dimostrazione di ciò che viene asserito la cosiddetta primavera araba nel Nord Africa e nel Medio Oriente, che è stata proprio agevolata dalla possibilità di essere collegati a internet. E tuttavia è anche certo che non in tutti gli Stati in questo momento internet sia effettivamente accessibile. In diversi Stati del Medio Oriente in cui internet è sottoposto a limitazioni, in altri Stati come l'Estonia, la Costa Rica, la Francia, che hanno dichiarato che internet è un diritto fondamentale. 

Quindi ci sono diverse posizioni da parte degli Stati, non c'è un approccio univoco e certamente non tutti gli Stati considerano come priorità assoluta il fatto di consentire l'accesso a internet.

È interessante però il fatto che nel rapporto di Frank Larue si riconosce che \textbf{è possibile introdurre delle limitazioni a internet. In particolare le limitazioni però devono essere eccezionali, legate a norme chiare e accessibili a tutti, e devono essere, laddove sono inserite, dirette a tutelare altri diritti di uguale rilevanza}. Ecco quindi che anche nel rapporto ONU ritornano gli stessi principi di cui abbiamo parlato, internet come strumento principe per veicolare la libertà di informazione, ma anche strumento che può avere delle conseguenze pericolose o dannose.

In quest'ottica è possibile introdurre delle limitazioni ma ciò che è fondamentale in paesi democratici, è che queste limitazioni siano delle limitazioni ben chiare, con delle finalità ben chiare e dirette a tutelare altri diritti fondamentali della stessa importanza, della stessa rilevanza, della libertà del diritto, della libertà d'espressione e che queste limitazioni siano introdotto con delle norme dello stato che devono essere estremamente chiare, accessibili a tutti e quindi facilmente individuabili.

Il fatto che nel rapporto si bilancino la libertà di espressione e delle possibilità di limitazioni, è conforme a quello che già avevamo potuto verificare nella lezione precedente dei principi sanciti dalle convenzioni internazionali, convenzione ONU, convenzione CEDU e costituzione italiana, rispetto alle possibili limitazioni del diritto alla libertà di espressione.

Lo spunto di riflessione è se internet sia in grado o meno di cambiare il rapporto fra libertà di espressione e altri diritti.

\subsection{Internet è un nuovo diritto fondamentale?}

Ultimo argomento di questa lezione è se internet è un nuovo diritto fondamentale. Alla luce di tutto quello che abbiamo potuto vedere, degli aspetti che abbiamo potuto esaminare, la domanda se internet di per sé sia o meno un diritto fondamentale è una domanda di estrema importanza.

I termini della questione sono: \textbf{l'accesso a internet è un diritto umano fondamentale di ultima generazione oppure è uno strumento che agevola il godimento di altri diritti?}

La questione non è una questione di soluzione semplice, certamente è un diritto fondamentale la libertà di espressione e l'abbiamo detto più volte. Certamente internet è uno strumento che consente di svolgere la libertà di espressione al massimo grado e quindi certamente avere l'accesso a internet consente di svolgere questo diritto fondamentale, ma da questo si può dedurre che l'accesso ad internet sia esso stesso un diritto?

Da ricordare che si suole dividere i diritti fondamentali in diritti di prima, seconda, terza generazione e così via. È chiaro che i diritti di ultima generazione, i diritti nuovi come questo sono dei diritti che si può pensare di tutelare soltanto dopo aver già garantito i diritti fondamentali primari.

Chi sostiene che l'accesso a internet sia un diritto fondamentale lo fa sul presupposto di una massima diffusione di internet, chi sostiene invece che internet sia soltanto uno strumento che agevola il godimento di altri diritti lo fa sul presupposto che internet in effetti è uno strumento e che internet in effetti non è diffuso ovunque e che non tutti gli stati, appunto come si diceva, hanno la stessa sensibilità rispetto al modo di fornire uno strumento.

Internet può essere, lo abbiamo detto, uno strumento attraverso cui violare altri diritti e quindi dato che si tratta di uno strumento che può essere usato positivamente ma anche negativamente, allora forse va considerato come uno strumento.

Le differenti opinioni possono essere facilmente individuabili anche tenendo conto di quelle che sono le normative che sono state adottate in diversi paesi. In Francia, alcuni anni fa, fu adottata la cosiddetta legge Hadopi che ormai è al tramonto che prevedeva l'obbligo per i fornitori di connettività, per i provider di sconnettere coloro che scaricavano musica o altri contenuti protetti dal diritto d'autore per più di tre volte. Quindi si dava agli internet service provider un obbligo di controllo che non è previsto in altri ordinamenti. 

Italia la cosiddetta legge Stanca considera un diritto per i disabili di avere l'accessibilità ai internet.

Nell'Unione Europea l'emendamento 138/46 in tema di diritto d'autore si occupa specificamente della violazione del diritto d'autore su internet.

Ecco quindi che gli aspetti che vengono trattati sono di vario genere.\par
Dal punto di vista dei fenomeni la primavera araba è stato proprio un esempio estremamente importante per il mondo per comprendere come l'utilizzo di internet nel bene e nel male abbia consentito di sviluppare una rivolta, un moto verso la libertà, verso un cambiamento di carattere democratico all'interno di quelle popolazioni. La maggior parte delle notizie che sono state comunicate, veicolate, che hanno consentito a quelle popolazioni di parlarsi e di andare avanti è stata resa possibile proprio dal web, proprio dalle comunicazioni su internet. E possiamo ricordare che ci fu durante la primavera araba proprio l'uccisione di un blogger proprio perché consentiva la comunicazione di informazioni che facevano sapere che cosa stava accadendo.

In periodi più recenti possiamo invece ricordare ciò che è accaduto e che sta accadendo in Turchia. Nel corso di quest'ultimo anno con l'avvicinarsi delle elezioni presidenziali è stata introdotta una legge di controllo all'interno della Turchia su internet e in particolare su youtube e twitter. Si tratta di norme che prevedono il controllo e che sono state ufficialmente motivate dalla necessità di reprimere violazioni e dalla necessità di tutelare dell'ordine pubblico.

In altri termini le ragioni che sono state date per l'introduzione di queste norme di carattere restrittivo sono delle ragioni che in astratto corrispondono a quei principi di cui abbiamo parlato prima riportati dal rapporto ONU. Ricordate che il rapporto ONU dice che riconosce la possibilità di introdurre delle limitazioni, purché queste limitazioni siano finalizzate a una tutela di altri diritti fondamentali, purché queste limitazioni siano introdotte con leggi ben chiare e comprensibili e accessibili a tutti. Il principio a cui si è fatto riferimento in Turchia era un principio di questo tipo.
%36:30
Quindi in teoria non c'era motivo per contestarlo. Il problema è che l'introduzione di queste limitazioni aveva come effetto quello di introdurre una forma di censura e quello che si è ritenuto motivo per l'introduzione di questa forma di censura era un motivo di carattere politico, un controllo sugli avversari politici.

In realtà questa normativa non ha resistito a lungo perché la Corte Costituzionale turca ha ordinato o ne ha dichiarato l'illegittimità consentendo la riapertura di YouTube e poi di Twitter. Quindi all'interno stesso della Turchia c'è stato un movimento di diritto che ha bloccato questi tentativi.

Per quanto riguarda la Turchia, c'è anche da dire che laddove dovesse effettivamente andare avanti la candidatura per l'ingresso all'interno dell'Unione Europea, la Turchia sarebbe comunque tenuta a rispettare quelli che sono i vincoli dettati dall'Unione Europea tra cui anche la tutela dei diritti fondamentali e quindi anche quello che riguarda la libertà di informazione. Non sarebbe più possibile introdurre delle limitazioni che nel resto d'Europa non sono ammissibili.

Ricordo il caso Data Gate di cui abbiamo parlato nella lezione precedente e che ricorderete è un caso nel quale vi è stata da parte della National Security Agency degli Stati Uniti la raccolta di informazioni che hanno riguardato diversi paesi e anche capi di Stato e rispetto ai quali c'è stata una presa di posizione dall'ONU.

Lo spunto di riflessione per quanto riguarda internet è libertà da una parte e governance dall'altra, come gestirli? Questa è la domanda che dovete porvi, questa è la domanda a cui è difficile rispondere e quindi si potrà ancora riflettere a lungo.