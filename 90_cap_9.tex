\chapter{Lezione 9 - Identità digitale}

Nella lezione di oggi parleremo di identità digitale. 

Gli argomenti della lezione di oggi:
\begin{itemize}
    \item identità digitale 
    \item identità digitale e pubblica amministrazione 
\end{itemize}

\textbf{L'identità digitale non è la semplice trasposizione elettronica di quella fisica.} 

Per parlare di identità digitale occorre definire:
\begin{itemize}
    \item l'identità personale 
    \item l'identità digitale
    \item il processo tecnologico di identificazione
\end{itemize}

Si tratta di tre argomenti del tutto diversi anche se evidentemente ci sono dei collegamenti stretti.

\section{Identità digitale}
\subsection{Identità personale}

\textbf{L'identità personale è l'insieme dei caratteri fisici e psicologici che rendono una persona quella che è e diversa da ogni altra.} 

Quando si parla di identità personale si fa riferimento ad una serie di elementi che sono stati individuati nel corso del tempo. L'identità personale è un insieme di caratteristiche che vengono anche prese in considerazione dall'ordinamento nei rapporti tra l'individuo e lo stato.
L'identità personale è un complesso di caratteristiche che rappresentano, che individuano la personalità. L'identità personale quindi è un coacervo, un insieme di elementi del tutto diversi che però individuano un soggetto, una persona come assolutamente unica.

Nel rapporto con lo stato vi sono diversi profili di tutela dell'identità personale:

\begin{itemize}
    \item l'individualità. L'identità personale tutelata è quella del singolo soggetto, del singolo individuo, diverso appunto da ogni altro. Il fatto di essere diverso e unico nel rapporto con gli altri è fondamentale per tutti quelli che sono i rapporti giuridici che vengono instaurati dalla persona con i consociati, coloro che vivono nello stesso ambiente.
    \item La fama. Un individuo può essere più o meno conosciuto nel gruppo dei consociati a cui fa riferimento e la sua fama, le modalità con cui è conosciuto sono per lui essenziali. 
    \item la credibilità
    \item la reputazione
    \item il credito inteso come rapporto con gli altri ma anche inteso come modalità di gestione di diritti, doveri e di attività nel rispetto alle altre persone.
\end{itemize}

Vicino al concetto di fama vi è la credibilità del soggetto e ancora la reputazione del soggetto. Queste due sono caratteristiche inscindibili che consentono di rappresentare l'individuo con se stesso e rispetto agli altri in un modo che è considerato più o meno positivo. Ognuno ha diritto a che la propria credibilità e la propria reputazione siano effettivamente quelle corrispondenti a ciò che l'individuo sente di essere.

Quindi in sintesi l'identità personale è un elemento caratterizzante di ogni persona ed è connesso alle garanzie che sono proposte dall'ordinamento, garanzie di carattere costituzionale e anche date da altre norme. In particolare si richiamano quelli che sono i principi del codice civile in tema di diritti della persona, in tema di individualità della persona. Si richiamano quelli che sono i principi sanciti dal codice penale per la tutela della persona rispetto non soltanto della persona fisica ma anche della persona come soggetto che ha una certa reputazione, una certa credibilità, una certa fama da difendere.

Da ultimo, nella più recente evoluzione, la tutela dell'identità personale è anche stata data con la tutela dei cosiddetti dati personali. La normativa, meglio nota come normativa tutela della privacy, si occupa proprio di difendere, di tutelare l'individuo rispetto a un utilizzo non corretto fatto dei dati personali che lo indicano, che lo individuano come soggetto unico al mondo nella gestione dei rapporti con gli altri.

Individuata quella che è l'identità personale si tratta di verificare e di capire che cos'è l'identità digitale e in che rapporto l'identità digitale si pone rispetto all'identità personale. 

\subsection{L'identità digitale}

\textbf{L'identità digitale è la rappresentazione virtuale di un'identità reale che è utilizzabile nell'interazione elettronica.} 
%5:30

Ecco quindi che individuiamo immediatamente quella che è la prima caratteristica di un'identità digitale, la rappresentazione virtuale. L'identità digitale non è qualcosa di diverso rispetto all'identità personale, si tratta sempre della rappresentazione, dell'individuazione di una persona che ha una sua realtà. Quello che interessa in questa relazione è che l'identità digitale è l'elemento che è utilizzabile in una interazione elettronica, cioè sui rapporti che vengono instaurati nella gestione di relazioni elettroniche su internet e quant'altro.

Quali sono i legami con l'identità reale? 

Si parla di anonimato o di possibilità di associazione tra le informazioni dell'identità digitale e la persona a seconda del modo in cui queste informazioni vengono proposte. Nel momento in cui le informazioni che individuano l'identità digitale sono delle informazioni che non consentono un facile collegamento con l'effettiva identità della persona, si può parlare di anonimato.

Quando invece questi elementi di identità digitale sono degli elementi che consentono un'associazione più o meno ampia con l'individuazione della persona fisica c'è un legame che si può ritenere più stretto.
L'identità digitale può essere forte oppure debole.

Questa distinzione tra identità forte e identità debole riguarda le caratteristiche distintive della persona individuata nel mondo digitale. Si parla di identità forte se questa identità digitale coinvolge un elevato numero di caratteristiche distintive. Si parla di identità debole se invece la descrizione dell'individuo contiene delle caratteristiche inferiori.

L'identità digitale è collegata con l'identità reale nella misura in cui consente l'accesso a servizi digitali da una parte e mette a rischio la privacy, la riservatezza della persona dall'altra.
 %8:35
L'accesso a strumenti, a servizi, a spazi nel mondo digitale è consentito attraverso l'identità digitale.

L'identità digitale è come detto che fa riferimento a informazioni di carattere personale. Quando si parla di informazioni c'è da dire che le informazioni personali possono essere replicate senza limiti e con una grande precisione. Quindi le informazioni che vengono utilizzate per costruire, per evidenziare, per individuare l'identità digitale sono informazioni che possono essere riprodotte nella costruzione di identità digitali diverse che consentono per esempio l'accesso ad altri servizi, ad altri spazi.

Un tema che si è posto frequentemente nei primi tempi in cui si parlava del rapporto fra identità digitale e identità personale è se si possa parlare di tante identità digitali quante sono le agglomerazioni di dati che consentono appunto di fare riferimento alla persona e che vengono utilizzate in varie situazioni.%10:13

In effetti non è corretto parlare di più identità digitali, l'identità è una ed è quella della persona. Questa identità della persona insieme alle caratteristiche, i dati personali e quant'altro che la  individuano ha un modo di essere rappresentato nel mondo digitale e quindi questa è l'identità digitale.

Nell'utilizzo dell'identità digitale per accedere a servizi, per fare qualunque tipo di attività nel mondo digitale  esiste un processo tecnologico di identificazione.

\textbf{Sono necessarie delle credenziali di identità affidabili dal punto di vista tecnico e giuridico.} Identificare una persona significa riconoscerne tratti caratteristici individuali. Nell'utilizzo di un'identità digitale non c'è la possibilità di una conoscenza diretta della persona e quindi di un riconoscimento diretto della persona, né c'è la possibilità di verificarne un documento di identità, per questa ragione e per le caratteristiche peculiari dei sistemi informatici, sono state introdotte delle credenziali di identità dedicate all'identità digitale.

Queste credenziali di identità sono affidabili se rispondono a determinate caratteristiche relative agli aspetti sia giuridici sia tecnici e sono quelle caratteristiche che consentono effettivamente al sistema a cui si chiede il riconoscimento dell'identità digitale di poter dire sì, effettivamente quell'identità digitale corrisponde ad una certa persona.

Il processo di identificazione prevede diversi passaggi:
\begin{itemize}
    \item verifica dell'autenticità. La verifica dell'autenticità delle credenziali significa verificare che quelle credenziali sono state rilasciate da un soggetto che è quello che appare e che si tratta di credenziali che non sono state falsificate dal punto di vista tecnico. Questa è una prima valutazione che possiamo definire di carattere tecnico.
    \item occorre verificare i contrassegni dell'autorità emittente. Le credenziali dell'autenticazione che vengono emesse da un soggetto generalmente contengono dei contrassegni che individuano l'autorità, il soggetto che ha emesso quest'ultimo. Nel processo di tecnologico di identificazione, nella verifica di affidabilità delle credenziali, occorre che queste siano state emesse da un'autorità che sia legalmente riconosciuta e che abbia un potere certificatorio rispetto alla relazione fra l'identità personale e l'identità digitale. Quindi la seconda verifica rispetto all'affidabilità dell'identità digitale è la verifica che riguarda l'esistenza e l'effettività dei contrassegni dell'autorità emittente.
    \item comparazione tratti caratteristici. Per essere effettivamente certi che quell'identità digitale corrisponde effettivamente alla soggetto è necessario comparare i tratti caratteristici, occorre comparare quelle che sono le informazioni contenute nell'identità digitale con le informazioni relative alla persona.
\end{itemize}

\subsubsection{livelli di autenticazione}
 
 \begin{itemize}
     \item qualche cosa che so, una password.
     \item qualche cosa che ho, un chip
     \item qualcosa che sono impronte biometriche
 \end{itemize}
 %15:05
 Queste diverse caratteristiche caratterizzano i diversi sistemi di autenticazione. Qualche cosa che so è la password, è il sistema di autenticazione più semplice, è quello previsto da tutta una serie di norme, tra cui la normativa della sicurezza sulla privacy.\par
 Qualche cosa che ho, un chip, ad esempio una smart card che contiene un chip in cui sono contenute le informazioni relative all'identità personale e digitale.\par
 Qualche cosa che sono, le impronte biometriche che è il sistema di autenticazione più sicuro, perché fa riferimento all'utilizzo di caratteristiche assolutamente personali, ad esempio l'impronta dell'iride, l'impronta digitale e quant'altro. L'utilizzo di questo tipo di credenziali richiede una particolare attenzione per evitare che ci sia una sovraesposizione della persona e quindi un utilizzo abusivo possibile di queste caratteristiche personali.\par
\subsubsection{processo di identificazione }
 
 Il processo di identificazione prevede:
 \begin{itemize}
     \item un'autenticazione, cioè capire quale soggetto, persona, server, postazione e terminale sta richiedendo l'accesso a un servizio.
     \item L'autorizzazione, cioè stabilire se un'identità ha il diritto di accedere alle risorse richieste.
     \item a verifica, cioè accertare nel caso di esseri umani la validità dei tratti caratteristici.
 \end{itemize}
 
 Spunto di riflessione: quale relazione tra l'identità reale e l'identità digitale?
 
 \section{Identità digitale e pubblica amministrazione.}
 Passiamo al secondo argomento.  Nel rapporto fra il cittadino e la pubblica amministrazione, l'identità personale  entra in gioco, ma anche l'identità digitale ha un ruolo particolare. Quali sono le norme di riferimento?\par
 \subsection{Norme di riferimento}
 Norme di riferimento:
\begin{itemize}
    \item la costituzione dove si parla della dignità della persona e in tutte le sue norme sulla persona. il codice civile, in particolare gli articoli, i principi di riferimento per l'individuazione della persona e la sua tutela. 
     \item Il codice della privacy, cioè il decreto legislativo 196/2003 che tutela i dati personali ed è qualcosa di relativamente recente. 
    \item Il codice dell'amministrazione digitale, decreto legislativo 82/2005, si tratta della normativa principale di riferimento di quella che è l'attività basata sull'informatica per la pubblica amministrazione. 
    \item il codice penale, che prevede una serie di norme che costituiscono reato a tutela della persona e anche a tutela della identità della persona, a partire dalla sanzione per i casi di sostituzione di persona, che ha un riflesso anche nel caso di sostituzione della persona digitale, per proseguire con quelle che sono le forme di tutela della fama, dell'onore, della reputazione della persona.  
\end{itemize}
 
 In concreto quali sono gli strumenti che individuano l'identità? 
 
 \begin{itemize}
    \item Carta di identità elettronica e carta nazionale servizi, dette CIE e CNS sono disciplinate dal codice dell'amministrazione digitale.
 \item SPID o meglio sistema pubblico di identità digitale. E' stato introdotto nel codice dell'amministrazione digitale dal cosiddetto decreto del fare del 2013 ed è la prima volta che si introduce all'interno di una normativa la definizione di identità digitale. 
 \item la PEC, posta elettronica certificata, è il sistema di posta elettronica che è equiparato alla posta raccomandata. 
 \item la firma digitale che è un sistema di identificazione e di validazione e messa in sicurezza di documenti informatici. 
 \end{itemize}
 \textbf{Tutti questi sistemi richiedono l'individuazione preliminare dell'identità personale.}\par
 Si tratta di sistemi di riconoscimento dell'identità digitale, di riconoscimento di quell'identità che viene utilizzata su internet nella gestione del rapporto con la pubblica amministrazione, che richiedono l'individuazione preliminare dell'identità personale e poi comportano la verifica della corrispondenza fra l'identità digitale e l'identità personale. 

\subsubsection{La carta di identità elettronica, CIE, e la carta nazionale servizi, CNS}
 
 La carta di identità elettronica, CIE, e la carta nazionale servizi, CNS, contengono informazioni identificative. All'interno dei sistemi di carta di identità elettronica e carta nazionale servizi, vi sono tutte le informazioni che consentono di identificare la persona. In particolare, queste informazioni sono contenute su un microchip o su una banda ottica in una smart card. Quindi le informazioni sono contenute su un supporto esterno al sistema informatico che per essere utilizzato richiede una apparecchiatura particolare. 
 PIN e PUC. Questi microchip o questi sistemi o smart card o altri sistemi che vengono utilizzati per il rilascio di carta di identità elettronica e carta nazionale servizi, hanno dei codici. Oltre ad avere le informazioni che riguardano la persona all'interno di microchip o banda ottica, vi sono dei codici, codice PIN e codice PUC, che devono essere utilizzati insieme al supporto esterno per poter consentire al sistema di verificare e validare la corrispondenza fra l'identità personale e l'identità digitale e per consentire l'accesso ai servizi della pubblica amministrazione. \par
 %22:52
 Questi sistemi sono inseriti all'interno del codice dell'amministrazione digitale, c'è da dire che il loro utilizzo non è mai veramente decollato un po' per la complessità della gestione e soprattutto per la difficoltà in questa fase di passaggio fra la gestione tradizionale e una gestione virtuale. La difficoltà di portare un grande numero di cittadini all'effettivo utilizzo di questi sistemi. \par
 Per questo con il decreto del fare è stato introdotto il \textbf{sistema pubblico per la gestione dell'identità digitale SPID.}\par 
 
 Il sistema pubblico per la gestione dell'identità digitale si fonda su quelle che sono le caratteristiche già individuate. È necessario che il sistema che si utilizza contenga le informazioni identificative ma anche di comunicazione.\par
 È la prima volta che è stato inserito all'interno di un testo di legge la definizione di identità digitale. In passato l'elaborazione che riguardava l'identità digitale era un'elaborazione esclusivamente dottrinaria, nella normativa si faceva riferimento  agli strumenti di utilizzo dell'identità digitale, ma non era mai stata individuata. Con l'introduzione del sistema pubblico per la gestione dell'identità digitale si fa uno sforzo per portare l'utilizzo dei sistemi a tutti i cittadini.\par
 Il cittadino secondo quello che è il sistema SPID potrà ottenere, nell'idea del legislatore, una o più identità digitali. In realtà potrà ottenere dei documenti digitali, dei passaporti digitali che conterranno alcune informazioni identificative obbligatorie, ad esempio il codice fiscale, il nome, il cognome, il luogo di nascita, la data di nascita, il sesso e potrà ottenere delle informazioni utili per comunicare con il soggetto titolare dell'identità, per esempio un numero di telefono o un indirizzo di posta elettronica.\par
 Questa identità conterrà una o più credenziali utilizzate per accedere ai servizi in modo sicuro.\par
 
 Il fatto di poter avere una o più identità digitali, uno più passaporti digitali, può servire a seconda del tipo di servizio a cui si accede, a seconda dell'utilizzo che si deve fare di quella identità.\par
 Per l'attribuzione delle identità digitali occorre un gestore dell'identità che è un soggetto pubblico o privato che, previo accreditamento presso l'Agenzia per l'Italia Digitale, si dovrà occupare di creare e gestire le identità digitali.\par
 
 E' stato previsto anche un gestore di attributi qualificati, cioè un soggetto che per legge è titolato a certificare alcuni attributi, un titolo di studio ad esempio, un'abilitazione professionale, un'appartenenza a un determinato albo.  \par
 Il cittadino che desidererà ottenere un'identità digitale dovrà rivolgersi a uno dei gestori di identità digitale accreditati, il quale dovrà procedere prima di tutto con un riconoscimento del cittadino attraverso una verifica diretta, una verifica personale. Ecco la relazione esistente fra l'identità personale e l'identità digitale, si capisce che l'identità digitale non è qualcosa di diverso ma è una modalità di espressione dell'identità personale. La verifica da parte del gestore prevederà anche il controllo in tempo reale della coerenza di tutti gli attributi contenuti nell'anagrafe nazionale della popolazione residente e questa forma di identificazione preliminare ed è ciò che serve per evitare di creare delle identità con attributi non corretti. 
 Fatta questa identificazione preliminare il gestore degli attributi qualificati potrà procedere per il rilascio dell'identità digitale.\par 
 Cosa conterrà?:
 \begin{itemize}
     \item credenziali di sicurezza, cioè le credenziali che consentono al sistema di verificare effettivamente la corrispondenza e la correttezza dei dati.
     \item il sistema di autenticazione è il sistema che permette appunto il riconoscimento, l'autenticazione del soggetto presso il sistema a cui il soggetto vuole accedere.
     \item la condivisione minima degli attributi prevede che vi siano delle informazioni che comunque e sempre devono essere inserite qualunque sia il tipo di identità digitale rilasciata nell'ambito del sistema pubblico di identità digitale. Ci sono delle caratteristiche individuali univoche, ad esempio il codice fiscale è certamente una caratteristica univoca in qunato al nome e cognome sono collegate anche altri dati quali il codice fiscale o la residenza ed ecco che le caratteristiche minime per l'individuazione univoca della persona sono state individuate.
 \end{itemize}

L'innovazione del sistema pubblico di identità digitale è che si cerca di staccarsi dalla necessità dell'utilizzo di una specifica tecnologia.\par    

Questo è un sistema innovativo introdotto da poco e bisognerà vedere effettivamente la concreta diffusione che questo avrà rispetto ai cittadini, perché già in passato, erano stati previsti dei sistemi per creare la relazione tra il cittadino e la pubblica amministrazione ma si è trattato di sistemi che non sono stati usati fino ad ora compiutamente.

\subsubsection{domicilio digitale PEC}
 Altro elemento che comporta e consente un rapporto con la pubblica amministrazione e prevede comunque la preliminare individuazione della persona è il domicilio digitale PEC, la posta elettronica certificata, che è un sistema di posta elettronica. 
 La differenza tra la posta elettronica certificata e la posta elettronica tradizionale è la parvenza di certezza che si dà rispetto alla partenza, all'arrivo e all'invio della comunicazione. Per questa ragione la PEC è utilizzata in sostituzione delle raccomandate con ricevuta di ritorno in tutte quelle situazioni in cui occorre avere la certezza della partenza e la certezza dell'arrivo della comunicazione.\par
 La posta elettronica certificata può costituire anche un domicilio digitale, cioè l'attribuzione di un indirizzo di posta elettronica certificata da parte di un soggetto autorizzato e la sua iscrizione all'interno dell'anagrafe nazionale delle pubbliche amministrazioni consente di avere un domicilio digitale sicuro e certo del cittadino.\par
 Questo può permettere al cittadino di dialogare con la pubblica amministrazione su un piano esclusivamente virtuale. Il cittadino può avere la certezza che la pubblica amministrazione riuscirà a comunicare con lui attraverso il sistema informatico con una destinazione sicura, così come l'amministrazione ha con certezza un'individuazione di un domicilio digitale dove appunto poter fare le proprie comunicazioni.
 Il domicilio digitale PEC è stato attribuito a tutti i cittadini registrati nell'anagrafe nazionale. Il sistema stenta a decollare; 
 le difficoltà sono legate all'effettiva diffusione fra i cittadini e la difficoltà per alcuni, soprattutto generazioni più anziane, di interagire concretamente con questi sistemi ormai non più nuovi ma per alcuni sempre troppo nuovi. \par
 Una delle grosse difficoltà nella gestione dell'informatizzazione della pubblica amministrazione è proprio il digital divide, il divario digitale esistente  ancora oggi fra i cittadini.
 
 \subsubsection{firma digitale}
 L'ultimo sistema che prevede l'individuazione della persona è la firma digitale, che è un sistema:
 \begin{itemize}
     \item con chiavi digitali private e pubbliche univoche. Il sistema di firma digitale contiene delle chiavi digitali, una chiave privata e una chiave pubblica che sono univoche che consentono, nella relazione fra mittente e destinatario, di fare le verifiche. 
     \item l'associazione ad un documento. L'associazione della firma ad un documento significa che nel momento in cui il documento viene firmato non c'è alcun dubbio rispetto al fatto che quel documento che viene inviato e quindi ricevuto da un determinato soggetto è certamente associato a quella firma e quindi alla persona che l'ha firmato. 
     \item 
 \end{itemize}
 
 Il sistema di firma digitale è uno dei primi sistemi che è stato sperimentato ed utilizzato e ancora ad oggi viene utilizzato per la gestione dei rapporti fra soggetti privati e soggetti pubblici, ma è anche un sistema utilizzato e fondamentale che consente di inviare documenti certi. Per certi si intende documenti che sono senza dubbio riconducibili ad un determinato soggetto e documenti dei quali c'è la certezza rispetto all'autenticità del contenuto.\par
 
 Il sistema di firma digitale è un sistema per l'invio sicuro di documenti che compie la sua completa funzione quando è associato alla trasmissione del documento stesso attraverso la posta elettronica certificata. L'utilizzo di entrambi questi strumenti consente di relazionarsi con la pubblica amministrazione e di inviare istanze, domande, documenti alla pubblica amministrazione con la certezza che questi documenti vengono effettivamente ricevuti. E quindi con la certezza di adempiere a tutti gli eventuali obblighi di comunicazione certa che sono previsti da diverse norme nella relazione con la pubblica amministrazione. \par
 Quindi in definitiva si può dire che l'accesso alla pubblica amministrazione attraverso l'utilizzo di sistemi informatici comporta da un lato l'accesso a servizi erogati in rete, dall'altro servizi per i quali è necessaria l'identificazione informatica. 
 \par
 In sostanza, nell'utilizzo dei sistemi informatici il rapporto tra il cittadino e la pubblica amministrazione è un rapporto complesso. L'utilizzo dell'identità digitale nel rapporto con la pubblica amministrazione ha una duplice veste. L'identità digitale consente di accedere, come detto, ai servizi erogati in rete, servizi non semplicemente di carattere informativo, non servizi semplicemente che permettono di dare informazioni o che sono rivolti a tutti, ma in particolare servizi per i quali è necessaria un'identificazione informatica perché si tratta di servizi indirizzati ad una certa persona.\par
 Spunto di riflessione di questa parte della lezione è quali sono gli strumenti per interagire con la pubblica amministrazione. \par
 L'identità digitale quindi è una rappresentazione dell'identità personale utile nella gestione delle relazioni informatiche.